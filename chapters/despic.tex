\documentclass[
    output=paper,
    colorlinks,
    citecolor=brown,
%    hidelinks,
%    showindex
]{langscibook}

\author{Miloje Despić \affiliation{Cornell University}\orcid{}}
\title{Size of the Moving Element Matters: LBE is not Scattered Deletion }
\abstract{In this paper I investigate the size of the moving element in Left Branch Extraction (LBE). On the traditional, Direct Extraction approach (e.g., \citealt{vanRiemsdijk1978; Corver 1990; Bošković2005} etc.) a left branch element (e.g., adjective, possessor, demonstrative etc.) is directly extracted out of an argument. I compare this type of analysis to the so-called Scattered Deletion approach, on which the entire nominal argument containing the left branch element moves and leaves behind a copy. Then, at PF, the operation of copy-deletion takes place and deletes copies in such a way that the non-left branch portion of the phrase is deleted in the higher copy, while the left branch element is deleted in the lower copy (e.g., \citealt{Fanselow2002; Pereltsvaig2008; Bondarenko2018} etc.). Thus, the two approaches differ in whether they involve syntactic movement of the whole nominal argument, and thus whether (a copy of) this argument is present at a high structural position in syntax and LF. I first examine LBE in Serbo-Croatian and I argue on the basis of variable binding, weak crossover effects and scope that there is no evidence that the nominal argument is located high in syntax/LF, which supports the Direct Extraction approach. What seems to be moving in syntax is just the left branch element, not the whole nominal containing it. I also extend my analysis to Japanese and show that apparent cases of LBE in this language (\citealt{TakahashiFunakoshi2013}) also do not involve scattered deletion.  
}

\begin{document}
\maketitle

\section{Introduction}
In this paper I investigate the size of the moving element in \textit{Left Branch Extraction} (LBE hereafter). On the standard approach to LBE (e.g.,
\citeauthor{Riemsdijk1978} \citeyear{Riemsdijk1978}, \citeauthor{Corver1990} \citeyear{Corver1990}, \citeauthor{Bošković2005} \citeyear{Bošković2005} etc.) a left branch element (e.g., adjective, possessor, demonstrative etc.) is extracted out of the nominal argument, as shown in (\ref{1de}). I will call this type of analysis the \textit{Direct Extraction} (DE) approach.

\begin{exe}
\ex \label{1de}
\textit{Direct Extraction}: [  XP …. [$_{YP}$  \sout{XP}  Y]  
\end{exe}

I will compare this approach to the so-called \textit{Scattered Deletion} (SD) approach (or ‘copy-and-deletion’ approach). On this analysis, the entire phrase containing the left branch element moves and leaves behind a copy. Then, at PF, the operation of copy-deletion takes place and deletes copies in such a way that the non-left branch portion of the phrase is deleted in the higher copy, while the left branch element is deleted in the lower copy (e.g., \citeauthor{FanselowĆavar2002} \citeyear{FanselowĆavar2002}, \citeauthor{Pereltsvaig2008} \citeyear{Pereltsvaig2008}, \citeauthor{BondarenkoColin2018} \citeyear{BondarenkoColin2018} etc.) This is illustrated in (\ref{2de}). 

\begin{exe}
\ex \label{2de}
\textit{Scattered Deletion}:  [  [$_{YP}$  XP  \sout{Y}]……[$_{YP}$ \sout{XP}  Y] ]  
\end{exe}

On this approach, the size of the moving element is actually bigger than it appears. In particular, the whole nominal argument undergoes syntactic movement, but this is obscured by the PF-deletion, which deletes everything but the left branch element in the higher copy. One advantage of the SD approach is that it can deal much more straightforwardly with the so-called \textit{extraordinary} LBE (e.g., \citeauthor{Bošković2005} \citeyear{Bošković2005}). As illustrated in (\ref{3de}), this type of LBE involves non-constituent movement, since the fronted preposition and adjective do not form a constituent to the exclusion of the noun under any analysis. 

\begin{exe}
\ex \label{3de}
\gll U veliku on uđ-e   sobu. \hspace{38mm} \citep[78]{Bošković2005}\\
in big he entered room\\\\
‘He entered the big room.’
\end{exe}

While structures like (\ref{3de}) require additional assumptions for the DE approach, they are handled easily on the SD approach. Namely, what is being moved in (\ref{3de}) is the whole PP \textit{u veliku sobu} ‘in big room’, after which the noun \textit{sobu} is deleted at PF. Thus, there is no constituency issue on this analysis.\footnote{The constituency issue does not arise on the \textit{Remnant Movement} approach to LBE (\citeauthor{FranksProgovac1994} \citeyear{FranksProgovac1994}, \citeauthor{bavsic2005nominal} \citeyear{bavsic2005nominal} etc.) either, since on this analysis (\ref{3de}) involves remnant PP movement. I do not discuss this approach here, but see \citet{murphy2020left} for a recent criticism of this approach to LBE. } On the DE approach, on the other hand, one would have to assume that the preposition in (\ref{3de}) adjoins to the adjective by some process. \citet{BorsleyJaworska1988} implement this a as a restructuring operation, \citet{Corver1992} assumes that the preposition undergoes lowering, while \citet{Bošković2005} suggests the AP moves to a position c-commanding the preposition (within the PP), after which the preposition adjoins to the adjective. 
Despite this initial attractiveness and simplicity of the SD approach, I will argue in this paper that the DE approach is actually correct. The logic of my argumentation is quite simple: the two approaches differ in whether they involve syntactic movement of the \textit{nominal argument}, and thus whether (a copy of) this argument is present at a high structural position in syntax and LF. I argue on the basis of variable binding, weak crossover effects and scope that there is no evidence that the nominal argument is located high in syntax/LF. What seems to be moving in syntax is just the left branch element, not the whole noun phrase containing it. In Section \ref{s2de}, I will present data from Serbo-Croatian, a well-studied LBE language, in favor of the DE approach.\footnote{\citet{BondarenkoColin2018} provide an interesting argument in favor of the SD approach on the basis of the behavior of LBE in parasitic gaps constructions in Russian. In particular, they argue that not only movement of the whole wh-phrase, but also movement of the wh-modifier via LBE, can license a parasitic gap. I do not discuss this argument here for space reasons and because SC does not have these constructions, which are constrained by factors like aspect and negation even in Russian. Bondarenko and Davis also note that there is a case-matching requirement, whereby only an accusative object can license an accusative gap. They also note that in case of certain QP objects, parasitic gaps can be licensed even without any overt movement whatsoever.} In Section \ref{s3de}, I will show that the apparent case of LBE in Japanese \citep{TakahashiFunakoshi2013} also does not involve scattered deletion. Section \ref{s4de} is the conclusion. 

\section{LBE in Serbo-Croatian } \label{s2de}

In \citet{Despić2011,Despić2013} I argued that Serbo-Croatian (SC hereafter) does not project DP and that the possessor in (\ref{4ade}) is a simple adjunct which c-commands into the clause and thus violates Condition C.\footnote{I assume here the approach to c-command under which the segment of NP does not block the c-command relation in question (e.g., \citeauthor{Kayne1994}, \citeauthor{Despić2011} \citeyear{Despić2011} etc.) , as in the definition below:
\begin{exe}
\ex (i)	X c-command Y iff X and Y are categories, X excludes Y, and every category that dominates X dominates Y (X excludes Y if no segment of X dominates Y).  
\end{exe}}

\begin{exe}
\ex \label{4de}
\begin{xlist}
\ex[*]{
    \label{4ade}
    \gll Njegov$_{i}$ najnoviji film    je  zaista razočarao  Kusturicu$_{i}$.\\
    His latest film is really disappointed Kusturica$_{\textsc{acc}}$\\
    \glt ‘His$_{i}$ latest film really disappointed Kusturica$_{i}$.’
}
\ex[\cmark]{
    \label{4bde}
    \gll Kusturicu$_{i}$ je     njegov$_{i}$  najnoviji film     zaista razočarao     $t$.\\
    Kusturica$_{\textsc{acc}}$ is his latest film really disappointed\\
}
\end{xlist}
\end{exe}

Now, whether or not SC has DP is actually not relevant for the main point of this paper. The important observation is the contrast between (\ref{4ade}) and (\ref{4bde}). That is, (\ref{4bde}) in which the R-expression is fronted is acceptable on the given co-indexation, unlike (\ref{4ade}). I argued that in (\ref{4bde}) \textit{njegov} ‘his’ does not c-command \textit{Kusturicu}, so no Condition C violation arises. Also, there is no Condition B violation in (\ref{4bde}) either since the pronoun is free in its binding domain (i.e., NP), given the definition of Condition B in (\ref{5de}), which I adopted:

\begin{exe}
\ex \label{5de}
Condition B: a pronoun is free in its own predicate domain (i.e., phrase). An element is free if it is not c-commanded by a coindexed NP.
\end{exe}

Again, regardless of whether or not SC has a DP or whether my assumptions about binding in SC were correct, the simple observation is that structures like (\ref{4ade}) which exhibit Condition-C-like effects, become acceptable when the nominal in the object position is fronted. The clearest example of this effect is the following SC idiom, in which the fronted quantifier can bind the pronominal possessor in the subject:

\begin{exe}
\ex \label{6de}
\gll Svakome$_{i}$  je njegova$_{i}$ muka najveća  t$_{i}$.\\
Everyone$_{\textsc{dat}}$ is his$_{\textsc{nom}}$ trouble$_{\textsc{nom}}$ greatest\\\\
‘To everyone$_{i}$ his$_{i}$ trouble is the greatest.’ \\
(‘Everyone$_{i}$ thinks that his$_{i}$ trouble is the greatest.’)
\end{exe}

This binding is of course impossible if the quantifier stays in situ:

\begin{exe}
\ex \label{7de}
\gll *Njegova$_{i}$ muka   je  najveća  svakome$_{i}$.\\
His trouble is greatest everybody$_{\textsc{dat}}$\\\\
\end{exe}

The same contrast can be observed between (\ref{8ade}) and (\ref{8bde}) which involve object QPs like \textit{svakog generala} ‘every general’:

\begin{exe}
\ex \label{8de}
\begin{xlist}
\ex \label{8ade}
\gll Svakog generala$_{i}$ njegovi$_{i}$ vojnici vole t$_{i}$.\\
Every$_{\textsc{acc}}$ general$_{\textsc{acc}}$ his$_{\textsc{nom}}$ soldiers$_{\textsc{nom}}$ love\\\\
‘Every general is loved by his soldiers.’

\ex \label{8bde}
\gll *Njegovi$_{i}$ vojnici vole svakog generala$_{i}$\\
His$_{\textsc{nom}}$ soldiers$_{\textsc{nom}}$ love every$_{\textsc{acc}}$ general$_{\textsc{acc}}$\\\\

\end{xlist}
\end{exe}

Importantly, quantifiers like \textit{svaki} ‘every’ can undergo LBE in SC, just like adjectives or demonstratives:

\begin{exe}
\ex \label{9de}
\gll Svaku$_{i}$ je Milan pročitao [ t$_{i}$  knjigu].\\
Every$_{\textsc{acc}}$ is Milan$_{\textsc{nom}}$ read book$_{\textsc{acc}}$\\\\
‘Milan read every book.’ 
\end{exe}

The question is then what happens if instead of fronting the whole QP \textit{svakog generala} ‘every general’ as in (\ref{8ade}), only the quantifier ‘every’ moves via LBE, as in (\ref{10de}). The two approaches to LBE discussed here make different predictions about this example. On the SD approach, there should be no difference in acceptability between (\ref{8ade}) and (\ref{10de}) (on the given co-indexation), since they look identical in syntax and LF  – the fact that the only left branch element appears fronted in (\ref{10de}) is a consequence of a PF operation. The DE approach, on the other hand, predicts (\ref{10de}) to be ungrammatical on the given reading, since the whole QP ‘every general’ is at no point of the derivation in position from which it can bind the pronominal possessor in the subject. It stays in the object position throughout the derivation and in that sense should be ungrammatical, just like (\ref{8bde}). As indicated in (\ref{10de}), the DE approach makes the correct prediction (although grammatical, (\ref{10de}) disallows the bound variable reading): 

\begin{exe}
\ex \label{10de}
\gll *Svakog$_{j}$   njegovi$_{i}$ vojnici vole     [t$_{j}$ generala]$_{i}$\\
Every$_{\textsc{acc}}$ his$_{\textsc{nom}}$ soldiers$_{\textsc{nom}}$ love {} general$_{\textsc{acc}}$\\\\
‘His soldiers love every general.’\\
(Cannot be interpreted: for every general x, x’s soldiers love x) 
\end{exe}

The following contrast also supports the DE approach. In (\ref{11de}) there is a Condition C violation, as in (\ref{4ade}), since the pronominal possessor c-commands the R-expression \textit{Emira Kusturice}. There is no improvement in (\ref{12de}), in which the adjective modifying the object NP in which the R-expression is embedded undergoes LBE. This is expected under the DE approach, since just like in (\ref{11de}), the pronoun c-commands the R-expression. This is not quite expected on the SD analysis, because the whole object NP, with the R-expression in it is assumed to be moving in syntax to the position in which the R-expression is no longer c-commanded by the pronoun. Thus LBE cannot ameliorate Condition C effects, in contrast to the movement of the whole object, which apparently can, as illustrated in (\ref{13de}) (see also (\ref{4bde})). I thank one of the reviewers for suggesting checking this contrast. 

\begin{exe}
\ex \label{11de}
\gll *Njegov$_{i}$  najnoviji film je  razočarao      velikog prijatelja Emira Kusturice$_{i}$.\\
His latest film is disappointed big friend Emir$_{\textsc{gen}}$ Kusturica$_{\textsc{gen}}$\\\\
‘His$_{i}$ latest film disappointed a great friend of Emir Kusturica$_{i}$.’

\ex \label{12de}
\gll *Velikog je njegov$_{i}$ najnoviji film  razočarao prijatelja Emira Kusturice.\\
Big is his latest film disappointed friend Emir$_{\textsc{gen}}$ Kusturica$_{\textsc{gen}}$\\\\
‘His$_{i}$ latest film disappointed a great friend of Emir Kusturica$_{i}$.’


\ex \label{13de}
\gll Velikog prijatelja Emira Kusturice$_{i}$ je njegov$_{i}$  najnoviji film razočarao.\\
Big friend Emira Kusturice$_{\textsc{gen}}$ is his latest film disappointed\\\\
‘His$_{i}$ latest film really disappointed a great friend of Emir Kusturica$_{i}$.’

\end{exe}

A potential problem for this particular argument would be that (\ref{12de}) seems to be already degraded regardless of the co-indexation. This is still a problem for the SD approach, which in principle predicts that any time the movement of the whole nominal creates a grammatical structure (i.e., (\ref{13de})), the corresponding LBE should as well (i.e., (\ref{12de})), all else being equal. A separate question for the DE approach (which I have to leave for future work) is then why (\ref{12de}) would be degraded to begin with; that is, why would LBE out of a complex nominal be more constrained.

Another argument in favor of the DE approach comes from scope interpretation. For many speakers (including myself), SC seems to be rigid scope language.\footnote{A fair number of speakers I consulted share the judgments reported here. However, there are also speakers who seem to allow both readings in (\ref{14ade}), which indicates that there might be two dialects of SC in this respect. At this point I leave a more careful examination of this split to future work and focus here on judgments from the first group.} For those speakers a sentence like (\ref{14ade}) has only the surface scope. To get the inverse scope, the object must overtly move for those speakers, as in (\ref{14bde}):\footnote{Note that in (\ref{14bde}) the reading where the existential quantifier takes scope over the universal quantifier is not easily available to all speakers. The situation is further complicated by the existence of the distributer \textit{po} in SC, which some speakers require to get the distributed reading.} 

\begin{exe}
\ex \label{14de}
\begin{xlist}
\ex \label{14ade}
\gll Jedan student je pročitao svaku knjigu. \hspace{18mm} \cmark ∃ > ∀ *∀ > ∃\\
One student is read every book\\\\
‘A student read every book.’ 


\ex \label{14bde}
\gll Svaku knjigu je jedan student pročitao. \hspace{18mm} \cmark ∃ > ∀ \cmark ∀ > ∃ \\
Every book is one student read\\\\
‘A student read every book.’ 

\end{xlist}
\end{exe}

Focusing on those speakers, the question is what happens if instead of moving the whole object as in (\ref{14bde}), only the quantifier \textit{svaku} ‘every’ is fronted as in (\ref{15de}). The SD approach predicts that this sentence should have the same interpretation as (\ref{14bde}), as on this analysis they would have identical LF representations; i.e., the whole QP object moves in syntax, just like in (\ref{14bde}). On the DE approach, the sentence in (\ref{15de}) can only have the low scope of ‘every book’ since the extracted quantifier ‘every’ is uninterpretable in the fronted position. It can only be interpreted in its original, lower position via reconstruction, which would make it similar to (\ref{14ade}). Speakers for which the contrast in (\ref{14de}) exists, can only have the low interpretation of the universal quantifier in (\ref{15de}), as predicted by the DE approach. Specifically, there is one student and s/he read every book. Fronting of \textit{svaku} ‘every’ has the effect of emphasizing that the student in question read every book and not perhaps just one half or two thirds of the books. 

\begin{exe}
\ex \label{15de}
\gll Svaku$_{i}$ je jedan student pročitao [ t$_{i}$ knjigu]. \hspace{14mm} \cmark ∃ > ∀ *∀ > ∃\\
Every is one student read {} book\\\\
‘A student read every book.’ 
\end{exe}

Why is the quantifier \textit{svaku} ‘every’ not interpretable in the fronted position? This is quite straightforward on \citeauthor{HeimKratzer1998}’s (\citeyear{HeimKratzer1998}) approach to quantifier interpretation and scope.  In fact they directly discuss examples like (\ref{16de}):

\begin{exe}
\ex \label{16de} John fed every bird.\\
\rightarrow{} LF:

\begin{forest}for tree=nice empty nodes
[S
[every][
[1][S
[John][VP
[fed\\ \underline{\textbf{$\langle$e, $\langle$e,  t$\rangle \rangle$}}][DP \leftarrow{} \textbf{\underline{t (?!)}}
[t$_{1}$\\ \underline{\textbf{e}}][bird\\ \underline{\textbf{$\langle$e, t$\rangle$}}]
]
]
]
]
]
\end{forest}
\end{exe}

Regarding structures like (\ref{16de}), according to \citet[212]{HeimKratzer1998}: “…we are not dealing with an interpretable structure here in the first place. The trace’s type e meaning combines with the noun’s type <e,t> meaning to yield a truth-value (!) as the meaning of the DP “t$_{1}$ bird”. This cannot be composed with the type <e,et> meaning of the verb, and thus the VP and all higher nodes are uninterpretable”. 

But even if we assumed that the trace left by movement of \textit{every} is of type <<e,t>,<<e,t>,t>>, just like every (and that the type mismatch with the transitive verb can be resolved in a usual way via some local movement), we would still have a problem with the highest S node. As shown in (17), the whole sentence would not be of type t, but rather of type <<e,t>,t>>. Thus the only position in which the quantifier can be interpreted is the low, object-internal position, as expected on the DE approach.\footnote{One of the reviewers reports that even though they find (\ref{14ade}) ambiguous, they find (\ref{15de}) unambiguous – universal quantifier still must have low scope. We can assume that the speakers of the dialect who find (\ref{14ade}) ambiguous have covert QR, which can apparently apply freely in sentences like (\ref{14ade}). In (\ref{15de}), on the other hand, the overtly moved modifier must reconstruct at LF, as discussed above, which apparently bleeds further QR of the whole object. I leave exploration of this possibility for further research. }  

\begin{exe}
\ex \label{17de}

\begin{forest}for tree=nice empty nodes
    [S \underline{\textbf{$\langle \langle$e, t$\rangle$, t $\rangle \rangle$}}
        [every\\ \underline{\textbf{$\langle \langle$e, t $\rangle$, $\langle \langle$e, t$\rangle$, t$\rangle \rangle$ }}]
        [\underline{\textbf{$\langle$e, t$\rangle$}}\\ (\textit{Predicate Abstraction} applies)
            [1]
            [S \underline{\textbf{t}}
                []
                [.......]
            ]
        ]
    ]
\end{forest}


\end{exe}

\section{LBE in Japanese} \label{s3de}

As discusses in \citet{TakahashiFunakoshi2013} (T\&F hereafter), Japanese in general does not allow LBE, which is shown in (\ref{18de})

\begin{exe}
\ex \label{18de}
\begin{xlist}
\ex \label{18ade}
\gll Taroo-ga [ dare-no     tegami]-o    sute-ta-no?\\
Taro-\textsc{nom} {} who-\textsc{gen} letter-\textsc{acc} discard-\textsc{pst}-\textsc{q}\\\\
‘lit. Taro discarded whose letter?’

\ex \label{18bde}
\gll *Dare$_{i}$-no  Taroo-ga  [ t$_{i}$      tegami]-o  sute-ta-no?\\
who-\textsc{gen} Taro-\textsc{nom} {} {} letter-\textsc{acc} discard-\textsc{pst}-\textsc{q}\\\\
‘lit. Whose$_{i}$, Taro discarded [a letter \textit{t}$_{i}$ ]?’\\
\hspace{72mm} (T \& F: 237)		

\end{xlist}
\end{exe}

However, T\&F observe that a PP within a nominal \textit{can} undergo LBE: 

\begin{exe}
\ex \label{19de}
\begin{xlist}
\ex \label{19ade}
\gll Taroo-ga   [ dare-kara-no    tegami]-o    sute-ta-no?\\
Taro-\textsc{nom} {} who-from-\textsc{gen} letter-\textsc{acc} discard-\textsc{pst}-\textsc{q}\\\\
lit. ‘Taro discarded a letter from who?’


\ex \label{19bde}
\gll Dare-kara$_{i}$-no   Taroo-ga  [ \textit{t}$_{i}$       tegami]-o  sute-ta-no?\\
who-from-\textsc{gen} Taro-\textsc{nom} {} {} letter-\textsc{acc} discard-\textsc{pst}-\textsc{q}\\\\
lit. ‘From who$_{i}$, Taro discarded [a letter \textit{t}$_{i}$ ]?’\\
\hspace{72mm} (T \& F: 237)	

\end{xlist}
\end{exe}

T\&F also show that (\ref{19bde}) is a result of syntactic movement. In particular, this PP LBE is island sensitive. First, (\ref{20de}) shows that PP LBE can take place across a clausal boundary.  

\begin{exe}
\ex \label{20de}
\begin{xlist}
\ex \label{20ade}
\gll Hanako-ga  [$_{CP}$ Taroo-ga [ dare-kara-no    tegami]-o sute-ta]-to omottei-ru-no?\\
Hanako-\textsc{nom} {} Taro-\textsc{nom} {} who-from-\textsc{gen} letter-\textsc{acc} discard-\textsc{pst}-that think-\textsc{prs}-\textsc{q}\\\\
‘lit. Hanako thinks that Taro discarded [a letter from who]?’

\ex \label{20bde}
\gll Dare-kara$_{i}$-no Hanako-ga [$_{CP}$ Taroo-ga  [ \textit{t}$_{i}$  tegami]-o  sute-ta]-to  omottei-ru-no?\\
who-from-\textsc{gen} Hanako-\textsc{nom} {} Taro-\textsc{nom} {} {} letter-\textsc{acc} discard-\textsc{pst}-that think-\textsc{prs}-\textsc{q}\\\\
‘lit. From who$_{i}$ Hanako thinks that Taro discarded [a letter \textit{t}$_{i}$ ]?’\\
\hspace{72mm} (T \& F: 239)	
\end{xlist}

\end{exe}

However, the extraction out of the relative clause island is not possible:

\begin{exe}
\ex \label{21de}
\begin{xlist}
\ex \label{21ade}
\gll Hanako-ga [[$_{RC}$[ dare-kara-no   tegami]-o  sute-ta] hito]-o sagasitei-ru-no?\\
Hanako-\textsc{nom} {} who-from-\textsc{gen} letter-\textsc{acc} discard-\textsc{pst} person-\textsc{acc} be.looking.for-\textsc{prs}-\textsc{q}\\\\
‘lit. Hanako is looking for a person that discarded a letter from who?’


\ex \label{21bde}
\gll *Dare-kara$_{i}$-no   Hanako-ga     [[$_{RC}$[ \textit{t}$_{i}$     tegami]-o  sute-ta hito]-o sagasitei-ru-no?\\
who-from-\textsc{gen} Hanako-\textsc{nom} {} {} letter-\textsc{acc} discard-\textsc{pst} person-\textsc{acc} be.looking.for-\textsc{prs}-\textsc{q}\\\\
‘lit. From who$_{i}$ Hanako is looking for a person who discarded [a letter \textit{t}$_{i}$]?
\hspace{72mm} (T \& F: 239)	

\end{xlist}
\end{exe}

In a nutshell, T\&F explain the contrast between (\ref{18bde}) and (\ref{19bde}) in the following way. They assume (i) that K(ase)P (i.e., projection of a Case-particle) is projected above NP in Japanese and (ii) that nominals and PPs are adjoined to host NPs (cf. \citealt{Bošković2005}, \citealt{Cheng2011}) (see (\ref{22de})). Thus, genitive elements within nominals are all NP adjuncts.  T\&F propose that while KPs with nominals are phases, KPs with genitive PPs are not phases:

\begin{exe}
\ex \label{22de}
\begin{forest}for tree=nice empty nodes
[KP
[][K'
[NP
[NP-\textsc{gen}][NP]
][K]
]
]
\end{forest}
\end{exe}

There are two potential options to consider with phasal KPs: (i) direct movement of the genitive nominal out of the KP (option 1) and (ii) successive cyclic movement of the genitive nominal through the KP edge (option 2). They are both ruled out by the combination of the PIC and Antilocality.  Option 1 is excluded via the PIC  \citep{Chomsky2000}, which states that an element that is moving out of the phase must move to the edge of the phase. Option 2 is also excluded because of the \textit{antilocality} (\citealt{Abels2003}, \citealt{Bošković2005}). That is, the moving element cannot move to the edge of the phase (thus satisfying the PIC), because that movement would be \textit{too} local; i.e., the first XP that actually dominates the adjunct NP-\textsc{gen} in (\ref{22de}) is KP. 

In the case of PP LBE, however, KP is not a phase by assumption, and the PP be can extracted without violating any of the above conditions. 

\begin{exe}
\ex \label{23de}
\begin{forest}for tree=nice empty nodes
[KP (KP $\neq$ phase)
[][K'
[NP
[PP, name=pp][NP]
][K]
]
]
	\draw[->] (pp.west) to [bend left=35] (180:3);
\end{forest}
\end{exe}

T\&F also assume that \textit{-no} in (\ref{18de}) is structural Case assigned by K while \textit{-no} in (\ref{19de}) is a linking element, attached to a PP by the Mod-Insertion rule (e.g., \citealt{KitagawaRoss1982}, \citealt{saito2008n}).

\subsection{Japanese PP LBE and Weak Crossover} \label{s3.1de}

(\ref{24de}) illustrates standard Weak Crossover Effects, which characterize A’-movement:

\begin{exe}
\ex \label{24de}
\begin{xlist}
\ex \label{24ade}
*Who$_{i}$ does his$_{i}$ mother love \textit{t}$_{i}$?

\ex \label{24bde}
Who$_{i}$ \textit{t}$_{i}$ seems to his$_{i}$ mother \textit{t}$_{i}$ to be intelligent?
\end{xlist}
\end{exe}

T\&F observe that PP LBE in Japanese behaves as A’-movement in this respect: 

\begin{exe}
\ex \label{25de}
\begin{xlist}
\ex \label{25ade}
\gll *Kinoo \textit{soko}$_{i}$-no syain-ga           \textbf{[dono-kaisya$_{i}$-kara-no} syootaizyoo]-o uketot-ta-no? \\
yesterday it-\textsc{gen} employee-\textsc{nom} which-company-from-\textsc{gen} invitation-\textsc{acc} receive-\textsc{pst}-\textsc{q}\\\\
 ‘lit. Its$_{i}$ employees received [invitations from which company$_{i}$]  yesterday?’


\ex \label{25bde}
\gll \textbf{*Dono-kaisya}$_{i}$\textbf{-kara-no}  kinoo  \textit{soko}$_{i}$-no syain-ga [ \textit{t}$_{i}$    syootaizyoo]-o uketot-ta-no?\\
which-company-from-\textsc{gen} yesterday it-\textsc{gen} employee-\textsc{nom} {} {} invitation-\textsc{acc} receive-\textsc{pst}-\textsc{q}\\\\
‘lit. From which company$_{i}$, its$_{i}$ employees received [ invitations \textit{t}$_{i}$ ] yesterday?’



\ex \label{25cde}
\gll \textbf{Dono-kaisya}$_{i}$\textbf{-kara-no}  kinoo Toyota-no  syain-ga  [ \textit{t}$_{i}$   syootaizyoo]-o uketot-ta-no?\\
which-company-from-\textsc{gen} yesterday Toyota-\textsc{gen} employee-\textsc{nom} {} {} invitation-\textsc{acc} receive-\textsc{pst}-\textsc{q}\\\\
‘lit. From which company$_{i}$, Toyota's employees received [invitations \textit{t}$_{i}$] yesterday?’\\
\hspace{82mm} (T \& F: 243)	
\end{xlist}
\end{exe}

The acceptability of (\ref{25cde}) indicates that (\ref{25bde}) is unacceptable because of the bound variable reading, since in (\ref{25cde}), the pronoun \textit{soko} ‘it’ is replaced by the referential expression \textit{Toyota}. Also, the unacceptability of (\ref{25bde}) is not due to the presence of \textit{kara} ‘from’. In (\ref{26ade}), \textit{kara} ‘from’ is a matrix element, and if it is moved (via scrambling) to the sentence-initial position (as in (\ref{26bde})), the bound variable construal of the pronoun becomes possible.

\begin{exe}
\ex \label{26de}
\begin{xlist}
\ex \label{26ade}
\gll *Kinoo \textit{soko}$_{i}$-no syain-ga           \textbf{ dono-kaisya$_{i}$-kara} [syootaizyoo]-o  uketot-ta-no?\\ 
yesterday it-\textsc{gen} employee-\textsc{nom} which-company-from invitation-\textsc{acc} receive-\textsc{pst}-\textsc{q}\\\\
`lit. Its$_{i}$ employees received [invitations] from which company$_{i}$ yesterday?’


\ex \label{26bde}
\gll \textbf{Dono-kaisya$_{i}$-kara} kinoo       \textit{soko}$_{i}$-no syain-ga \textit{t}$_{i}$  [syootaizyoo]-o uketot-ta-no?\\
which-company-from yesterday it-\textsc{gen} employee-\textsc{nom} {} invitation-\textsc{acc} receive-\textsc{pst}-\textsc{q}\\\\
‘lit. From which company$_{i}$, its$_{i}$ employees received [invitations] \textit{t}$_{i}$ yesterday?’\\
\hspace{79mm} (T \& F: 244)	

\end{xlist}
\end{exe}

Furthermore, T\&F show that the presence of \textit{-no} in (\ref{25bde}) has nothing to do with its unacceptability. As shown in (\ref{27de}), genitive marked PPs can bind a variable pronoun if LBE does not apply to them: 

\begin{exe}
\ex \label{27de}
\gll Kimi-wa \textbf{[dono-kaisyai-kara-no} \textit{soko}$_{i}$-no syain-e-no syootaizyoo]-o mi-ta-no?\\
you-\textsc{top} which-company-from-\textsc{gen} it-\textsc{gen} employee-to-\textsc{gen} invitation-\textsc{acc} see-\textsc{pst}-\textsc{q}\\\\
‘lit. You saw [an invitation from which company$_{i}$ to its$_{i}$ employees]?’\\
\hspace{82mm} (T \& F: 244)	
\end{exe}

Now, although PP LBE results in weak-cross-over-like effects, moving the whole phrase containing the PP does not. This is the crucial contrast for the purposes of this paper:

\begin{exe}
\exi{(25)}
\begin{xlist}
\exi{b.} 
\gll \textbf{*Dono-kaisya}$_{i}$\textbf{-kara-no}  kinoo  \textit{soko}$_{i}$-no syain-ga [ \textit{t}$_{i}$    syootaizyoo]-o uketot-ta-no?\\
which-company-from-\textsc{gen} yesterday it-\textsc{gen} employee-\textsc{nom} {} {} invitation-\textsc{acc} receive-\textsc{pst}-\textsc{q}\\\\
‘lit. From which company$_{i}$, its$_{i}$ employees received [ invitations \textit{t}$_{i}$] yesterday?\\
\hspace{79mm} (T \& F: 243)	


\end{xlist}

\ex \label{28de}
\gll \textbf{[Dono-kaisya$_{i}$-kara-no}      syootaizyoo]-o   kinoo       \textit{soko}$_{i}$-no syain-ga             uketot-ta-no?\\
which-company-from-\textsc{gen} invitiation-\textsc{acc} yesterday it-\textsc{gen} employee-\textsc{nom} receive-\textsc{pst}-\textsc{q}\\\\
\end{exe}



All Japanese speakers I consulted agree with the contrast between (\ref{25bde}) and (\ref{28de}) (see also \citealt{AranoOda2019}). Again, this is surprising on the SD approach to LBE, because on this analysis in both (\ref{25bde}) and (\ref{28de}) the whole phrase containing the PP is moved in syntax.

One of the reviewers points out correctly that (\ref{28de}) should not be grammatical, given T\&F’s structure of Japanese nominals. KP dominates the PP \textit{dono-kaisyai-kara-no} ‘from which company’, which is adjoined to the NP, so since the PP does not c-command the pronoun, it should not be able to bind it, contrary to fact. But this seems to be a more general property of variable binding from the NP modifier position in Japanese. According to my informants, in both (\ref{29de}) and (\ref{30de}) \textit{dono-kaisya}$_{i}$ ‘which company’ embedded in the subject binds the possessive pronoun modifying the object, regardless of whether the wh-possessor is an NP possessor (e.g., (\ref{29de})) or a PP possessor (e.g., (\ref{30de})).

\begin{exe}
\ex \label{29de}
\gll [\textbf{Dono-kaisya$_{i}$-no}          shacho]-ga soko$_{i}$-no syain-o shikat-ta-no?\\
which-company-\textsc{gen} CEO-\textsc{nom} it-\textsc{gen} employee-\textsc{acc} scold-\textsc{pst}-\textsc{q}\\\\
‘Which company$_{i}$’s CEO scolded it$_{i}$’s employees?’ 

\ex \label{30de}
\gll \textbf{[Dono-kaisya$_{i}$-kara-no}       syootaizyoo]-ga soko$_{i}$-no shacho-ni todoi-ta-no?\\
which-company-from-\textsc{gen} invitation-\textsc{nom} it-\textsc{gen} CEO-\textsc{dat} arrive-\textsc{pst}-\textsc{q}\\\\
‘An invitation from which company$_{i}$ arrived to it$_{i}$’s CEO?’ 

\end{exe}

It seems to me that either KP (regardless of whether it is a phase or not) does not count as a category for purposes of c-command (the first category that properly dominates wh-possessors in (\ref{29de}) and (\ref{30de}) would actually be the first one that properly dominates KP), or that the wh-possessor always moves out at LF to a position from which it could bind the pronoun and, crucially, this movement would not be constrained by the PIC and Antilocality, as the overt LBE movement in (\ref{18de}) is. This assumption would be necessary to account for (\ref{29de}). The latter option would very similar to Kayne’s (1994) treatment of variable binding in similar English construction and the contrast between (\ref{31de}) and (\ref{32de}):

\begin{exe}
\ex \label{31de}
Every girl$_{i}$’s father thinks she$_{i}$ is a genius.

\ex \label{32de}
*Every girl’s father admires herself.

\end{exe}

Kayne (1994) argues that the possessor QP moves to a higher DP position at LF from which only variable-binding, but not anaphor-binding, is possible. This is, however, somewhat orthogonal to the main goal of this paper, which is to show that there are deep difference between LBE and movement of the whole nominal, which I believe, the contrast between (\ref{28de}) and (\ref{25bde}) illustrates well. 

Note finally that SC exhibits a similar contrast. Just like in the case of QP fronting from Section \ref{s2de}, binding is possible only if the whole wh-phrase moves overtly, as in (\ref{33bde}). If the wh-phrase stays in situ (e.g., (\ref{33bde})), or if \textit{kog} ‘which’ undergoes LBE (e.g., (\ref{33cde})), binding is not possible.

\begin{exe}
\ex \label{33de}
\begin{xlist}
\ex \label{33ade}
\gll *Njegovi$_{i}$ roditelji su izgrdili kog dečaka$_{i}$?\\
His parents are scolded which boy\\\\
‘Which boy$_{i}$ did his$_{i}$ parents scold?’\\
(ungrammatical on this coindexation)


\ex \label{33bde}
\gll  [Kog dečaka]$_{i}$ su njegovi$_{i}$ roditelji izgrdili  \textit{t}$_{i}$ ?\\
Which boy are his parents scolded\\\\
 ‘Which boy$_{i}$ did his$_{i}$ parents scold?’ 
 
\ex \label{33cde}
\gll *Kog$_{i}$    su   njegovi$_{j}$ roditelji izgrdili [ \textit{t}$_{i}$ dečaka]$_{j}$ ?\\
Which are his parents scold {} {} boy\\\\
 ‘Which boy$_{i}$ did his$_{i}$ parents scold?’\\      (ungrammatical on this coindexation)

\end{xlist}
\end{exe}

As noted by a reviewer, (\ref{33ade}) should be ungrammatical regardless of the coindexation, since the wh-phrase stays in-situ. To control for this, we can add another wh-phrase, which does not have to move, as below.

\begin{exe}
\ex \label{34de}
\begin{xlist}
\ex \label{34ade}
\gll [Kog dečaka]$_{i}$ su njegovi$_{i}$ roditelji izgrdili  \textit{t}$_{i}$  kad?\\
Which boy are his parents scolded {} when\\\\
‘Which boy$_{i}$ did his$_{i}$ parents scold when?’ 

\ex \label{34bde}
\gll *Kog$_{i}$ su  njegovi$_{j}$ roditelji izgrdili [  t$_{i}$ dečaka]$_{j}$ kad?\\
Which are his parents scold {} {} boy when\\\\
‘Which boy$_{i}$ did his$_{i}$ parents scold when?’\\
(ungrammatical on this coindexation)
\end{xlist}
\end{exe}

In (\ref{34de}, at least one wh-phrase moves to the front and there is still a contrast in binding. Even if all wh-elements move to the front, as in (\ref{35de}), the pronominal possessor cannot be bound if \textit{dečaka} ‘boy’ does not move (\ref{35bde}).

\begin{exe}
\ex \label{35de}
\begin{xlist}
\ex \label{35ade}
\gll Kad su [kog dečaka]$_{i}$ njegovi$_{i}$ roditelji izgrdili  \textit{t}$_{i}$  ?\\
When are which boy his parents scolded\\
‘Which boy$_{i}$ did his$_{i}$ parents scold when?’ 

\ex \label{35bde}
\gll *Kad su kog$_{i}$    njegovi$_{j}$ roditelji izgrdili [  \textit{t}$_{i}$ dečaka]$_{j}$?\\
When are which his parents scold {} {} boy\\\\
‘Which boy$_{i}$ did his$_{i}$ parents scold when?’\\
(ungrammatical on this coindexation)
\end{xlist}
\end{exe}

\section{Conclusion} \label{s4de}
In this paper I have investigated the size of the element that undergoes LBE. On the DE approach, what moves is exactly what we see overtly fronted – a left branch element (e.g., adjective, demonstrative, possessive etc.) and nothing more. On the SD approach, the size of the moving element is actually bigger then what is overtly evident. In addition to the left branch element (and the phrase immediately dominating it, such as AP), the modified nominal is also moved in the syntax, but it is not pronounced (overly realized) at PF. But in terms of its syntactic and semantic properties, LBE structures should not differ from structures in which the whole object is overtly moved on this analysis – the only difference between them is whether or not the moved noun is overtly realized. I have tried to show, using these two types of structures, that the SD approach in its basic form is not on the right track. That is, there are significant syntactic and semantic differences between the LBE structures and those in which the whole object moves.  I have used variable binding, weak crossover effects and scope properties to make this point. Empirically speaking, I have focused on data from SC, a well-studied LBE language, but I have shown that the same point can be made even in a language like Japanese. 

\section*{Acknowledgements}
This paper is missing acknowledgements. 

{\sloppy
    \printbibliography[heading=subbibliography,notkeyword=this]
}

\end{document}