%%% !TEX program = xelatex

\documentclass[output=paper,colorlinks,citecolor=brown,
% hidelinks,
% showindex
]{langscibook}

\author{
    Uli Sauerland%
        \affiliation{Osaka University; Leibniz-Centre General Linguistics}
        % \orcid{0000-0003-2175-535X}
    and 
    Kazuko Yatsushiro%
        \affiliation{Leibniz-Centre General Linguistics}
        % \orcid{0000-0002-8060-0385}
}

%Thanks to Aron Hirsch, one anonymous reviewer

%\author{A. Anonymous\affiliation{Café Sacher, Vienna}}

\title{Domain Size Matters: An Exceptive that Forms Strong NPIs}
\abstract{The study of exceptives provides important insights into the nature of negative polarity items (NPIs).
Specifically, the NPI nature of exceptive phrases in English can be transparently related to the compositional operation of shrinking the domain of a quantifier.
But we show that  exceptive phrases in Japanese form strong NPIs while English exceptives form weak NPIs.
We account for the difference between Japanese and English by the suggesting that exceptives in the two languages select different exhaustification operators.
Our result may provide a new avenue towards understanding the difference between strong and weak NPIs.}


%move the following commands to the "local..." files of the master project when integrating this chapter
% \usepackage{tabularx}
% \usepackage{langsci-basic}
% \usepackage{langsci-optional}
% \usepackage{langsci-gb4e}
%\bibliography{localbibliography}
%\newcommand{\orcid}[1]{}
%\pagenumbering{arabic}

%\IfFileExists{../localcommands.tex}{%hack to check whether this is being compiled as part of a collection or standalone
%   % add all extra packages you need to load to this file
\usepackage{tabularx,multicol}
\raggedcolumns

\usepackage{url}
\urlstyle{same}

\usepackage{listings}
\lstset{basicstyle=\ttfamily,tabsize=2,breaklines=true}

\usepackage{langsci-basic}
\usepackage{langsci-optional}
\usepackage{langsci-lgr}

% Wood's chapter
\usepackage{jimsstyle}

\usepackage{subcaption}
\usepackage{color,colortbl}
\definecolor{Gray}{gray}{0.8}%to get a color for the table
\usepackage{expex}
\usepackage[linguistics]{forest}
\forestset{qtree edges/.style={for tree={parent anchor=south, child anchor=north}}}
\usepackage{graphbox}
\usepackage{leipzig}
\usepackage{multirow}
\usepackage{needspace}
\usepackage{newunicodechar}
\usepackage{pifont}
\usepackage{pgf}
\usepackage{pst-node}
\usepackage{qtree}
  \qtreecenterfalse
\usepackage{rotating}
% \usepackage{skak}
\usepackage{soul}
\usepackage{tabto}
\usepackage{textgreek}
\usepackage{tikz}
\usetikzlibrary{calc,decorations,shapes.misc,decorations.pathreplacing,arrows}
\usetikzlibrary{backgrounds, matrix, positioning}
\usetikzlibrary{}
\usetikzlibrary{}
\usepackage{tikz-qtree,tikz-qtree-compat}
\usepackage{tipa} % must be imported before linguex, but we do not use linguex here
\usepackage{todonotes}
\usepackage{tree-dvips}
\usepackage[normalem]{ulem}%for strikeout
\usepackage[Symbolsmallscale]{upgreek} 
\usepackage{verbatim}
\usepackage{xcolor}
\usepackage{xspace}
% \usepackage{xtab}

\usepackage{langsci-gb4e}

%   \newcommand{\orcid}[1]{}


% pesetskycommands.tex
%ExPex stuff

\newcommand{\pexcnn}{\pex[exno={~},exnoformat={X}]}  % pexc with no examplenumber and no increment
\lingset{everygla=\normalfont,{aboveglftskip=0pt}}
\newcommand{\ptxt}[1]{#1\par}
\let\expexgla\gla
\AtBeginDocument{\let\gla\expexgla\gathertags}

%other stuff
\newunicodechar{⟶}{{\symbolfont{⟶}}}
%\newunicodechar{✓}{{\Checkmark}} % checkmark
\newunicodechar{✓}{{\ding{51}}} % checkmark
\newcommand{\gap}{\ \underline{\hspace{10pt}\phantom{x}}\ }
\newcommand{\ix}[1]{\textsubscript{\it{#1}}}  % subscript

\def\gethyperref#1{\hyperlink{#1}{\getref{#1}}}%
\def\getfullhyperref#1{\hyperlink{#1}{\getfullref{#1}}}%

\newcommand{\denote}[1]{⟦{#1}⟧}

%   %% hyphenation points for line breaks
%% Normally, automatic hyphenation in LaTeX is very good
%% If a word is mis-hyphenated, add it to this file
%%
%% add information to TeX file before \begin{document} with:
%% %% hyphenation points for line breaks
%% Normally, automatic hyphenation in LaTeX is very good
%% If a word is mis-hyphenated, add it to this file
%%
%% add information to TeX file before \begin{document} with:
%% %% hyphenation points for line breaks
%% Normally, automatic hyphenation in LaTeX is very good
%% If a word is mis-hyphenated, add it to this file
%%
%% add information to TeX file before \begin{document} with:
%% \include{localhyphenation}
\hyphenation{
affri-ca-te
affri-ca-tes 
agree-ment
anaph-o-ra
an-ti-caus-a-tive
an-ti-caus-a-tives
an-te-ced-ent
Bak-er
caus-a-tive
Christo-poulos
clas-si-fi-er
claus-al
Co-lum-bia
com-ple-ment-izer
com-ple-ments
con-tin-u-a-tive
de-di-ca-ted
De-mo-cra-tas
Dor-drecht
du-ra-tive
Ex-folia-tion
ex-tra-gram-mat-ical
fi-nite-ness
ger-und
ger-unds
Gro-ninger
in-trac-ta-ble
Jap-a-nese
judg-ment
Judg-ment
Lysk-awa
Ma-khach-ka-la
Ma-rantz
Mat-thew-son
Max-Share
merg-er
Pap-u-an
Per-elts-vaig
post-ver-bal
phe-nom-e-non
pre-dic-tion
Pre-dic-tion
Rich-ards
Sa-len-ti-na
to-pic
Ya-wa-na-wa
Wil-liam-son
}

\hyphenation{
affri-ca-te
affri-ca-tes 
agree-ment
anaph-o-ra
an-ti-caus-a-tive
an-ti-caus-a-tives
an-te-ced-ent
Bak-er
caus-a-tive
Christo-poulos
clas-si-fi-er
claus-al
Co-lum-bia
com-ple-ment-izer
com-ple-ments
con-tin-u-a-tive
de-di-ca-ted
De-mo-cra-tas
Dor-drecht
du-ra-tive
Ex-folia-tion
ex-tra-gram-mat-ical
fi-nite-ness
ger-und
ger-unds
Gro-ninger
in-trac-ta-ble
Jap-a-nese
judg-ment
Judg-ment
Lysk-awa
Ma-khach-ka-la
Ma-rantz
Mat-thew-son
Max-Share
merg-er
Pap-u-an
Per-elts-vaig
post-ver-bal
phe-nom-e-non
pre-dic-tion
Pre-dic-tion
Rich-ards
Sa-len-ti-na
to-pic
Ya-wa-na-wa
Wil-liam-son
}

\hyphenation{
affri-ca-te
affri-ca-tes 
agree-ment
anaph-o-ra
an-ti-caus-a-tive
an-ti-caus-a-tives
an-te-ced-ent
Bak-er
caus-a-tive
Christo-poulos
clas-si-fi-er
claus-al
Co-lum-bia
com-ple-ment-izer
com-ple-ments
con-tin-u-a-tive
de-di-ca-ted
De-mo-cra-tas
Dor-drecht
du-ra-tive
Ex-folia-tion
ex-tra-gram-mat-ical
fi-nite-ness
ger-und
ger-unds
Gro-ninger
in-trac-ta-ble
Jap-a-nese
judg-ment
Judg-ment
Lysk-awa
Ma-khach-ka-la
Ma-rantz
Mat-thew-son
Max-Share
merg-er
Pap-u-an
Per-elts-vaig
post-ver-bal
phe-nom-e-non
pre-dic-tion
Pre-dic-tion
Rich-ards
Sa-len-ti-na
to-pic
Ya-wa-na-wa
Wil-liam-son
}

%    \bibliography{localbibliography}
%    \togglepaper[23]
%}{}

%custom footer for preprints
% \papernote{\scriptsize\normalfont
%     Uli Sauerland and Kazuko Yatsushiro.
%     Domain Size Matters: An Exceptive that Forms Strong Negative Polarity Items.
%     To appear in:
%     XXX and XXX volume.  
%     Change volume title.  
%     Berlin: Language Science Press. [preliminary page numbering]
% }

%\usepackage{ulingex,ulingmsc}
%\bibliography{localbibliography}

\let\l\lambda
%\def\refp#1{(\ref{sy#1})}
\def\M#1{\textsc{#1}}
\def\exh{\ifmmode\mathop{\textbf{exh}}\else\textbf{exh}\xspace\fi}
\def\pex{\ifmmode\mathop{\textbf{pex}}\else\textbf{pex}\xspace\fi}
\def\K{\ifmmode\mathop{\textsf{\textbf{K}}}\else\textsf{\textbf{K}}\xspace\fi}
\def\existsx{\ifmmode\mathop{\exists x}\else$\exists x$\xspace\fi}
\def\forallx{\ifmmode\mathop{\forall x}\else$\forall x$\xspace\fi}

\setcounter{secnumdepth}{2}

% \usepackage{todonotes}


\begin{document}
%\listoftodos

\maketitle


\section{Introduction}

A frequent route of inquiry in linguistics and other fields of science is to understand complex properties of a structure by first decomposing the structure into its pieces and to then analyze how the different pieces give rise to the complex properties.
The understanding of the complex behavior of negative polarity items (NPIs) has been greatly advanced by a series of papers on exceptives by \cite{fintel93}, \cite{gajewski08b}, and \cite{hirsch16b} that follows this  route.
Specifically, the underlying assumption of this line of work has been that the NPI nature of some English exceptives derives from their compositional structure and its interaction with the environment it occurs in.%
\il{Japanese}%add "Japanese" to language index for this page
\il{English}%add "English" to language index for this page
%\ea Susi plays few games but Siedler.\z
%
The idea is that the function of an exceptive can only make sense in an environment that licenses NPIs, i.e.\ an \emph{antitone} environment.\footnote{We use the traditional order-theoretic terms \emph{isotone} and \emph{antitone} \citep{birkhoff40a} because the are more concise and widely accepted than their synonyms.
Specifically the terms \emph{upward entailing} or \emph{upward monotone} and \emph{downward entailing} or \emph{downward monotone}  are synonyms of \emph{isotone} and \emph{antitone} respectively that are used widely in linguistics.}
We introduce this idea by means of example (\ref{syex:noplayer}) here, and then in more general terms in section \ref{sysc:analysis}.

Polarity phenomena relate closely to quantification in language.
Quantificational meaning depends on the size of the domain of a quantifier.
An exceptive phrase in a quantificational statement provides a way to restrict the domain of the quantifier.  Consider for example the English example (\ref{syex:noplayer}).

\ea \label{syex:noplayer} 
   No player but Susi has access to the ocean.\z

The exceptive phrase \emph{but Susi} in (\ref{syex:noplayer}) has a number of semantic effects. The primary one is to reduce the size of a domain as illustrated below:
If $s$ representing Susi was an element of the domain of players, the exceptive \emph{except Susi} forms a smaller domain where $s$ is no longer an element.

\begin{eqnarray*}
\text{player}: && \{s, j, u, k\} \\
\text{player but Susi}: && \{j, u, k\}
\end{eqnarray*}

The immediate effect of the exceptive is then that (\ref{syex:noplayer}) expresses a generalization over a domain that Susi is no longer part of:
(\ref{syex:noplayer}) requires that players $j$, $u$, and $k$ have no access to the ocean, but exempts Susi.

In fact, (\ref{syex:noplayer}) provides more information about Susi than merely exempting her. We infer from (\ref{syex:noplayer}) that  Susi does have access to the ocean. This inference can be derived as an implicature of the domain size reduction by the exceptive: Roughly speaking, we assume that the speaker's reason for exempting Susi must be that (\ref{syex:noplayer}) would be false without the exceptive.  That means that the speaker must believe that the property \emph{has access to the ocean} must be false of each player in $\{j, u, k\}$, but true of at least one player from $\{s, j, u, k\}$, which entails that it must be true of $s$. In other words, Susi must have access to the ocean.

The implicature provides an insightful approach to the NPI-distribution of exceptive in English.  Briefly compare the ungrammatical (\ref{syex:someplayer}) with (\ref{syex:noplayer}).  

\ea \label{syex:someplayer} 
   *Some player but Susi has access to the ocean.\z

The line of explanation we adopt is based on the fact that the implicature predicted for (\ref{syex:someplayer}) stands in contradiction to the primary meaning of (\ref{syex:someplayer}): Namely, the primary meaning asserts that one player from $\{j, u, k\}$ has access to the ocean.  But the implicature requires that no player from the set $\{s, j, u, k\}$ has access to the ocean. Both cannot be true at the same time.  The ungrammaticality of (\ref{syex:someplayer}) follows from the contradiction we just observed if we make two further assumptions: 1) implicatures of the type we observe here with exceptives are obligatory inferences and 2) contradictions of the type we observe here can lead to the perception of ungrammaticality.  Both assumptions, surprising as they may initially seem, have been justified by a number of interesting further predictions they make (\citealt{chierchia13a}, and others).  We adopt therefore the approach just outlined, which we discuss in more detail in Section \ref{sysc:analysis} below.

Our main concern in this paper is Japanese data with \emph{sika} such as (\ref{syex:playerwa}).
\emph{Sika}-phrases have been analyzed as exceptives by \cite{alonso-ovalle04a}, \cite{yoshimura2007b}, \cite{kawahara08a}, and \cite{shimoyama11}.\footnote{Other authors like \cite{hasegawa11a} argue against an exceptive analysis of \emph{sika}. Items similar to \emph{sika} seem to also exist at least in Korean (\emph{pukkey}, \citealt{sells01a} and others) and French (\emph{ne \dots\ que}, \citealt{fintel07a} and others), and have also been analyzed as exceptives at least in the works cited here.}  But these works have assume that \emph{sika} is construed with a silent universal quantifier, while we argue that a silent existential quantifier makes better predictions.

\ea \label{syex:playerwa}
\gll Pureeyaa-wa Susi-sika umi-ni akusesu-ga nai.\\
player-\M{top} Susi-\M{sika} ocean-to access-\M{nom} \M{neg}\\
\glt `No player but Susi has access to the ocean.'\z
%
\emph{Sika}-phrases share the basic NPI distribution of English exceptives.
Note that the Japanese example does not contain a negative quantifier corresponding to \emph{nobody}, but instead contains \emph{sika} and the sentential negation \emph{nai} (`not').  The distribution of \emph{sika} is restricted as shown by the ungrammatical (\ref{syex:playeraru}), where the sentential negation of (\ref{syex:playerwa}) is left out.
\ea \label{syex:playeraru}
\gll *Pureeyaa-wa Susi-sika umi-ni akusesu-ga aru.\\
player-\M{top} Susi-\M{sika} ocean-to access-\M{nom} exist\\\z


The contrast between (\ref{syex:playerwa}) and (\ref{syex:playeraru}) in Japanese closely resembles the one between (\ref{syex:noplayer}) and (\ref{syex:someplayer}) in English.
But we show below that overall the distribution of \emph{sika}-phrases is actually more restrictive than that of \emph{but}-exceptives in English. In Section \ref{sysc:negq}, we discuss the relation of \emph{sika}-phrases in Japanese and one frequent way of expressing negative quantifiers in Japanese, an inderminate phrase like \emph{nani} (`what') with the particle \emph{mo} (`all', `also'), and the sentential negation.
In Section \ref{sysc:data}, we propose that the difference in distribution between \emph{but}-exceptives and \emph{sika}-phrases is parallel to that between weak and strong NPIs.
We then present in Section \ref{sysc:analysis} a semantic proposal based on the work of \cite{gajewski11a} to account for the strong NPI distribution of \emph{sika}-phrases, and also account for the difference between \emph{sika}-phrases and negative concord.  We conclude with a summary of the stipulations required for our analysis.



\section{The Distribution of \emph{Sika}-Exceptives}\label{sysc:data}

In this section, we argue that Japanese \emph{sika}-phrases have the distribution of strong NPIs while English \emph{but}-exceptives have the distribution of weak NPIs. As we  saw in the introduction, exceptives, like other NPIs, are licensed only in the immediate scope of operators that are antitone in a sense to be specified.
In this section, we compare the distribution of \emph{sika}-phrases with that of \emph{but}-exceptives, other weak NPIs like \emph{ever}, on the one hand, and that of strong NPIs like \emph{in weeks} in English, on the other.
In the following two subsections, we first consider environments that license both strong and weak NPIs, and then environments that only license weak NPIs.
The generalization we substantiate is that \emph{sika}-phrases share the distribution of English strong NPIs, rather than that of \emph{but}-exceptives. In the final subsection, we show that \emph{sika}-phrases are licensed in many environments where also negative concord items are licensed in Japanese, but that there are also differences between the two.




\subsection{Strong NPI licensing environments}

Environments that license strong NPIs like \emph{in weeks} in English are the immediate scope of negation, the scope of \emph{without}, and the scope of negative universal quantifiers like \emph{nobody}.  Furthermore the quantifier \emph{few} is in some cases capable of licensing strong NPIs, but the matter is a bit complex (see the discussion by \citealt{chierchia13a}) and since there is no  lexical item corresponding to \emph{few} in Japanese, we will not discuss \emph{few} in this paper. Any environment that licenses strong NPIs also licenses weak NPIs as far as we know.\footnote{In Slavic, there may be exceptions.}  English \emph{but}-exceptives are licensed in the scope of negation as (\ref{syenglish:negation1}) illustrates.

\ea \label{syenglish:negation1}
Susi didn't choose any number but 11.\z

The parallel example (\ref{syjapanese:negation1}) shows that a \emph{sika}-phrase associated with the object noun \emph{player-o} is licensed by clausal negation as well.  

\ea \label{syjapanese:negation1}
\gll Susi-wa suuzi-wa 11-sika eraba-na-katta\\
     Susi-\M{top} number-\M{top} 11-\M{sika} choose-\M{neg}-\M{past}\\
\glt `Susi didn't choose any number but 11.'\z

Example (\ref{syjapanese:negation2}) shows furthermore that also a bare \emph{sika}-phrase is licensed in the same structural configuration. In other words, the Japanese \emph{sika}-phrase does not need an associated NP that defines its domain of alternatives.

\ea \label{syjapanese:negation2}
\gll Susi-wa 11-sika eraba-na-katta\\
     Susi-\textsc{top} 11-\textsc{sika} choose-\M{neg}-\textsc{past}\\
\glt `Susi chose only 11.'\z


One difference between English and Japanese arises with subjects:
While a clause-mate negation does not license an NPI in the subject position in English (with some exceptions, see \citealt{uribe95a}), NPIs in subject position are generally licensed by clausal negation in Japanese.
We think such differences relate to cross-linguistic differences in the position of the subject of the type \cite{wurmbrand06a} discusses.
Thus the fact that in examples like (\ref{syjapanese:subject}) \emph{sika} is licensed is expected and should not be attributed to differences between English \emph{but} and Japanese \emph{sika}, but to different structural properties of the two languages.\footnote{Also negative concord items are licensed in the subject position of a negated verb in Japanese:
\ea \label{syNegQ-1}
\gll Dare-mo sushi-o tabe-na-katta.\\
Ind-\M{mo} sushi-\M{acc} eat-\M{neg}-\M{past}\\
\glt `Nobody ate sushi.'\z}

\ea *Anybody but Susi didn't eat sushi.\z

\ea \label{syjapanese:subject}
\gll Susi-sika sushi-o tabe-na-katta.\\
Susi-\M{sika} sushi-\M{acc} eat-\M{neg-past}\\
\glt `Nobody but Susi ate Sushi.'\z


The requirement to be in the immediate scope of the licensor is an important property of polarity licensing, but one that does not distinguish between strong and weak licensing.
Therefore we do not explore this here in detail, but to note that negation in restructuring environments \citep{wurmbrand01b} can license a  \emph{sika}-phrase associated with the embedded object position \citep{muraki78}.

%\ea \gll Bill-ga [John-ga Mary-to-sika au yoo(ni)] nozoma-nakat-ta.\\
%Bill-NOM John-NOM Mary-with-NPI meet C hope-NEG-TNS\\
%\glt `Bill (Neg) hoped [John meet [(NPI) only Mary]].'
%(adapted from \cite{uchibori2000}: Ch.5; Appendix 2.2, (16))\z

\ea \gll Susi-ga [Kazuko-ga kanozyo-to-sika kaado-o kookan-suru yooni nozoma-naka-tta.]\\
Susi-\M{nom} Kazuko-\M{nom} she-with-\M{sika} card-\M{acc} trade-do \M{comp} hope-\M{neg}-\M{past}\\
\glt `Susi didn't hope for Kazuko to trade cards with anyone but her.'\z


The argument of \emph{without} is another environment where NPIs are licensed in English.
There is no general counterpart in Japanese for the English expression \textit{without} across all its possible uses.  But in some cases the suffix \emph{nasi} as in \emph{hituzi-nasi} (`without sheep') can be used.  A \emph{sika}-phrase is licensed within this structure, as shown below.
\ea \label{syex:nasi}
    \gll Susi-wa 11-sika-nasi-de ka-tta.\\
        Susi-\M{wa} 11-\M{sika}-without-at  win-\M{past}\\
    \glt    `Susi won without anything but 11.'\z
%       
Licensors such as \emph{without} are relevant for the classification of polarity sensitive words.  We return to this topic below.

In English, \emph{but}-exceptives are licensed in the scope of a negative quantifier as the following example shows. 

\ea Nobody chose any resources but sheep.\z
%
The Japanese counterpart in (\ref{syex:kawamori}), however, is not acceptable (\citealt{kawamori01a}). This is curious given that there is a sentential negation that could license \emph{sika}. 

\ea \label{syex:kawamori}
\gll *Dare-mo sigen-wa hituzi-sika tora-na-katta.\\
who-\M{mo} resource-\M{top} sheep-\M{sika} take-\M{neg-past}\\
\glt `Intended: Nobody took any resources but sheep.' \z

When the subject is changed to a universal quantifier as in (\ref{syex:daremoga}), however, the sentence becomes grammatical, suggesting that it is the existence of both negative quantifier and the \emph{sika}-phrase, both of which require a sentential negation, that affects the grammaticality.

\ea \label{syex:daremoga}
\gll Dare-mo-ga sigen-wa hituzi-sika tora-na-katta.\\
who-\M{mo}-\M{nom} resource-\M{top} sheep-\M{sika} take-\M{neg-past}\\
\glt `Everybody took no resources but sheep.' (lit: `Everybody didn't take any resources but sheep')\z



\subsection{Weak NPI licensing environments}

Now consider environments that license only weak NPIs, but not strong ones \citep{zwarts98a,gajewski11a}.
Relevant environments in English are the restrictor of universal determiner quantifiers such as \emph{every}, the restrictor of negative universal quantifiers such as \emph{no}, the restrictor of free choice \emph{any}, and conditional clauses.  
As can be seen below, the strong NPI like \emph{in weeks}, for instance, is not licensed in any of these environments as the following examples illustrate \citep{hoeksema05m}.

\ea a. *Every player who won in weeks celebrated.\\
    b. *No player who won in weeks celebrated.\\
    c. *Susi can beat any player who won in weeks.\\
    d. *If Susi has won in weeks, she is happy.\z

We will see that \emph{but}-exceptives are licensed in all of them while \emph{sika}-phrases are licensed in none of them, as we consider exceptives in these four environments in turn.

\paragraph{Restrictor of a positive universal} 
Following two examples show that \emph{but}-exceptive is licensed when it is in the restrictor of a positive universal, as illustrated by (\ref{syex:player}), or in the relative clause if the restrictor of \emph{every} is formed by a relative clause, as illustrated by (\ref{syex:rolled}).

\ea \label{syex:player} 
   Every player \textbf{but Susi} has access to the ocean.\z

\ea \label{syex:rolled}
    Every number that anybody \textbf{but Susi} likes was rolled.\z

Before we consider universal quantifiers in Japanese, let us clarify the general structure of the structures we focus on.
First, there are a number of quantificational expression with positive universal force in Japanese, just like in English.
In this paper, we primarily focus on quantificational expressions formed with \emph{indeterminate pronouns}  (\citealt{nishigauchi90, shimoyama06, yatsushiro09b} among others), but the generalizations we develop hold, as far as we know, for the other quantificational expressions as well.
Second, the Japanese quantifiers that we focus on are created by combining an indeterminate pronoun and a dedicated suffix, as  illustrated in the following table.
The most well-known use of the indeterminate pronoun is as a \emph{wh}-phrase. 
We list below the expressions that are relevant for the following discussion.

\begin{center}
  \begin{tabular}{rcccc}
  {} & \emph{Wh} & Universal: \textit{mo} & Negative: \textit{mo} & Free choice: \textit{demo}\\\toprule
  who & dare & dare-mo-\M{case} & dare-mo & dare-demo\\
  what & nani & nani-mo-\M{case} & nani-mo & nan-demo\\
  which & Dono & dono-\M{noun}-mo & dono-\M{noun}-mo & dono-\M{noun}-demo\\
  \end{tabular}
\end{center}

Let us now consider whether \emph{sika}-phreases are licensed in the same environment as \emph{but}-exceptives.
The following example is the Japanese literal translation of (\ref{syex:player}).  The example is ungrammatical because of the presence of the  \emph{sika}-phrase.  This may be because the environment the \emph{sika}-phrase occurs in does not license it, though it may also be attributed to an impossibility to attach the \emph{sika}-phrase to a quantified noun phrase.

\ea \gll *Dono pureeyaa-mo Susi-sika umi-ni akusesu-ga aru\\
which player-\M{mo} Susi-\M{sika} ocean-to access-\M{nom} exists.\\
\glt `Intended: Every player \textbf{but Susi} has access to the ocean.\z

We do observe, however, that the same contrast between English and Japanese obtains also for the literal translation of (\ref{syex:rolled}) in (\ref{syex:rolledJ}) where the \emph{sika}-phrase occurs inside of a relative clause.
Since in this case the \emph{sika}-phrase does not occur with an associated nominal phrase, only the lack of a licensing environment can explain the ungrammaticality of (\ref{syex:rolledJ}).

\ea \label{syex:rolledJ}
\gll *Susi-sika suki-na dono-kazu-mo de-ta.\\
Susi-\M{sika} like-cop which-number-\M{mo} turn.out-\M{past}\\\z

This contrasts with an example in which a negation is added to (\ref{syex:rolledJ}) to the relative clause internal verb.

\ea \gll Susi-sika suki-ja-nai dono-kazu-mo de-ta.\\
Susi-\M{sika} like-cop-neg which-number-\M{mo} turn.out-\M{past}\\
\glt `Every number that nobody but Susi likes was rolled.'\z

\paragraph{The restrictor of negative universals}   In English, the negative determiner \emph{no} licenses a \emph{but}-exceptive in its restrictor as the following example illustrates (recall also \ref{syex:noplayer}).
\ea{} No family members but Susi took a development card. \z

The  Japanese counter-part (\ref{syex:kazoku}), however, is ungrammatical.

\ea \label{syex:kazoku}
\gll *Dono-kazoku-no itiin-mo Susi-sika kaado-o tora-na-katta\\
which-family-\M{gen} member-\M{mo} Susi-\M{sika} card-\M{acc} take-\M{neg-past}\\
\z

Recall that the grammatical example (\ref{syex:playerwa}) in the introduction involved a bare noun and a sentential negation though the natural English translation involved a negative quantifier. The contrast of (\ref{syex:kazoku}) to its English counter-part and to (\ref{syex:playerwa}) both point to the lack of licensing of a \emph{sika}-phrase in the restrictor of a negative universal quantifier.  To further strengthen this point, consider also some slightly different structures.
Example (\ref{syJapanese-NegQ1}) shows that a \emph{sika} phrase is also ungrammatical in the restrictor of the Japanese equivalent of \emph{nobody}.\footnote{The negative quantifier in (\ref{syJapanese-NegQ1}) can be distinguished from a universal quantifier by the absence of a Case marker and the prosody of the quantifier. See the further discussion in Section \ref{sysc:negq}.}

\ea \label{syJapanese-NegQ1}
\gll *Dare-mo Susi-sika 11-ni mati-ga na-katta.\\
who-\M{mo} Susi-\M{sika} 11-to town-\M{nom} \M{neg}-\M{past}\\
\glt Intended: `Nobody but Susi had a town on 11.'\z

Example (\ref{syex:norelclause}) shows that a \emph{sika}-phrase also is not licensed in a relative clause that is part of the restrictor of \emph{dare-mo + na} (`nobody') in Japanese.  In sum, we conclude that \emph{sika}-phrases are never licensed within the restrictor of a negative universal in Japanese.

\ea \label{syex:norelclause}
\gll *Susi-to-sika kaado-o kookan-si-ta dono-pureeyaa-mo kata-na-katta.\\
Susi-with\M{sika} card-\M{acc} trade-do-\M{past} which-player-\M{mo} win-\M{neg}-\M{past}\\
\glt intended: `No player who traded with anyone but Susi won.'
\z

As was the case with universal quantifier, adding a negation to the relative clause internal position licenses the \emph{sika}-phrase.

\ea 
\gll Susi-to-sika kaado-o kookan-si-naka-tta dono-pureeyaa-mo kata-na-katta.\\
Susi-with\M{sika} card-\M{acc} trade-not-\M{past} which-player-\M{mo} win-\M{neg}-\M{past}\\
\glt `No player who didn't trade with anyone but Susi won.'
\z


%When we add a \emph{sika}-phrase to the subject, however, the sentence becomes ungrammatical.
%
%\ea \gll *Dare-mo Susi-sika 11-no kaado-o eraba-na-katta.\\
%who-\M{mo} Susi-\M{sika} 11-\M{gen} card-\M{acc} choose-[ast\\
%\glt `Intended: Nobody but Susi chose the card with 11.'\z
        
\paragraph{Scope of Free Choice Items}
Free choice \emph{any/anyone} licenses \emph{but}-exceptives, whether they are in the subject position, as in (\ref{syex:freechoice1}), or the rest of the clause, as in (\ref{syex:freechoice2}). 
\ea \label{syex:freechoice1}
Anyone but Susi would take the resource with 6 over 11.\z

\ea \label{syex:freechoice2}
Susi would trade a card for two cards with anyone but Uli.\z

Example (\ref{syJ:freechoice1}) exemplifies free choice in Japanese. The counter part of free choice \emph{any} can be created by combining an indeterminate pronoun with the suffix \emph{demo}. 

\ea \label{syJ:freechoice1}
\gll Susi-wa dare-to-demo iti-mai-no kaado-o ni-mai-no kaado-to kookan-suru.\\
Susi-\M{top} who-with-\M{demo} 1-\M{cl}-\M{Gen} card-\M{acc} 2-\M{cl}-\M{gen} card-with exchange-do\\
\glt `Susi trades a card with two cards with anyone.'\z

But free choice does not license \emph{sika}-phrases. Specifically when we add a \emph{sika}-phrase to (\ref{syJ:freechoice1}), the sentence becomes ungrammatical:

\ea \label{syex:ulisika}
\gll *Susi-wa dare-to-demo Uli-to-sika iti-mai-no kaado-o ni-mai-no kaado-to kookan-suru.\\
Susi-\M{top} who-with-\M{demo} Uli-with-\M{sika} 1-cl-\M{Gen} card-\M{acc} 2-cl-\M{gen} card-with exchange-do\\
\glt `Intended: Susi trades one card for two cards with anyone but Uli.'\z

Since (\ref{syex:ulisika}) may be ungrammatical for independent  reasons, consider also (\ref{syex:jfreechoice}), where the \emph{sika}-phrase  occurs in a relative clause inside the restrictor of free-choice \emph{demo}.
Since (\ref{syex:jfreechoice}) is also ungrammatical, we conclude that \emph{sika}-phrases are never licensed in the restrictor of free choice items.
%\ea \label{syex:jfreechoice}
%\gll *Susi-to-sika kaado-o kookan-suru dare-demo katta.\\
%Susi-\M{sika} card-\M{acc} trade-do who-\M{mo} win-\M{neg}-\M{past}\\
%\glt intended: `Anybody who trades a card with anyone but Susi would win.'
%\z


\ea \label{syex:jfreechoice}
\gll *Susi-wa [hituzi-sika motteiru] dare-to-demo kaado-o kookan-suru.\\
Susi-\M{top} sheep-\M{sika} have who-with-\M{demo} card-\M{acc} exchange-do\\
\glt `Intended: Susi would trade with anyone who has any card but sheep.'\z

The grammaticality improves with a relative-clause internal negation, as shown below.

\ea 
\gll Susi-wa [hituzi-sika motte-inai] dare-to-demo kaado-o kookan-suru.\\
Susi-\M{top} sheep-\M{sika} have-\M{neg} who-with-\M{demo} card-\M{acc} exchange-do\\
\glt `Intended: Susi would trade with anyone who has any card but sheep.'\z




%\ea \label{syevery-restrictor}
%Every number but two was rolled.\z


%Japanese universal quantifier is created by combining indeterminate pronoun with a suffix \emph{mo} (\cite{shimoyama11, yatsushir09b}.

%\ea \gll Dono-kazu-mo de-ta.\\
% which-number-\M{mo} come.out-\M{past}\\
%\glt `Every number came/was rolled.'\z


%Addition of \emph{sika}-phrase makes the sentence ungrammatical.

%\ea \gll Dono-kazu-mo 2-sika de-ta.\\
%which-number\M{mo} 2-\M{sika} come.out-\M{past}\\
%\glt `Intended: Every number but 2 came/was rolled.'\z
   
\paragraph{Conditionals}
Conditional clauses are another environment where a \emph{but}-exceptive is licensed like other weak NPIs.
\ea \label{syif-clause}
If any number but 7 is rolled, Susi will win.\z

But the Japanese counterpart of (\ref{syif-clause}) in (\ref{syex:jcond}) is ungrammatical.  Again the \emph{sika}-phrase patterns with strong NPIs rather than weak NPIs.

\ea\label{syex:jcond}
   \gll *Suuzi-wa 7-sika de-tara, Susi-ga katu.\\
        number-\M{top} 7-\M{sika} come.out-\M{cond} Susi-\M{nom} wins\\
    \glt `Intended: If any number but 7 is rolled, Susi wins.'\z

\paragraph{Questions}
Questions are another environment where weak NPIs can be licensed, but strong NPIs never are.\footnote{The licensing of NPIs in questions presents several theoretical complications.  See \cite{nicolae15b} for an account in terms of monotonicity.} In questions, too, \emph{but}-exceptives can be licensed, as shown in (\ref{syquestion}).
\ea \label{syquestion}
Does Jonathan need any other cards but sheep?\z
Japanese counterpart, shown in \ref{syj:question}, is ungrammatical, showing that question does not license \emph{sika}-phrase.
\ea \label{syj:question}
\gll *Jonathan-wa hituzi-no kaado-sika iri-masu-ka?\\
Jonathan-\M{Top} sheep-\M{gen} card-\M{sika} need-polite-\M{Q}\\
\glt `Intended: Does Jonathan need any other cards but sheep?'\z



%\subsection{disjunction in Japanese}

%\ea \gll Kai-ga burokkorii-ka guriin piisu-o tabe-ta.\\
%Kai-Nom broccoli-or green peas-Acc eat-past\\
%\glt `Kai ate broccoli or green peas.\z

%True when:

%1. Kai ate broccoli.

%2. Kai ate green peas.

%3. Kai didn't eat both broccoli and green peas.

%Under negation:

%\ea \gll Kai-ga burokkorii-ka guriin piisu-o tabe-naka-tta.\\
%Kai-Nom broccoli-or green peas-ac eat-neg-past\\
%\glt `Kai didn't eat broccoli or Kai didn't eat green peas.'\z

%True when:

%Kai didn't eat broccoli, but ate green peas.

%Kai didn't eat green peas, but ate broccoli.\\
%  $\leadsto$ Kai did eat one of broccoli and green peas.

%False when:

%Kai didn't eat broccoli and Kai didn't eat green peas.

%English?

%Kai didn't eat broccoli or green peas.

%True when:

%Kai didn't eat broccoli and Kai didn't eat green peas.

%(also possible when Kai ate one of them??)




%\subsection{Combining \emph{ka} and \emph{sika}}

%\ea \gll Kai-wa burokkorii-ka guriin-piisu-sika tabe-nak-atta.\\
%Kai-top broccoli-KA green-peas-SIKA eat-neg-past\\
%\glt `Kai didn't eat anything other than broccoli or green peas.\z

%What do we expect the interpretation to be?

%\emph{Ka} is a positive polarity item, and hence, we should have the following interpretation:

%\ea Kai didn't eat anything other than broccoli, or Kai didn't eat anything other than green peas. $\leadsto$ Kai ate only one of the two.\z
    
    
%The \pprev\ instead is true in the following:
%\ea Kai didn't eat anything other than broccoli or green peas.\z


%1. Kai ate broccoli
%2. Kai ate green peas
%3. Kai ate broccoli and green peas
    

%\ea  not [broccoli ka green peas]-sika $\l y$ . Kai $y$ eat\\
%It's not the case that Kai ate anything other than broccoli or green peas.\z

%Why doesn't \emph{ka} need to scope out of \emph{not}?  Rescuing effect:

%\ea $\neg$ [$\exists y$ ($\neg$ [$y =$ broccoli \lor $y =$ peas] \land eat ($y$) (Kai))]\z


\subsection{Negative Quantifiers}\label{sysc:negq}

\cite{aoyagi94}, \cite{kawahara08a}, \cite{tanaka97a}, \cite{yoshimura2007b} among others draw parallels between the distribution of \emph{sika} and negative concord in Japanese.
Also the data we have reviewed so far have shown a number of similarities between the two phenomena: Both \emph{sika}-phrases and negative concord items are licensed in the immediate scope of a clausal negation ((\ref{syex:NPIgood}) and (\ref{syex:NPIbad}) and can be licensed when they occur in the subject position (\ref{syex:NPIsubject}).


\ea \label{syex:NPIgood}
\gll Susi-wa dare-to-mo kaado-o kookan-si-na-katta.\\
Susi-\M{top} who-with-\M{mo} card-\M{acc} exchange-do-\M{neg}-\M{past}\\
\glt `Susi didn't exchange cards with anyone.'\z

\ea \label{syex:NPIbad}
\gll *Susi-wa dare-to-mo kaado-o kookan-sita.\\
Susi-\M{top} who-with-\M{mo} card-\M{acc} exchange-do-\M{neg}-\M{past}\\\z

\ea \label{syex:NPIsubject}
\gll Dare-mo Uli-o kaado-o kookan-si-na-katta.\\
who-\M{mo} Uli-with-\M{mo} card-\M{acc} exchange-do-\M{neg}-\M{past}\\
\glt `Nobody exchanged cards with Uli.'\z




And both types of items cannot be licensed by weak NPI licensing environments.  Furthermore the class of expressions that license \emph{sika} and negative concord item exhibit further overlap, namely, the expressions \emph{nasi} (`without'), \emph{iya} (`hate'), and \emph{dame} (`not good') license both \emph{sika} and negative concord items.

Example (\ref{syex:nasi}) above showed that \emph{nasi} (`without') can license \emph{sika}-phrases.  \emph{Nasi} can also license negative concord items in Japanese:

\ea \gll Kyoo-no geemu-wa hituji-igai [nani-mo nasi] datta.\\
today-\M{gen} game-\M{top} sheep-other.than what-\textsc{mo} without was\\
\glt `Today's game was without anything other than sheep'\z


\cite{hasegawa11a} observes that also \emph{iya} (`dislike/hate')  licenses \emph{sika}-phrase:

\ea \gll Rokku-sika iya.\\
rokku-\M{sika} hate \\
\glt `I only like rock.'\z

We add to this the observation that  \emph{dame} (`not good') also licenses a \emph{sika}-phrase as shown in (\ref{syex:dame}).\footnote{Other predicates like \emph{kirai} (`dislike') and \emph{warui} (`bad') do not license \emph{sika}-phrase even though they have similar meanings to \emph{iya} and \emph{dame} respectively.
Their inability to licence \emph{sika} is shown by the following two examples:
\ea \gll *Otya-sika kirai.\\
green.tea-\M{sika} hate \\
\glt `Intended: I only like green tea.' (from \citealt{hasegawa11a})\z

\ea \gll *Hituzi-sika warui-desu.\\
sheep-\M{sika} bad-copular\\
\glt `Intended: Nothing but sheep is good.'\z
}
\emph{dame} (`not good') also licenses negative concord items:

\ea \label{syex:dame}
   \gll Nani-mo iya / dame-desu\\
   what-\M{mo} hate / not.good-\M{cop}\\\z

Nevertheless we think an account of \emph{sika}-phrases as negative concord item is not viable because there are also differences in the distributions of the two types of items.  
The parallels we just observed we attribute to the presense of a silent negative marker licensing negative concord in the above.
The main difference between \emph{sika}-phrases and negative concord lies in how many \emph{sika}-phrases and  negative concord items can be licensed by a single negation, namely a single negation can license multiple negative concord items.  
But we show in the following that, if a negation licenses one \emph{sika}-phrase it cannot license any other \emph{sika}-phrases.  
Example (\ref{syex:negconc}) illustrates that a single negation can license multiple negative concord items.

\ea \label{syex:negconc}
 \gll Dare-mo nani-mo tabe-na-katta.\\
who-\M{mo} what\M{mo} eat-\M{neg}-\M{past}\\
\glt `Nobody ate anything.' (literally: `Nobody didn't eat nothing.')\z

A single sentential negation, however, cannot license two \emph{sika}-phrases as \cite{aoyagi94}, \cite{kawahara08a}, \cite{miyagawa16a} and others observe.  
For example, (\ref{sytwo-sika}) is unacceptable.\footnote{\cite[(28b)]{miyagawa16a} claim that two occurrences of \emph{sika} can be licensed in a syntactic adjunct configuration.  But the translation they offer reveals that the structure must be bi-clausal.  If multiple licensing of \emph{sika} was possible in (\ref{syex:miyag}), it should be understood similar to \emph{`I have been to Karaoke with Shiori alone only once'}.
\ea \label{syex:miyag}
\gll Karaoke-e-wa itido-sika Shiori-to-sika it-ta koto-ga nai.\\ karaoke-to-\M{top} one.time-\M{sika} Shiori-with-\M{sika} go-\M{past} experience-\M{nom} \M{neg}\\
\glt ‘I have been to karaoke only once, only with Shiori.’ 
\z
}

\ea \label{sytwo-sika}
\gll *Susi-sika 11-sika eraba-na-katta.\\
Susi-\M{sika} 11-\M{sika} choose-\M{neg}-\M{past}\\
\trans `Intended: Nobody but Susi chose nothing but 11.'\z

When either one of the occurrences of \emph{sika} is replaced by another particle that means \emph{only}, such as \emph{dake} (`only'), the sentence becomes grammatical again.\footnote{Furthermore, \emph{dake} can be suffixed to both subject and object.

\ea \gll Susi-dake-ga 11-dake-o eran-da.\\
Susi-\M{dake}-\M{nom} 11-\M{dake}-\M{acc} choose-\M{past}\\
\glt `Only Susi chose only 11.'\z}

\ea \gll Susi-sika 11-dake-o eraba-na-katta.\\
Susi-\M{sika} 11-\M{dake}-\M{acc} choose-\M{neg}-\M{past}\\
\glt `Nobody but Susi chose only 11.'\z

\ea \gll Susi-dake-ga 11-sika eraba-na-katta.\\
Susi-\M{dake}-\M{nom} 11-\M{sika} choose-\M{neg}-\M{past}\\
\glt `Only Susi didn't choose anything but 11.'\z

These data show, then, that the reason the sentence in (\ref{sytwo-sika}) is ungrammatical is because the sentential negation fails to license two \emph{sika}-phrases.  
Furthermore example (\ref{syex:kawamori}) shows that a single negation cannot license both a negative concord item and a \emph{sika}-phrase.

As we have seen above, the licensing environment of \emph{but}-exceptives and \emph{sika}-phrases are different: \emph{sika}-phrases are licensed only in the contexts that license strong NPIs, while \emph{but}-exceptives can occur in both strong and weak NPI-licensing environments.  In the next section, we propose a theoretical analysis of this generalization.


\def\textsb#1{\textsf{\textbf{#1}}}

\section{Analysis}\label{sysc:analysis}

In this section, we propose an account of the distribution of \emph{sika} on the basis of an exceptive semantics.
We first summarize a version of the analysis of \emph{but}-exceptives in English that accounts for their weak NPI status.
We then present a modification of that analysis to account for the strong NPI distribution of \emph{sika}-exceptives. In a nutshell, we propose that, while both exceptives require obligatory exhaustification, they select for different exhaustification operators.


We adopt the analysis of \cite{hirsch16b} for \emph{but}-exceptives, which integrates the insights of \cite{fintel93} and \cite{gajewski08b}.
The concept for \emph{but} Hirsch proposes is the following:\footnote{We adapted the analysis of \cite{hirsch16b} to the Meaning First Approach \citep{sauerland20b}, and assume that a concept \textsb{but} exists that is (at least frequently) realized by `but' in English (i.e. \textbf{but} $\longrightarrow$ `but').  This corresponds to the statement `$[\![\textrm{but}]\!] = \textsb{but}$' in an interpretive approach such as that of \cite{heim98}.} 

\ea \label{syex:but}
    \textsb{but} = $\l x^e  \l y^e$ . $x$ and $y$ do not overlap.\z

In what follows, primarily one special case is relevant. Namely, if exceptive \emph{but} applies to two atomic individuals, $x$ and $y$, it requires that the two individuals be non-identical.  
Consider again the example (\ref{syex:player}) repeated in (\ref{syex:playerR}).

\ea \label{syex:playerR} 
   Every player but Susi has access to the ocean.\z

According to (\ref{syex:but}), which atomic individuals satisfy the scope of the quantifier \emph{every}?
The intersection of the two $et$-predicates expressed by \emph{player} and \emph{but Susi} results in the predicate $\l x\ .\ \text{player}(x) \wedge x \neq \text{Susi}$.
Concretely, if the set of players is $\{j, u, k, \text{Susi}\}$, \emph{player but Susi} is true only of $j$, $u$, and $k$.  
For the sentence (\ref{syex:playerR}), we derive therefore the inference that $j$, $u$ and $k$ each has access to the ocean.
But this inference captures only part of the meaning of (\ref{syex:playerR}).
The full meaning contribution of the exceptive amounts at least to the following three inferences:

\ea \label{syex:inferences}
    1. Every player other than Susi has access to the ocean.\\
    2. Susi is a player.\\
    3. Susi does not have access to the ocean. \z

The lexical meaning of \emph{but} and the other sentence parts only predict inference 1 of this list.
To capture inferences 2 and 3, \cite{hirsch16b} adopts the exhaustification operator \textbf{exh} \citep{chierchia13a,fox07b}.
At this time, several different versions of \textbf{exh} are being discussed in the literature including operators with different acronyms but a similar core semantics.
We focus on the version in (\ref{syex:exh}) for our purposes in this paper.
\textbf{exh} takes a set of alternatives $A$ and a proposition $p$, and asserts $p$ while it negates all excludable alternatives to $p$.

\ea \label{syex:exh}
\textbf{exh}$(A)(p) \Leftrightarrow p \land \forall q \in \text{excludable}(p,A) \ \neg q$\z

The set of alternatives is, we assume, determined as in focus semantics since the set of focus alternatives and scalar alternatives are closely related to each other (\citealt{gotzner19a} and others).
The concept \emph{excludable} is one of the major points of contention in the theory of exhaustification, and some aspects of the controversy are relevant to the understanding of polarity licensing.
Specifically, \cite{chierchia13a} proposes that NPIs are ungrammatical outside of antitone environments because the application of \textbf{exh} gives rise to obligatory logical contradictions (cf.\ \citealt{crnic14a}).
But \textbf{exh} can only give rise to logical contradictions if the notion of excludability is sufficiently lax.
For concreteness, we adopt the idea that all non-weaker alternatives are excludable:

\ea \label{syexcludable}
$\text{excludable}(p,A) = \{q \in A \mid p\  \not\rightarrow\ q\}$\z


Consider now how the addition of \textbf{exh} completes the account of (\ref{syex:playerR}) on the basis of the following representation:

\ea \label{syex:playerLF}
     \textbf{exh}$(A)$ [ every player but Susi$_F$ has access to the ocean ]\z

Assume still that the set of players is $\{s, j, k, u\}$ and that these also determine the alternatives under consideration.
$j$, $k$ and $u$ all lead to excludable alternatives, and therefore the following truth/falsity requirements arise:

\ea \label{syex:truefalse}
    true: $\forall x$: [player ($x$) $\land$ $x$ $\neq s$ ] $\rightarrow$ ocean-access($x$)\\
    false: $\forall x$: [player ($x$) $\land$ $x$ $\neq j$ ] $\rightarrow$ ocean-access($x$)\\
	  false: $\forall x$: [player ($x$) $\land$ $x$ $\neq k$ ] $\rightarrow$ ocean-access($x$)\\
    false: $\forall x$: [player ($x$) $\land$ $x$ $\neq u$ ] $\rightarrow$ ocean-access($x$)\z

If we compare the requirements in (\ref{syex:truefalse}) to the three inferences in (\ref{syex:inferences}), the first line accounts directly for the inference 1 of (\ref{syex:inferences}) while inferences 2 and 3  follow from (\ref{syex:truefalse}) in a less direct fashion.
For inference 2, consider that the inference 1 would entail that all players have access to the ocean if Susi was not a player.
But then none of the falsity-requirements in (\ref{syex:truefalse}) could be false. Therefore Susi must be a player.
By the same line of reasoning, we can also infer from (\ref{syex:truefalse}) that Susi must not have access to the ocean, i.e.\ inference 3 of (\ref{syex:inferences}).

In this way, the addition of \textbf{exh} completes the account of the meaning of (\ref{syex:playerR}).  
But \textbf{exh} also provides an account of the restricted distribution of \emph{but}-exceptives. Consider for example the following:

\ea *Some player but Susi has access to the ocean.\z

The truth and falsity conditions derived in the same scenario as above have existential quantifiers in the place of the universals of (\ref{syex:truefalse}):



\ea \label{syex:contradiction}
    true: $\exists x$: [player ($x$) $\land$ $x$ $\neq s$ ] $\land$ ocean-access($x$)\\
    false: $\exists x$: [player ($x$) $\land$ $x$ $\neq j$ ] $\land$ ocean-access($x$)\\
	  false: $\exists x$: [player ($x$) $\land$ $x$ $\neq k$ ] $\land$ ocean-access($x$)\\
    false: $\exists x$: [player ($x$) $\land$ $x$ $\neq u$ ] $\land$ ocean-access($x$)\z

But it is easy to see that the four requirements cannot be simultaneously satisfied, i.e.\  are logically contradictory.

Consider how the account extends to an example with an existential quantifier like \emph{any}, as in (\ref{syenglish:negation2}),  on the basis of representation (\ref{syenglish:negation1LF}).
Though the exceptive is also attached to an existentially quantified nominal, namely \emph{any player}, \textbf{exh} can take scope above negation and therefore a contradiction does not arise.

\ea \label{syenglish:negation2} 
Susi didn't trade with any player but Jonathan.\z

\ea \label{syenglish:negation1LF}
\textbf{exh}$(A)$ [ Susi didn't trade with any player but Jonathan$_F$ ]\z


To see that (\ref{syenglish:negation1LF}) is not contradictory, assume the same set of players as above with $j$ representing Jonathan and that Susi-trade-with is the predicate $\lambda x$ . Susi traded with $x$.  Then (\ref{syenglish:negation1LF}) amounts to the following truth and falsity requirements, which are structurally parallel to (\ref{syex:contradiction}), but because of the negation the opposite truth values are required in each of the four lines.  Therefore the requirements (\ref{syex:anyonebut}) are logically consistent, and require that Susi traded with Jonathan and did not trade with any of the other players.

\ea \label{syex:anyonebut}
    false: $\exists x$: [player ($x$) $\land$ $x$ $\neq j$ ] $\land$ Susi-traded-with($x$)\\
    true: $\exists x$: [player ($x$) $\land$ $x$ $\neq s$ ] $\land$ Susi-traded-with($x$)\\
	  true: $\exists x$: [player ($x$) $\land$ $x$ $\neq k$ ] $\land$ Susi-traded-with($x$)\\
    true: $\exists x$: [player ($x$) $\land$ $x$ $\neq u$ ] $\land$ Susi-traded-with($x$)\z


Let us now turn to \emph{sika}.  As we saw in Section \ref{sysc:data}, \emph{sika}-phrases can occur both with an associated noun or without any associated noun, but they cannot occur with a quantified noun phrase.
This suggest that \emph{sika}-phrases have a quantificational force as part of their lexical meaning unlike \emph{but}.
\cite{alonso-ovalle04a}, \cite{kawahara08a} and \cite{yoshimura2007b} suggest that \emph{sika} has universal quantificational force.
But we instead will assume \emph{sika} has existential force, i.e.\ along the lines of English (\ref{syenglish:negation1}) just as e.g.\ \cite{wurmbrand08b} reconsidered the logical force of \emph{nor}.
The lexical entry in (\ref{syex:sika}) provides an argument position for the associated noun phrase $R$, which we assume is filled by a null general noun when there is no overt associate.\footnote{Specifically, we assume that there is a concept \textsb{general-noun} = $l x \in D_e\ .\ x$ which is realized by a null phoneneme.  If the content of the $R$-argument of \textsb{sika} is unpronounced \textsb{general-noun} must occupy that position.}

\ea \label{syex:sika} 
   \textsb{sika} = $\l x \in D_e\ \l R \in D_{et}\ \l S \in D_{et}\ \exists y \in D_e$ . $x \neq y \land R(y) \land S(y)$\z

Consider how (\ref{syex:sika}) accounts for the interpretation of (\ref{syex:playerwa}) (repeated in \ref{syex:playerwaR}).

\ea \label{syex:playerwaR}
\gll Pureeyaa-wa Susi-sika umi-ni akusesu-ga nai.\\
player-wa Susi-\M{sika} ocean-{to} access-\M{nom} \M{neg}\\
\glt `No player but Susi  has access to the ocean.'\z

To be non-contradictory, negation must take scope above \emph{sika} as sketched in (\ref{syex:playerwaLF}) where we abstract away from  aspects of (\ref{syex:playerwaR}) that are not relevant for our purposes such as tense,  topic marking and the internal structure of the verbal complex.
Note that our interpretation of the constituent \emph{player-wa Susi-sika} is identical as for the English \emph{any player but Susi}.
We assume following the work on \emph{but}-exceptives that \emph{sika} must be associated with an occurrence of \textbf{exh} and that the noun that \emph{-sika} attaches to must receive focus, which \textbf{exh} must obligatorily associate with.

\ea \label{syex:playerwaLF}
    \textbf{exh}$(A)$ [  \text{-sika} (Susi$_F$) (\text{player}) $\lambda x$ [ $x$ umi-ni akusesu-deki ] ] nai \z

The interpretation of (\ref{syex:playerwaLF}) is parallel to that of (\ref{syenglish:negation1LF}) in the aspects relevant to the acceptability of \emph{sika}.
The truth and falsity requirements of (\ref{syex:playerwaLF}) are shown in (\ref{syex:oceanaccess}).
(\ref{syex:oceanaccess}) are consistent; namely, a scenario where Susi has  access to ocean and none of the other plays does satisfies all four requirements.

\ea \label{syex:oceanaccess}
    false: $\exists x$: [player ($x$) $\land$ $x$ $\neq s$ ] \& ocean-access($x$)\\
    true: $\exists x$: [player ($x$) $\land$ $x$ $\neq j$ ] \& ocean-acces($x$)\\
	  true: $\exists x$: [player ($x$) $\land$ $x$ $\neq k$ ] \& ocean-acces($x$)\\
    true: $\exists x$: [player ($x$) $\land$ $x$ $\neq u$ ] \& ocean-acces($x$)\z


The existential lexical entry for \emph{sika} directly predicts that a version of (\ref{syex:playerwa}) without negation in (\ref{syex:playeraru}) is ungrammatical by virtue of being an obligatory contradiction. In contrast, a lexical entry for \emph{sika} with universal force \citep{alonso-ovalle04a,kawahara08a,yoshimura06a} would lead us to expect that a \emph{sika}-phrase should be acceptable even when there is no negation in the sentence since the restrictor of a universal can license \emph{but}-exceptives.\footnote{\cite{shimoyama11} also associates \emph{sika} with universal force, but stipulates that it is a negative concord universal.  \cite{sells01a} and \cite{fintel07a} assume existential force for the Korean and French exceptives \emph{pakkey} and \emph{que} respectively.} 

The analysis up to know predicts the same distributional restriction for \emph{but}-exceptives and \emph{sika}-phrases.
But the environments where \emph{but}-exceptives and \emph{sika}-phrases can occur are not exactly the same as we discussed in Section \ref{sysc:data}.
Specifically, we showed that \emph{sika}-phrases are restricted to strong NPI licensing environments. 
For example,  while a \emph{but}-exceptive can occur in the restrictor of a universal quantifier as in (\ref{syex:butseven}), a \emph{sika}-phrase cannot.

\ea \label{syex:butseven}
Every player who rolled anything but seven built a city.\z

Why do English and Japanese differ with respect to examples like \ref{syex:butseven}?
The answer to this question we think lies in work on the distinction between weak and strong NPIs by \cite{gajewski11a} and \cite{chierchia13a}.
We present in the following an application of their ideas to exceptives.  

The central insight of \cite{gajewski11a} is that the distinction between the presupposition and the assertion component of meaning plays a role in the distinction between weak and strong NPIs (see also \citealt{homer08h}).
We use the \emph{fraction} notation in the following to designate a proposition  following \cite{harbour14a}: The fraction $a/p$ or $\scriptstyle\frac{a}{p}$ with numerator $a$ and denominator $p$ denotes the trivalent proposition with presupposition $p$ and truth condition $a$.\footnote{The role of numerator and denominator is the inverse of the notation one of us used in earlier work \citep{sauerland05f}. As Harbour notes, the inversal leads to the mnemonic corolarries that $a/1 = a$ and that $a/0$ is undefined.} Using this notation, the lexical entry for English \emph{every} in (\ref{syex:every}) captures that the universal quantifier carries an existential presupposition.

\ea \label{syex:every}
$\displaystyle \textsb{every} = \l R\ \l S\ \frac{\forallx [R(x) \rightarrow S(x)]}{\existsx R(x)}$\z

We furthermore adopt from \cite{chierchia13a} the proposal that two different exhaustification operators predict the distribution of strong and weak NPIs.
We recast this implementation by replacing the excludability as defined in (\ref{syexcludable}) with the following two notions for \textbf{exh}$_{S}$ and \textbf{exh}$_W$ respectively.
The two are distinct only with respect to the concept of excludability, which we define as excludable$_S$ and excludable$_W$ respectively:\footnote{In this definition, the strong notion doesn't always denote a superset of the weak one (i.e. $\text{excludable}_S(a/p,A) \supset \text{excludable}_W(a/p,A)$, also not superset relationship obtains in the opposite directiong, but that is desirable). For example, if $p = p' = x \lor y$, $a = x$, and $a' = y$ for two logically independent propositions $x$ and $y$ with $a'/p'$ in the alternative set $A$ of $a/p$,  then
$a'/p' \not\in \text{excludable}_S(a/p,A)$, but  $a'/p' \in \text{excludable}_W(a/p,A)$. Because only a superset relationship would make a contradiction more likely to arise, \cite{chierchia13a} doesn't predict that strong NPIs necessarily occur in a proper subset of the environments that license weak NPIs. 
We leave the question whether such environments can be constructed to be explored in future work.}

\begin{eqnarray*}
\text{excludable}_S\left(\frac{a}{p},A\right) &=& \left\{\frac{a'}{p'} \in A \mathrel{\Big|}  p  \not\rightarrow\ p'\right\}\\[1ex]
\text{excludable}_W\left(\frac{a}{p},A\right) &=& \left\{\frac{a'}{p'} \in A \mathrel{\Big|}  a \land p \land p' \  \not\rightarrow\ a' \land p \land p'\right\}
\end{eqnarray*}

The two operators \textbf{exh}$_{S}$ and \textbf{exh}$_{W}$ replacing \textbf{exh} from (\ref{syex:exh}) above are defined on the basis of these notions, but both specify the presuppositional and assertive component separately. The strong notion applies exclusion only in the presuppositional component.

\begin{equation*}
\textbf{exh}_S(A)\left(\frac{a}{p}\right) \Leftrightarrow \frac{a}{p \land \forall \frac{a'}{p'} \in \text{excludable}_S\left(\frac{a}{p},A\right) \neg p'}
\end{equation*}

The weak notion, on the other hand, applies exclusion in the assertive component only.  

\begin{equation*}
\textbf{exh}_W(A)\left(\frac{a}{p}\right) \Leftrightarrow \frac{a \land \forall \frac{a'}{p'}  \in \text{excludable}_W\left(\frac{a}{p},A\right) \ \neg a'}{p}
\end{equation*}

To understand the distribution of NPIs, we need to understand under what conditions  \textbf{exh}$_S$ and \textbf{exh}$_W$ give rise to a contradiction and more specifically, how this contradiction can be resolved.  Before we look at the concrete case, some general considerations: We only need to consider an item that is blocked  in an isotone environment -- i.e.\ gives rise to a contradiction.
The exclusion of a single alternative must be consistent with assertion or presuppposition of the uttered sentence by the definitions of excludability.  Therefore, the contradiction must derive from multiple alternatives.  Furthermore the contradictory alternatives cannot stand in a logical entailment relation to each other since otherwise one would be sufficient to trigger a contradiction.

Concretely, consider (\ref{syex:butseven}) now.  From (\ref{syex:every}), the sentence meaning prior to exhaustification (\ref{syap}) follows, where we indicate $a$ and $p$ corresponding to the preceeding discussion. 

\ea \label{syap}$\displaystyle \frac{a}{p} = 
\frac{\forall x\ [\ \exists n \neq 7\  \textsb{roll}(n)(x)  \rightarrow \textsb{build(city)}(x)\ ]}{\exists x\  \exists n \neq 7\  \textsb{roll}(n)(x)}$
\z

The relevant  alternatives for the exhaustification of (\ref{syap}) are of the form (\ref{syap11}), corresponding to $a'$ and $p'$ as indicated for an $m \in \{2, \dots, 12\}$ with $n \neq 7$.

\ea \label{syap11}$\displaystyle \frac{a'(m)}{p'(m)} = 
\frac{\forall x\ [\ \exists n \neq m\  \textsb{roll}(n)(x)  \rightarrow \textsb{build(city)}(x)\ ]}{\exists x\  \exists n \neq m\  \textsb{roll}(n)(x)}$
\z

For example with $m = 11$, the negation of the presupposition of (\ref{syap11}), $\neg p'(11)$ can be paragraphed as \emph{`Nobody rolled anything but 11.'}  It is possible to see that the conjunction of all $\neg p'(m)$ with $m \neq 7$ contradicts the presupposition $p$ of (\ref{syap}).   At the same time, the negated assertion $\neg a'(11)$ is paraphraseable as \emph{`Somebody rolled a number other than 11 and didn't built a city.'} The conjunction of all $\neg a'(m)$ with $m \neq 7$ is consistent with $a$ of (\ref{syap}) as can be seen from a scenario where somebody rolled a 7 and didn't build a city.

In sum, we have shown that exceptives are expected to be licensed in an environment when they associate with \textbf{exh}$_W$, but not when they associate with \textbf{exh}$_S$.  We showed this only for the specific case of the restriction of a presuppositional universal quantifier, but this suffices for our purposes here.  We showed that despite the same lexical meaning, different distributions are predicted for exceptives if they select different \textbf{exh} operators. 
In other words the difference between English \emph{but}-exceptives and Japanese \emph{sika}-phrases relates to the still unexplained different \textbf{exh}-operator selection properties of strong and weak NPIs if we are correct.  We can hope therefore that the different exceptives may help to understand this difference even better in the future.

We think our proposal can also shed light on the one difference between negative concord items in Japanese and \emph{sika}-phrases.  Specifically, We noted that a single negation cannot license multiple \emph{sika} phrases in the following Japanese example (repeated from \ref{sytwo-sika} above): 

\ea\label{sytsrep}
\gll *Susi-sika 11-sika eraba-na-katta.\\
Susi-\M{sika} 11-\M{sika} choose-\M{neg}-\M{past}\\
\trans `Intended: Nobody but Susi chose nothing but 11.'\z

The ungrammaticality of (\ref{sytsrep}) follows from our proposal if the \emph{sika}-phrase itself intervenes for the licensing of strong NPIs, i.e.\ if the semantics of \emph{sika} imposed a presupposition on their scope.  A presuppositional semantics is indeed what \cite[p.~461--2]{fintel07a} for independent reasons propose for the French exceptive \emph{que}. We propose the following revised lexical entry for \emph{sika} to replace  (\ref{syex:sika}):\footnote{\cite{fintel07a} discuss two different proposals and we follow the second suggestion here concreteness. Their other proposal would in our terms in (\ref{syex:sikarev}) adopt the presupposition $S(x)=1$.  Neither their evidence nor ours provides decisive evidence to distinguish between the proposals.}

\ea \label{syex:sikarev} 
  $\displaystyle \textsb{sika} = \l x \in D_e\ \l R \in D_{et}\ \l S \in D_{et}\ \frac{\exists y \in D_e\ .\ x \neq y \land R(y) \land S(y)}{\exists y \in D_e\ S(y)}$\z

Adding the presupposition predicts that in the configuration sketched below the lower \emph{sika}-phrase is not licensed.

\ea $\exh_S\ [\ \neg\ [\  \textsb{Susi-sika}\  [\ \textsb{11-sika} \dots\ ]\ ]\ ]$ \z

In sum, our account of \emph{sika} as an exceptive with existential force predicts the strong-NPI distribution of \emph{sika}-phrases and also the difference between negative concord items and \emph{sika}-phrases. 


%Takahashi

%\ea \gll *Dare-ga ringo-sika tabe-na-katta-no?\\
%Who-Nom apple-\M{sika} eat-neg-\M{past}-Q\\
%\glt `Intended: Who ate nothing but apples?'\z

%But, it seems that by changing \emph{ringo} to a wh-phrase, the sentence becomes grammatical.

%\ea \gll Dare-ga dono-kudamono-sika tabe-na-katta-no?\\
%Who-Nom which-fruit-\M{sika} eat-neg-past-\M{Q}\\
%\glt `Who ate only which fruit?'\z

%Environment where \textit{sika} is allowed:


%Alonso-Ovalle \& Hirotani's argument for an analysis as an exceptive---interaction with scalar\\
%\emph{2 or less than 2} vs. \emph{2 or more than 2}

%\ea \gll John-ga banana-o 2-tu-ika-sika tabe-nak-atta.\\
%John-Nom 2-cl-less.than-SIKA eat-neg-past\\
%\glt `John only ate two or less than two bananas.'

%\ea \gll \st John-ga banana-o 2-tu-ijoo-sika tabe-nak-atta.\\
%John nom banana-Acc 2-cl-more.than.SIKA eat-neg-past\\



%\cite{aoyagi94} (cited in \cite{takita11a}) note that \emph{sika} can occur with an overt NP-complement

%\ea \gll Taroo-ga ringo-sika   kudamono-o    tabe-na-katta\\
 %  Taroo-nom apple-sika fruits-acc eat-neg-past\\
  % \glt `Among fruits, Taroo ate only apples'
   






%To spell out such an exceptive analysis, we adopt the lexical entries for \emph{but} by \cite{hirsch16b} for \emph{sika} (see also \cite{fintel93,gajewski08b}):

%\ea \[but\] = \[sika\] = $\l X  \l Y$ . \neg Overlap(X, Y)\z
  
%We assume that the exhaustification operator \textbf{exh} must obligatorily associate with the focus on the argument of \emph{sika}.

%\ea \textbf{exh(A)}(not [ Mika sika Harry Potter$_F$ reads])\z


%But why does \emph{sika} need negation? It is not the requirement of \emph{only}

%\ea \gll Mika-ga Harry Potter-dake-o yon-da.\\
%Mika-Nom Harry Potter-only-acc read-past\\
%\glt `Mika read only Harry Potter.'\z

%\ea \gll Mika-ga Harry Potter-dake-o yoma-nak-atta.\\
%Mika-nom Harry Potter-only-acc read-neg-past\\
%\glt `Mika didn't read only Harry Potter.' \z


%Two possibilities:

%\emph{sika} is an NPI (\cite{shimoyama11} (and many syntacticians)
%\emph{sika} is a negative concord item (\cite{alonso-ovalle04a} \cite{kawahara08a})	


 

%Argument for \emph{sika} being a negative concord item 

%Downward entailing does not by itself license X-sika: Restrictor of universal quantifier (\cite{alonso-ovalle04a})

%\ea \gll *Rokku-sika ensoosi-ta dono bando-mo syoo-o kakutoku sita\\
%rock-SIKA play-past which band-MO award-acc receive did\\\z

%Intervention effect (\cite{alonso-ovalle04a})

%\ea \gll *John-sika nani-o utawa-nakat-ta-no?\\
%John-SIKA what-Acc sing-neg-past-Q\\\z



%Syntactic proposal by \cite{kawahara08a}:

%Sika-phrases move to Spec of Neg
%the movement is incurred by an uninterpretable focus feature and is regulated by island constraints




\section{Conclusion}

In this paper, we argued that \emph{sika}-phrases in Japanese should be analyzed as exceptives associated with existential force.  While this proposal is new for Japanese \emph{sika}, similar proposals have been made for \emph{pukkey} in Korean \citep{sells01a} and for \emph{que} in French \citep{fintel07a}.  Our focus has been to furthermore derive the distribution of \emph{sika}-phrases from this proposal.  We have shown that \emph{sika}-phrases have the distribution of strong NPIs in contrast to \emph{but}-exceptives in English, which have the distribution of weak NPIs.  We also showed that the distribution of \emph{sika}-phrases is different from that of negative concord items in Japanese.   To derive the basic properties of \emph{sika}-phrases and their distribution within the theory of NPI licensing of \cite{gajewski11a} and \cite{chierchia13a}, we introduced the following four stipulations:

\begin{itemize}
    \item the restrictor of \emph{sika}-phrases can remain silent, 
    \item the quantificational force associated with \emph{sika}-phrases must remain silent and is always existential,
    \item \emph{sika} must be related to strong exhaustification operator $\exh_S$, and
    \item \emph{sika} introduces an existential presupposition on its scope.
\end{itemize}

At least the first three, and possibly all four stipulations do not apply to English \emph{but}-exceptives, while at least three of the stipulations apply also to Korean \emph{pukkey} and French \emph{que}.  

In future work, we hope to understand whether the four stipulations above can be derived from fewer assumptions.
We think the contrast between the \emph{but}- and \emph{sika}-exceptives is a novel case of a pair of strong and weak NPI that seem to have the same core meaning, perhaps even more so than English NPI pairs such as the strong \emph{in weeks} and the weak \emph{ever}. This case may be of further theoretical interest since exceptives  give rise to their polarity property in a more transparent fashion than other NPIs.



\section*{Abbreviations}
\begin{tabularx}{.5\textwidth}{lQ}
\textsc{acc}  & Accusative       \\
\textsc{cl}   & classifier      \\
\textsc{comp} & complementizer    \\
\textsc{cop}  & copular        \\
\textsc{gen}  & genitive       \\      
\end{tabularx}
\begin{tabularx}{.5\textwidth}{lQ}
\textsc{ind} & indeterminate pronoun         \\
 \textsc{neg} & negation               \\
 \textsc{nom} & nominative             \\
\textsc{npi} & negative polarity item \\
\textsc{top} & topicalization        \\
\end{tabularx}

\section*{Acknowledgements}

This paper would have been impossible without Susi Wurmbrand's friendship and support for over more than a quarter of a century, including letting us win at \emph{Die Siedler of Catan} occasionally.

We thank Aron Hirsch, Masao Ochi, two seminar audiences at Osaka University, and two anonymous reviewers for their helpful comments.  This project has received funding from the European Research Council (ERC) under the European Union's Horizon 2020 research and innovation programme (grant agreements No 787929 and No 856421) and from Osaka University, International Joint Research Promotion Program, project \emph{On Development of Logical Language and Mathematical Concepts}, PI Yoichi Miyamoto.

\printbibliography[heading=subbibliography,notkeyword=this]

\end{document}

%
%
%\ea \gll Hanako-wa ima [Taroo-ga ringo-sika kudamono-o tabe-na-katta]-to kigatuita.\\
%Hanako-top now Taroo-nom apple-SIKA fruit-Acc eat-neg-past-that noticed\\
%\glt `Hanako noticed just now that Taro didn't eat any fruits but apples.'
%
%\begin{itemize}
%\item Hanako noticed just now that Taro ate apples.
%\item Hanako noticed just now that Taro didn't eat any fruits other than apples
%\item NOT: Hanako noticed just now that apples are a type of fruits
%\end{itemize}
%


