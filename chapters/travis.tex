\documentclass[output=paper,colorlinks,citecolor=brown,
% hidelinks,
% showindex
]{langscibook}
\author{Lisa deMena Travis\affiliation{McGill University}\orcid{}}
\title{Heads first: the rest will follow}
\abstract{The goal of this paper is to find a stable place for head movement within narrow syntax.  The first step is to introduce a typology of movement proposed by Travis and Massam \citeyearpar{Travis:2021} that extends beyond the better known \=A- and A-movement of dependents (limb movement) to include not only movement of elements along the extended projection (spinal movement), but also roll-up movement that violates anti-locality (labelled C-movement).  Viewing head movement within this context, a case is made that head movement shares the characteristics of Spinal C-movement, the only distinction being the level of projection that is moved.  Candidates are then proposed to fit other cells of the typology – \=A- and A-movement of both limbs and spines – to make a complete, though speculative, picture.  As a final step, suggestions are made for rethinking the Extension Condition and E-merge/I-merge in order to create a grammatical system that includes rather than excludes head movement.}

\begin{document}
\maketitle

%%%

\vspace{1cm}
\noindent [Head movement] is illegitimate.  Head-movement is not formulable in any framework addressing the conditions of genuine explanation.\\
 \hspace*{\fill} Chomsky \citeyear{Chomsky:2021}, 42:19

\section{Introduction}

Head movement has had a precarious position in the realm of narrow syntax for a few decades (see, e.g.\ Chomsky \citeyear{Chomsky:2001a}:37-39) but this was not always the case.\footnote{By `precarious'  I am not suggesting that  head movement does not have its defenders.  It does (too many to cite here).  But, unlike phrasal movement, it has many detractors (also too many to cite here).  As the introductory quotation indicates, Chomsky, in his 2021 WCCFL plenary talk, dismisses head movement as illegitimate, and it is not hard to find a significant number of syntacticians who agree.} For example, when Rizzi proposed Relativized Minimality, head movement fit nicely into his typology of movements with respect to the locality of movement.  

\needspace{5\baselineskip}
\ea Relativized Minimality (Rizzi \citeyear{Rizzi:1990}: taken from Rizzi \citeyear{Rizzi:2001a})
    \ea \ldots{} X \ldots{} Z \ldots{} Y\\
    \ex Y is in a Minimal Configuration (MC) with X iff there is no Z such that\\
        \ea Z is of the same structural type as X, and\\
        \ex Z intervenes between X and Y\\
        \z
    \ex The typology must involve at least two irreducible\label{ex:RMc}
    distinctions:\\
        \ea between heads and phrases and, in the latter class,\\
        \ex between positions of arguments (A-positions) and of non-arguments
    (\=A-positions).
        \z
    \z
\z

\noindent 
While part of the typology, it was nevertheless clear that head movement had less in common with the two other movement types as it differed in size as seen in (\ref{ex:RMc}). The former, by definition, was movement of an X$^0$ and the latter was movement of an XP.  It was equally clear in Rizzi's characterization, however, that it was part of the same family of movements.  In this paper, I argue that the distinction of size continues to matter, but I also argue that this difference in size crucially does  not exclude head movement from the narrow syntactic movement family.  Basically, I argue that a grammatical system can be created that not only allows for but also explains the diversity that we find in movement.

I propose that, to best understand the status of head movement, other under-represented types of movement also have to be included.  These other movements differ from the more familiar XP movements not in the size of what moves but rather in what part of the syntax structure moves and how.  Once these more diverse movement types are included, head movement looks less like an outlier.  I start by reporting on results outlined in Travis and Massam \citeyearpar{Travis:2021} (T\&M), introducing an \=A and A distinction for XP movement that targets XPs along an extended projection (Spinal Movement), in this case VP.  The T\&M typology also includes a third type of spinal movement, labelled C-movement, which is more local than A-movement.   Once Spinal Movement and C-movement are added to the picture, we will see that head movement actually shares characteristics with XP movement, in particular C-Spinal Movement.\footnote{This connection has already been made by Pearson \citeyearpar{Pearson:2000}.}  Having established the background from T\&M, I then speculate on ways in which the emerging movement typology might be expanded to include a variety of types of head movement, just as we find a variety of types of XP movement.  Once the typology is extended to include a wider diversity of movements, I suggest a re-envisioned version of the grammatical system itself that naturally legitimizes this range of diversity.
 
%A DIFFERENT ENDING\\
%There has been a problem with fitting head movement into the system that has been created with the properties of XP movement in mind.  An alternative is to start by recognizing the central role of heads in generating syntactic structure (initial merge, holder of probing features), and allow equal importance when moving into the domain of movement.  I end with some speculations on what such a system would look like as a first step in moving towards and more egalitarian system.

\section{Expanding the movement landscape}

In Travis and Massam \citeyearpar{Travis:2021}, VP fronting is investigated in an effort to determine the difference between the type of VP movement one finds in more well-studied languages such English and German as opposed to the type of VP movement that has been proposed for Austronesian languages such as Niuean and Malagasy.  The study reaches three conclusions. First, VP movement may be expected to have different characteristics from DP movement as it involves movement not of dependents of the lexical head of the clause (i.e.\ limbs), but rather movement of (extended) projections of this head (i.e.\ parts of the spine).  Second, certain types of VP fronting can be argued to correlate either with \=A movement (English/German VP fronting) or with A movement (Niuean). Third, the properties of VP fronting in Malagasy, which are similar to the roll-up movement, seen for example in Cinque's (\citeyear{Cinque:2005}, \citeyear{Cinque:2014}) work, point to a third type of VP-fronting.  This third type of movement is labelled C-movement for reasons to be outlined in Section~\ref{sec:feature}.

\subsection{Adding Spinal Movement}

In investigating VP fronting, it is important to point out that focus has shifted away from DP movement and \textsc{wh}-movement, the more commonly studied movements.  These latter movements will be referred to as Limb Movement.\footnote{This distinction becomes very apparent, for example, in Ott \citeyearpar{Ott:2010} where Ott argues that a variety of different X(P)s along the V-headed extended projection can be fronted.}  The tree in (\ref{ex:extproj}) below highlights the differences between XP limb constituents (in boxes) and XP spinal constituents (in gray).  In this context, VP movement is Spinal Movement.

\qtreecenterfalse
\ea \label{ex:extproj}\footnotesize{ \Tree [.\colorbox{Gray}{XP} \framebox{Spec}  [.\colorbox{Gray}{X$'$} \colorbox{Gray}{X} [.\colorbox{Gray}{YP} \framebox{Spec}  [.\colorbox{Gray}{Y$'$} \colorbox{Gray}{Y} \framebox{\colorbox{Gray}{ZP}}  ]]]]}
\z

Note that, without more information on category type (lexical vs.\ functional),  ZP might either be a limb (if Z is the highest member of an extended projection) or part of the spine (if Z is a lower member of an extended projection).  T\&M look only at predicate fronting as an example of Spinal Movement showing that it can come in (at least) two varieties, one parallel to A-Limb Movement and one parallel \=A-Limb Movement. 

\subsubsection{\=A-Spinal Movement}

A case of predicate fronting well-known in the literature is the type of VP-fronting found in English examples such as (\ref{ex:EngVP}).  Similar examples may be found in other languages such as German (\ref{ex:VPGerm}) and Javanese (\ref{ex:VPJavanese}).

\ea VP fronting \label{ex:VPfront}
    \ea  \ldots{} and {[}\textsubscript{CP} {[}\textsubscript{VP} do their homework ] {[}\textsubscript{TP} they will \underline{\hspace{.5cm}} {]]} \hfill \textsc{English}\label{ex:EngVP}\\
    \ex[] { 
	\gll {[}\textsubscript{CP} [\textsubscript{VP} Das Buch gelesen ]  hat  [\textsubscript{TP} Peter  gestern \underline{\hspace{.5cm}} ]] \label{ex:VPGerm}\\
	~ ~ the book read ~ has ~ Peter yesterday\\
	\glt `Peter  read  the  book  yesterday' \hfill \textsc{German}\\
	}
% 	\z
\ex[] { 
		\gll {[}\textsubscript{CP} [\textsubscript{VP} nggotong watu-ne ] [\textsubscript{TP} cak Kholiq iso \underline{\hspace{.5cm}} ]]
	\label{ex:VPJavanese}\\
	~ ~ \textsc{av}.lift rock-\textsc{def} ~  ~ Mr. Kholiq \textsc{circ.pos}\\
	\glt `Lift the stone, Kholiq can.'  \hfill \textsc{Paciran Javanese}\\
	\hspace*{\fill} Vander Klok (2016: 213) 
}
\z
\z 

Such fronting fits quite nicely into the description of \=A-movement, as the fronted constituent appears to move to Spec, CP.\footnote{This is an oversimplification especially since we will eventually shift to a system where movement type is determined not by landing site, but rather by trigger feature, as in van Urk \citeyearpar{vanUrk:2015a}. I am assuming, however, that fronting of the sort seen in (\ref{ex:VPfront}) is triggered by an \=A-type feature, and such movements typically, though not exclusively, move to a position in the C domain.}  This is most clearly seen in the German example in  (\ref{ex:VPGerm}), where movement of the VP triggers movement of the auxiliary to second position.  Also, like typical \=A-movement, the movement comes with discourse effects.

%refs for A' spinal movement (to Spec, VP??)

\subsubsection{A-Spinal Movement}

It is more difficult to find a case of A-movement of the predicate, but this is likely to be due to the types of languages that have been most studied.  Massam and Smallwood \citeyearpar{Massam:1997}, however, propose that Niuean's predicate initial order is produced by the fronting of the predicate to Spec, TP.  Below we first see a sentence with VSO order (\ref{ex:NnoPNI}), suggesting that Niuean's word order is derived by head movement of the verb around the subject.  But as argued in Massam \citeyearpar{Massam:2001}, this order is misleading. If the object is indefinite (a case of Pseudo-Incorporation in Massam's terms), as in (\ref{ex:NPNI}), the object is fronted with the verb. The VOS order in (\ref{ex:NPNI}) shows that Niuean's word order is, in fact, VP first.  This XP movement of the predicate is masked when a definite object has moved out of the predicate and only the remnant of the VP has been fronted as in (\ref{ex:NnoPNI}).\footnote{More accurately, these are referential objects, but here I am abstracting away from such details.  See Massam \citeyearpar{Massam:2020} for more information.}


\ea[] {Niuean Pseudo Noun Incorporation (Massam \citeyear{Massam:2001}:157) \label{ex:Niuean}}
	\ea[] {
	\gll [ Takafaga t$_k$ ] t\=umau n\=i e \textbf{ia} [ \uuline{e} \uuline{tau}  \uuline{ika} ]$_k$\label{ex:NnoPNI}\\
	~ hunt ~ ~ aways \textsc{emph} \textsc{erg} he ~ \textsc{abs} \textsc{pl} fish\\
	\glt`He is always fishing.' \hfill [ V t$_k$ ]$_j$ \textbf{S} \uuline{O}$_k$ t$_j$ 
	}
	\ex[] {
	\gll [ Takafaga  \uuline{ika} ] t\=umau n\=i a \textbf{ia}\label{ex:NPNI}\\
	~ hunt fish ~ always \textsc{emph} \textsc{abs} he  \\
	\glt `He is always fishing.' \hfill [ V  \uuline{O} ]$_j$  \textbf{S}  t$_j$
	}
	\z
\z

VP fronting in Niuean is set up in Massam and Smallwood \citeyearpar{Massam:1997} as EPP driven movement to Spec, TP that correlates with EPP driven DP movement to Spec, TP in a language like English.  Further, this movement occurs in discourse neutral sentences, much like DP movement to Spec, TP, making it an obvious candidate for A-spinal movement. 

\subsection{Adding C-movement}

Staying within the Austronesian language family, we can find another language that has been argued to have predicate fronting: Malagasy (see Pearson \citeyear{Pearson:1997}, \citeyear{Pearson:2018}, Rackowski \citeyear{Rackowski:1998}, Rackowski \& Travis \citeyear{Rackowski:2000}).  While one might expect it to be  like Niuean, it is different in important ways.  This difference leads to the proposal of a third type of movement, adding to the A vs. \=A typology.  In this section I review the arguments from T\&M for C-movement and briefly summarize the feature-based typology that was used in that paper to describe the three types of movement.


\subsubsection{Very local VP movement}

To see the difference between Niuean predicate fronting and Malagasy predicate fronting, we look at what happens with definite object shift.  Pearson \citeyearpar{Pearson:2000} discusses object shift in the context of distinguishing between two types of VO languages -- \textsc{direct} (e.g.\ English, Icelandic) and \textsc{inverse} (e.g.\ Malagasy, Zapotec).  The former group creates the verb first order through head movement, while the latter group fronts the V through roll-up VP movement.  To take just one point of comparison, in direct languages (those that have head movement within the VP shell structure) definite object shift is to the left.  This is familiar in the literature with examples from Icelandic as shown below, where \textit{greinina} `the article' appears to the right of negation and the quantifier in (\ref{ex:icelandica}), and to the left in (\ref{ex:icelandicb}).  

\ea Icelandic (Holmberg \citeyear{Holmberg:1986}: 166)
    \ea[] {
    	\gll Hvers vegna lasu st\'udentarnir ekki allir \textbf{greinina}\label{ex:icelandica}\\
    	why ~ read the.students not all the.article\\
    	\glt `Why didn't all the students read the article?'\footnote{A translation is not given in the original article.  This translation is taken from \url{https://citeseerx.ist.psu.edu/viewdoc/download?doi=10.1.1.491.8626&rep=rep1&type=pdf}.}
    	}
    \ex[] {
    	Hvers vegna lasu st\'udentarnir \textbf{greinina} ekki allir \label{ex:icelandicb}
    	}
    \z
\z

Malagasy, an inverse language, where the V-initial VP is created through successive roll-up movement of the VP, has definite object shift to the right.   We examine the relevant data below.  We see first in (\ref{ex:bareNPgasy}) that the object appears adjacent to the verb. In this position it may  appear with or without a determiner.  In (\ref{ex:MalOS}), where an adverb intervenes between the object and the verb, the determiner is required.  Descriptively, it looks like only definite objects may move rightward over the adverb (whereas in Icelandic a definite object may move leftward). 

\needspace{5\baselineskip}
\ea Malagasy (Inverse) rightward object shift
	\ea[] {
		\gll Nijinja \textit{(ny)} \textit{vary} \textbf{haingana} ny mpamboly\label{ex:bareNPgasy}\\
		\textsc{pst-at.}cut (\textsc{det}) rice quickly \textsc{det} farmer\\
		\glt `The farmer harvested (the) rice quickly.' 
		}
	\ex[] { 
	\gll Nijinja   \textbf{haingana} *(\textit{ny}) \textit{vary} ny mpamboly\label{ex:MalOS}\\
	\textsc{pst-at.}cut  quickly ~\textsc{det} rice \textsc{det} farmer\\
	\glt `The farmer harvested *(the) rice quickly.' 
	}
	\ex[] {
		[[ V t$_k$ ]$_j$ \textbf{Adv} \uuline{O}$_k$ t$_j$ ]$_m$ \textbf{S} t$_m$
	}
	\z
\z

 Pearson's explanation for why movement of the definite object appears to be rightward has to do with the roll-up movement that defines inverse languages.\footnote{Pearson situates his analysis within the assumptions of the Linear Correspondence Axiom of \cite{Kayne:1994}, but the Malagasy object shift data would require a solution for any system that restricts movement to the left, such as Abels \& Neeleman \citeyearpar{Abels:2012}.}  Pearson proposes, as we have seen for Niuean, that the definite object moves leftward, and then the remnant VP, now containing only the V, moves to the left of the object (and the adverb).  While similar, predicate fronting in Malagasy is not identical to  predicate fronting in Niuean -- it is more local.  Rather than moving to a position in front of the subject, leaving the object to remain to the right of the subject as in Niuean, the moving predicate in Malagasy has a landing site between the position of the moved object and the subject.  Subsequent movement, then, displaces both the verb and the definite object to the left of the subject, creating a VOS order with definite objects  (see (\ref{ex:MalOS})) rather than the VSO order that we have seen in the case of Niuean (see (\ref{ex:NnoPNI})).  The distinction then is between the movement of the Niuean predicate over both the moved object and the subject, and the more local iterative movement of the Malagasy predicate.  This distinction is shown in the two tree structures below.

\begin{figure}
    \textcolor{red}{These trees have a problem. There should not be children labelled `!', but instead the left sibling constituents of these nodes should have a box around them.}
    \centering
    % \scriptsize{
    % \Tree  [.YP \textbf{S} [.Y$'$ Y [.XP [.VP \textbf{V} \sout{O} ] !{\qframesubtree} [.XP \textbf{O} [.X$'$  X \sout{VP} ]]] !{\qframesubtree} ]]\hspace{.4cm}
    \Tree 
        [.TP 
            [.VP\textsubscript{Pred} 
                \textbf{V} 
                \sout{O} 
            ].VP
            !{\qframesubtree} 
            [.T$'$ T\textsubscript{Pred}  
                [.YP \textbf{S} 
                    [.Y$'$ 
                        Y 
                        [.XP 
                            \textbf{O} 
                            [.X$'$  
                                X 
                                \sout{VP} 
                            ]
                        ]
                    ]
                ]
            ]
        ]
    \Tree
        [.YP 
            [.XP 
                [.VP 
                    \textbf{V} 
                    \sout{O} 
                ] 
                !{\qframesubtree} 
                [.XP 
                    \textbf{O} 
                    [.X$'$  
                        X 
                        \sout{VP} 
                    ]
                ]
            ] 
            !{\qframesubtree}   
            [.YP 
                \textbf{S} 
                [.Y$'$ 
                    Y 
                    \sout{XP} 
                ]
            ]
        ]
    % }
    \caption[Niuean vs.\ Malagasy VP fronting]{Niuean vs.\ Malagasy VP fronting\footnotemark{}}
    \label{ex:NvsM}
    \label{ex:NvsMtrees}
\end{figure}
\footnotetext{Labels of X(P) and Y(P) are used to abstract away from details that are not necessary to make the point needed here.}
  
In T\&M this distinction, as well as the extreme locality of roll-up movement in Malagasy, are taken to indicate that there is a third type of movement to add to A- and \=A-movement, labelled C-movement (for reasons that will become apparent in the next section).  This movement has two distinguishing characteristics -- it is more local than both A and \=A-movement, and it is roll-up, meaning that the moved element gains more material with each iteration of movement.  This is clear in the Malagasy tree in (\ref{ex:NvsM}), where VP moves into Spec, XP and then it is XP (not VP) that is targeted for the next movement.  In the next section, a feature-based account of movement locality  is outlined and used to account for the three types of Spinal Movement we have just seen.\footnote{There is, in fact, roll-up movement in Niuean, but lower in the predicate.  See Massam (\citeyear{Massam:2010},  \citeyear{Massam:2020}), and Travis \& Massam \citeyearpar{Travis:2021} for details.}  

\subsubsection{Feature-based locality\label{sec:feature}}

It is possible to make sense of these three types of movement, even predicting the existence (and characteristics) of C-movement, by adopting a feature-based account of movement such as the one outlined in van Urk \citeyearpar{vanUrk:2015a}. This feature-based system crucially divides A- and \=A-movement not by their landing position (which in the past was the standard assumption, resulting in the usage of the A vs.\ \=A labels), but rather by the probing feature. Such a  system  captures three ways in which \=A- and A-movement are distinguished.  While \=A-movement can target a variety of elements, (traditional) A-movement targets only DPs.  While \=A-movement can skip potential \=A-movement targets, A-movement can only target the closest DP.  Finally, constructions with \=A-movement have specific discourse consequences and do not generally occur in discourse neutral contexts.  A-movement, however, is part of the basic grammatical system with no discourse effects.\footnote{This sort of description may be too simplistic (for example, there may be \=A- and A- scrambling), but, for the purposes of introducing the system, I present the most canonical uses of these two movements.}  

The characteristics that distinguish  A-movement from \=A-movement are captured in the following manner.  For A-movement, the observation is that the locality, category sensitivity, and obligatory nature of A-movement are explained by the fact that T will always have the relevant feature (obligatory), that the relevant feature is D (category sensitive), and that this feature is inherent to every DP (local).  In the structure below, we see the probing D feature in T and the goal D feature in every DP.  The probe, then, will always target the closest DP, and this DP will always be an intervener for any less local DP.\footnote{It may be the case that sub-features of the DP must also be available for probing in order to explain why some local DPs are overlooked (e.g.\ dative DPs in German, Susi Wurmbrand, p.c.).  There are also  Austronesian constructions where it appears that there is A-movement of a non-local DP.  See, for example, the discussion of Acehnese Object Voice constructions in Legate \citeyearpar{Legate:2014}: Chapter 3 for details and a possible account.  Another solution is to say that these constructions are not created by movement (see e.g.\ Travis \citeyear{Travis:2006a}).}

\ea A-movement (obligatory and local) -- inherent feature\\ 
{[}\textsubscript{\colorbox{Gray}{[D]}} They ]  [\textsubscript{T:\framebox{D}} will ] [\textsubscript{\colorbox{Gray}{[D]}} \sout{they} ] put [\textsubscript{\colorbox{Gray}{[D]}} it ] on [\textsubscript{\colorbox{Gray}{[D]}} the table].
\z

\hspace{\parindent} \=A movement is distinguished by the optional nature of the probing feature.  It is optional in the probe and it is optional in all of the possible goals.  As the feature isn't restricted to any one category, it is not category sensitive.  In the example below, both DPs and PPs can be targeted.  In a case where the furthest possible target has the feature, none of the intervening possible targets will intervene since none of these will host the relevant feature.  This results in \=A movement appearing to be less local.  

\ea \=A-movement (optional and less local) -- optional (movable) feature\\
\vspace{.2cm}
[\textsubscript{\colorbox{Gray}{[wh]}} What ] [\textsubscript{C:\framebox{wh}} will ] {[} they  ] put [ it ] [ on [\textsubscript{\colorbox{Gray}{[wh]}} \sout{what} ]]? \label{ex:wh5}
\z

\subsubsection{Non-cyclicity in C-movement}

This feature-based system is not only simple and intuitive, it nicely accommodates the newly proposed C-movement.  T\&M argue that a feature system that targets the common categorial feature of an extended projection (hence the name C-movement) will result in the right properties for the type of predicate fronting found in languages like Malagasy.  To do this, the extended projection structure of Grimshaw \citeyearpar{Grimshaw:2000} is assumed.  In this system, all heads share a categorial feature, here [verbal].  The heads differ in an F feature, which indicates the position of a head along a functional hierarchy.  In a structure such as that represented in (\ref{ex:Grimshaw}) below, the relevant categorial feature that is probed for C-movement would be the  [verbal] feature that all heads share.

\begin{figure}
    \centering
    \Tree  
        [.CP\\{[}~\textbf{verbal}~{]}\{F2\} 
            [.C\\{[}~\textbf{verbal}~{]}\{F2\} ] 
            [.IP\\{[}~\textbf{verbal}~{]}\{F1\}  
                [.I\\{[}~\textbf{verbal}~{]}\{F1\} ] 
                [.VP\\{[}~\textbf{verbal}~{]}\{F0\} 
                    [.V\\{[}~\textbf{verbal}~{]}\{F0\} ] 
                    [.DP\\{[}~nominal~{]}\{F1\} ]
                ]
            ]
        ]
    \caption{\protect\citet[118]{Grimshaw:2000}}
    \label{ex:Grimshaw}
\end{figure}

The proposal is that C-movement is triggered by an obligatory probing C-feature.  This set-up will have two effects.  First, the movement will be very local, as every projection on the spine will inherently have the relevant feature.  Second, the movement will be roll-up movement.  Since XPs are being targeted in this movement\footnote{The distinction between XP and X movement will become important below.} and every maximal projection will have this feature, when movement to a Spec position takes place, this moved constituent will never be the closest target, as the projection which dominates the Spec will also have that feature.\footnote{An anonymous reviewer points out that  Rackowski  \& Richards \citeyearpar{Rackowski:2005} treat Spec, XP and XP itself as being equidistant w.r.t.\ a c-commanding goal. While this requires more research, my impression is that their need to specify this relation arises from issues internal to their account.}  We can see how this works below.   In the first step, a \textit{v} (probe) in X targets a V (goal) in YP, triggering movement to Spec, XP.  In the next movement a \textit{v} (probe) in W targets a V (goal) in XP, triggering movement to Spec, WP. Crucially, YP in Spec, XP isn't the goal because XP itself acts as a closer goal.  C-movement is, then, very local (in fact violating anti-locality for principled reasons\footnote{An anonymous reviewer notes that anti-locality is also derived from principles. I suggest that this points to another case where the current system of principles needs to be open to re-examination in order to accommodate a larger set of languages and the syntactic mechanisms that they use.}) and it is roll-up (again for principled reasons).

\begin{figure}
    \textcolor{red}{Same here: These trees have a problem. There should not be children labelled `!', but instead the left sibling constituents of these nodes should have a box around them.}
    \centering
    \Tree 
        [.WP\textsubscript{V} 
            ~~ 
            [.W$'$\textsubscript{V}
                W\textsubscript{V:v} 
                [.XP\textsubscript{V} 
                    ~~ 
                    [.X$'$\textsubscript{V}
                        X\textsubscript{V:v} 
                        [.YP\textsubscript{V} 
                            ~~ 
                            [.Y$'$\textsubscript{V} 
                                Y\textsubscript{V:v} 
                                Z\textsubscript{N}P 
                            ]
                        ] !{\qframesubtree} 
                    ]
                ]
            ]
        ]
    \Tree 
        [.WP\textsubscript{V} 
            ~~ 
            [.W$'$\textsubscript{V} 
                W\textsubscript{V:v} 
                [.XP\textsubscript{V} 
                    [.YP\textsubscript{V} 
                        ~~ 
                        [.Y$'$\textsubscript{V} 
                            Y\textsubscript{V:v} 
                            Z\textsubscript{N}P 
                        ]
                    ] !{\qframesubtree} 
                    [.X$'$\textsubscript{V} 
                        X\textsubscript{V:v} 
                        \sout{YP} 
                    ]
                ] !{\qframesubtree} 
            ]
        ]
    \caption{\textcolor{red}{Caption}}
    % \label{fig:my_label}
\end{figure}

\subsection{Interim summary}

Above I have summed up the main findings from T\&M, and below I give a table that outlines the VP fronting typology that has been proposed.  

\begin{table}
\caption{XP movement (adapted from Travis \& Massam \citeyear{Travis:2021})}
%\label{tab:1:frequencies}
\footnotesize
\begin{tabular}{ l  l  l }
\hline
\hline
& \multicolumn{2}{c}{XP} \\
\hline
&\multicolumn{1}{c}{\textsc{Limb}} & \multicolumn{1}{c}{\textsc{Spine}}   \\
\hline
\hline
\=A &  \textsc{wh-}, Focus & VP fronting (English)   \\
%& \cellcolor{Gray} English &  &&  \\
 %~~~& \cellcolor{Gray}~~~& ~~~& ~~~&~~~\\
\hline
%$ ~~~&\cellcolor{Gray} ~~~& ~~~& ~~~&~~~\\
A  & Derived Subject&  VP fronting (Niuean)    \\
%&\cellcolor{Gray} English &  &  &  \\ %CHECK SLAVIC
% ~~~& \cellcolor{Gray}~~~& ~~~& ~~~&~~~\\
\hline
%~~~& ~~~& ~~~& ~~~& \cellcolor{Gray}~~~\\
C  & ***  &  VP fronting (Malagasy) \\
% ~~~& ~~~&~~~& ~~~& \cellcolor{Gray}~~~\\
\hline
\hline
\end{tabular}
\end{table}

\normalsize

We have two types of VP fronting (Spinal Movement) that match the better-known \=A and A-movement of limb XPs.  Since C movement only becomes apparent when one looks specifically at movement of spinal XPs, it is understandable why it is a latecomer to the XP movement typology.  Given that, by definition, C-movement probes for a spinal feature, it will only target projections of the spine.  This explains the unfilled cell of C-movement of a limb.\footnote{The link between C-movement and spinal movement becomes less obvious when the Spec contains projections of the same categorial type, for example the DP possessor in Spec, DP. This issue requires a longer discussion.}    Further, C-movement will always target the closest projection of the spine, accounting for its extreme locality and the roll-up nature of the movement. The operation does not move material from one Spec position to another one. Rather, the projection above the landing site of one movement becomes the target of the next movement because the dominating projection is the closest XP goal with the relevant feature.\footnote{The absence of movement of the fronted VP out of a Spec position is discussed in more detail in Travis and Massam \citeyearpar{Travis:2021}) with respect to the observation that predicate fronting in Niuean does not undergo Spec, TP to Spec, TP (subject to subject) raising.}   The fact that C-movement is hyper-local and non-cyclic (roll-up) is crucial in facilitating the full member status of head movement in the movement typology as described above.  

\section{The repatriation of head movement}

Two questions come to mind concerning the status of head movement in narrow syntax -- (i) why did head movement become the black sheep of the family, and (ii) how was the awkwardness of this situation dealt with?\footnote{Roberts \citeyearpar{Roberts:2010} nicely lays out the issues. See also D\'ek\'any \citeyearpar{Dekany:2018} for a clear overview of the status of head movement}   In response to the first question, it is clear that the more different that head movement looks, the more precarious its position.  Head movement (i) does not affect meaning (or rarely does -- see Lechner \citeyearpar{Lechner:2006}, Roberts (\citeyear{Roberts:2010}:Ch1) for evidence to the contrary), (ii) does not move cyclically (in that it does not excorporate -- though see Roberts \citeyearpar{Roberts:2010} for a different view on excorporation), (iii) is not anti-local in the sense of Grohmann \citeyearpar{Grohmann:2003b}, and (iv)  does not obey the Extension Condition.  Chomsky (\citeyear{Chomsky:2015}:12) writes `head raising is a unique operation, with special properties'. 

Given all of these, at best, differences, and, at worse, violations, what is to be done?  One possibility is to say that head movement is, in fact, XP (remnant) movement (e.g. Koopman \& Szabolcsi \citeyear{Koopman:2000}).  Another possibility is to make head movement behave as much like XP movement as possible (i.e., not violate the Extension Condition) even though the fit remains uncomfortable (e.g.\ Matushansky \citeyear{Matushansky:2006}).  The problem with some such solutions is that questions relating to the apparent lack of semantic effects or the hyper-locality  may still remain.  A more drastic possibility is to say that it is not, in fact, a syntactic movement at all (see Chomsky \citeyear{Chomsky:2001a} or Harley \citeyearpar{Harley:2003a} for different versions of this).  A very different tack to take, however, is to examine the system itself.   It is this path I explore in this paper.  I argue that a grammatical system has been created which, by its very principles, excludes head movement.  The reason for this situation is that the system was created to account for the set of movement rules that are most prevalent in the languages most studied, that is, phrasal cyclic movement of dependents of the syntactic structure.  My claim is that if a system had been created to account for a wider diversity of movements, including not only XP and X$^0$ movement, but also dependent (limb)  and extended projection (spinal) movement, the status of head movement would not have been treated with such suspicion. 

I start by re-examining the apparent exceptional nature of head movement with the aim of showing that the observed differences are derivable by the inherent nature of canonical head movement.\footnote{At this point I am concentrating on garden variety head movement such as V-to-T-to-C.  Later in the discussion I  return to less canonical cases.}  It is, after all, movement of a head and not an XP, and of a spinal element, not a limb.   The point that I will make is that a typology that naturally includes Spinal Movement and C-movement will also include head movement. As mentioned earlier, XP C-movement has distinguishing properties that follow from the feature being targeted.  It is extremely local and it is roll-up.  These are also two salient properties of head-movement.\footnote{Pearson \citeyearpar{Pearson:2000} also makes the connection between Malagasy roll-up movement and head movement.}  

D\'ek\'any \citeyearpar{Dekany:2018}, in her overview paper on head-movement, lists three constraints on head movement.  First,  she mentions the roll-up characteristic of head movement as being a constraint against excorporation, meaning that it is not cyclic (the same element cannot undergo further movement).   Second, she mentions the hyper-locality of head movement (the Head Movement Constraint).  Third she points to the fact that head movement is clause bound, i.e.\ does not occur beyond the border of an extended projection.  All of these traits follow naturally from the T\&M feature-driven C-movement.  Roll-up and hyper-locality follow in a movement driven by the shared features of an extended projection.  Movement across an extended projection  falls out in the same way.  C-movement must be triggered by a head that shares the same extended projection feature as its complement.\footnote{This raises the question of where incorporation in the sense of Baker \citeyearpar{Baker:1988c} fits in the typology.  This is being left for future work.}  Given this, head movement of the sort seen in, for example, verb raising in Italian, appears to be C-movement, differing from VP roll-up movement in Malagasy only in the level of projection moved, i.e.\ a head rather than a phrase.  Proceeding along these lines and following T\&M, I extend the typological table to include head movement as shown below.

\begin{table}
\caption{Movement typology with XP and X$^0$}
\label{tab:table}
\footnotesize
\begin{tabular}{ l  l  l  l  l }
\hline
\hline
& \multicolumn{2}{c}{XP} & \multicolumn{2}{c}{X$^0$} \\
\hline
&\multicolumn{1}{c}{\textsc{Limb}} & \multicolumn{1}{c}{\textsc{Spine}} & \multicolumn{1}{c}{\textsc{Limb}} & \multicolumn{1}{c}{\textsc{Spine}} \\
\hline
\hline
\=A & \cellcolor{Gray}\textsc{wh-}, Focus & VP fronting (English) &  &  \\
%& \cellcolor{Gray} English &  &&  \\
 %~~~& \cellcolor{Gray}~~~& ~~~& ~~~&~~~\\
\hline
%$ ~~~&\cellcolor{Gray} ~~~& ~~~& ~~~&~~~\\
A  & \cellcolor{Gray}Derived Subject&  VP fronting (Niuean)   &  &  \\
%&\cellcolor{Gray} English &  &  &  \\ %CHECK SLAVIC
% ~~~& \cellcolor{Gray}~~~& ~~~& ~~~&~~~\\
\hline
%~~~& ~~~& ~~~& ~~~& \cellcolor{Gray}~~~\\
C  & ***  &    VP fronting (Malagasy) &  &  \cellcolor{Gray} V-movement\\
% ~~~& ~~~&~~~& ~~~& \cellcolor{Gray}~~~\\
\hline
\hline
\end{tabular}
\end{table}


The grey cells are those movements most commonly presented in any discussion of, say, English syntax, i.e.\ \=A (\textsc{wh-})movement, A (NP-)movement, and head (V-)movement.  Viewing just these cells, one could easily come to the conclusion, as has been done, that XP movement must be Limb Movement that is anti-local, while head movement must be Spinal Movement that is hyper-local.  Given this family picture in which head movement looks nothing like either of its siblings, it is not surprising that some suspicion should arise.  However, once Spinal Movement, and in particular C-Spinal Movement, becomes part of the group, head movement now has a sibling with shared traits.

With this in mind, I continue to examine the typology and in particular the empty cells of Table (\ref{tab:table}), as one naturally wonders why certain movement types are missing.  Below I only suggest directions that might be taken in searching for likely candidates, looking first at Spinal Head Movement and then at Limb Head Movement.  Before proposing possible candidates, I want to make two points.  The first point is that, as with XP movement, we might  expect not to find C-movement of a head from the limb, given that C-movement is triggered by the probe and the goal having a shared extended projection feature. The second is that the head movements we are looking for (\=A- and A-movement of limb and spine) will violate the Head Movement Constraint, given that we do not expect  \=A and A movement to have the hyper-locality property of C-movement.\footnote{Many of the conclusions I come to in this section are preceded by similar conclusions reached in Roberts \citeyearpar{Roberts:2010}.  One the main differences here is the  addition of spinal C-movement to the movement typology.}
 
\section{Spinal Head Movement}

In this section, I suggest that predicate clefting might be an example of \=A-Spinal Head Movement, and that Slavic Long Head Movement (LHM) might be a case of A-Spinal Head Movement.  I acknowledge that both of these proposals require a more in-depth study and that at this point I am just holding them up as possibilities.\footnote{Both of these movements are discussed in Roberts \citeyearpar{Roberts:2010} (along with long head movement of Breton), which he also categorized as \=A-movement and A-movement, respectively.  While movement in Breton is, according to Roberts, to C, he nevertheless categorizes it as A-movement.  In his system as well as the one being outlined here, the type of movement is determined by features not by landing site.}


\subsection{\=A-Spinal Head Movement}

Koopman \citeyearpar{Koopman:1984} argued that (i) predicate clefting of the sort seen in the Vata data in (\ref{ex:VataPC})   is an example of \=A verb movement, and (ii) short V-movement as seen in the Vata data in   (\ref{ex:VataVmove}) is an example of A verb movement.  

\ea[] {
\gll \textbf{l\=i} [\textsubscript{TP} \textvbaraccent{O} d\=a s\textvbaraccent{a}k\'a \textbf{l\=i} ]\label{ex:VataPC}\\
eat ~ s/he \textsc{perf.aux} rice eat\\
\glt `S/he has \textbf{eaten} rice.' \hfill Vata: adapted from Koopman (\citeyear{Koopman:1984}:38)
}
\z

\ea \label{ex:VataVmove}
    \ea[] {
    \gll [\textsubscript{TP} \'a l\=a s\textvbaraccent{a}k\'a \textbf{l\=i} ]\\
    ~ we \textsc{perf.aux} rice eat\\
    \glt `We have eaten rice'
    }
    \ex[] {
    \gll [\textsubscript{TP} \'a \textbf{l\`i}$_k$ s\textvbaraccent{a}k\'a t$_k$ ] \label{ex:VataVmove}\\
    ~ we ate rice\\
    \glt `We ate rice'  \hfill Vata: adapted from Koopman (\citeyear{Koopman:1984}:42)
    }
    \z
\z

I borrow from Koopman the claim that V movement in the predicate cleft construction is, in fact, a case of \=A head movement of the verb.  I will assume, however, that movement of the V into T as in (\ref{ex:VataVmove}) is a case of C-movement as it appears to be similar to the type of head movement of the verb we find in languages such as English, German, and Italian.  

The process of predicate clefting has many of the earmarks of \=A-movement.  The trigger is high in the extended projection of the clause, it has the discourse effect of focusing the verb, and the movement is relatively long-distant.  When compared to its close sibling of \=A VP movement, it is quite similar in that it moves to the left of the subject, suggesting that it moves to a position within the CP domain.\footnote{While I bring up landing site as possible support for the determination of the movement type, it is neither a necessary nor sufficient condition in the determination of movement type.  It is the clustering of properties that is important.}

Many questions can be raised pertaining to predicate clefting that I leave unanswered here.  On such question is whether predicate clefting is, in fact, movement (see Cable \citeyear{Cable:2004}), and, if it is movement, whether it moves a head and not an XP.  One reason to believe that the fronted element is an XP is that in Yiddish predicate clefts, the clefted material is arguably in a Specifier position, which triggers V2 effects.  In (\ref{ex:Yiddish}) below, the verb \textit{essen} in sentence initial position triggers movement of the auxiliary \textit{hot} to second position.\footnote{Ideally a case can made for some instance of predicate clefting where (i) only a head may move and (ii) it is clear that movement is to a head position.}

\ea[] { 
\gll Essen hot Max  gegessen a fish\label{ex:Yiddish}\\
to.eat has Max eaten a fish\\
\glt `As for eating, Max has eaten a fish.'\hfill Yiddish (Cable \citeyear{Cable:2004}:2)
}
\z

Further, some predicate clefts may move both more than just the V (see e.g.\ Vicente \citeyearpar{Vicente:2009}).  And finally there is the question of why predicate clefts require pronunciation at the tail of the chain (see e.g.\ Trinh \citeyear{Trinh:2009}).  Ideally, a case can be made for some instance of predicate clefting where (i) only a head may move and (ii) it is clear that movement is to a head position within the C domain.  Whether such a case can be found remains to be seen.

\subsection{A-Spinal Head Movement}

Turning now to A-Spinal Head Movement, it could be that some cases of Long Head Movement (LHM) in Slavic, where a non-finite verb appears to move over auxiliaries, might be good candidates.\footnote{LHM is much more complicated than will be presented here.  See Harizanov \& Grivanova \citeyearpar{Harizanov:2018}, Lema \& Rivero \citeyearpar{Lema:1989}, Rivero \citeyearpar{Rivero:1991a},  \citeyearpar{Rivero:1994}, Roberts \citeyearpar{Roberts:2010} for more details.}  In order to have  the expected properties for A-movement, we would want the movement to  not interact with discourse issues such as focus or question formation, and we might expect the probe for the movement to be in the inflectional domain.  In Rivero \citeyearpar{Rivero:1994}, there appear to be two types of LHM.  The Romanian example shown in (\ref{ex:LHMtoC}) below is not the best candidate for A-movement, as the LHM brings with it a new discourse function, either exclamatory or interrogative.

\ea\label{ex:LHMtoC}
    \ea[] {
    \gll L-ar \textbf{bate} Dumnezeu.\\
    him-would-3s punish God\\
    \glt `God would punish him.'
    }
    \ex[] {
    \gll \textbf{Bate}-l-ar Dumnezeu.\\
    punish-him-would-3s God\\
    \glt `God would punish him!' \hfill Romanian: Rivero \citeyear{Rivero:1994}:86
    }
    \z
\z

In other cases, however, certain contexts force the use of LHM without triggering discourse effects.  In the Present Perfect in Bulgarian, the main verb must precede the Auxiliary.\footnote{Rivero \citeyear{Rivero:1994} also gives an example with the Past Perfect where the Aux-V order is also possible.}

\ea
    \ea[]  {
    \gll \textbf{Pro\v{c}el} s\v{u}m knigata.\\
    read I-have book-the\\
    \glt `I have read the book.'
    }
    \ex[*] {S\v{u}m \textbf{pro\v{c}el} knigata. \hfill Bulgarian: Rivero \citeyear{Rivero:1994}:89}
    \z
\z

As with the proposal that predicate cleft constructions exemplify \=A-Spinal Movement, the proposal that Slavic long head movement, at least in certain cases, exemplifies A-Spinal Movement is still speculative.  Many questions still have to be addressed.  For example, it is curious that this movement only occurs when the subject is either missing or post-verbal (see Rivero \citeyear{Rivero:1994}:322 and fn 1).\footnote{King \citeyearpar{King:1996} offers an alternative analysis for some cases of LHM that needs to be looked into.  She argues that at least certain cases of LHM are not movement at all but rather encliticization of the auxiliaries onto the verb.}   It is tempting to say that A-movement of the verb satisfies the same EPP feature that subject movement satisfies, accounting for why the subject no longer raises to Spec, TP.  This follows in the spirit of the proposal in Massam \citeyearpar{Massam:2001a} that Niuean VP fronting, which is A-Spinal Movement of an XP, satisfies the EPP feature in Niuean.  More work, however, is required to do a complete study on this phenomenon and determine its appropriateness for A-spinal movement.

\section{Limb Head Movement}

 While most cases of traditional head movement involve moving heads along the extended projection (Spinal Movement), there are cases in the literature of proposed head movement from limbs.  Here I suggest that argument cliticization may be an instance of A-Spinal Head Movement.  Some have proposed that clitization of the Romance type is head movement (e.g.\ Roberts \citeyearpar{Roberts:2010} and Preminger \citeyearpar{Preminger:2019}). I, however, will look at a phenomenon in Malagasy where cliticization more closely resembles the type of A-movement of DPs that we are familiar with.  Lastly, stretching a bit, I suggest that the type of \textsc{wh}-feature movement proposed in Cheng \citeyearpar{Cheng:2000b} and Donati \citeyearpar{Donati:2006} or the question particle movement of Hagstrom \citeyearpar{Hagstrom:2000} could be seen as \=A-Spinal Head Movement.

\subsection{A-Limb Head Movement}

Before beginning the search for examples that will represent A-Limb Movement of a head, we need to outline what properties we are looking for, and for this we turn to the sort of DP movement that is familiar to us, i.e.\ DP raising to Spec, TP.\footnote{See Baker \& Hale \citeyearpar{Baker:1990} for arguments for a similar movement analysis for Breton.}  As discussed in section (\ref{sec:feature}), DP movement does not trigger discourse effects, and it attracts the closest DP.  As tempting as it is to subsume Romance cliticization into this category of movement, given that, unlike DP raising to Spec, TP, multiple DPs may cliticize, I take a different direction. Instead I turn to a process that occurs in some Austronesian languages, wherein  the highest non-subject argument in the clause arguably moves.  I illustrate this with data from two languages, Indonesian and Malagasy.

Indonesian has two different means to promote an object to the subject position -- the passive (\ref{ex:BIpassive}) and Object Voice (OV: (\ref{ex:BIOV})).\footnote{For more details, see Chung's paper on the two passives of Indonesian (Chung \citeyear{Chung:1976a}). I have given an active translation for the OV constructions to distinguish OV from the passive, but I underline the grammatical subject in the translation.  See Chung \citeyearpar{Chung:1976a} for arguments that the sentence initial DP is the subject in the OV construction in Indonesian, and Legate \citeyearpar{Legate:2014} for similar arguments for the subject of OV in Achenese.}

\needspace{5\baselineskip}
\ea Indonesian: Active/Passive/Object Voice\footnote{These examples are taken from Guilfoyle, Hung \& Travis \citeyearpar{Guilfoyle:1992}.  Their consultants allowed proper names in this position, while this was not allowed by Chung's Indonesian consultants.  These examples have been checked for Indonesian with Jozina Vander Klok, who points out that proper names are allowed if they have 1st or 2nd person referents.}
    \ea[] {
    \gll Ali/saya/kamu meN-baca buku itu\\
    Ali/\textsc{1sg}/\textsc{2sg} \textsc{act}-read   book \textsc{det}\\
    \glt `\underline{I/you} read the book.'\hfill Active
    }
    %\needspace{5\baselineskip}
    \ex[] {
    \gll Buku itu di-baca oleh Ali/saya/kamu\label{ex:BIpassive}\\
    book \textsc{det} \textsc{pass}-read by Ali/\textsc{1sg}/\textsc{2sg}\\
    \glt `\underline{The book} is read by me/you (Ali).'\hfill Passive
    }
    \ex[] {
    \gll Buku itu Ali/saya/kamu baca\label{ex:BIOV}\\
    book \textsc{det} Ali/\textsc{1sg}/\textsc{2sg} read\\
    \glt `I/you (Ali) read \underline{the book}.'\hfill Object Voice
    }
    \ex[*] {
    \gll Buku itu lelaki itu baca\label{ex:OVbad}\\
    book \textsc{det} man \textsc{det} read\\
    \glt `The man read \underline{the book}.'
    }
    \z
\z

It is the OV sentence in (\ref{ex:BIOV})  that is relevant for our purposes, and in particular, the placement of the Agent in the pre-verbal position.   There is speaker variation with regard to what forms can appear in this position, but there is at least one variety that allows all pronouns (see Chung \citeyear{Chung:1976a}:60-61, taken from  Macdonald and Dardjowidjojo \citeyear{Macdonald:1967}:235).  Crucially, it is not possible to front a whole DP, as shown in (\ref{ex:OVbad}).\footnote{When the Agent is a full DP, only the passive construction can be used to place the Theme in the subject position.} 

Guilfoyle et al.\ \citeyearpar{Guilfoyle:1992} account for this position of the Agent through D-movement from the Agent DP in Spec, VP (which would update to Spec,\textit{v}P) to T as shown in an updated tree below.\footnote{Other updating is needed but this would require a much longer discussion.  Clearly there is a subject DP in Spec, TP raising the question of how this DP comes to appear in this position.  Further, auxiliary type heads appear between the subject and the moved D, so it must be that D  moves to a head lower than T. Also note that the framework of Bare Phrase Structure is not being assumed.} 

\begin{figure}
    \centering
    \Tree 
        [.TP 
            \qroof{Buku itu}.DP  
            [.T$'$ 
                [.T 
                    [.D ku$_k$ ] 
                    [.V pukul$_i$ ] 
                ] 
                [.\textit{v}P 
                    [.DP 
                        {D\\t$_k$} 
                    ] 
                    [.\textit{v}$'$ 
                        {\textit{v}\\t$_i$} 
                        \qroof{... Vt$_i$ ...}.VP 
                    ]
                ]
            ]
        ]
    \caption[Caption missing]{\textcolor{red}{Caption missing}}
    % \label{fig:my_label}
\end{figure}

This movement, then, has the properties of what we  expect to find in \=A-Limb Head Movement -- it is obligatory (has no discourse effects) and it targets the closest DP.  

%It is interesting to note that in Acehnese, a closely related language, a full DP can appear in this preverbal position indicating that the same Agent movement in this language is phrasal rather than head movement.

%\needspace{5\baselineskip}
%\ex. Acehnese: Active/Passive/Object Voice
%\ag.  [ uleue nyan ] di-kap l\^on\\
% ~ snake \textsc{dem} ~ \textsc{3.fam}-bite \textsc{1sg}\\
%`\underline{The snake} bit me.'\hfill \textsc{active}: Legate (\citeyear{Legate:2012}:497) \\
%\bg. l\^on di-kap [ l\'e  uleue nyan ] \\
%\textsc{1sg} bite ~ by snake \textsc{dem} \\
%`\underline{I} was bitten by the snake.'\hfill \textsc{passive}: Legate (\citeyear{Legate:2012}:497) \\
%\cg. l\^on [ uleue nyan ] kap\\
%\textsc{1sg} ~ snake \textsc{dem} ~ bite\\
%`The snake bit \underline{me}.'\hfill \textsc{object voice}: Legate (\citeyear{Legate:2012}:498) 

%\ag.  Sie ji-taguen


Malagasy has a similar process, labeled  N-bonding by Keenan \citeyearpar{Keenan:2000}.  Malagasy is a VOS language with a complex voice system (often referred to as the Philippine-type voice system).  Changes in the verb form indicate the semantic role of the sentence-final subject.\footnote{There are debates about whether the sentence final DP is a subject or a topic, but I follow the more traditional literature and adopt the label of subject.}  One of these forms is similar to the Object Voice construction in Indonesian (often labeled Theme Topic in the syntactic literature on Malagasy).  We see below what could be called the Active or the Agent Voice, where there is no N-bonding, followed by three cases of N-bonding: one with a pronoun, one with a proper name, and one with a common noun.\footnote{Orthographic conventions of Malagasy are used here where an apostrophe appears between the verbal form and a determiner, a hyphen between the verbal form and a proper name, and word final i is written as y. See Ting \citeyearpar{Ting:2021} for a syntactic and phonological account of N-bonding.}

\ea N-bonding in Malagasy
\ea[] {
\gll \textit{Mamaky} ny boky \textit{aho}.\\
\textsc{pres.av}.read \textsc{det} book \textsc{1sg}\\
\glt`\underline{I} read the book.'\hfill \textit{m-an-vaky}
}
\ex[] {
\gll \textit{Vaki\textbf{ko}} ny boky.\label{ex:Npron}\\
\textsc{pres.ov}.read-\textsc{1sg} \textsc{det} book\\
\glt `I read \underline{the book}.'\hfill \textit{vaki-na} + \textit{ko}
}
\ex[] {
\gll \textit{Vakin-\textbf{dRasoa}} ny boky.\\
`\textsc{pres.ov}.read-Rasoa \textsc{det} book\\
\glt `Rasoa reads \underline{the book}.'\hfill \textit{vaki-na} + \textit{Rasoa}\\
}
\ex[] {
\gll \textit{Vakin'\textbf{ny}} vehivavy ny boky.\\
\textsc{pres.ov}.read'\textsc{det} woman \textsc{det} book\\
\glt `The woman reads \underline{the book}.'\hfill \textit{vaki-na} + \textit{ny}
}
\z
\z

In all three cases of N-bonding, an element has undergone some morphological merger with the verb. It has been argued that this merger is similar to the D$^0$ movement posited for Indonesian above (e.g.\ Travis \citeyear{Travis:2006}).\footnote{The assumption is that whether head movement results in prefixation or suffixation is determined by morpho-phonology and not syntax.}  In more recent work, Paul \& Travis \citeyearpar{Paul:2019} have provided an argument for D-movement using the Augmented Pronoun Construction (APC), as in \textit{we women}.  The Malagasy APC differs from the English APC in that it can be used for all pronominal forms, while in English it only appears with 1st and 2nd person plural pronouns: \textit{we women}, \textit{you women}, *\textit{I woman}, *\textit{you woman}, *\textit{she woman}, *\textit{they women}.  The Malagasy APC also provides insight into the N-bonded pronoun of the type we have seen in (\ref{ex:Npron}).  In (\ref{ex:NAPC}) the 	`bonded' pronoun that we have seen in (\ref{ex:Npron}) must be doubled if it is followed by a nominal complement.

\ea {
\gll Vaki\textit{ko} *(\textit{izaho}) ankizy ny boky.\label{ex:NAPC}\\
\textsc{pres.ov}.read-\textsc{1sg}  \textsc{1sg.nom} child \textsc{det} book\\
\glt `I child read \underline{the book}.'\hfill Malagasy APC
}
\z

Paul \& Travis claim that the lower copy of the movement must be pronounced under certain circumstances, but what is important here is to demonstrate that the D$^0$ in (\ref{ex:Npron}) has undergone movement and not simply morphological merger under adjacency.\footnote{See Levin \citeyearpar{Levin:2015} for an adjacency account of N-bonding.}

Again, this candidate for A-Limb Head Movement requires more support before being confirmed. As with the predicate cleft construction, it is important to determine the relevant circumstances for copy pronounciation since this is not what we find in the closely related DP movement.

\subsection{\=A-Limb Head Movement}

In order to find a possible candidate for \=A-Spinal Head Movement, we want to first find a process which interacts with the discourse, which displaces something small enough to be considered a head, and which can be assumed to move to a position within the C domain.  One candidate for this position already exists in the literature in slightly different forms.  Cheng \citeyearpar{Cheng:2000b} argues that some cases of \textsc{wh-}movement are, in fact,  feature movement.  Hagstrom (\citeyear{Hagstrom:2000}, \citeyear{Hagstrom:2004}) argues that there is particle movement in \textsc{wh}-constructions in Japanese,  and Donati \citeyearpar{Donati:2006} argues for head movement of a quantifier in English free relatives and in comparatives.  

Below I give an example from Cheng \citeyearpar{Cheng:2000b} showing a case of partial \textsc{wh-}movement in German.  

\ea
\ea[] {
\gll [ \textbf{Mit} \textbf{wem} ] glaubt Hans [\textsubscript{CP} \textbf{da\ss{}} [\textsubscript{IP} Jakob jetzt \textbf{t$_i$} spricht ]]\label{ex:fullXP}\\
~ with whom ~ thinks Hans ~ that ~ Jakob now ~ talks\\
\glt `With whom does Hans think that Jakob is now talking?'
}
\ex[] {
\gll \textbf{Was}$_i$  glaubt Hans [\textsubscript{CP} [ \textbf{mit} \textbf{wem} ]$_i$ [\textsubscript{IP} Jakob jetzt \textbf{t$_i$} spricht ]]\label{ex:partialX}\\
\textsc{wh} thinks Hans ~ ~ with whom ~ ~ Jakob now ~ talks\\
\glt `With whom does Hans think that Jakob is now talking?'
}
\z
\z

Cheng proposes that while there is full movement of the \textsc{wh}-XP in (\ref{ex:fullXP}) to the matrix Spec, CP, in (\ref{ex:partialX}) there is only partial movement to the embedded Spec, CP followed by feature movement to the scopal position in the matrix clause as realized by \textit{was}.  `I propose that partial \textsc{wh}-movement involves overt movement of part of the \textsc{wh}-word (hence partial), namely, the \textsc{wh}-feature of the \textsc{wh}-word.' (Cheng \citeyear{Cheng:2000b}:77).

For a similar type of analysis, Hagstrom \citeyearpar{Hagstrom:2000} investigates \textsc{wh}-constructions in Japanese, coming to the conclusion that these constructions are formed through movement of the \textit{ka} particle: `a question particle \textit{ka} undergoes syntactic movement from a clause internal position (by the  \textit{wh}-word) to the clause periphery (i.e.\ into the complementizer position).'
A relevant example from Japanese is given below.

\ea[] {
\gll dare-ga t$_i$ hon-o kaimasita ka$_i$\\
who-\textsc{nom} ~ book-\textsc{acc} bought.\textsc{polite} Q\\
\glt `Who bought the book?'\hfill Japanese: Hagstrom (\citeyear{Hagstrom:2000}:ex.(1))
}
\z

Hagstrom argues that by positing \textit{ka} movement, we can account for the intervention effects of other uses of \textit{ka}, as shown with a disjunctive  \textit{ka} in (\ref{ex:disjunctiveka}) and an indefinite  \textit{ka} in (\ref{ex:indefiniteka}).

\needspace{5\baselineskip}
\ea Disjunctive \textbf{ka} \label{ex:disjunctiveka}
\ea[?*] {
\gll {[} John-\textbf{ka} Bill ]-ga \textit{nani-o}  \textbf{t$_i$} nomimasita \textbf{ka}?\\
~ John-or Bill- \textsc{nom} what-\textsc{acc} ~ drank Q?\\
}
\ex[] {
\gll\textit{nani-o}  \textbf{t$_i$} [ John-\textbf{ka} Bill ]-ga nomimasita \textbf{ka}? \label{ex:Jdisj}\\
what-\textsc{acc} ~  ~ John-or Bill- \textsc{nom} drank Q?\\
\glt `What did John or Bill drink?' \hfill Japanese: Hagstrom (\citeyear{Hagstrom:2000}:ex.(2))\\
}
\z
\z

\ea Indefinite \textbf{ka} \label{ex:indefiniteka}
\ea[??]{
\gll dare\textbf{ka}-ga \textit{nani-o} \textbf{t$_i$} nomimasita \textbf{ka}?\\
someone what-\textsc{acc} ~  drank Q?\\
}
\ex[]  {
\gll \textit{nani-o} \textbf{t$_i$} dare\textbf{ka}-ga nomimasita \textbf{ka}? \label{ex:Jindef}\\
 what-\textsc{acc} ~ someone  drank Q?\\
\glt `What did someone drink?' \hfill Japanese: Hagstrom (\citeyear{Hagstrom:2000}:ex.(3))
}
\z
\z

Only when the \textsc{wh}-XP has scrambled out of the c-command domain of the competing \textit{ka} form, as in (\ref{ex:Jdisj}) and (\ref{ex:Jindef}), is the construction completely grammatical.

A third candidate for \=A-head movement from a limb is found in Donati \citeyearpar{Donati:2006}.  She argues that certain \textsc{wh}-constructions in English and Italian, such as free relatives, are created through head movement of a \textsc{wh}-head, as illustrated below.  What distinguishes these constructions from regular \textsc{wh}-constructions is that they have what she labels an `anti-pied piping restriction'.  In the (a) examples, a full DP has moved in a free relative, which accounts for the ungrammaticality.  The (b) examples show that embedded \textsc{wh}-questions have no such restriction. The (c) examples show that the string is grammatical when, in her terms, just a head moves to create the free radical

%\needspace{5\baselineskip}
\ea 
\ea[*] {
\gll Manger\`o [quanto pane] vorrai [t]\\
I-will-eat how-much bread you-will-want\\
}
\ex[] {
\gll Mi chiedo [quanto pane] vorrai [t]\\
I wonder how-much bread you-will-want\\
}
\ex[] {
\gll Manger\`o [quanto] vorrai [t]\\
I-will-eat what you-will-want\\
}
\z
\z

\ea 
    \ea[*] {I shall visit [what town] you will visit [t]}
    \ex I wonder [what town] you will visit [t]
    \ex I shall visit [what] you will visit [t]
    \z
\z

Again, while I have outlined three possible candidates for \=A-Limb Head Movement, more work needs to be done to vet these more thoroughly.  For example, the following questions arise: why, in German partial movement, is the \textsc{wh}-XP not able to remain in situ but must move to a Spec, CP, unlike in Japanese and English/Italian free relatives? What features make it possible for disjunct \textit{ka} and indefinite \textit{ka} to interfere with Q-\textit{ka} movement without making them targets themselves?  In the case of Donati's proposal, it is crucial that movement of the \textsc{wh} head lands in a D and not a C, suggesting that the constructions she investigates differ from those found in Cheng's and Hagstrom's work.

\section{Going head first}

My aim has been to argue for a more inclusive movement typology -- one in which head-movement is an equal member.  In doing so, a related goal is to add to the possible properties that movement can express and to have these properties follow from the features that trigger movement.  If done right, those properties of head movement that made it look different from its better known siblings are now part of the set of possible properties and are shared by at least some other family members.  If these proposals in the end turn out to be on the right track, we might have solved  some of the problems arising from the distinct nature of head-movement.  But we are still left with the problem of the Extension Condition, given below.\footnote{Note that Feature Cyclicity, leading to tucking in, as laid out in Richards \citeyearpar{Richards:1997a}, already includes some of the points I make in the discussion to come.  One important point made in that paper (Richards \citeyear{Richards:1997a}:57) is that cyclicity can be derived by having features satisfied when they enter the derivation, without any recourse to the Extension Condition.  I thank an anonymous reviewer for leading me back to this.} 

\ea \textbf{Extension Condition} (Chomsky \citeyear{Chomsky:1995})\\
Merge should be effected at the root.
\z

Since head movement clearly violates this condition, we can see that the system needs not just expansion, but some fundamental rethinking.  

One option to make sense of the Extension Condition is to start with E-merge, which consists of merging $\alpha$ and $\beta$ and creating $\gamma$. Any additional element added to this figure will have to be merged to the root $\gamma$, resulting in the Extension Condition. Looking now at I-merge and XP movement, we see that movement can be accounted for in a similar fashion by requiring that the moved element undergo I-merge in a manner similar to E-merge, whereby the moved element must merge with the root node.  There are many things that fall out nicely by making I-merge parallel to E-merge (generation and transformation are reduced to one process, the c-command requirement on movement does not need to be stipulated).  However, the drawback is that head movement is now put in an uncomfortable position due to the fact that a moved head does not merge with the root node (pace Matushansky \citeyear{Matushansky:2006}).

However, if we give our story a different beginning, we arrive at a different endpoint. Since features have become the driving force of syntax, it seems reasonable to begin our story with features and what they do.  And since  features have received the most attention in the domain of movement, it also seems reasonable to start with I-merge.  A simple view of movement is that a (probe) feature in a head interacts with a (goal) feature within its domain, and that this interaction  will trigger movement.  Since movement is triggered, it must be that there is a requirement that the probe feature and the goal feature be in a local configuration -- one more local than the non-movement configuration, hence the need for movement.  Now we ask two questions: what moves and where does it move to?  As pointed out by others (e.g.\ both Cheng \citeyear{Cheng:2000b} and Donati \citeyear{Donati:2006}), the most economical element to move would be the goal feature itself, or perhaps the head that is the minimal syntactic realization of that feature. Starting at this end of the story, head movement becomes the default.   

Continuing with head movement for the moment, we  look at the question of where this head moves to.  It seems that one could say that it must be as local as possible to the probing feature and that this can be achieved by adjunction to the head that contains the probe.  Now adjunction to the head becomes the default.  

It is at this point that we turn to XP movement.  We know that there has to be some mechanism that determines whether X or XP movement is triggered, but for the time being I assume that it is (unsatisfyingly) some diacritic. So now we imagine that we have the same probe-goal relationship, but instead of moving a head, it is the XP that dominates the feature that will move, due to the diacritic. This XP too will want to adjoin to the head, but if there is a restriction on adjunction relationships, perhaps the LCA of Kayne \citeyearpar{Kayne:1994}, such that XPs can only adjoin to XPs, then the moved element will attach to the root of the structure.  In this story, extension is not only not required, but is not even the default process.  Extension occurs when the default of adjoining to the head hosting the probe feature is ruled out.  In sum, head movement is the default, but in some cases it must be the XP (controlled by a diacritic).  Secondly, adjunction to the head is the default, but when ruled out by the LCA, it must be adjunction to the root node.

Obviously this scenario needs to be worked out, and the details relating to E-merge have not been addressed,  in particular issues of complementation and adjunction.  The point, however, is to begin to imagine a scenario where the full diversity of movements are taken into account as the defining characteristics of syntactic movement are being determined. 

%\section{Co-existence of XP and X movement: size matters}

%\exg. %\ag. Manasa tsara *(ny) lamba ho an'ny savony \textbf{ny} \textbf{ankizy} \\
%\textsc{pres.at}.wash well \textsc{det} clothes with \textsc{det} soap \textsc{det} child\\
%`The child is washing *(the) clothes well with the soap.'\\
 %Anasan'ny ankizy (ny) lamba \textbf{ny} \textbf{savony} \\
%\textsc{pres.ct}.wash'\textsc{det} child  \textsc{det} clothes  \textsc{det} soap \\
%`The child is washing (the) clothes with the soap.'

% \ex. Head movement of V in Malagasy\\
%\tiny{\Tree [.T$'$ [.T \colorbox{yellow}{\node{2}{V}} T ] [.VoiceP \qroof{Agt}.DP [.VoiceP [.ZP \qroof{\colorbox{yellow}{\node{1}{V}} O$_{indef}$}.VP [.ZP \qroof{tsara}.AdvP [.Z$'$ Z \sout{VP} ]]] [.Voice$'$ Voice \sout{ZP} ]]]]}
%\anodecurve[bl]{1}[b]{2}{2cm}





\section{Conclusion and next steps}

The purpose of this paper was to examine why head movement has become marginalized and to re-evaluate its position in the typology of syntactic movement once a wider range of movements is added to the theoretical landscape.  I have argued that creating a view of grammar that only takes into account XP Limb Movement marginalizes head movement for the wrong reasons.  In the end, I hope to have shown that head movement is not principally excluded from the narrow syntax.  Once the properties of Spinal Movement and C-movement are added to the range of possibilities, head movement has a clear position in the system.  Further, it is predicted that there will be cases of head movement that violate the Head Movement Constraint, but they themselves should show particular characteristics that fit with either their A- or \=A-status.  While this work is still preliminary and cuts through many other phenomena such as predicate clefting, cliticization, and \textsc{wh}-movement, I argue that it is worthwhile to at least try to create a grammatical system that includes the complete range of movement types from square one.

\section*{Abbreviations}
\begin{tabularx}{.5\textwidth}{lQ}
\textsc{3s} & 3rd person singular              \\
\textsc{av} & Actor Voice\\
\textsc{cir.pos} & circumstantial possibility (modal)   
\end{tabularx}
\begin{tabularx}{.5\textwidth}{lQ}
\textsc{emph} & emphatic\\
\textsc{ov} & Object Voice \\
\textsc{perf} & perfect \\
\textsc{pres} & present 
\end{tabularx}

\section*{Acknowledgements}
This paper benefited from the financial support of SSHRC grant 435-2016-1331, feedback from The Parameters Workshop 2019 at McGill, as well as from Tom Leu, Diane Massam, Maddie O'Reilly-Brown, and two anonymous reviewers.  Some comments will be given more attention in future work and all mistakes remain mine.  I am also honoured to be part of a tribute to Susi.  She is an exceptional linguist, who has contributed to our deeper understanding of grammatical systems as well as to the dignity and accessibility of the field of linguistics.


\printbibliography[heading=subbibliography,notkeyword=this]
\end{document}