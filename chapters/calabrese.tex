\documentclass[output=paper]{langscibook}
\author{Andrea Calabrese\affiliation{University of Connecticut}}

\title[The morphosyntax of andative forms in the Campiota vernacular]
      {The morphosyntax of andative forms in the Campiota vernacular: The synthetic behavior of restructuring roots}
\abstract{This article investigates the syntactic and morphological properties of andative Motion Verb Constructions -- i.e., constructions that are composed of the motion verb go and a main lexical verb -- in Campiota, a southern Italian Salentino dialect.  Campiota displays two of such constructions; one is mono-clausal/mono-eventive and the other bi-clausal/bi-eventive.  It will be shown that both constructions share the same root /ʃ-/B-/ ‘GO’ with its idiosyncratic morpho-phonological properties, including its suppletive patterns.  The same motion verb root, thus, displays a lexical use and an affixal one, which it will be argued results from a semantic bleaching operation.  In its lexical use the motion verb root may select argument structure and a full clause; when used as an affix, it is part of the full extended projection of the lower verb and has special morphological behavior: it can be reduplicated and is attached to the participle in participial compound tenses.  It will be argued that the relation between the lexical verb GO and its bleached affixal counterpart in Campiota MVCs is better understood if bleaching may entail an operation -- referred to here as Syntactic Truncation -- in which the higher motion verb selects a vP constituent and, therefore, all the projections of the lower verb are prevented from being projected.  The characteristic properties of MVCs in other Italo-Romance varieties will also be investigated: this will lead to an analysis of the restructured and non-restructured Infinitival MVCs and the MVCs with double inflections found in these other varieties and it will be shown how they correlate to the two andative MVCs in Campiota.  }

% First, we need counters to hold temporary values:
\newcounter{nexttmp}    % Counter for the Next example
\newcounter{nnexttmp}   % Counter for the NNext example
\newcounter{lasttmp}    % Counter for the Last example
\newcounter{llasttmp}   % Counter for the LLast example
% These replicate the linguex commands. I can't promise they won't break things.
\newcommand{\Next}{\setcounter{nexttmp}{\value{equation}}\stepcounter{nexttmp}(\thenexttmp)\xspace}
\newcommand{\NNext}{\setcounter{nnexttmp}{\value{equation}}\addtocounter{nnexttmp}{2}(\thennexttmp)\xspace}
\newcommand{\Last}{\setcounter{lasttmp}{\value{equation}}(\thelasttmp)\xspace}
\newcommand{\LLast}{\setcounter{llasttmp}{\value{equation}}\addtocounter{llasttmp}{-1}(\thellasttmp)\xspace}


%\bibliography{localbibliography-calabrese.bib}
%\newcommand{\orcid}[1]{}
%\pagenumbering{roman}

\begin{document}
\SetupAffiliations{mark style=none}
\maketitle

\section{Introduction}

%FOR FINAL VERSION: CHECK THE ENGMAS IN EXAMPLES!!

An examination of the syntactic and morphological properties of Motion Verb Constructions (MVCs) (\citealt{cruschina2013a, cruschina2021a}), i.e., constructions that are composed of a motion verb (go, come, pass (by), etc.) and a main lexical verb, in southern Italian dialects, reveals different types of morphosyntactic structures.\footnote{For ease of recognition I decided to use the commonly used Italianized form Campiota to refer to this variety, instead of the most proper vernacular Kampiotu.} Specifically, the higher verbal form in such constructions can in principle be a lexical verb, but can also be analyzed as an aspectual marker. In the case of the verb GO, \cite{cruschina2013a}, along the lines of Cinque (1999, 2006), refers to this aspect as the “andative”, which signals that a distance, possibly also temporal, away from the speaker, must be covered for the action to be realized or executed, matching in this way the directional properties of this verb lexical semantics, (see \cite{cinque2006a} for further references on this aspect, cf. also \cite{heine2018a}).\footnote{Further bleaching of the original movement of GO meaning in MVCs  may lead to the development of a future tense. This is a cross-linguistically common grammaticalization path (see \cite{bybee1994a}). However, there is no southern Italian dialect in which GO has developed the temporal function that it has in many other Romance varieties, such as Spanish, Portuguese, and French, where in MVCs the GO element functions as a future marker (see, e.g., \cite{squartini1998a}, among many others). }   The Salentino “andative” verbal forms have been recently investigated in \cite{ledgeway2016a} and in \cite{manzini2017a}, \cite{cardinaletti2019a} (see also \cite{andriani2017a, cardinaletti2001a, cardinaletti2003a, cruschina2013a, cruschina2021a}, \cite{caro2015a, caro2018a, caro2019a, caro2015b, manzini2005a} on related constructions in other southern Italian and Sicilian dialects). 

In this article I will focus on Campiota, the Salentino dialect of Campi Salentina, in the northern part of the Lecce province, which displays two andative Motion Verb Constructions (cf. \ref{ac2}--\ref{ac4}). They have approximately the same meaning, insofar as Campiota speakers may translate the same Italian sentences in either ways (\ref{ac1a}\,=\,\ref{ac2}), (\ref{ac1b}\,=\,\ref{ac3}), (\ref{ac1c}\,=\,\ref{ac4}) and readily switch from one construction to the other despite the striking differences they may have in some cases ((\ref{ac4}), for example): 

\ea \label{ac1}
\ea[] {\label{ac1a} 
\gll stasera vado a coricarmi presto\\
tonight go-PRS.1SG. to go.to.bed-INF-self.CL earlier\\ 
\glt `Tonight I will go to go to bed earlier'
  }
\ex[] {\label{ac1b} 
\gll stasera vado a comprarlo\\
tonight go-PRS.1SG. to  buy-INF-it.CL \\
\glt `Tonight I will go to buy it'
}
\ex[] {\label{ac1c} 
\gll ieri sono andato a comprarlo \\
tonight BE-PRS.1SG. go-PTCP-M.SG to buy-INF-it.CL\\
\glt `yesterday I went to buy it'
  }
\z
\ex \label{ac2}
\ea[] {\label{ac2a} 
\gll stasira au ku    mme  kurku  mprima\\
 tonight go-PRS.1SG. ku  self.CL go.to.bed-PRS.1SG earlier\\
  }
\ex[] {\label{ac2b}
\gll stasira me bba kkurku mprima\\
tonight self.cl GO- go.to.bed -PRS.1SG. earlier  \\
\glt `Tonight I will go to bed earlier'
}
\z
\ex \label{ac3}
\ea[] { \label{ac3a}
\gll stasira   au     ku  llu  kkattu\\
 tonight   go-PRS.1SG.   ku   it.CL buy-PRS.1SG  \\
} 
\ex[] { \label{ac3b}
\gll stasira   lu    bba   kkattu\\
tonight   it.CL  GO-  buy-PRS.1SG \\
\glt `Tonight I will  buy it'
}
\z
\ex \label{ac4}
\ea[] { \label{ac4a}
\gll jeri su  ʃutu ku llu kkattu\\
  yesterday  BE-PRS.1SG.   go-PTCP-M.SG ku  it.CL buy -PRS.1SG\\
}
\ex[] { \label{ac4b}
\gll jeri l’ addʒu  ʃʃa kkattatu\\
yesterday  it.CL  HAVE-PRS.1SG. GO- buy-PTCP-M.SG\\
\glt  `yesterday I went to buy it.'}
\z
\z

As shown below\footnote{In order to demonstrate the mono-clausality/bi-clausality of Campiota MVCs I will use the diagnostics proposed by \cite{cardinaletti2003a} to test the mono-clausality/bi-clausality of Doubly Inflected Constructions (DIC) and the related infinitival MVC in Sicilian dialects (see below for more discussion and Section 3.7 on the DICs).},  the first construction is a bi-clausal/bi-eventive structure where a fully-fledged bundle of Tense and Agreement features is morphologically realized on the matrix verb. The matrix verb GO selects for a clause introduced by the complementizer /ku/ which is similar to an independent infinitive in Italian or to a subjunctive in the Balkan languages (see \cite{calabrese1993a}), and may involve a reduced CPs (FinP in Rizzi’s 1997 terms).  I will refer to it as the Full-fledged MVC.

The second construction is a mono-clausal/mono-eventive structure where it is the lower verb that is morphologically marked with the full-fledged bundle of Tense and Agreement features.\footnote{See \cite{prete2020a}, \cite{todaro2018a} for a semantic analysis of the single event interpretation of MVCs in Sicilian dialects.}  The verbal GO element in it appears as an uninflected particle and behaves as an affix attached to the lower verb.\footnote{See \cite{cruschina2013a}, \cite{cruschina2021a} and \cite{ledgeway2016a} on the invariant forms as the final morphological step of a grammaticalization due to different morpho-phonological reduction processes.}  I will refer to it as the Reduced MVC.   

A crucial issue will be the morpho-syntactic status of the verbal GO element in Campiota Reduced MVCs. \cite{cardinaletti2003a} argue that the motion verbs that appear as the higher verbs in mono-clausal MVCs are “lexical categories merged as functional heads” in the extended projection of the lower verb. On the other hand, they define these verbs as “semi-lexical verbs” because, while it is true that they lack, or have lost their canonical lexical properties, they still retain their motion semantics. The morpho-syntactic status of the higher motion verbs in mono-clausal MVCs is, however, not fully investigated in their work.  This is what I will do in this article building on their analysis.  Specifically, I will propose that the GO element in Campiota Reduced MVCs is actually an affix (see Section 3.7 for the DICs studied by \cite{cardinaletti2003a}). However, as expected in Cardinaletti \& Giusti’s analysis, this affixal element shares semantic and crucially morpho-phonological properties -- in particular, the same suppletive allomorphy --  with its lexical counterpart.  So, I will also propose that both Campiota MVC constructions share the same root /ʃ-/B-/ ‘GO’ with its idiosyncratic morpho-phonological properties, including its suppletive patterns. 

It follows that same motion verb root displays a lexical use and an affixal one:  in its lexical use it may select argument structure and a full clause; when used as an affix, it is part of the full extended projection of the lower verb and has special morphological behavior: it can be reduplicated and is attached to the participle in participial compound tenses.  I will argue that the relation between the lexical verb GO and its bleached counterpart in Campiota MVCs is better understood if the bleaching involves an instance of an operation -- referred to here as Syntactic Truncation -- in which the higher motion verb selects a vP constituent and, therefore, all the projections of the lower verb are prevented from being projected (\cite{wurmbrand2014a, wurmbrand2015, wurmbrand2017verb}). This relation is instead not adequately captured in an approach like \cite{cinque2001a, cinque2006a} where all restructuring verbs are always functional heads. If we adopt this hypothesis, on the one hand, the fact that the reduced and full-fledged andative constructions of  Campiota share the same lexical root becomes a matter of pure coincidence. In this approach, in fact, the sharing of the root can only be motivated historically, but not synchronically in which case one could in fact expect totally different lexical exponents. On the other hand, the semantic interchangeability of these constructions also becomes a problem in the sense that one must also assume that there is an andative interpretation of the full-fledged MVC, which implies that the "andative" GO can be an aspectual functional head not only in reduced MVCs but also in full-fledged ones, something which is not explainable in this approach.

The article is organized as follows:  Section 2 discusses the basic facts concerning the Campiota MVCs starting with the diagnostics for their biclausality/mono-causality, and shows that the reduced MVCs are mono-clausal and the full-fledged ones biclausal (§2.1).  The following sections deal with the special properties of these MVCs:  §2.2 discusses the ku-clauses that are embedded in full-fledged MVCs; §2.3 shows that the ʃʃa/bba piece that characterizes reduced MVCs can be analyzed as having the same root of the andative element appearing  the full-fledged MVCs in so far as they share the same basic morpho-phonological properties; §2.4 investigates the progressive aspect constructions which often co-occur with the reduced andative MVCs. §2.5 deals with the reduplication process targeting the ʃʃa/bba piece in reduced MVCs and §2.6 with the peculiar position of the ʃʃa/bba piece in compound tenses where it appears attached to the lower participle.  Section 3 provides an analysis of the Campiota facts. §3.1 shows how abstract syntactic structures are converted into surface morpho-syntactic head-complexes in Distributed Morphology (\cite{halle1993a}), the morphology model adopted here. §3.2 discusses the process of syntactic truncation with converts full-fledged MVCs into reduced ones. §3.3 accounts for the progressive structure morpho-syntax, and §3.4 for the formation of periphrastic structures, and for the peculiar positioning of the ʃʃa/bba piece in compound tenses. §3.5 analyzes the reduplication process characterizing the reduced MVCs. The relation between the ku-clauses embedded in full-fledged MVCs in Campiota and the infinitival clauses of the same constructions in other Italo-Romance varieties is dealt with in §3.6. The final Section 3.7 deals with the Doubly-Inflected MVCs of other southern Italo-Romance varieties in which both the higher motion verb and the lower verb share the same inflectional features. A brief conclusion ends the paper.

\section{Facts}
\subsection{Diagnostics for clausality and the Motion verbs construction in Campiota}

 The first diagnostic used by \cite{cardinaletti2003a} is clitic climbing (\cite{rizzi1976a}). In fact, since clitic pronouns are clause-bound and target the first T-layer above them, they provide a good diagnostics for mono-clausality/bi-clausality. Clitic climbing to the higher GO element in the MVCs only occurs in the reduced ones, as in (\ref{ac5aii}), (\ref{ac6aii})-(\ref{ac7aii}) (vs. (\ref{ac5bii}), (\ref{ac6bii})-(\ref{ac7bii})) but not in the Full-fledged ones (see (\ref{ac5ai}), (\ref{ac6ai})-(\ref{ac7ai}) vs. (\ref{ac5bi}), (\ref{ac6bi})-(\ref{ac7ai})); in the latter case the pronoun follows the connecting element /ku/ and procliticizes onto the lower verb:\footnote{The attentive reader will note that there are alternations in the length of the word-initial consonants in the following examples, e.g. \textit{kurkamu} in (\ref{ac5ai}) vs. \textit{kkurkamu} in (\ref{ac5aii}).  They are governed by the Raddoppiamento Sintattico process discussed in section 2.3.  The word-initial geminate in the verb \textit{kkattare} ‘buy’ (\ref{ac6}) and (\ref{ac7}), however, is underlying.} 

\ea \label{ac5}
    \ea \label{ac5a}
        \ea[] { \label{ac5ai}
        \glt Stasira    ʃamu     ku  nne    kurkamu      mprima\\
        tonight   go-PRS.1PL  ku   self.CL  go.to.bed-PRS.1PL earlier\\
        }
        \ex[] { \label{ac5aii}
        \gll Stasira   ne    ʃʃa   kkurkamu        mprima\\
        tonight   self.cl   GO-  go.to.bed -PRS.1PL    earlier\\
        }
        \z
    \ex \label{ac5b}
        \ea[*]{\label{ac5bi} Stasira ne ʃamu ku kkurkamu mprima}
        \ex[*]{\label{ac5bii} Stasira ʃʃa nne kurkamu mprima\\
          `Tonight we will go to bed earlier'}
          \z
    \z
\ex \label{ac6}
    \ea \label{ac6a}
        \ea[] { \label{ac6ai}
        \gll Stasira  ʃamu     ku  llu    kkattamu\\
        tonight  go-PRS.1PL  ku    it.CL  buy-PRS.1PL\\ 
        }
        \ex[] { \label{ac6aii}
        \gll Stasira  lu   ʃʃa   kkattamu\\
        tonight  it.CL GO-  buy-PRS.1PL\\
        }
        \z
    \ex \label{ac6b}
        \ea[*]{\label{ac6bi} Stasira lu ʃamu ku kkattamu}
        \ex[*]{\label{ac6bii} Stasira ʃʃa  llu kkattamu\\
             `Tonight we will  buy it'}
        \z
    \z
\ex \label{ac7}
    \ea \label{ac7a}
        \ea[] { \label{ac7ai}
        \gll jeri   simu     ʃuti       ku   llu   kkattamu\\
        yester.  BE-PRS.1PL  go-PTCP-M.PL ku   it.CL buy-PRS.1PL\\
        }
        \ex[] { \label{ac7aii}
        \gll jeri    l’   imu         ʃʃa  kkattatu\\
             yesterday it.CL  HAVE-PRS.1PL    GO- buy-PTCP-M.SG \\
        }
        \z
    \ex \label{ac7b}
        \ea[*]{\label{ac7bi} jeri lu simu  ʃuti  ku   kkattamu}
        \ex[*]{\label{ac7bii} jeri imu lu ʃʃa  kkattatu/*jeri imu ʃʃa llu  kkattatu\\
        `yesterday I went to buy it'}
        \z
    \z
\z

If a reduced MVC is monoclausal and the GO verbal element is a functional head, we expect that the latter cannot project its argument structure. This leads to Cardinaletti and Giusti’s second diagnostics: while a Full-fledged MVC allows arguments -- such as the directional  locative in (\ref{ac8a}) and the mean of transportation in (\ref{ac8b}) -- or lexically selected clitic clusters (\ref{ac8c}), a Reduced MVC does not allow either (cf. (\ref{ac9a}, \ref{ac9b}, \ref{ac9c})),:

\ea \label{ac8}
    \ea[] {\label{ac8a}
    \gll  au      rittu   a mare ku  mme  ddifrisku\\
    go.PRS.1SG  towards to sea   ku  self.CL freshen.up-PRS.1SG\\
    \glt `I am going to the sea.shore to get some fresh air'
    }
    \ex[] {\label{ac8b}
    \gll  au           ku   ffatiu    ku  la makina\\
    go.PRS.1SG  ku   work.1p.sg with the car\\
    \glt `I go  to work with the car'
    }
    \ex[] { \label{ac8c}
    \gll oɲɲi ssira   se   ne     ʃʃia    ku  ffatia      fore \\
    every evening self.CL from-it.CL go.IPF.3SG  ku  work.PRS.3SG to-the-c. \\
    \glt `Every evening he used to go to work in the countryside'
    }
    \z
\ex \label{ac9}
    \ea[*] {\label{ac9a}
    \gll mme  bba    ddifrisku     rittu   a   mare\\
    self.CL go.PRS.1SG  freshen.up-PRS.1SG towards to  sea\\
    \glt ‘I am going to the square to get rest’
    }
    \ex[] {\label{ac9b}
    \gll *bba   ffatiu       ku   la makina \\
     GO   work-PRS.3SG with the car\\
    }
    \ex[] {\label{ac9c}
    \gll oɲɲi ssira    (*se ne)    ʃʃa  ffatiava     fore \\
    Every evening (self.cl from-it.cl) GO – work-IPF.1SG to-the-c.\\
    }
    \z
\z

Strong support for a biclausal analysis of the Full-fledged MVC comes from the third diagnostics proposed by \cite{cardinaletti2001a, cardinaletti2019a}). Discussing Salentino examples from Lecce, \cite{cardinaletti2019a} observe that Full-fledged MVCs refer to two different events, while reduced ones have a single event interpretation. Here I adapt their examples in the Campiota vernacular.  By stating (\ref{ac10a}) with a reduced MVC, the speaker not only claims that she goes to buy chicory but, crucially, that she actually buys it every day. For this reason, the continuation which implies that the event of buying has not taken place is ungrammatical. This is not the case in the Full-fledged MVC (\ref{ac10b}) where the two verbs have separate Tenses:

\ea \label{ac10}
    \ea[] { \label{ac10a}
    \gll bba kkattu       le tʃikorie oɲɲi dʒurnu (*ma nu lle trou  mai)\\
      GO buy-PRS.3SG  the chicories every day  (but not them.cl find.PR3SG  never)\\
    }
    \ex[] { \label{ac10b}
    \gll au    ku  kkattu       le tʃikorie oɲɲi dʒurnu (ma nu lle trou mai)\\
       go.1p.sg ku buy.PRS.3SG the chicories every day (but not them.cl find.PR3SG  never)\\
    \glt `I go to buy the chicories every day but I never find them'
    }
    \z
\z

In conclusion, a reduced MVC is a mono-clausal / mono-eventive structure: the higher verbal element GO appears in a reduced and uninflected form; it lacks arguments, and there is obligatory clitic climbing onto it.   Tense+Agr are morphologically realized only on the lower verb.  The Full-fledged MVC is a bi-clausal / bi-eventive structure: the higher GO selects a clause introduced by the complementizer /ku/, shows full Tense+Agr realization and cannot be reduced morphologically; it has arguments, and does not allow clitic climbing. 

\subsection{ku-clauses}

In Full-fledged MVCs, the higher GO verb selects a clause introduced by /ku/.  A brief discussion of such clauses is required.  A /ku/-clause can be analyzed as a reduced subordinate clause (FinP) with an independent TP.  In this regard, they are parallel to the well-established subordinate clauses found in Balkan languages (Albanian, Romanian, Greek) which replaces infinitival clause (See \cite{calabrese1993a, rivero1994a, manzini2005a, roberts2003a} a.o.). Ku embedding in Salentino varieties, covers all obligatory control and raising environments as well non-obligatory control contexts and subjunctive contexts in general. The particle /ku/ is typically used to introduce clauses embedded under verbs of ordering, desiring, warning, requesting, urging, fearing, etc., as well as purpose clause, and “before that”-clauses (but not “after that”-ones). Thus, as detailed by \cite{calabrese1993a}, /ku/ introduces the clausal complement of verbs which express an attitude towards, or an attempt to bring about, an event, or eventuality, which is yet to come.\footnote{In this sense, the eventuality identified in the clause introduced by /ku/ does not refer to a specific point of reference in time: it does not have a deictic tense.  Thus, in order to acquire a time reference so that it can be interpreted, the tense of this clause must refer to the time reference of the matrix clause (see section 3.6 for analysis).}

With certain verbs, the ku-clauses may alternate with the full finite clause introduced by ka ‘that’. 

\ea\label{ac11}
    \ea \label{ac11a}\gll speru   lu  Karlu   ku  bbene      kraj \\
       hope.PRS.1SG    the  Karlu   ku  come.PRS.3SG  tomorrow\\
    \ex \label{ac11b}\gll speru       ka   lu  Karlu ene       kraj \\
     hope.PRS.1SG    ka   the  Karlu  come.PRS.3SG  tomorrow\\
   \glt ‘I hope that Charles comes tomorrow’
    \z
\z

Note that the position of the subject of the lower verb is above ku in (\ref{ac12}) (see also (\ref{ac11a}). This makes this finite clause different from the embedded clause introduced by the complementizer \textit{ka}, which occurs above the subject position (see (\ref{ac11b}) above). 

\ea\label{ac12}
\gll ojju   {lu Marju} ku {*lu Marju} (bb)ene  kraj \\
 want.PRS.1SG {the Marju} ku {the Marju}  come.PRS.3SG tomorrow\\
\glt ‘I want Mario to come tomorrow’
\z

Following \cite[p36]{calabrese1993a}, I take the subject to be in the usual preverbal subject position, where it receives nominative case.  One can hypothesize that the connecting element ku occurs in a position of the IP field, which \cite{roberts2003a} takes to be MoodP, the same position as infinitival \textit{to} in English and subjunctive \textit{na} in Greek.

The independent subject position of the \textit{ku} clause can, but does not have to be, anaphoric to the subject of the main predicate:

\ea \label{ac13}
    \ea \label{ac13a}
        \gll ojju          ku   bbeɲɲu      kraj\\
   want.PRS.1SG   ku    come.PRS.1SG  tomorrow\\
    \glt   ‘I want to go away’
    \ex \label{ac13b}
        \gll ojju          ku   bbene       kraj \\
   want.PRS.1SG   ku   come.PRS.3SG  tomorrow\\
    \glt ‘I want that he comes tomorrow’
    \z
\z

When it is anaphoric, other languages requires the clause to be infinitival (cf. (\ref{ac14a}) in Italian and (\ref{ac14b}) in English):

\ea \label{ac14}
    \ea[*] { \label{ac14a}
    \gll Voglio   che      venga    domani  /voglio     venire domani\\
         want.PRS.1SG che  come.PRS.1SG tomorrow /want.PRS.1SG  come.INF tomorrow\\
    \glt ‘I want to come tomorrow’
    }
    \ex[*]{\label{ac14b}I want that I come tomorrow/I want to come tomorrow.}
    \z
\z

 As observed by \cite{calabrese1993a}, tense distinctions are neutralized in ku-clauses; so, sequence-of-tenses restrictions are absent in such a clause.  As discussed in \cite{calabrese1993a} the ku-clause contains a Tense anaphoric to the Tense in the higher verb (see Footnote 6) as is the case in subjunctives (cf. \cite[p46-48]{calabrese1993a} and \cite[p652]{manzini2005a}).

\ea \label{ac15}
    \ea[]{\label{ac15a}
    \gll ulia        ku   llu    kkattu \\
     want.IPF.1SG ku   it-CL-  buy-PRS.1SG\\
    \glt ‘I would have liked to buy it’}
    \ex[*]{\label{ac15b}
    \gll ulia       ku   llu     kkattava \\
     want.IPF.1SG ku   it.CL-  buy.IPF.1SG \\}
    \z
\z
     
Aspectual contrast, though, may occur in the ku-clause:

\ea \label{ac16}\gll ulia         ku  ll’   ia    kkattatu     mprima  (cf. (\ref{ac15}))\\
  want.IPF.1SG ku  it.CL- HAVE-past buy-PRS.1SG earlier\\
 \glt ‘I would have liked to have bought it’
\z

Note, however, that aspectual contrasts are not possible in the Fully-Fledged MVC:\footnote{The use of the progressive sta makes this sentence more felicitous (see Section 2.4):

\ea \label{fn7ex} {sta ʃia ku llu kkattu}
\z

It is to notice that in alternative to the present in the embedded clause in (\ref{ac17a}) , also the imperfect could be used as in (\ref{fn7ex2}) in striking contrast with (\ref{ac15b}). The reasons for this are unclear to me at this moment:

\ea \label{fn7ex2}{sta ʃia ku llu kkattava}
\z}

\ea\label{ac17}
    \ea[] {\label{ac17a} 
    \gll ʃia       ku llu   kkattu \\
    go-IPF.1SG ku it-cl.  buy-PRS.1SG  \\
    \glt ‘I was going to buy it earlier’}
    \ex[*] {\label{ac17b}
    \gll ʃia    ku llu  ll’addʒu kkattatu / l’ia kkattatu\\
    go-IPF.1SG  ku  it.CL  buy-PRS.1SG / it.CL.have-PRS.1SG buy-PTCP / it.CL have-IPF.1SG buy-PTCP\\
    }
    \z
\z 

\textcolor{red}{something is wrong with alignment \Last}

It is also to note that the subject of the embedded clause in this type of construction must always be anaphorically bound by the subject of the matrix clause:

\ea \label{ac18}\gll Lu Rontsui   ʃiu        proi, *j  ku  llu    kkatta\\
  The R.     go-PRF.3SG  proi, *j  ku  it-cl.  buy-PRS.1SG  \\
  \glt ‘Oronzo went to buy it’
\z

The connecting ku element may be absent in ku-klauses (see \cite{ledgeway2015a} for an investigation of ku-omission patterns in Salentino).  Note that in this case the neutralized temporal patterns discussed above are also found in the absence of ku. Note also that in this case the clitic pronoun remains on the lower verb:\footnote{In Northern varieties of Salentino (e.g. in the dialect of Mesagne) clitic climbing may occur when ku is absent (\cite{calabrese1993a, terzi1992a, terzi1994a, terzi1996a})):

\ea \label{fn8ex}\gll nɔ  lu  vɔɟɟu  ffattsu   ccui    \\ 
 not it.CL  want.1p.sg  do.PRS.1SG  anymore\\
 \glt `I no longer want to do it’
\z 

}$^,$\footnote{However, /ku/ omission is dispreferred in Campiota in Full-fledged MVC:

\ea \label{fn9ex}
    \ea \label{fn9exa} \gll stasira au      ku  mme    kurku       mprima\\
     tonight go-PRS.1SG  ku self.CL  go.to.bed-PRS.1SG  earlier\\
    \ex \label{fn9exb}\gll stasira  au       ku  llu   kattu\\
   tonight  go-PRS.1SG  ku  it.CL buy- PRS.1SG \\
    \z
\ex \label{fn9ex2}
    \ea \label{fn9ex2a}???stasira au me kurku mprima
    \ex \label{fn9ex2b}??stasira au llu kattu 
    \z
\z
  
In some other varieties ku-deletion is accepted:  Carmiano: ʃamu nne kurkamu GO-PRS.1PL self.cl   go.to.bed -PRS.1PL. Note the absence of clitic climbing:  *nne ʃamu kurkamu.  Other varieties of the same type, especially from southern Salento, display neutralization of AGR distinction for the higher motion verb in the singular, as in \ref{fn9ex3}.  Also these varieties do not allow clitic climbing if the higher verb is inflected: *nne ʃamu kurkamu:  

\textcolor{red}{table missing}}

\begin{comment}

\ea \label{fn9ex3}
\begin{table}[htb] \small
\begin{tabular}{llll}
bba        & mme     & korku             & ( Tricase) \\
GO-PRS.1SG & self.cl & go.to.bed-PRS.1SG &            \\
bba        & tte     & korki             & 2SG        \\
bba        & sse     & korka             & 3SG        \\
ʃamu       & nne     & korkamu           & 1PL        \\
ʃati       & bbe     & korkati           & 2PL        \\
ane        & sse     & korkane           & 3PL       
\end{tabular}
\end{table}
\z

\end{comment}

\ea \label{ac19}
    \ea {\label{ac19a}
        \gll ulia        ku   llu    mandʒu \\
    want. IPF.1SG   ku   it.CL  eat. PRS.1SG \\
    }
    \ex {\label{ac19b}
    \gll ulia            llu    mandʒu \\
    want. IPF.1SG       it.CL  eat. PRS.1SG\\
    \glt ‘I would like to eat it’ 
    }
    \z
\z
    
It is to observe finally that the fact that clauses that would be infinitival in languages like Italian and English are replaced by ku-clauses in Salentino is not due to the morphological absence of infinitival morphology in this Romance variety (see \cite{calabrese1993a}). In fact, infinitival forms can actually be used when a verb is a complement of restructuring verbs such as modal or aspectual ones: must, be able, begin, finish, continue, stay, try, etc. In this case, we are dealing with a restructuring configuration, in which the restructuring verb behaves as a functional head (see \cite{rizzi1976a, rizzi1978a, cinque2001a, cinque2006a, wurmbrand2001a,wurmbrand2004a,wurmbrand2015,wurmbrand2017verb}).  As expected, clitic climbing is obligatory in this case:

\ea \label{ac20}
    \ea {\label{ac20a}
        \gll llu    pottsu      kkattare   kraj\\
     it.CL  can-PRS.1SG buy-INF   tomorrow\\
     \glt ‘I can buy it tomorrow’
    }
    \ex{\label{ac20b}
        \gll ll’   addʒu       kkattare    kraj\\
     it.CL  must-PRS.1SG buy-INF   tomorrow\\
    \glt  ‘I must buy it tomorrow’
    }
    \ex{\label{ac20c}
        \gll llu   ntʃiɲɲu       a   ffare     kraj\\
     it.CL  begin-PRS.1SG  a   do-INF   tomorrow\\
     \glt ‘I begin to do it tomorrow’
    }
    \ex{\label{ac20d}
        \gll llu   spittʃu      te   pulittsare  stasira\\
     it.CL  finish-PRS.1SG te   clean-INF  tonight\\
     \glt ‘I finish to clean it tonight’
    }
    \z
\z

Many of these verbs can also appear with ku-clauses -- with no apparent meaning changes. In this case no climbing is possible, as expected:\footnote{Note that ku-deletion is again dispreferred in these cases in Campiota;

\ea \label{fn10ex}
    \ea \label{fn10exa}???ntʃiɲɲu llu fattsu kraj  
    \ex \label{fn10exb}???spittʃu llu pulittsu stasira
    \z
\z

} 

\ea \label{ac21}
    \ea \label{ac21a}\gll ntʃiɲɲu     ku  llu   fattsu     kraj  (cf. (20)c) \\
     begin-PRS.1SG  ku  it.CL do-PRS.1SG  tomorrow\\
    \ex \label{ac21b}\gll spittʃu      ku  llu   pulittsu     stasira (cf. (20)d) \\
     finish-PRS.1SG ku it.CL clean-PRS.1SG  tonight\\
    \z
\z


\subsection{Reduced MVC}

As already discussed, Reduced MVCs are characterized by an uninflected morphological piece ʃʃa/bba attached to the lower inflected verb.   

Note, first of all, that the ʃʃa/bba piece and the verb to which it is attached are tightly connected and cannot be separated by adverbials or other materials. 

\ea \label{ac22}
    \ea \label{ac22a}\gll au       sempre  ku  llu   leggu       allu  bar\\
    go-PRS.1SG   always   ku  it-CL  read-PRS.1SG to-the bar\\
    \ex \label{ac22b} *lu bba ssempre leggu allu bar\\
    \ex \label{ac22c}\gll lu    bba lleggu       sempre  allu   bar \\
    it-CL   GO  read-PRS.1SG  always  to-the   bar \\
  \glt  “I always go to read it at the bar’
  \z
\z

The complex piece ʃʃa/bba plus the following verb appears to be a single morpho-syntactic constituent.  This is also shown by the fact that it can be syntactically moved as a single syntactic piece in imperative forms. Note that even in these imperative forms the clitic cannot appear between the andative form and the following verb:

\ea \label{ac23}
    \ea \label{ac23a}\gll bba  kkatta=lu!      (cf. lu   bba  kkatti)\\
    GO buy-IMPER.2SG=it-CL   {} it-CL GO  buy-PRS.2SG \\
    \glt ‘go buy it’   \hspace{8em}         ‘you go buy it’
  \ex \label{ac23b}\gll ʃʃa  kurkamu=ne!    (cf. ne   ʃʃa   kurkamu)\\
    GO go.to.bed-IMPER.1PL=self-CL  { } self-CL GO  go.to.bed-PRS.1PL \\
    \glt ‘let’s go to bed’  \hspace{3.25cm}   ‘we go to bed’
  \ex  \label{ac23c}*bba -lu kkatta!  
  \ex \label{ac23d}*ʃʃa -ne kurkamu!  

    \z
\z

The ʃʃa/bba piece does not correspond to any surface verbal forms.  It is analyzed below. To understand its nature and composition it is necessary to consider the forms of ʃire as the main verb, and analyze them:

\ea \label{ac24}ʃire as a main verb:
    \ea \label{ac24a}Present:\\
       \ea\label{ac24ai}~\\
            \begin{table}[htb]
            \begin{tabular}{llll    }
              & Today & go-PRS. & to.the.country.side          \\
            1SG & Oʃe   & áu      & fore                          \\
            2SG & Oʃe   & ái      & fore                         \\
            3SG & Oʃe   & áe      & fore                         \\
            1PL & Oʃe   & ʃámu    & fore                         \\
            2PL & Oʃe   & ʃáti    & fore                         \\
            3PL & Oʃe   & áune    & fore                      \\
            \end{tabular}
            \end{table}
        \ex\label{ac24aii}~\\
        \begin{table}[htb]
        \begin{tabular}{lllll}
         & Today & not     & go-PRS              & to.the.c. \\
        1SG & Oʃe   & nu      & bbáu                & fore      \\
        2SG & Oʃe   & nu      & bbái                & fore      \\
        3SG & Oʃe   & nu      & bbáe                & fore      \\
        1PL & Oʃe   & nu      & ʃʃámu               & fore      \\
        2PL & Oʃe   & nu      & ʃʃáti               & fore      \\
        3PL & Oʃe   & nu      & bbáune              & fore     
        \end{tabular}
        \end{table}
        \z  
    \ex \label{ac24b}Imperfect:\\
    \gll      quannu era    vaɲɲone, ʃía      fore    ogni dʒurnu\\
     when be.IPF.1SG child  go-IPF-1SG to.country.side every day\\
     \glt ‘When I was a child, I used to go to the country side every day’
    \ex  \label{ac24c}Perfect: \\
    \gll jeri      ʃívi      a lla  kasa   te l’ isabbella\\
     yesterday  go-PRF.1SG to the house of the Isabella.\\
    \glt ‘Yesterday I went to Isabella’s home’
    \ex \label{ac24d}Infinitive: \\
     \gll pottzu  ʃíre    aɖɖai  moi\\
     be.able go-INF  there now\\
     \glt ‘I can go there now’
    \z
\z

Comparing forms such as ʃámu/ ʃíamu with counterparts in other verbs such as  kattámu/kattá(v)amu, rumpímu/rumpíamu tenímu/teníamu, the simplest segmentation leads to consider /-mu/ the suffixal marker for the 1st plural. We are therefore left with the bases  ʃá-/ ʃía-, kattá-/kattá(v)a-, rumpí-/rumpía-, tení-/tenía-.  It can be assumed that the marker of the imperfect is  the suffixal element /-(v)-a- /.\footnote{The element /-a-/ in this case is the thematic vowel of the imperfect node.  The exponent of the imperfect is actually /-v-/ which is deleted, as in this case, unless it is between identical vowels.  See below for a brief discussion of the allomorphy of this element.} This further segmentation gives us the following verbal themes:  ʃa-/ ʃi-, katta-, rumpi-, teni-  (note that in these forms I am removing the accent that is determined by rules that are not relevant to the discussion carried out here). One problem now is the state of the final vowel of the theme. The comparison of  katta-, rumpi-, teni, with other themes such as  manda-, parla-, etc.,  endi, pendi-,  etc., shows that the final vowel is common in different sets of  themes.  This final vowel can then be segmented into an element that is traditionally called  the  "thematic vowel". To explain the distribution of thematic vowels we must say that it is lexically  conditioned:  some verbs take the thematic  vowel  /-a-/, and others the thematic vowel  /-i-/. At this point it can be observed that the lexical meaning of the  theme  is  due to the piece that precedes the thematic vowel. This piece is traditionally called the root. So in the case of ʃámu/ ʃíamu  we have the root /ʃ-/, in addition to  katt-, rump-, ten-, mand-, parl-, end-, pend,etc. -,  for the other verbs mentioned above.  Note that  ʃire  is also characterized by a change in the quality of the thematic vowel between the forms of the present and those of the non-present, as in the imperfect, perfect, infinite: the thematic vowel is /-a-/ in the present  (eg. [[ʃ-]$_{Root}$  a-]$_{Thematic Vowel}$ ) otherwise it is  /-i-/ (eg. [[ʃ-]$_{Root}$  i-]$_{ThematicVowel}$).

We can now consider the other forms of the verb ʃire  in  (1).  It is evident that the root does not always appear in the same form as it happens to other verbs.  There is a surface alternation between radical allomorphs such as  [Ø-], [bb-], [ʃ-]  and  [ʃʃ-]  (e.g.,  Ø-a-u/bb-a-u/ʃ-a-mu/ ʃʃ -a-mu).  

In order to account for these alternations we need to deal with some aspects of the phonology of consonants in Salentino, and specifically in Campiota.   I begin with with the phonology of voiced labial obstruents, since it is of fundamental importance to understand the alternations involving the root GO.  

Note first of all that there are no single voiced stops in Salentino.  They were affected by a process of lenition that turned them into fricatives ([b]$\rightarrow$[v], [g]$\rightarrow$ [j/Ø]), although they could also devoiced ([d]$\rightarrow$[t], [g]$\rightarrow$[k], (see \cite{calabrese1987a} for discussion and an analysis).  Voiced stops were preserved only when geminated [bb/dd/gg]. I will consider only the labial ones here and neglect the other ones.

The outcome [v] of single /b/ is neutralized with the etymological single voiced labial fricative [v] in Campiota as shown by the fact that  the latter does not alternate with a geminated [vv]\footnote{Other Northern Salentino varieties, such as that of Latiano for example, do indeed have geminated [vv] (Cf. Latiano: veni come- PRS.3SG ‘he comes’ vs. ku vveni ‘ku he comes’ (\cite{urgese2003a}).}  but with a geminated [bb] (see below for examples).

At the same time, in Campiota, single voiced labial fricatives are deleted unless they are found between identical vowels in word medial position.  For example, the imperfect suffix /-v-/ is systematically deleted after the TV [i] of the second conjugation although preserved after the TV [a] of the first conjugation when  another suffixal low vowel follows: fatʃ-i-[Ø]-a-mu make-TV-IPF-TV-1pl ‘we were making’, fin-i-[Ø]-a-mu finish-TV-IPF-TV-1pl ‘we were finishing’, but kkatt-a-[v]-a  buy-TV-IPF-TV-1sg ‘I was buying’, kkatt-a-[v]-a-mu ‘buy-TV-IPF-TV-1pl ‘we were buying’\footnote{Note 2nd Sing Imperfect kkatt-a-[v]-i ‘buy-TV-IPF-TV-2sg‘ you(sg) were buying” without [v]-deletion.  This form points out to a derivation in which [v]-deletion precedes the independently needed rule of TV -deletion before vowels (TV$\rightarrow$Ø/V), i.e., kkatt-a-[v]-a-i$\rightarrow$[v]Del: n/a$\rightarrow$kkatt-a-[v]-a-i$\rightarrow$TVDel$\rightarrow$kkatt-a-[v]-Ø-i}  (cf. also ʃ-i-v-i ‘GO-Perf-1sg vs. ʃ-i-Ø-u ‘GO-Perf-3sg).  In the same way, word-initial [v] is deleted in Campiota: inire  ‘come’, itire ‘see’, cf. the Salentino dialect of Latiano: vinire ‘come’ vitire ‘see’ (\cite{urgese2003a}).   This created situations where one observes alternations between [Ø] and [bb]. To understand these alternations one must consider the so-called Raddoppiamento Sintattico (RS) (‘Syntactic Doubling’), another process that characterizes Salentino as well as other central and southern Italo-Romance varieties.  It triggers gemination of word-initial consonants in certain phonological and morphological contexts (\cite{chierchia1986a, loporcaro1997}).  In Salentino, it is triggered by morphemes such as kkju ‘more’, pi ‘for’, ku ‘with’, nu ‘negation’, elements such ku ‘ku’, sta ‘STAY$_{[+progr]}$’, bba/ ʃʃa ‘GO$_{[+and]}$’ etc.

\ea\label{ac25}
    \ea\label{ac25a}
        \ea \gll ete  pattʃu\\   
    be.PRS.3SG crazy\\
    \glt    ‘he is crazy’
        \ex \gll  ete     kju  ppattʃu\\
    be.PRS.3SG more crazy   \\
    \glt ‘he is more crazy
        \z
    \ex\label{ac25b}
        \ea  \gll kraj \\          
    tomorrow \\
        \ex \gll pi kkraj\\   
      for tomorrow\\
        \z
    \ex\label{ac25c}
        \ea \gll lu tene\\      
        it.CL hold-PRS.3SG \\
        \glt he holds it       
        \ex \gll lu   sta ttene\\
        it.CL STAY[+progr] hold-.3SG\\
        \glt ‘he is holding it’
        \z
    \ex\label{ac25d}
        \ea \gll le     kattsa   \\ 
            them.CL crack-PRS.3SG\\ 
            \glt ‘s/he cracks them’     
        \ex \gll ole ku kkattsa   mennule\\
        want-PRS.3SG ku crack-PRS.3SG almonds\\
        \glt ‘s/he wants to crack almonds’
        \z
    \z
\z  

The RS rule is proposed in \figref{ac26}.  It inserts a skeletal position after diacritically marked morphemes.  Automatic resyllabification and melodic spreading as in \figref{ac26b} leads to gemination.

\begin{figure}
    \caption{\label{ac26}\color{red}Please provide a caption}
    \begin{subfigure}[b]{.5\linewidth}\centering
    \begin{forest}
        [R
            [N
                [∅ $\rightarrow$ X\quad /\quad X{$]$}$^{[+RS]}$\quad \underline{\hspace{2em}}
                ]
            ]
        ]
        \end{forest}
    \caption{\label{ac26a}\color{red}Please provide a subcaption}
    \end{subfigure}\begin{subfigure}[b]{.5\linewidth}\centering
    \begin{forest}
		[,phantom
			[R,tier=t3,name=r [N,tier=t2 [X {]\textsuperscript{[+RS]}},tier=t1]]]
			[X, tier=t1,name=x]
			[σ,calign=child, calign child=2 
					  [X,tier=t1 [{[+cons]},name=cons]] 
			   		  [R,tier=t3 [N,tier=t2 [X,tier=t1]]]]
		]
		\draw [-{Triangle[]}] (r.south) to[out=270,in=90] (x.north);
		\draw [-{Triangle[]}] (cons.north) to[out=90,in=270] (x.south);		
	\end{forest}
	\caption{\label{ac26b}\color{red}Please provide a subcaption}
	\end{subfigure}
\end{figure}

We can now consider the surface alternations between [Ø] and [bb] in Campiota.  Consider the words in (\ref{ac27}) which etymologically have an initial /b/ or /v/ (cf. Italian battere ‘beat’, basso ‘short’, venire ‘come’, always display a geminated [bb] in RS environments:

\ea\label{ac27}
    \ea\label{ac27a}
        \ea \label{ac27ai}\gll lu atte\\
            it.CL beat-PRS.3SG\\  
            \glt `s/he beats'  
        \ex \label{ac27aii}\gll lu   sta     bbatte\\
            it.CL STAY[+progr] beat-PRS.3SG\\
            \glt `s/he is beating it'
        \z
    \ex\label{ac27b}
        \ea \label{ac27bi}\gll aʃʃu\\
                short\\
        \ex \label{ac27bii}\gll kju     bbaʃʃu\\
                more    short\\
        \z
    \ex\label{ac27c}
        \ea \label{ac27ci}\gll eɲɲu\\
            come-PRS.1SG\\ 
            \glt `I come'   
        \ex \label{ac27cii}\gll ojju      ku   bbeɲɲu\\
            want-PRS.1SG ku   come-PRS.1SG\\
            \glt `I want to come' 
        \z
    \z
\z

I assume that they all contain an underlying abstract labial obstruent /B/, i.e., [+consonantal, −sonorant, +labial] segment, which is assigned the feature [−continuant] when geminated (\ref{ac28}) but is otherwise deleted\footnote{The same alternations occur with vowel initial words that were etymologically onsetless (cf. Italian: alto `high', alzare `lift').  One must assume that they were reanalyzed as having an initial [B]:

\ea \label{fn14ex}
    \ea \label{fn14exa}
        \ea \label{fn14exai}\gll ete autu\\
            be-PRS-3sg high\\
            \glt ‘he is high’    
        \ex \label{fn14exaii}gll ete kju bbautu\\
            be-PRS-3sg more high\\
            \glt ‘he is higher’
        \z
    \ex \label{fn14exb}
        \ea \label{fn14exbi}\gll lu ausu\\
            it.CL lift-PRS.1SG\\
            \glt ‘I lift it’  
        \ex \label{fn14exbii}\gll lu sta bbausu\\
            it.CL STAY[+progr]  lift-PRS.1SG\\
            \glt ‘I want to lift it’
        \z
    \z
\z

}

\ea \label{ac28}\textcolor{red}{28 missing}
\z

\ea \textcolor{red}{29 missing}
\z

We now have all the machinery need to account for the surface alternations between [Ø-], [bb-], [ʃ-]  and [ʃʃ-]  displayed by the root GO in (\ref{ac1}) . The first step is the observation that in this case there is a root suppletivism as in Italian GO verbal forms where one finds the suppletive alternant /vad-/ in the present forms with the exception of the 1st and 2nd plural, otherwise one find the alternant /and-/(cfr.vado/vai/va/andiamo/andate/vanno,  andavo,  andare ‘GO-PRS-1SG/2SG/3SG/ 1PL/2PL/3PL,GO-IMPF-1SG,GO-INF’).\footnote{The same pattern is found in most other Italo-Romance varieties. \cite{calabrese2012a, calabrese2015a}) (following \cite{embick2010a}, accounts for it in terms of impoverishment of the special diacritic triggering suppletion. Due to space restrictions, I cannot deal with this issue further here.}For the root GO in Campiota, one can propose the underlying suppletive alternants  /B-/  and  /  ʃ-/. As seen in  (\ref{ac24})  the alternant  /ʃ-/  has a wider distribution than that of the alternant  /B-/ which is restricted only to the present with the exception of the first and second plurals (i.e. like that of the Italian  vad- with respect to  and-).  This alternant can therefore be considered the basic elsewhere suppletive exponent of the root GO, as shown in the VI in (\ref{ac30}):\footnote{\textcolor{red}{there is no FN16.}}

\ea\label{ac30}
    \ea \label{ac30a}/B-/ $\longleftrightarrow$ GO / \underline{\hspace{2em}}  [-Past]$_{T^0}$
    \ex\label{ac30b} /ʃ-/ $\longleftrightarrow$ GO
    \z
\z

The geminated instances of these exponents are due to RS, cf. (\ref{ac24aii}) where they are triggered by the negation particle /nu/. Furthermore, single /B-/ is deleted by the rule in (\ref{ac28}) and is assigned the feature [-continuant] by the rule in (\ref{ac29}) when it is geminated.

The alternations for the root GO observed in the present forms in (\ref{ac24}) can now be analyzed as in (\ref{ac31}):

\ea\label{ac31}
    \ea \label{ac31a}{[[[B-]}$_{Root}$-a]$_{TV}$]$_{v^0}$ -u]$_{T^0 [-Pst]+AGR}$ $\rightarrow$ (29)$\rightarrow$ [[[Ø-]$_{Root}$-a]$_{TV}$]$_{v^0}$ -u]$_{T^0 [-Pst]+AGR}$        (cf. au in (24)ai))
    \ex \label{ac31b}{[[[B-]}$_{Root}$-a]$_{TV}$]$_{v^0}$ -u]$_{T^0 [-Pst]+AGR}$ $\rightarrow$ RS (26)$\rightarrow$ [[[BB-]$_{Root}$-a]$_{TV}$]$_{v^0}$ -u]$_{T^0 [-Pst]+AGR}$      $\rightarrow$(28)$\rightarrow$ [[[bb]$_{Root}$-a]$_{TV}$]$_{v^0}$ -u]$_{T^0 [-Pst]+AGR}$   (cf. bbau in (24)aii)) 
    \ex \label{ac31c}{[[[ʃ -]}$_{Root}$-]$_{TV}$]$_{v^0}$ -mu]$_{T^0 [-Pst]+AGR}$ (cf. ʃamu in (24)ai))
    \ex \label{ac31d}{[[[ʃ-]}$_{Root}$-a]$_{TV}$]$_{v^0}$ -mu]$_{T^0 [-Pst]+AGR}$ $\rightarrow$RS (26)$\rightarrow$[[ʃʃ-]$_{Root}$-a]$_{TV}$]$_{v^0}$ -mu]$_{T^0 [-Pst]+AGR}$           (cf. ʃʃamu in (24)aii))
    \z
\z

Once this is analysis is done, we can observe that the root suppletive alternants found in the reduced MVC have the same distribution of their counterparts in the full-fledged MVCs and in the uses of  ʃire ‘GO’ as a main verb with the proviso that the TV of the reduced MVC is always /a/.\footnote{The extension of the Thematic vowel /a/ to all forms of GO in the Reduced MVC is characteristic of Campiota.  Other northern  Salentino varieties, such that of Latiano (Urgese pc.), display the same alternation in TV a/i one observes when GO is the main verb: Present: Stasira mmi va kurku mprima/ nni ʃa kurkamu mprima vs. Non-present: oɲɲi sera mmi ʃi kurkava alle noe/ m’ aggju ʃʃi kurkari.}

Now, as noted above, the initial consonant of the exponent of GO in Reduced MVC is systematically geminated in Campiota.  So, we not only always have bb but also ʃʃ in this case.\footnote{
This is again a characteristic feature of Campiota.  Thus, in the northern Salentino varieties of Latiano, the initial consonant of the Reduced MVC andative exponent is geminated only in RS contexts, otherwise it is single.  See examples in the preceding note.} I assume that this is due to the fact that the rule in \figref{ac26a} was generalized in this context as (\ref{ac32}) case of analogical levelling due to the frequent occurrence of this allomorph in a RS environment, e.g. after progressive \textit{sta}.\footnote{Note that Salentino allows onset geminates (cf. (\ref{fn19ex})) the imperative of the verb kkattare ‘buy’:

\ea[\#] {\label{fn19ex}
\gll kkatta-lu\\
buy-Imperative.2SG it.CL\\
\glt ‘buy-it’
}
\z

So the onset geminate consonant generated by the application of (\ref{ac32}) (and (\textcolor{red}{32})) is left untouched.
}

\ea \label{ac32}
  \begin{forest}
    [σ,calign=child, calign child=2 
                  [X,tier=t1,name=xl] 
                  [R,tier=t3 [N,tier=t2 [X,tier=t1,name=xr]]]]	
    \node [right=5pt of xr.base,anchor=base west, inner xsep=0pt] {{]\textsuperscript{[+andative]}}};
    \node [left=5pt of xl.base,anchor=base east, inner xsep=0pt] {∅ → X \quad / \quad [\_\_\_\_};	    
  \end{forest}
\z

\subsection{Progressive constructions}

A description of the behavior of the andative requires discussion of the progressive since the andative forms are more felicitous when used with the progressive.
Whereas progressive aspect in standard Italian is realized periphrastically with the auxiliary verb stare ‘stay’ followed by the verbal root inflected as a gerund (i.e., +ndo) (\textit{sto mangiando una mela} stay-Pres.1sg eat-TV-Gerund an apple ‘I am eating an apple’(see section 3.4 below on periphrastic structures), in Campiota, as in other Salentino varieties of Italo-Romance, this aspect is realized with a construction parallel to that of the Reduced MVCs, in which the invariant piece sta is attached to the main verb where inflectional T and AGR contrasts are marked:

\ea\label{ac33}
    \ea\label{ac33a} \gll  ne   sta   kkurkamu \\ 
    self-CL  STAY   go.to.bed-PRS.1PL\\
  \glt ‘we are going to bed’
    \ex\label{ac33b} \gll se    sta   kkurkane\\
   self-CL  STAY go.to.bed-PRS.3PL\\
     \glt ‘they are going to bed’
 \z
\z

There is also another construction displaying the verb stare, in addition to its use as a main verb (see (\ref{ac34})). This construction is parallel to a Full-Fledged MVC insofar as here the inflected verb stare governs a clause introduced by /ku/. However, differently than a Full-Fledged MVC that is readily interchangeable with a Reduced one from a semantic point of view, this construction does not have a progressive meaning, rather an inceptive or inchoative one that can be translated in English with “being about to”.

\ea \label{ac34}\gll kraj      stau       a  kkasa  tutta  la  matina\\
    tomorrow   stay-PRS.1SG at home all   the  morning\\
 \glt ‘Tomorrow I will stay home all morning.
\z

\ea\label{ac35}
    \ea \label{ac35a}\gll stamu      ku  nne     kurkamu\\
    stay-PRS.1Pl ku self-CL  go.to.bed-PRS.1PL\\ 
    \ex \label{ac35b}\gll staune      ku  sse     kurkane\\
    stay-PRS.3PL ku self-CL  go.to.bed-PRS.3PL\\
    \glt ‘We are about to go to bed/ they are about to go to bed’
    \z
\z

As in the case of the Reduced MVC, in progressive constructions, clitic climbing is required (\ref{ac36}), and no adverbial can occur between /sta-/ and the verb, differently than in its Italian counterpart \textit{lo sta già facendo} (\ref{ac37}):

\ea\label{ac36}
    \ea\label{ac36a} \gll ne   sta     kkurkamu    /  se    sta   kkurkane\\
    self-CL  STAY go.to.bed-PRS.1PL / self-CL STAY go.to.bed-PRS.3Pl\\
    \ex[*]  {\label{ac36b} sta   nne kurkamu     /   *sta   sse  kurkane}
    \z
\z

\ea\label{ac37}
    \ea[*] {\label{ac37a}\gll lu   sta    ddʒa    face\\
     it-CL STAY already  do-PRS.3SG\\
     }
    \ex \label{ac37b}lu sta fface dʒa / dʒa lu sta fface\\
     ‘He is already doing it
    \z
\z

Although durative or frequentative present forms do not require the presence of /sta-/ as in (\ref{ac38}), present forms normally appear with /sta-/ (\ref{ac38}):

\ea \label{ac38}\gll  au        fore         oɲɲissira\\
   go.PRS.1SG  to-the.countryside every night\\
 \glt  ‘I go to the country side every night’
\z

\ea \label{ac39}\gll sta    bbau       fore\\
   STAY- go.PRS.1SG  to-the.countryside \\
\glt   ‘I am going to the country side’
\z

As observed earlier, andative forms are more felicitous when used with the progressive. The same properties discussed earlier for the non-progressive Reduced MVCs hold for the progressive ones, in particular clitic climbing and the same suppletive patterns:

\ea \label{ac40}Clitic climbing 
    \ea \label{ac40a}
        \ea \gll stau       ku  mme   bba  kkurku      mprima\\
      stay-PRS.1SG ku self-CL  GO- go.to.bed-PRS.1SG earlier\\
      \glt ‘I am going to bed earlier’
        \ex \gll stau       ku   llu   bba   kkattu\\
      stay-PRS.1SG  ku   it-CL  GO-  buy-PRS.1SG\\
      \glt ‘I am going to buy it’
        \z
    \ex \label{ac40b}me sta bba kkurku mprima/lu sta bba kkattu
    \z
\z

\ea \label{ac41}Clitic Climbing \& Suppletion:
    \ea \label{ac41a}\gll stau    ku     llu   bba  kkattu/llu sta bba kkattu \hfill PRS.1SG\\
    stay-PRS.1SG ku  it-CL  GO-  buy-PRS.1SG\\
    \glt stai ku llu bba kkatti/llu sta bba kkatti  \hfill     PRS.2SG\\
    stae ku llu bba kkatta/llu sta bba kkatta   \hfill    PRS.3SG\\
    stamu ku llu ʃʃa kkattamu/llu sta ʃʃa kkattamu \hfill  PRS.1PL\\
    stae ku llu ʃʃa kkattati/ llu sta ʃʃa kkattati  \hfill    PRS.2PL\\
    staune ku llu bba kkattane/ llu sta bba kkattane \hfill PRS.3PL\\
    \z
\z

\ea \label{ac42}
    \ea \label{ac42a}\gll stia       ku mme   ʃʃa   kkurkava \\
    stay-PRS.1SG ku self-CL  GO- go.to.bed-PRS.1SG\\
    \ex \label{ac42b}\gll me    sta    ʃʃa   kkurkava\\
    self-CL STAY GO- go.to.bed-PRS.1SG\\
    \glt ‘I was going to go to bed’
    \z
\z


\subsection{Elative reduplication of the andative}

The andative particle that appears in Reduced MVC can be reduplicated “per rafforzare il suo significato” ‘to strengthen its meaning” as reported by a Campiota speaker.  I will refer to it as the elative reduplication of the andative.\footnote{I must admit that I am unable to express the meaning difference brought about the reduplication in the translation and to make explicit what “rafforzare il suo significato” ‘strengthen its meaning” really conveys in this context.  Note that in some speakers, reduplication appears to be obligatory.  Perhaps, as a morphological device, reduplication simply emphasizes the presence of the reduced construction, and its semantic effect, which has become obligatory for some speakers. In any case, given my doubts and unclarity about the semantic purport of reduplication, I decided to neglect referring to meaning changes in the translation.}  It is shown in (\ref{ac43}) (Observe that elative reduplication does not interfere with clitic climbing):

\ea \label{ac43}
    \ea \label{ac43a}
        \ea \label{ac43ai}\gll oʃe    me      bba   kkurku         mprima\\
       Today self.cl    GO-   go.to.bed -PRS.1SG.   earlier\\
        \ex \label{ac43aii}$\rightarrow$ oʃe    me    ʃʃa  bba   kkurku         mprima
        \z
    \ex \label{ac43b}
        \ea \label{ac43bi}\gll oʃe    me   sta     bba  kkurku      mprima\\
       Today self.cl  STAY  GO-  go.to.be-PRS.1SG.   earlier\\
        \ex   \label{ac43bii}$\rightarrow$ oʃe    me   sta   ʃʃa bba  kkurku mprima\\
       ‘Today I am going to bed earlier’  
        \z
    \z
\z
 
The order of the reduplicative element appears to be fixed:

\ea \label{ac44}???oʃe me bba ʃʃa kkurku mprima
\z

The sequences bba bba/ ʃʃa ʃʃa are disallowed: the second andative element always appears as bba regardless of what the base should be, and the first one as ʃʃa:

\ea \label{ac45}
    \ea \label{ac45a}\gll *me    sta     bba   bba kkurku      mprima\\
    self.cl  STAY   GO-   go.to.bed -PRS.1SG.   earlier\\
    \glt me sta  ʃʃa bba kkurku mprima\\
     ‘I am going to bed earlier’
    \ex \label{ac45b}\gll *ne    sta     ʃʃa   ʃʃa kkurkamu       mprima\\
    self.cl  STAY     GO-  go.to.bed -PRS.1PL.   earlier\\
    \glt nne sta ʃʃa bba kkurkamu mprima\\
    ‘we are going to bed earlier’
    \ex \label{ac45c}\gll *me   sta  ʃʃa ʃʃa kkurkava      kwannu ete rriatu\\
    self.cl  STAY  GO- go.to.bed -IPF.1SG.when  be--PRS.3SG arrive-PP\\
    \glt me sta ʃʃa bba kkurkava kwannu ete rriatu\\
    ‘I was going to bed when he arrived’
    \ex \label{ac45d}\gll *m   addʒu     ʃʃa ʃʃa  kkurkare    subbra lu tivanu\\
     self.cl  must-PS-1SG  GO- go.to.bed -INF  on    the  sofa.bed\\
    \glt m addʒu ʃʃa bba kkurkare subbra lu tivanu\\
    ‘I must go to bed on the sofa bed’
    \z
\z

\subsection{ʃire in periphrastic construction with an auxiliary:}

As already noted, in Reduced MVC, tense and aspectual contrasts appear on the lower verb as shown in (\ref{ac46}) by comparing full-fledged MVCs to reduced ones:

\ea\label{ac46}
    \ea\label{ac46a}
        \ea \label{ac46ai}\gll sta    bbau     ku   llu   kkattu\\
        STAY go-PRS.1SG  ku   it.CL buy-PRS.1SG   \\
        \ex \label{ac46aii}$\rightarrow$ \gll lu    sta    bba  kkattu\\
        it.CL  STAY GO  buy-PRS.1SG \\
        \glt ‘I am going to buy it’
        \z
    \ex\label{ac46b}
        \ea  \label{ac46bi}\gll sta    ʃʃia       ku   llu    kkattu \\
       STAY go-IPF.1SG  ku   it.CL  buy-PRS.1SG  \\
        \ex  \label{ac46bii}$\rightarrow$ \gll lu    sta    bba   kkattava\\
       it.CL   STAY GO   buy--IPF.1SG\\
       \glt ‘I was going to buy it’
        \z
    \ex\label{ac46c}
        \ea  \label{ac46ci}\gll ʃivi       ku   llu   kkattu \\
       go-PRF.1SG  ku   it.CL  buy- PRS.1SG \\
        \ex  \label{ac46cii}$\rightarrow$ \gll lu     ʃʃa   kkattai\\
       it.CL   GO  buy-PRF.1SG\\
       \glt ‘I went to buy it’
        \z
    \z
\z

\footnote{\textcolor{red}{fn21 missing in the original file}}

Now observe what happens in the case of compound tenses with an auxiliary and a participle: the participial morphology that in the full-fledged MVC appears on the andative verb appears on the lower verb in the reduced one; the andative ʃʃa  appears between the auxiliary and the participle and attaches to the latter; at the same time, the auxiliary\footnote{ʃire appears to be the only unaccusative verbs that selects the auxiliary essere ‘be’ in Campiota.  All other unaccusative verbs appear to select aire ‘have’, e.g. enire ‘come’ addʒu inutu ‘I have come’, partire ‘leave’, addʒu partutu ‘I have left’, murire ‘die’ a muertu ‘he has died’. Optionally also ʃire can select aire ‘have’: addʒu ʃutu.  No optionality is possible in the case of the reduced MVCs in (\ref{ac47}).}  is obligatorily selected from the lower verb and displays the tense and mode features of the higher andative verb of the full-fledged MVC:

\ea\label{ac47}
    \ea \label{ac47a}
        \ea \label{ac47ai}\gll su        ʃutu        ku  llu   kkattu\\
       be-PRS.1SG.    go-PTCP-M.SG  ku   it.CL buy -PRS.1SG\\ 
        \ex \label{ac47aii}$\rightarrow$ \gll l’   addʒu       ʃʃa   kkattatu\\
       it.CL have-PRS.1SG.    go  buy-PTCP-M.SG\\
      \glt ‘I went to buy it’
        \z
    \ex\label{ac47b}
        \ea \label{ac47bi}\gll era      dʒa    ʃutu      ku mme   kurku     quannu  e’       rriatu \\
       be-IPF.1SG already go-PTCP-MSG ku self.cl go.to.bed-PRS.1SG
       when    be-PRS.3SG  arrive-PP\\
        \ex    \label{ac47bii}$\rightarrow$  \gll m’   ia        dʒa   ʃʃa  kkurkatu        quannu  e’        rriatu\\
       self.cl  have-IPF.1SG already GO  go.to.bed-PTCP-M.SG  
       when    be-PRS.3SG  arrive-PP\\
       \glt ‘I had already gone to bed when he arrived’
        \z
    \ex\label{ac47c}
        \ea \label{ac47ci}\gll lu Pissu ulia   ku    (bb) era    dʒa   ʃutu         ku sse    kurka\\
       the P.  have-3SG want-PTCP ku be-IPF.3SG already go-PTCP-MSG    ku self.cl   go.to.bed-PRS.1SG\\
        \ex  \label{ac47cii}  $\rightarrow$ \gll lu Pissu ulia   ku       ss   ia      dʒa         ʃʃa  kkurkatu\\
      the P.  have-3SG want-PTCP ku self.cl  have-IPF.3SG already 
       GO go.to-bed PTCP-MSG\\
      \glt ‘Pissu would have liked to have already gone to bed’
        \z
    \z
\z

\subsection{Summary}

As shown in the previous sections, full-fledged and reduced MVCs have a close relationship despite their clear structural differences. This is demonstrated not only by the fact that these constructs can be easily exchanged semantically but also by the fact that they seem to share the same root /ʃ-/B-/ with all its idiosyncratic suppletive properties. At the same time, when this root is in a full-fledged MVC, it behaves like a lexical root in being capable of selecting an argument structure.  In contrast, when it is in a reduced MVC, it becomes a functional element syntactically and semantically integrated into the extended projection of the adjacent verb. In this case it behaves morphologically like an affix so that it can appear attached to the participle and in a position lower than that of the auxiliary in structures such as that of l'addʒu ʃʃa kkattatu (see (\ref{ac47})).  In this case, it can also undergo a morphological operation such as reduplication in \textit{oʃe me ʃʃa bba kurku mprima} (see (\ref{ac43})). Any analysis of these constructions must explain how theit root can be converted from a verbal position capable of syntactic projection into a functional element included in the extended projection of the lower verb and behaving like an affix.

\section{Analysis}
\subsection{A morphosyntactic analysis of lexical and functional restructuring}

The theory of Distributed Morphology proposes a piece-based view of word formation, in which the syntax/morphology interface is made as transparent as possible by incorporating hierarchical structure into morphology. Thus, it assumes the input to morphology to be syntactic structure where morphosyntactic and semantic features (or feature bundles) are distributed over nodes forming morphemes (see \cite{halle1993a}). Morphology manipulates these syntactic structures and eventually converts them into linear sequences of phonological representations:
 
\ea \label{ac48}
  The grammar\\
  \begin{forest} for tree = {l sep=2\baselineskip, s sep=3em}
   [(Syntactic derivation)
       [PF,edge label={node[midway,left,font=\small\itshape]{Morphology}}]
       [LF]
   ]
  \end{forest}
\z

The derivation of all morphological forms then takes place in accordance with the architecture given in (\ref{ac48}). Roots and other morphemes are combined into larger syntactic objects, which are moved when necessary (Merge, Move). Words, i.e., X$^0$-complexes, are generated by head movement operations. These X$^0$-complexes are the (abstract) morphosyntactic representations which are the input to phonological spell out.  During phonological spell out, phonological realizations are assigned to the terminal nodes via the cyclic application from the inside out of a process called Vocabulary Insertion. By this process, individual Vocabulary Items (VI)  --  rules that pair a phonological exponent with a morphosyntactic context -- are consulted, and the most specific VI that can apply to an abstract morpheme is inserted (in the so-called Elsewhere (Subset, Paninian) ordering).  Finally, after vocabulary insertion, morpho-phonological and phonological rules apply.  These rules eventually determine the surface allomorphy of words.  I will not deal with these rules here.

Along the lines of \cite{wurmbrand2015}, I will assume the verbal functional structure in (\ref{ac49}) which expresses the basic core temporal, aspectual and modal structure of eventualities:

\ea \label{ac49}
\gll [\textsubscript{MoodP} Mood$^0$ [\textsubscript{TenseP} T$^0$ [\textsubscript{AspP} Asp$^0$ [\textsubscript{VoiceP} Voice$^0$ [\textsubscript{vP} v$^0$ [\textsubscript{√p} √Root$^0$ ]]]]]]\\
     {} [+/−irrealis] {} {[+/−past, +/−future]} {}  {[+/−perfect., +/−result.]} {} [+/−passive] {} {} {} {} {}\\
\z

Additional functional heads may be provided by bleached lexical roots (=verbs triggering syntactic truncation/restructuring).\footnote{It can be hypothesized that the root in this case is subject to an operation of semantic impoverishment (=bleaching) that affects the root semantics in such a way that 1) it removes its ability to identify and describe an independent eventuality but 2) it preserves its abstract logical framework (see \cite{roberts2010a} for a more detailed discussion).  This logical framework can describe aspectual, or other, properties of another eventuality.  Thus, the impoverished form of the root ANDARE in Salentino loses its ability to refer to a separate event of movement and comes to indicate an aspectual property -- the “andative" one -- of the eventuality described by the lower verb.}  I will refer to these bleached roots with the term semi-lexical roots as Cardinaletti and Giusti do for andative verbs of the reduced MVCs. The semi-lexical roots express additional "nuances" of the eventualities.   I assume that the progressive and the andative are nuances of this type.

Semi-lexical roots have the property of being syntactically merged as normal lexical roots; therefore, from the formal point of view, they can select sentences and arguments and project a functional structure.  Their other essential property, however, is that of being able to lose this ability. Thus, once inserted they can trigger an operation of syntactic truncation / restructuring that will be discussed later. 

Before doing that, however, I want to consider  the question of how a universal hierarchical structure like that in (\ref{ac49}) is mapped onto surface morphological forms. In \cite{calabrese2019a}, I try to account for this mapping of functional structure into surface verbal forms. This model also accounts for when periphrastic morphology occurs. Here I will introduce this model by illustrating the derivation of simple forms such as Campiota kkattavamu ‘buy.IPF.1PL’.

As mentioned above, X$^0$-complexes, are generated by head movement operations. Along the lines of \cite{calabrese2014a}, I assume that morphological operations and syntactic derivation are cyclically interleaved. So, head-movement operations, during what I call morphological spell out (\cite{calabrese2019a}), may first generate X$^0$-complexes, i.e., words, that can then be targeted by other syntactic head-movement operations such as V-to C movement, etc. The word-forming head movement operation are the only one of relevance here.  

I assume that the affixal properties of functional heads during morphological spell-out follow from the morphological requirement in (\ref{ac50}):

\ea \label{ac50}Synthetic morphology constraint:       
Each functional head X$^0$ in an extended projection, with the exception of the topmost one,\footnote{This account for why the functional head C is not a verbal affix but an independent particle even if often cliticized to a verb or to another adjacent word.  When C is targeted by head-movement in V-to-C operations, V is always a fully formed verbal complex.} must be adjoined to a root or to a Y$^0$ complex including a root.
\z

In this system, syntactic representations in violation of (\ref{ac50}) are repaired through the operation in (\ref{ac51}), from \cite{harizanov2019a} (for the sake of simplicity, the alternative operation of head lowering is not covered in this paper since it is not directly relevant to the analysis developed here; see \cite{calabrese2019a} for more discussion):\footnote{In this approach, a single mechanism -- the synthetic morphology constraint (\ref{ac50}) --  with head raising (and head lowering) as the associated repair implements word formation. Such an approach is simpler, and more parsimonious, than other approaches such as that of \cite{bjorkman2011a} where m-word formation (head movement in her theory) is associated with infl-agreement, or \cite{pietraszko2017a} where word formation can be implemented by the mechanism of c-selection with m-word formation (head movement in her theory) as an additional strategy. It is closer to what has been proposed by \cite{arregi2018a} with a single operation (Generalized Head raising) including both head raising and lowering.}

\ea \label{ac51}A syntactic complementation relation [$_{XP}$ ...X$^0$  ... [$_{YP}$ ... Y$^0$   [$_{ZP}$ ... ]  ]  may be  realized in the morphology as a complex head by:\\
\textbf{Head Raising}:\\
{[}$_{XP}$ ...X$^0$ ... [$_{YP}$ ... Y$^0$   [$_{ZP}$ ... ] ] ] $\rightarrow$ [$_{XP}$ ... [$_{X^0}$ Y$^0$ X$^0$] [$_{YP}$ ...[$_{ZP}$ ... ] ] ]\\
(where X$^0$ and Y$^0$ are heads, X$^0$ c-commands Y$^0$, and there is no head Z$^0$ that c-commands Y$^0$ and is c-commanded by X$^0$)
\z

Given the syntactic structure in (\ref{ac52}), head raising generates the structure in (\ref{ac53}):

\ea \label{ac52}
    \begin{forest}
        [XP
            [X$^0$]
            [YP
                [Y$^0$]
                [\dots]
            ]
        ]
    \end{forest}
\ex \label{ac53}A word-generated by head raising:\\
\begin{forest}
    [XP
        [X$^0$
            [X$^0$]
            [Y$^0_i$]
        ]
        [YP
            [t$_i$]
            [\dots]
        ]
    ]
\end{forest}
\z

Therefore, given the structure in (\ref{ac49}),  head raising to satisfy (\ref{ac50}) will create the structure in (\ref{ac54}) by moving the root or a constituent including the root in a roll up fashion upwards cyclically and adjoining them to the functional head in the extended projection:\footnote{The positioning of the exponent of the head as a suffix/prefix is due to information associated with the exponent and not a morphosyntactic property (see below).}

\begin{figure}
    \caption{\label{ac54}\color{red}Please provide a caption}
	\begin{forest} for tree = {fit=band}
		[MoodP
		  [Mood$^0$
		  	[$\text{T}^0_1$
		  		[$\text{Asp}^0_k$
		  			[$\text{Voice}^0_j$
		  				[$\text{v}^0_i$
		  					[$\surd{}\text{Root}_h$]
		  					[$\text{v}^0_i$]
		  				]
		  				[$\text{Voice}^0_j$]
		  			]
		  			[$\text{Asp}^0_k$]
		  		]
		  		[$\text{T}^0_l$]
		  	]
		  	[Mood$^0$]
		  ]
		  [TP
		  	[$\text{t}_l$,name=tl]
		  	[AspP
		  		[t$_k$,name=tk]
		  		[VoiceP
		  			[t$_j$,name=tj]
		  			[vP
		  				[t$_i$,name=ti]
		  				[$\surd{}$P
		  					[t$_h$,name=th]
		  					[\dots]
		  				]
		  			]
		  		]
		  	]
		  ]
		]
	\draw [->] (th) to [in=0,out=180] (ti);
	\draw [->] (ti) to [in=0,out=180] (tj);
	\draw [->] (tj) to [in=0,out=180] (tk);		
	\draw [->] (tk) to [in=0,out=180] (tl);			
	\end{forest}
\end{figure}

Three important operations are needed to derive the surface structure of Romance verbal forms including the Campiota ones. Two of them insert ornamental\footnote{Ornamental means that they do not have syntactico-semantic functions or content  but only a morpho-phonological one.}  morphological pieces such as AGR (\cite{halle1993a, bobaljik2000a}) and Thematic Vowels (\cite{oltra-massuet2005a}). The third delinks nodes with non-overt exponents through a pruning operation.

The rules inserting ornamental pieces are the following.  The rule in \ref{ac55} inserts AGR. The two rules in (\ref{ac56a}) and (\ref{ac56b}), instead, insert Thematic Vowels (TV) in Italo-Romance verbal forms. One rule adjoins a TV to v$^0$  (see (\ref{ac56a}). It applies early in the derivation before vocabulary insertion (and the subsequent pruning operations discussed below). Another rule of TV insertion applies after vocabulary insertion and the pruning operations; hence it adjoins a TV only to overt functional heads (\ref{ac56b}):

\ea \label{ac55}AGR-insertion:\\
 Given a Complex X$^0$ not including inherent phi-features, adjoin AGR$_V$ to its highest X$^0$ (to be revised later).
\z

\ea \label{ac56}
    \ea \label{ac56a}(it applies before VI)\\
        \begin{forest}
         [v$^0$,name=v0
           [v$^0$] [TV]
         ]
         \node [left=1em of v0.base west,anchor=base east] {v$^0$ →};
        \end{forest}
    \ex \label{ac56b}(X$^0$\,=\,functional; it applies after VI and pruning, if α is an overt exponent)\\
        \begin{forest}
          [X$^0$,name=X0
            [X$^0$] [TV]
          ]
        \node [left=1em of X0.base west,anchor=base east] {X$^0$ →};
        \end{forest}
    \z
\z

So, (\ref{ac55}) and (\ref{ac56a}) apply in the case of complex head structure in (\ref{ac1}).  Hence, \figref{ac57} is generated in the case of the form \textit{kkattavamu} ‘buy-IPF-1PL’(before vocabulary insertion).

\begin{figure}
\caption{\label{ac57}\color{red}Please insert a caption}
	\begin{forest}
		[Mood\textsuperscript{P}
			[Mood$^0$
				[$\text{T}^0_h$
					[$\text{Asp}^0_k$
						[$\text{Voice}^0_j$
							[v\textsuperscript{0}
								[$\surd{}\text{Root}^0_i$ \textsc{buy}]
								[v\textsuperscript{0}
									[v\textsuperscript{0}]
									[TV]
								]
							]
							[$\text{Voice}^0_j$]
						]
						[$\text{Asp}^0_k$ {[−perf]}]
					]
					[$\text{T}^0_h$ {[+past]}]
				]
				[Mood\textsuperscript{0} {[−irrealis]}]
			]
			[\textsc{agr 1pl},edge=dotted]
		]		
	\end{forest}
\end{figure}

Now let's move onto the third operation necessary to derive the surface structure of verb forms. The complex head in  \figref{ac57} is the basic structure of the Italo-Romance verb forms, including Campiota Salentino, before the insertion of the lexical entries. It is an agglutinative structure, i.e., a cumulation of morphological nodes. However, in Italo-Romance, functional categories such as aspect, tense and mood are no longer represented as independent morphological pieces as in the Latin pluperfect subjunctive form laud-a-vi-s-se-mus, i.e., [[[[laud-]$_{Root}$ [-a]$_{TV}$]$_{v^0}$-v-[-i]$_{TV}$]$_{[+perf]-Asp^0}$] -s]$_{[+past]}-T^0$  -s -[-e]$_{TV}$ ]$_{[+irr]-Mood0}$]-mus][$_{1pl-AGR}$  ‘praise.PLUPRF.SUBJ1PL’ (see Calabrese (2020) for discussion of the constituency of Latin verbs)). On the contrary, a single morpheme /-v-/ appears for the string Aspect$_{[-perfect]}$ +T$_{[+Past]}$+Mood$_{[-irreal]}$ in the Campiota kkattavamu.  An operation that can account for this is null node pruning proposed by \cite{calabrese2019a}.\footnote{Pruning was originally proposed by \cite{embick2010a} only for non-overt category defining nodes. Following \cite{christopoulos2017a} and \cite{christopoulos2018a}, \cite{calabrese2019a} extended it to all types of non-overt category nodes and reformulated it as in (\ref{ac36}) and used it 1) to simplify the phonological realization of morphosyntactic structures, 2) to account for the convergence of possibly complex morphosyntactic structures and their possibly simpler PF surface shape, and also 3) crucially to explain the fact that phonologically null exponents -- regardless of their marked/unmarked status --  appear not to act as interveners for morphological locality (cf. \cite{embick2010a, calabrese2019a}). Null node pruning also provides an alternative to fusion (cf. \cite{halle1993a}).}
 
It consists of the cyclic delinking of nodes with non-overt exponence. Note that an exponent /- Ø-/ is automatically  inserted  into a independently motivated  terminal node when there is no vocabulary insertion rule that assigns to this node a phonologically overt exponent.  After this pruning operation, the features that were become floating by pruning are docked upwards onto the highest adjacent node.  In a system like the Salentino one discussed so far where only the terminal node [-perfective, +past]$_{T^0}$ is assigned overt exponence, i.e., /-v-/, all other nodes are assigned Ø as in (\ref{ac58}), and pruned away, given their cyclic bottom up order.

\ea \label{ac58}
    \ea Ø $\longleftrightarrow$ v$^0$
    \ex Ø $\longleftrightarrow$ Voice$^0$
    \ex Ø $\longleftrightarrow$ Asp$^0$
    \ex Ø $\longleftrightarrow$ [-perfective, +past]$_{T^0}$
    \ex Ø $\longleftrightarrow$ Mood$^0$
    \z
\z

Phonological spell out operates cyclically node-by-node bottom up.  TV insertion and vocabulary insertion -- where in addition to overt exponents, Øs are inserted, followed by their pruning and feature docking -- will generate the cyclic derivation in (\ref{ac39}). In (\ref{ac39}), some of the verbal functional nodes are fused together due to pruning  --  in cyclic steps, due to the cyclic nature of vocabulary insertion.

\begin{figure}\small
\caption{\label{ac59}\color{red}Please provide a caption}
\begin{subfigure}[b]{.5\linewidth}\centering
\begin{forest}
[,shape=coordinate
[Voice$^0$
  [v$^0$,name=v0
    [$\surd$Root\\/kkat/]
    [v$^0$
      [v$^0$] [TV]
    ]
  ]
  [Voice$^0$]
]
[,phantom]
]
\draw [dashed] (v0.south west) |- (v0.north) -| (v0.south east);
\end{forest}
\caption{}
\end{subfigure}\begin{subfigure}[b]{.5\linewidth}\centering
\begin{forest}
[,shape=coordinate
  [Voice$^0$
	[v$^0$,name=v0
	[Root\\/kkat-/]
	[v$^0$
	[v$^0$\\∅,name=v0empty,edge label={node[midway,rotate=33]{||}}] [TV\\/-a-/,name=TVa]
	]
	]
	[Voice$^0$]
  ]
  [,phantom]
]
\draw [dashed] (v0.south west) |- (v0.north) -| (v0.south east);
\draw [-{Triangle[]}] (v0empty) to[in=270,out=270] (TVa);
\end{forest}
\caption{}
\end{subfigure}\bigskip\\\begin{subfigure}[b]{.5\linewidth}\centering
\begin{forest}
[,shape=coordinate
  [Asp\textsuperscript{0}
    [Voice\textsuperscript{0},name=V
      [v$^0$
      	[$\surd$Root\\/kkatt-/]
      	[v$^0$ [TV\\/-a-/]]
      ]
      [Voice$^0$\\{[aPass]}\\∅,edge label={node[midway,rotate=-33]{||}},name=voice0]
    ]
    [Asp\textsuperscript{0},name=asp0]
  ]
  [,phantom]
]
\draw [-{Triangle[]}] (voice0) to[in=270,out=345] (asp0); 
\draw [dashed] (V.south west) |- (V.north) -| (V.south east);
\end{forest}
\caption{}
\end{subfigure}\begin{subfigure}[b]{.5\linewidth}\centering
\begin{forest}
[,shape=coordinate
  [T$^0$
    [Asp$^0$,name=asp
      [v$^0$
        [$\surd$Root\\/kkatt-/]
        [v$^0$
          [TV\\/-a-/]
        ]
      ]
      [Asp$^0$\\{[aPass, −Perf]}\\∅,edge label={node[midway,rotate=-33]{||}},name=asp0]
    ]
    [T$^0$\\{[+Past]},name=t0]
  ]
  [,phantom]
]
\draw [dashed] (asp.south west) |- (asp.north) -| (asp.south east);
\draw [-{Triangle[]}] (asp0) to[in=270,out=345] (t0);
\end{forest}
\caption{}
\end{subfigure}
\end{figure}
{\renewcommand{\thefigure}{\arabic{figure} (continued)}
\addtocounter{figure}{-1}
\begin{figure}\small
\begin{subfigure}[b]{.5\linewidth}\centering
\setcounter{subfigure}{4}
\begin{forest}
[Mood$^0$
  [T$^0$,name=t0
    [v$^0$
      [$\surd$Root\\/kkatt/]
      [v$^0$
        [TV\\/-a-/]
      ]
    ]
    [T$^0$
      [T$^0$\\{[aPass, −Perf, +Past]}\\/-v-/]
    ]
  ]
  [Mood$^0$\\{[-irr]}]
]
\draw [dashed] (t0.south west) |- (t0.north) -| (t0.south east);
\end{forest}
\caption{}
\end{subfigure}\begin{subfigure}[b]{.5\linewidth}
\begin{forest}
[Mood$^0$
  [T$^0$,name=t0
	[v$^0$
	[$\surd$Root\\/kkatt-/]
	[v$^0$
	[TV\\/-a-/]
	]
	]
	[T$^0$
	[T$^0$\\{[aPass, −Perf, +Past]}\\/-v-/]
	[TV\\a]
	]
  ]
  [Mood$^0$\\{[-Irr]}]
]
\draw [dashed] (t0.south west) |- (t0.north) -| (t0.south east);
\end{forest}
\caption{}
\end{subfigure}\bigskip\\\begin{subfigure}[b]{.5\linewidth}\centering
\begin{forest}
[Mood$^0$
  [Mood$^0$,name=mood
    [T$^0$
      [v$^0$
        [$\surd$Root\\/kkatt-/]
        [v$^0$ [TV\\/a/]]
      ]
      [T$^0$
        [T$^0$\\/-v-/]
        [TV\\/-a-/]
      ]
    ]
    [Mood$^0$\\{[-irr]}\\∅,name=mood0irr]
  ]
  [\textsc{agr}\\\textsc{1pl},name=agr]
]
\draw [dashed] (mood.south west) |- (mood.north) -| (mood.south east);
\draw [-{Triangle[]}] (mood0irr) to[in=270,out=354] (agr);
\end{forest}
\caption{}
\end{subfigure}\begin{subfigure}[b]{.5\linewidth}
\begin{forest}
[Mood$^0$
  [T$^0$
    [v$^0$
      [$\surd$Root\\/kkatt-/[katt,tier=word,no edge]]
      [v$^0$ [TV\\/a/,l sep=0pt [-a-,tier=word,no edge]]]
    ]
    [T$^0$
      [T$^0$\\-v-[-v-,tier=word,no edge]]
      [TV\\-a-[-a-,tier=word,no edge]]
    ]
  ]
  [\textsc{agr}\\{[−irr, \textsc{1pl}]}\\-mu/ [-mu,tier=word,no edge]]
]
\end{forest}
\caption{}
\end{subfigure}
\caption{\color{red}Please provide a caption}
\end{figure}}

From now on, I will represent derivations such as that in (\ref{ac59}) as in \figref{ac60} where all of the different cyclic steps are compacted together.  The final output is given in \figref{ac61} where for simplicity, I replace the complex fused [v$^0$+TV] with TV, [Voice0+Asp$^0$+T$^0$] with T$^0$, and  [Mood0+AGR with AGR). Furthermore, I relabel the topmost headless Mood0 with T$^0$.

\begin{figure}
\caption{\label{ac60}\color{red}Please provide a caption}
\begin{forest}
[Mood$^0$
  [Mood$^0$
    [T$^0$,name=t0
      [Asp$^0$,name=asp
        [Voice$^0$,name=voice
          [v$^0$,name=v
            [$\surd\text{Root}_i$ [/katt-/]]
            [v$^0$
              [v$^0$\\/∅/,name=v0empty,edge label={node[midway,rotate=33]{||}}]
              [TV\\a,name=tva]
            ]
          ]
          [Voice$^0$\\/∅/,name=voice0empty,edge label={node[midway,rotate=-33]{||}}]
        ]
        [Asp$^0$\\{[−perf]}\\/∅/,name=asp0,edge label={node[midway,rotate=-33]{||}}]
      ]
      [T$^0$
        [T$^0$\\{[+past]}\\v]
        [TV/-a-/]
      ]
    ]
    [{[-irr]}\\∅,name=irr]
  ]
  [\textsc{agr}\\\textsc{1pl}\\-mu,name=agr1pl]
]
\draw [-{Triangle[]}] (v0empty) to [out=270,in=270] (tva);
\draw [-{Triangle[]}] (voice0empty) to [out=345,in=270] (asp0);
\draw [-{Triangle[]}] (irr) to [out=345,in=270] (agr1pl);
\foreach \i in {t0,asp,voice,v}
  {\draw [dashed] (\i.south west) |- (\i.north) -| (\i.south east);}
\end{forest}
\todo[inline] {One arrow from Asp⁰ missing – where should this one go?}
\end{figure}

\begin{figure} 
\caption{\color{red}\label{ac61}Please provide a caption}
\begin{forest}
    [T$^0$
        [T$^0$
            [v$^0$
                [$\surd{}\text{Root}_i$ [/katt-/,tier=word]]
                [TV [/-a-/,tier=word]]
            ]
            [T$^0$
                [T$^0$ [{[−perf, +past]} [/-v-/,tier=word]]]
                [TV [/-a-/,tier=word]]
            ]
        ]
        [\textsc{agr} [/-mu/,tier=word]]
    ]		
\end{forest}
\end{figure}

\subsection{A morphosyntactic analysis of the reduced MVC}

As shown in the preceding sections, Full-Fledged MVC and Reduced MVC, despite their clear structural differences, have a close relationship. Putting aside the striking semantic interchangeability between these two constructions, they share what appears the same root /ʃ-/B-/ ‘GO’ with its idiosyncratic suppletive properties. At the same time, whereas this root is clearly fully lexical when it is in the  Full-Fledged MVC -- and therefore it is characterized by the ability to select a CP and argumental structure, it becomes syntactically and semantically functional, and thus integrated in the extended projection of the lower verb, when it is in a Reduced MVC.  At this point, I need to account for how the andative root of a reduced MVCs can be converted from its syntactic projecting verbal position into a functional head included in the extended projection of the lower verb.

I assume that the Reduced MVC is derived from the structure underlying the Full-Fledged MVC. As shown below, this accounts for the presence and preservation of the higher functional structure of the GO verb which is identical in the Full-Fledged MVC and in the Reduced MVC:

\ea \label{ac62}
    \ea \label{ac62a}
        \ea\label{ac62ai}\gll ʃiunu ku llu ttʃitunu\\
        go-PRF.3PL ku it.CL kill.PRS-3PL\\
        \ex \label{ac62aii}$\rightarrow$ \gll lu    ʃʃa   ttʃiseru\\
        it.CL  GO  kill-PRF.3PL\\
        \glt ‘They went to kill him’
        \z
    \ex\label{ac62b}
        \ea \label{ac62bi}\gll ʃianu ku  llu kkattanu\\
        go-IPF.3PL ku it.CL buy.PRS-3PL\\
        \ex \label{ac62bii}$\rightarrow$ \gll lu ʃʃa  kkattavanu\\
        it.CL GO  buy-IPF.3PL\\
        \z
    \z
\z

I assume that in a full-fledged MVC, the GO root can appear not only in its regular lexical version but also in its semantically bleached form, so that in this case this construction may have an andative meaning identical to that of the reduced MVC. This explain the possibility of a semantic interchangeability between the two constructions. In this case, however, there is no application of syntactic reduction, an operation that characterizes only the reduced MVCs.

The hypothesis is thus that semi-lexical verb roots are inserted as normal lexical roots capable of projecting an extended projection, selecting argument structure and governing clauses referring to events.  In their bleached form, however, they can further undergo an operation of syntactic truncation such as that proposed in \cite{wurmbrand2014a, wurmbrand2015, wurmbrand2017verb}), that is, an operation of stripping of the structure associated with the bi-eventual interpretation of the previous construction. Thus, the temporal and aspectual structure of the lower proposition is removed, although not the v$^0$ of the lower verb which must be left intact given the preservation of verbalizers in the lower verb in the examples in (\ref{ac63}):

\ea \label{ac63}
    \ea \label{ac63a}\gll lu     ʃʃa  mmattsiʃu  (denominal from mattsa ‘club’+ verbalizer -iʃ-)\\
 it-CL GO  give.beating-PRS1SG\\
 \glt ‘I am going to give him a beating’
    \ex \label{ac63b}\gll ne    ʃʃa   ddifriskamu (deadjectival from frisko ‘fresh’)\\
 self.CL  GO  refresh\\
 \glt ‘We are going to refresh ourselves’
    \z
\z

Through stripping the "bleached" andative head becomes a functional head. As a functional head, it selects a reduced constituent vP, as in Wurmbrand's syntactic truncation; all unlicensed structure is then erased, including the level of complementer  phrase CP. As a functional head, the andative GO also loses the ability to select argument structure and to project its level vP. Thus, this bleached root and its extended projection become part of the extended functional projection of the lower verb: 

\ea\tabcolsep=0.5ex\resizebox{\linewidth}{!}{\begin{tabular}[t]{@{}l@{~}lllll@{}}
    a. & ... [\textsubscript{AspP} Asp$^0$ & [\textsubscript{vP} v$^0$ & [\textsubscript{√P} GO[+and]] & [\textsubscript{CP} C$^0$ [\textsubscript{TP} T$^0$ [AspP Asp$^0$ & [\textsubscript{vP} v$^0$ [\textsubscript{√P} √Root] ... ]]]]]]\\
	b. & ... [\textsubscript{AspP} Asp$^0$ & \sout{[\textsubscript{vP} v$^0$} & [\textsubscript{√P} GO[+and]] & \sout{[\textsubscript{CP} C$^0$ [\textsubscript{TP} T$^0$ [AspP Asp$^0$} & [\textsubscript{vP} v$^0$ [\textsubscript{√P} √Root] ... ]\sout{]]]]}]\\
	c. & ... [\textsubscript{AspP} Asp$^0$ &                        & [\textsubscript{√P} GO[+and]] &                                                           & [\textsubscript{vP} v$^0$ [\textsubscript{√P} √Root] ... ]]
\end{tabular}}
\z

Now, the andative functional head is in violation of (\ref{ac50}).  When after syntactic stripping, word-forming head movement applies, the v$^0$ complex including the lower root, is raised to this head position (From now on, for graphic simplicity, I will omit mentioning the nodes Mood0 and Voice0 that not only are always assigned Ø in Campiota, and therefore pruned but also do not play any role in the analysis)

\begin{figure}
  \caption{\color{red}Please provide a caption}
  \begin{forest} for tree = {fit=tight}
	[TP
	  [T$^0$\\{[βpast]}]
	  [AspP
	  	[Asp$^0$\\{[αperf]}]
	  	[$\surd$
	  		[$\surd$GO{[+and]},name=go
	  			[$\text{v}^0_i$
	  				[$\surd\text{Root}^0_l$]
	  				[$\text{v}^0_i$]
	  			]
	  			[GO {[+and]}]
	  		]
	  		[vP
	  			[t$_i$,name=ti]
	  			[$\surd$p
	  				[t$_l$,name=tl]
	  				[\dots]
	  			]
	  		]
	  	]
	  ]
	]	
	\draw [->] (tl) to[in=0,out=180] (ti);	
	\draw [->] (ti) to[in=0,out=180] (go);	
  \end{forest}
\end{figure}

Further cyclic movement to the higher functional heads, subsequent pruning and insertion of AGR and TVs generate (\ref{ac66}) (with the resulting structures in \figref{ac67}) shown here after VI insertion.\footnote{An additional VI is needed to account for the perfect form in (\ref{ac66b}).  It is given below:

\ea \label{fn29ex} /-s-/ $\longleftrightarrow$ [+perf]/Roots \underline{\hspace{2em}}
\z

(\ref{fn29ex}) is an instance of root-conditioned allomorphy. the aspectual exponent /-s-/ requires root-information to be inserted. Although it is not really relevant in the context of this paper, an important issue raises at this point: that of morphological locality.  Morphological locality is assumed to require adjacency: the issue is if it is structural (\cite{bobaljik2012a}, \cite{calabrese2019a}) or linear (\cite{embick2010a}).  If it is structural, the andative node should act as an intervener in the interaction between the root and the aspectual node, contrary to the facts. It must be linear then, since in this case the andative node does not interfere with the allomorphic interaction between the root and the following aspect node. It follows that if it is linear, linearization must occur cyclically but crucially preserving structural information. Thus, when the [+perf] node is reached, one must know what the TV is, so that it can be deleted. Further discussion of this topic is not possible here and must be left to future research.} Note that the regular application of (\ref{ac56b}) adjoins a TV to the andative head, as expected given its functional status.\footnote{An anonymous reviewer wonders if two verbs can be coordinated under a single ʃʃa/bba since according to the structure in \figref{ac67} this should be impossible. As the matter of facts, this is the case: So, the Italian sentence in (\ref{fn30exa}) can only be translated with a full-MVC in (\ref{fn30exb}), or with a coordination including two ʃʃa/bba pieces as in (\ref{fn30exc}) but not with a coordination under a single one as in (\ref{fn30exd}):

\ea[]{\label{fn30exa}\gll Ora lo andiamo a pulire e ricucire \\
                         now it.CL Go-PRS.1pl a clean and re-sew\\
                        \glt ‘Now we go to clean it and re-sew it again’}
\ex[]{\label{fn30exb}\gll moj  ʃamu      ku llu    pulittzamu     e   ku llu ripittsamu\\
                        now GO-PRS.1PL ku  it.CL clean-PRS.1PL and ku it.CLsew-PRS-1PL again\\}
\ex[]{\label{fn30exc} \gll moj lu   ʃʃa  pulittzamu    e    lu  ʃʃa ripittsamu\\
                     now it CL  GO- clean-PRS.1PL and it-CL sew-PRS-1PL again\\}
\ex[*]{\label{fn30exd} moj lu ʃʃa pulittzamu e  ripittsamu}
\z
}\footnote{The verb \textit{ttʃidere} ‘kill’ is athematic. The v$^0$-TV is pruned and deleted in this case (see \cite{calabrese2015a, calabrese2019a}) for discussion and analysis).  A phonological rule delete /d/ before /s/.
}

\ea \label{ac66}
\textcolor{red}{missing 66}
\ea \label{ac66a}
\ex \label{ac66b}
\z
\z

\begin{figure}
\caption{\label{ac67}\color{red}Please provide a caption}
\begin{subfigure}[b]{\linewidth}\centering
  \begin{forest}
  	[T$^0$
  		[T$^0$
  			[And$^0$
  				[And$^0$ [/\textit{ʃʃ}-/] [TV\\/a/]]
  				[v$^0$
  					[Root$_i$ [katt,tier=word,name=katt]]
  					[TV [a,tier=word]]
  				]
  			]
  			[T$^0$
  				[T\\\relax {[+perf, +past]} [v,tier=word]]
  				[TV[a,tier=word]]
  			]
  		]
  		[\textsc{agr} [nu,tier=word]]
  	]
  	\node [below=1pt of katt] {[\textit{ʃʃa kkattavanu}]};
  \end{forest}
\caption{\color{red}Please provide subcaption}
\end{subfigure}\medskip\\\begin{subfigure}[b]{\linewidth}\centering
  \begin{forest}
  	[T$^0$
  	  [T$^0$
  	    [And$^0$
  	    	[And$^0$
  	    		[/\textit{ʃʃ}-/]
  	    		[TV\\/a/]
  	    	]
  	    	[v$^0$ [Root$_i$ [/\textit{tʃid}-/,tier=word,name=tschid]]]
  	    ]
  	    [T$^0$
  	    	[T\\{[+perf, +past]} [/-s/,tier=word]]
  	    	[TV [e,tier=word]]
  	    ]
  	  ] 
  	  [\textsc{agr} [/ru/,tier=word]]
  	]
  	\node[below=1pt of tschid] {[\textit{ʃʃa ttʃiseru}]};
  \end{forest}
\caption{\color{red}Please provide subcaption}
\end{subfigure}
\end{figure}

Observe that the andative morpheme is a prefix in Figures~\ref{ac66}--\ref{ac67}.  One can assume that the linear order of the Full Fledged MVC is preserved in this case.  This can be accounted for by hypothesizing that the And0 exponent is marked as being antitropal, i.e. a prefix (cf. \cite{bye2012a}).
\footnote{It is to be observed that the order of restructuring modal/aspectual vs. andative verbs is fixed, and appears to be independent of the morphosyntactic environment (i.e., independent of whether or not the andative appear in a Full-Fledged or Reduced MVC):

\ea \label{fn32ex}
    \ea \label{fn32exa}\gll sta bbau   ku ntʃiɲɲu   ku llu  fattsu\\
STAY GO-PRS.1SG ku  begin do-PRS.1SG  ku it.CL do-PRS.1SG\\
\glt *ntʃiɲɲu ku bbau ku llu fattsu
    \ex \label{fn32exb}\gll lu  sta  bba   ntʃiɲɲu      a   ffare\\
it.CL  STAY GO-PRS.1SG begin- PRS.1SG  a   do-INF\\
\glt *lu ntʃiɲɲu a ʃʃa ffare/*lu ntʃiɲɲu a bba ffare/ *lu sta ntʃiɲɲu a ʃʃa ffare/ *Lu  ntʃiɲɲu a sta ʃʃa ffare
    \z
\z

}

This can be interpreted as a general property of bleached roots in Salentino to be added to the VIs in (\ref{ac30}) insofar as also progressive sta behaves in the same way. \footnote{As also observed by \cite{ledgeway2016a} for the variety of Lecce, a reduced MVC and a clausal one can co-occur in one and the same sentence (here adapted for the Campiota variety).

\ea \label{fn33ex}
    \ea \label{fn33exa}\gll simu      ʃuti        ku   llu     ʃʃa {} kkattamu\\
  simu      ʃuti       ku   llu    ʃʃa  bba kkattamu\\
be.PRS.1PL go-PTCP.M.PL  ku   it.CL.    GO  buy-PRS.1PL\\
\glt  ‘we went to buy it’
    \ex \label{fn33exb}\gll sta   bbau     ku  me   ʃʃa bba kkurku      subbra lu tivanu\\
    STAY go-PRS.1SG ku  self.cl  GO- go.to.bed-PRS.1SG on  the sofa.bed\\
    \glt ‘I am going to go to be on the sofa bed’
    \z
\z

\textcolor{red}{ugly}

}

\ea\label{ac68}
The exponents of bleached roots are antitropal
\z

\subsection{The progressive}

The same analysis can be proposed for the progressive.  The root for STAY may be fully lexical or bleached. One can assume the basic Full-Fledged structure in (\ref{ac69}) in the case of the lexical root.  This structure is associated with the inceptive meaning.

\ea \label{ac69}...[$_{AspP}$ Asp [$_{vP}$ [STAY]$_{Root}$ [$_{CP}$ C [$_{TP}$ ... [...V$_2$...]$_{T^0}$….]]]]
\z

In the case of the bleached [+progressive] STAY root, the same stripping operations discussed above for the Reduced MVC generates the Progressive construction in (\ref{ac70b}):

\ea \label{ac70}
    \ea \label{ac70a}...[$_{AspP}$ Asp [$_{vP}$ [STAY$_{[+prog]}$]$_{Root}$ [$_{CP}$ C [$_{TP}$ ... [...V$_2$...]$_{T^0}$….]]]]
    \ex \label{ac70b}...[$_{AspP}$ Asp \sout{[$_{vP}$} [STAY$_{[+prog]}$]$_{Root}$ \sout{[$_{CP}$ C [$_{TP}$ ...} [...V$_2$...]$_{T^0}$….]]]]
    \z
\z

The application of cyclic head movement to the projection of the lower verb generates the structure in \figref{ac72} for the sentence in (\ref{ac71}) which also contains a bleached andative form (Also the exponent /sta/ must be marked as being antitropal, i.e. a prefix because of (\ref{ac68}). 

\ea \label{ac71}
    \gll ne     sta    ʃʃa  kkurkavamu\\
   self.CL STAY GO  go.to.bed-IPF-1PL\\
\z

\begin{figure}
  \caption{\label{ac72}\color{red}Please provide a caption}
  \begin{forest}
  	[T$^0$
  		[T$^0$
	  		[$\surd{} + \text{prog}$
	  			[$\surd{} + \text{prog}$
	  				[$\surd{} + \text{prog}$\\/\textit{st}-/]
	  				[TV\\/-\textit{a}-/]
	  			]
	  			[And$^0$
	  				[And$^0$
	  					[And$^0$\\/\textit{ʃʃ}-/]
	  					[TV\\/-\textit{a}-/]
	  				]
	  				[v$^0$
	  					[$\surd\text{Root}_i$ [/\textit{kurk}-/,tier=word]]
	  					[v$^0$
	  						[TV [/-\textit{a}-/,tier=word]]
	  					]
	  				]
	  			]
	  		]
	  		[T$^0$
	  			[T$^0$\\{[−perf, +past]} [/-\textit{v}-/,tier=word]]
	  			[TV [/-\textit{a}-/,tier=word]]
	  		]
  		]
  		[\textsc{agr 1pl} [/-\textit{mu}/,tier=word]]
  	]
  \end{forest}
\end{figure}

\subsection{Periphrastic constructions}

I now turn to the derivation of the sentence in (\ref{ac73}) where a periphrastic construction with an auxiliary and a participle is present:

\ea\label{ac73}
\gll l’    iti         ʃʃa   kkattatu\\
   it-CL  have-Pres.2PL    GO  buy-PTPL-MSG\\
\z

One needs to explain why the piece \textit{ʃʃa/bba} behaves morphologically as an affix thus appearing attached  to the participle and lower than the auxiliary in a structure.

In the preceding pages, I assumed that verbal synthetic forms are due to the cyclic application of head movement which is able to convert the extended functional projection of a verb into a single complex X$^0$ (i.e., a single word involving a root plus affixes). If this is correct, one can also plausibly assume that, in contrast, periphrastic verbal forms  --  in which similar verbal extended functional projections are broken into different complex X$^0$ (i.e. different words: auxiliaries and other verbal morphological pieces such as participles, gerunds and infinitives)  --  are due to the failure of the application of this operation to certain functional heads. In fact, this approach to periphrasis formation, which was at first formulated in \cite{embick2000a}, has been more recently fully developed by \cite{bjorkman2011a, pietraszko2016a, fenger2020a, calabrese2019a}.\footnote{An obvious advantage of such approaches over purely lexical ones that assume that periphrastic formation is just due to paradigmatic gaps (see \cite{kiparsky2004a} for example) is that the periphrastic structure, and the subsequent formation of auxiliaries, follows the hierarchial functional structure: it is expected that when there is a higher and lower head, the lower head will end up on the verb, whereas the higher head ends up on the auxiliary.}  In Bjorkman and Pietrasko’s works, the failure of functional heads to combine with the verb is due to the action of certain nodes (or better the feature complexes of those nodes) as interveners (\cite{rizzi1990a}) in syntactic processes  --  such as Agree\footnote{In Bjorkman’s system this is done via a version of Agree (\cite{chomsky2000a, chomsky2001}), namely Upward Agree (see \cite{merchant2011a}, a.o.); in Pietraszko’s system this happens through a type of selection, similar to cyclic agree (\cite{bejar2009a}).}   --  that lead to head movement. For example, the v-complex may not raise to Tense because (marked) Aspect features intervene for the Tense feature to be agreed with and checked. In Calabrese’s model, in contrast, the failure of head movement is formalized in terms of morphological filters disallowing combinations of functional head features: movement is blocked if such combination may be generated. Fenger proposes that head movement may be blocked by phasal boundaries such as that between the verbal thematic complex which includes Aspect and the higher T-C complex (see \cite{bo2014a}, \cite{wurmbrand2017verb})  --  some form of phasal extension would be required to account for the cases where movement crosses these boundaries.  A thorough discussion, comparison and selection among these different theories is far beyond the goals of this paper.  What matters here is that periphrasis is the result of blocking of head movement. A simple way of implementing this, without taking a stand with respect to the above mentioned theories, is to propose that head movement\footnote{Here we are dealing only with head raising. The same blocking could also occur with head lowering, which is not considered here.}  from one head position in the extended functional verb projection to the one directly higher up may be parametrized with a parameter allowing or not allowing movement between these positions. If movement to the higher up position is blocked, the complex X$^0$ head that was cyclically constructed up to that point remains stuck in the lower position. This leads to a periphrastic formation in which the extended functional projection is split in at least two X$^0$ complexes (i.e., in two words): a lower one, i.e, blocked X$^0$ complex, and a higher one including the higher functional heads of the projection. The head movement parameters may have their deeper grounds in the theories mentioned above but choosing what they are will not be an issue here.
 
Consider the derivation of the periphrastic construction in (\ref{ac74}):

\ea \label{ac74}\gll siti    ʃuti        a   kkasa\\
   be-PRS-2PL  go-PTCP-MPL  to  home\\
   \glt ‘We went home’
\z

As proposed in the works quoted above, it is derived by blocking head movement of the lower complex with Asp$^0$ to the higher T$^0$ node.\footnote{Lowering of the higher T$^0$ onto the lower Asp$^0$ complex must be prevented. A detailed discussion of how periphrastic verbal constructions are derived in Italian, is unfortunately not possible here due to space restrictions; the reader is referred to \cite{calabrese2019a} for this.} It follows that the higher T$^0$ is in violation of (50).  A dummy root -- the AUX root -- is therefore inserted as a “holder” for T$^0$ (\cite{bjorkman2011a} Given the analysis just proposed, the participle is essentially a tenseless, moodless verbal Asp$^0$ constituent (see \cite{calabrese2020a} on the derivation of perfect participle forms in Italo-Romance and Latin).  
 In order to understand the morphological properties, of the constituents of periphrastic constructions, we also need to look into their agreement patterns.  As proposed in \cite{calabrese2019a}, an important feature of all verbal complexes X$^0$ is that they are assigned an AGR node which is adjectival in participle forms --  analyzed as complex Asp$^0$ heads in \cite{calabrese2019a}, following \cite{embick2000a, embick2004a} --  but is otherwise verbal, where verbal AGR$_v$ requires person and number features, and adjectival AGR$_{Adj}$ requires gender and number features (and case features in languages with overt morphological case). The rule for AGR insertion proposed in that work is the following: \footnote{An important issue I cannot address fully here is that of the root-adjacent TV in auxiliaries.  Given that v$^0$ should not be present in the AUX constituent the relevant TV should not be there.  Many Italian dialects indeed do not have it: consider the  Lexical verb /AUX counterpart’ in the case of HAVE in Sicilian: \textit{av-i-ti/a-ti}, and  in Neapolitan: \textit{av-i-te/a-te} ‘have-PRS-2PL where we have the structures [[[[av-]$_{Root}$ [-e]$_{TV}$]$_{v^0}$- te]$_{[1pl-AGR+T^0}$  vs. [[[[av-]$_{Root}$ ]v$^0$- te]$_{[1pl-AGR+T^0}$ . However, in standard Italian such a distinction is absent: a thematic vowel is present when avere occurs as a main verb but also when aver is an auxiliary forms in (\ref{fn38exb})
 
 \ea \label{fn38ex}
    \ea \label{fn38exa}avete una bella casa (av-e-te)\\
   ‘you have a beautiful home’
    \ex \label{fn38exb}avete mangiato (av-e-te)\\
   ‘you have eaten’
    \z
 \z
 
To account for what happens in this case, \cite{calabrese2019a, calabrese2020a} proposes  that this is an instance of a morphological condition. Morphological conditions, according to him, may introduce ornamental nodes such as Thematic Vowels but also what appears to be syntactically void functional heads. They are the ways in which the outcomes of analogical, or purely morphological, changes are integrated in the PF derivation, and the means by which abstract syntactic structures are converted into surface morpho-phonological forms where one finds pieces that do not have a true syntactic motivation. In the case of the auxiliaries, a morphological structure condition formally generalizes verb structure to AUX --  a purely morphological change  --  by inserting a syntactically void TV. However, it is unclear if a TV is present in the auxiliaries avire and essere in Campiota. Here I will assume that there no TV in these auxiliaries in this variety.  Note that an Italian restructuring auxiliary such as andare ‘GO’ discussed below displays it as expected. and that Campiota restructuring auxiliaries such as putire, spittare, etc. also display it as in Italian (cf. \figref{ac93}).}

\ea \label{ac75} Adjoin AGR to the highest X$^0$ of a Complex X$^0$ included in the extended projection of v$^0$
    \ea \label{ac75a}where AGR is adjectival if the highest X$^0$ is Asp$^0$\\
        Otherwise:
    \ex \label{ac75b}verbal.
   \z 
\z

We can now derive the surface forms. Blocking and Aux-insertion are shown in (\ref{ac76a}).  The outcome with further operations of AGR insertion, pruning, etc. is shown in (\ref{ac76b}).  note that the lower Asp$^0$ complex is assigned adjectival AGR$_{Adj}$ and therefore displays participial morphology (see \cite{calabrese2020a} for further discussion of participial morphology in Italian and Latin, and \cite{calabrese2019a} for discussion of the structure of the auxiliary form):\footnote{The exponent of [+perf] Asp$^0$, when it is the topmost functional node in a X$^0$-complex -- i.e., in a past participle --  is /-t-/.  The v$^0$-TV in this case is /-u-/.} 


\ea \label{ac76}
\textcolor{red}{missing 76}
    \ea\label{ac76a}
    \ex \label{ac76b}
    \z

\z

Let us turn to the sentence in (\ref{ac73}) l’iti ʃʃa kkattatu, which includes the bleached andative GO.  The blocking of upward head movement of the complex X$^0$ including this element is shown in \figref{ac77}.


\begin{figure}
\caption{\label{ac77}\color{red}Please provide a caption}
  \begin{forest}
    [TP,name=tp
      [T\textsuperscript{0}\\{[−past]},name=t0]
      [AspP
        [Asp\textsuperscript{0},name=asp
          [$\surd$
            [GO{[+and]}]
            [$\text{v}^0_i$
              [$\surd{}\text{Root}^0_l$]
              [$\text{v}^0_i$]
            ]
          ]
          [Asp\textsuperscript{0}\\{[+perf]}]
        ]
        [$\surd{}$P
            [t$_i$,name=ti]
            [vP
              [t$_i$,name=ti2]
              [$\surd$p
                [t$_l$,name=tl]
                [\dots]
              ]
            ]
        ]
      ]
    ]
  \draw[-{Triangle[]}] (tl)  to[in=0,out=180] (ti2);
  \draw[-{Triangle[]}] (ti2) to[in=0,out=180] (ti);
  \draw[-{Triangle[]}] (ti)  to[in=0,out=180] (asp);
  \path let \p1=(tp), \p2=(t0) in coordinate (i1) at (\x1,\y2);
  \path let \p1=(t0), \p2=(asp) in coordinate (i2) at (\x1,\y2);
  \draw [double] (i1) -- (i2);
  \end{forest}
\end{figure}

AUX insertion in the higher T$^0$, AGR insertion in the lower and higher verbal X-complexes and the relevant TV insertion, subsequent VI insertion followed by the other relevant operations, all applying cyclically, generate \figref{ac78}. 

\begin{figure}
\caption{\label{ac78}\color{red}Please provide a caption}
\begin{forest}
[TP
  [T$^0$,name=T0
    [T$^0$
      [Root\textsubscript{\textsc{aux}}\\/i-/]
      [T$^0$\\{[−past]}\\∅,edge label={node[midway,rotate=-33]{||}},name=t0]
    ]
    [\textsc{agr}\\\textsc{1pl}\\/-ti/,name=ti]
  ]
  [AspP
    [Asp$^0$,name=ASP
      [Asp$^0$
        [$\surd$GO{[+and]},name=GO
          [GO{[+and]}
            [GO{[+and]}\\/\textit{ʃʃ}-/]
            [TV\\/-a-/]
          ]
          [v$^0$,name=V0
            [$\surd\text{Root}_i$ [/katt-/,tier=word,anchor=south]]
            [v$^0$
              [v$^0$\\/∅/,name=empty,tier=word,edge label={node[midway,rotate=33]{||}},anchor=south]
              [TV\\/-a-/,name=tva,tier=word,anchor=south]
            ]
          ]
        ]
        [Asp$^0$\\{[+perf]} [/-t-/,tier=word,anchor=south]]
      ]
      [AGRAdj\\MscSg [/-u/,tier=word,anchor=south]]
    ]
  ]
]
\draw [-{Triangle[]}] (empty) to[in=270,out=270] (tva);
\draw [-{Triangle[]}] (t0) to[in=270,out=345] (ti);
\foreach \i in {T0,ASP,GO,V0}
{\draw [dashed] (\i.south west) |- (\i.north) -| (\i.south east);}
\end{forest}\\
\relax [\textit{iti ʃʃa kkattatu}]
\end{figure}

The structure in \figref{ac78} accounts for the properties of this construction. Firstly, the andative head belongs to the lower Asp$^0$ complex which is converted into a participle through application of (\ref{ac75a}); therefore, after vocabulary insertion, the andative piece ʃʃa attaches to this participle as a prefix and therefore appears below the auxiliary. It also follows that the participial morphology that in the full-fledged MVC appears on the higher andative verb appears instead on the lower verb in the reduced one since this verb is part of lower Asp$^0$ complex.  At the same time, the T$^0$ (and Mood$^0$) features that appeared on the higher andative verb are now assigned to the auxiliary. Finally, since T$^0$ and the inserted auxiliary belongs to the same extended projection of the lower v$^0$, auxiliary selection will be sensitive to the properties the latter. This also explains why the auxiliary is obligatorily selected by the lower verb.

\subsection{Reduplication}

An account of the properties of reduplication requires assuming that it applies before VI insertion insofar as the reduplicant is not sensitive to the phonological properties of the base:

\ea \label{ac79}\gll ʃʃa  bba  kkurka te\\
    GO. go.to.bed-Imperative.2SG CL2SG\\
   \glt ‘go to bed’
\z

The most adequate way of accounting for this type of reduplication is by means of a fission-like operation (\cite{calabrese1988a, noyer1992a, arregi2012a, calabrese2014a}): 


\ea\label{ac80}
\textcolor{red}{missing 80}
\z

The surface outcomes of the reduplication are governed by an OCP-like constraint that blocks the sequence of identical exponents *ʃʃ-a ʃʃ-a, (see \cite{pescarini2010a} on the role of a similar constraint in clitic clusters).  I assume that the higher position can be assigned only the default exponent / ʃʃ-/. If the base contains / bb-/, there is no problem and / ʃʃ -/ is inserted.  If the base contains / ʃʃ-/, instead, there is a violation of the relevant OCP constraint.  Two repairs are possible as shown in (\ref{ac80}), before TV insertion): 1) the entire inserted GO node can be deleted which simply results in the absence of reduplication (i.e., a case of obliteration (cf. \cite{arregi2012a}); 2) only the terminal element of the inserted node is deleted, which results in the insertion of the alternative GO exponent /B-/ (cf. \cite{pescarini2010a} for this type of repairs) ([bb-] after application of the rules (\ref{ac32}) and (\ref{ac28}):


\ea
\textcolor{red}{missing 81}
\z

\subsection{Full-Fledged MVC and infinitival MVC constructions in Italo-Romance}

In most Western Romance varieties, when verbs such as GO (and COME) feature in MVCs, they are typically followed by an infinitive. We will refer to this construction as the Infinitival MVC. Here is an example from the Apulian dialect spoken in Bari (\ref{ac82}).  But the same structures are found in standard Italian (\ref{ac83}): 

\ea \label{ac82}\gll Mə  vògg’    a  ’ccattà  u  cappìddə  névə.   \hfill (Bari, Apulia)\\
   me  go.PRS.1SG to  buy.INF  the  hat    new {}\\
   \glt ‘I go buy a new hat.’ \hfill (Andriani 2017: 231)
\z

\ea \label{ac83}\gll Andava      a     mangiarlo      \hfill (Italian)\\
     go.IMPF.3SG   to    eat.INF=it {}\\
    \glt ‘I went to eat it.’ 
\z

An analysis of the Full-Fledged MVC requires an analysis of these Infinitival MVC constructions. Let us thus turn to an Infinitival MVC constructions without restructuring in Italian like that in (\ref{ac84}).  The sentence in (\ref{ac84}) has the basic syntactic structure in (\ref{ac85}) in the model developed here.  If there is no restructuring (e.g., no clitic climbing), the verb GO selects a full CP, an instance of a purpose clause. No clitic climbing can occur in this case and the verb GO is a full lexical verb that can select argumental structure (cf. (\ref{ac84}))

\ea \label{ac84}\gll Andava   a casa   a    mangiarlo       \hfill (Italian)\\
    go.PRF.3SG to house  to   eat.INF=it.CL\\
    \glt ‘I went home to to eat it.’
\ex \label{ac85}...[$_{AspP}$ Asp [$_{vP}$ [GO]$_{Root}$ [$_{CP}$ C [$_{TP}$ ... [...V2...]$_{T^0}$...
\z

The sentence in (\ref{ac84}), however, can be restructured as shown by the clitic climbing and removal of argumental structure:

\ea \label{ac86}\gll Lo   andava    (*a casa) a mangiare  \hfill    (Italian)\\
    it.CL go.IMPF.3SG  to house to  eat-INF=it.CL\\
    \glt ‘I was going  (*home) to to eat it.’
\z

Along the lines of the analysis proposed earlier for Salentino, I assume that restructuring involves stripping the temporal and aspectual structure of the lower proposition, with subsequent integration of the restructuring root in the extended verbal projection of the lower verb, as in (\ref{ac70b}).  However, there is a fundamental difference between Salentino and the other Romance varieties in the case of andative MVCs. Whereas in Salentino, head movement merges the lower v$^0$ complex with the andative GO element -- whereby this becomes an affix -- this does not occur in restructuring infinitival MV constructions as shown in \figref{ac87}.  I assume that head movement is parametrically blocked here as in the periphrastic constructions discussed in §3.4 (cf. (\ref{ac66}) where head movement to Andative GO is allowed)

\begin{figure}
\caption{\label{ac87}\color{red}Please provide a caption}
\begin{forest} for tree = {fit=band}
    [,shape=coordinate[,phantom]
    [TP,edge=dashed
        [T$^0$\\{[+past]}]
        [AspP
          [Asp$^0$\\{[−perf]}]
          [$\surd$P
            [$\surd$ [GO{[+and]},name=go]]
            [vP
              [v$^0$,name=v0
                [$\surd\text{Root}^0_h$]
                [v$^0$]
              ]
              [$\surd$p
                [t$_h$,name=th]
                [\dots]
              ]
            ]
          ]
        ]
    ]
    ]
    \draw [-{Triangle[]}] (th) to[in=0,out=180] (v0);
    \draw [double] ($(go.north east) !.25! (v0.north west)$) -- ($(go.south east) !.25! (v0.south west)$);
    \draw [-{Triangle[]}] (v0) -- +(-3em,0pt);
\end{forest}
\end{figure}

The further derivational steps that eventually lead to the surface (\figref{ac87}) are discussed below, cf. Figures~\ref{ac88} and~\ref{ac93}.  Being a functional head, the motion verb is in violation of (\ref{ac50}).  However, being also a root makes a difference.  Thus I propose that, in this case, it is licensed as is, and thus does not be adjoined to another root as requested by (\ref{ac50}) -- it thus becomes an auxiliary in itself -- and can therefore be the host of the higher functional heads, as shown in \figref{ac88}.  

\begin{figure}
\caption{\label{ac88}\color{red}Please provide a caption}
\begin{forest} for tree = {fit = band}
  [TP
    [T$^0$
      [$\text{Asp}^0_j$
        [$\surd_i$ [GO{[+and]}]]
        [$\text{Asp}^0_j$\\{[−perf]}]
      ]
    [T$^0$\\{[+past]},name=t0]
    ]
  [AspP
    [t$_j$,name=tj]
    [$\surd$P,name=surdp
      [t$_i$,name=ti]
      [vP
        [v$^0$,name=v0
          [$\surd\text{Root}^0_h$]
          [v$^0$]
        ]
        [$\surd$p
          [t$_h$,name=th]
          [\dots]
        ]
      ]
    ]
  ]
  ]
\draw [-{Triangle[]}] (th) to[in=0,out=180] (v0);
\draw [-{Triangle[]}] (ti) to[in=0,out=180] (tj);
\draw [-{Triangle[]}] (tj) -- ++(-2em,0);
\path let \p1=(surdp), \p2=(ti) in coordinate (i1) at (\x1,\y2);
\path let \p1=(ti), \p2=(v0) in coordinate (i2) at (\x1,\y2);
\draw [double] (i1) -- (i2);
\draw [-{Triangle[]}] (v0) to [out=180,in=315] ($(i1) !0.5! (i2)$);
\end{forest}
\end{figure}

Now, insofar as the entire complex is a single extended functional projection, therefore a single clausal structure, clitic climbing to a hight clitic landing site is allowed, as in (\ref{ac86}). 

There is still an issue that needs to be addressed here, though, in order to account for how (\ref{ac86}) is converted to the surface MVC in (\ref{ac84}): specifically, we need to understand why the lower verbal X$^0$ -complex in \figref{ac88} is characterized by infinitival morphology. The issue is the morphological nature of the infinitive. Now, the infinitive, with the gerund, is, by definition, the “uninflected” verbal form and occurs in a wide variety of embedded constructions, as observed by \cite{wurmbrand2014a}.  Thus, an infinitive can appear in an embedded full clause as [CP [TP/FutP [AspP [vP [VP ]]], but also in a restructured embedded constituent one as [vP [VP ]].  In addition, Wurmbrand observes that an infinitive occurs in embedded future clauses  [TP/FutP [AspP [vP [VP ]]]. Importantly, for all these constructions, \cite{wurmbrand2014a} also showed that the different temporal properties of the infinitive do not correlate with a difference between control and ECM/raising. It follows that there is no syntactic functional verbal element, or other syntactic property, that can account for the surface distribution of the infinitive. Here I propose that this distribution can be readily determined in the morphological component. Note, in particular, that in all of the infinitival constituents mentioned above, we are dealing with independent morphological words, specifically verbal complex X$^0$s. In \cite{calabrese1993a}, it is proposed that the infinitive is the morphological realization of the  AGR$_v$ and that it is therefore sensitive to  AGR$_v$ features. On the one hand, the  AGR$_v$ properties of inflected verbal forms are associated with the feature [-anaphoric], which triggers explicit morphological marking of phi-features.  Otherwise, the  AGR$_v$ lacks explicit marking of phi-features, and can co-occur with anaphorically-bound PRO subjects, with overt NPs, and with subjectless structures. In this case we have the infinitive. This then means that the infinitive is the default elsewhere realization of AGR$_v$:

\ea\label{ac89}
    \ea \label{ac89a}\{φ$_1$, φ$_2$, φ$_3$, etc.\}$\longleftrightarrow$[-anaphoric  AGR$_v$, Phi-features, etc. ]$_{AGR_v}$
       (where φ$_1$, φ$_2$, φ$_3$, etc. are exponents of AGR in inflected V forms, such as /-u/, /-i/ etc.)
    \ex \label{ac89b}/-re/ $\longleftrightarrow$  [   ]$_{AGR_v}$\\
     (Infinitive)
    \z
\z

The distribution of infinitives can be captured if one assumes that the presence of [-anaphoric] AGR is associated with the presence of a deictic, i.e., [-anaphoric], tense, as stated in (\ref{ac90}). So, the infinitive occurs as a default when Tense is non-deictic, i.e., anaphorically dependent on the Tense of the matrix verb and the subject anaphorically bound (i.e., [+anaphoric AGR] (see the analysis of EQUI-clauses in \cite{calabrese1993a}), or when Tense is simply missing as in the future infinitives or in constructions with restructuring:

\ea \label{ac90} {[}-anaphoric{]}$_{\text{T}^0}$ $\rightarrow$ [-anaphoric]\textsubscript{AGR}
\z

Infinitives therefore have a morphosyntactic structure such as that in (\ref{ac91}), where X$^0$ is the highest non-Asp$^0$ functional head.  Assuming that this head is non-overt in this context, it will be pruned and therefore fused into a single node with the higher AGR as  in (\ref{ac91}). 

\ea\label{ac91}
\begin{forest} for tree = {fit=band}
[X$^0$
  [v$^0$
    [Root [/mangi-/,tier=word]]
    [TV [/-a-/,tier=word]]
  ]
  [$\text{X}^0 + \textsc{agr}_v${[$\notin$-anaphoric]}
          [/-re/,tier=word]
  ]
]
\end{forest}
\z

We can therefore have the derivation in \figref{ac93} for the surface form andava a mangiare in (\ref{ac86})\footnote{The TV adjacent to the andative root is inserted as discussed in footnote 40.} where, as proposed in \cite{cruschina2021a} the connecting preposition (the linker) is inserted by the rule in (\ref{ac92}) as an instance of ornamental morphology and is therefore devoid of any syntactic and semantic content. 

\ea \label{ac92}
\begin{forest}
 [XP,name=xp [Linker] [XP]]
 \node[left=of xp.base west,anchor=base] {X$^0$ →};
\end{forest}
\z

\begin{figure}
\caption{\label{ac93}\color{red}Please provide a caption}
\begin{forest}
[TP
	[T$^0$
	  [T$^0$
	    [$\text{Asp}^0_j$
	      [GO{[+and]$_j$}
	        [GO{[+and]$_j$}\\/and-/]
	        [TV\\/-a-/]
	      ]
	      [$\text{Asp}^0_j$\\{[+perf]}\\∅,edge label={node[midway,rotate=-33]{||}},name=asp]
	    ]
	    [T$^0$
	      [T\\{[+past]}\\/-v-/,name=tpast]
	      [TV\\/-a-/]
	    ]
	  ]
	  [AGR\\3SG\\∅]
	]
	[AspP
	  [t$_j$]
	  [$\surd$P
	    [t$_i$]
	    [vP
	      [Linker\\/a/]
	      [vP
	        [v$^0$
	          [v$^0$
	            [$\surd\text{Root}^0_h$\\mangi-]
	            [v$^0$
	              [v$^0$\\∅,edge label={node[midway,rotate=33]{||}},name=v0]
	              [TV\\/-a-/,name=tva]
	            ]
	          ]
	          [\textsc{agr}$_v$\\/-re/]  
	        ]
	        [$\surd$p
	         [\phantom{xyz}] [\phantom{xyz}]
	        ]
	      ]
	    ]
	  ]
	]
]
\draw[-{Triangle[]}] (asp.south) to[out=270,in=270] (tpast.south);
\draw[-{Triangle[]}] (v0.south) to[out=270,in=270] (tva.south);
\end{forest}
\end{figure}

We can turn back to Salentino at this point. Whereas the Salentino counterpart of Romance restructured infinitival MVCs is a Reduced MVC, the Salentino counterpart of non-restructured one, is a Full-Fledged MVC involving a ku-clause.  As proposed in \cite{calabrese1993a}, this is due to the fact that [+anaphoric]$_{T^0}$ is not possible in this language (See \cite{calabrese1993a} for an account).  Thus, in presence of T$^0$, given (\ref{ac90}), AGR will always be [-anaphoric] thus disallowing the infinitival clause.

\ea\label{ac94}
 ... [$_{AspP}$ Asp$^0$ [$_{vP}$ [GO]$_{Root}$ [$_{CP}$ C [$_{TP}$ ... [ ... V$_2$... ]$_{T^0}$...
\z

\ea \label{ac95} \gll  ʃivi  ku  llu kkattu ( cf. andai a mangiarlo)\\
  go-PRF-lsg  ku  it.CL buy-lPRSlsg. \\
 \glt ‘I went to buy it '
\z

\subsection{Infinitival forms in Campiota}

Before turning to the Doubly Inflected MVCs of other southern Italian dialects, I need to discuss restructuring verbs that take infinitival complements in Campiota. In fact, as observed in §2.2, infinitives are indeed possible in the complements of restructuring verbs such as modal or aspectual or ones: \textit{must},\textit{ be  able},  \textit{begin},  \textit{finish},  \textit{continue},  \textit{stay},  \textit{try}, etc. cf. (\ref{ac96}). Given that stripping of tense and aspectual structure occurs in these cases according to the analysis developed above, the same basic structure of restructured infinitival clauses proposed above for the MVC in Romance is found here, i.e., the structure derived in \figref{ac93}. Insofar as this is the same structure of the Reduced MVC, we must account for why infinitive-taking restructuring verbs do not behave like GO and STAY. My proposal here is to extend to these cases the analysis just proposed for the infinitival MVC in Romance: Only GO and STAY can undergo merging  by  head  movement  with  the  lower  v$^0$.  Instead, all other restructuring verb roots are parametrically prevented from undergoing that operation, and therefore cannot be merged with the lower v$^0$ complex via head-movement. This results in structures similar to that of the andative GO in ltalian in \figref{ac93}. Crucially in the this structure, T$^0$ is not present in the lower piece; (\ref{ac90}) will therefore not apply and an unspecified AGR$_v$ will be inserted.  Given (\ref{ac89b}) this results in the insertion of infinitive exponence.

\ea \label{ac96}
    \ea \gll lu      pottsu     kkattare\\
    it.CL can-PRS.1SG buy-INF\\
    \glt `I can buy it'
    \ex \gll m’    addʒu      kurkare\\
    self.CL  must-PRS.1SG   go.to.bed-INF \\
    \glt `I must go.to.bed'
    \ex \gll lu ntʃiɲɲa       a  ffare\\
    it.CL begin-PRS.3SG a  do-INF \\
    \glt `I begin to do it'
    \z
\z

\subsection{Doubly Inflected Construction}

In addition to the Salentino Reduced and Full-fledged MVC, and common Romance Infinitival MVCs, another option for motion verb constructions is found in southern Italian varieties. Following \cite{cruschina2013a}, I use the name Doubly lnflected Construction (DIC) for this other kind of MVCs, where the two verbs, usually connected by the linker ‘a’,\footnote{See \cite{calabrese2021a} for further discussion of this linker.}  act as a single predicate and share the very same inflectional features. Example of DICs are provided below, where both the higher motion verb and the lower one are in the 1st person singular (\ref{ac97}), and in the 3rd person singular (\ref{ac98}), of the present indicative:\footnote{The motion verbs that most typically appear in DIC are the local equivalents of go, come, come by/ pass and send. Other verbs may enter the construction as V1 is some dialects. See \cite{caro2018a, caro2019a} for a review of the additional motion verbs that can occur in DIC in different Sicilian varieties. On the special properties of send as V1, which involves both a motion and a causative semantics, see \cite{todaro2018a} and \cite{prete2020a}.}

\ea \label{ac97}
\gll Vaju    a   pigghiu    u   pani.    (Marsala, Sicily)\\
   go.PRS.1SG to   take.PRS.1SG  the  bread \\
 \glt  ‘I go to fetch the bread.’ \hfill (Cardinaletti \& Giusti 2001: 373)
\z

\ea \label{ac98}\gll  U   veni      a  piglia   dopu.      (Mussomeli, Sicily)\\
   him  come.PRS.3SG  to  collect.PR.3SG later\\
\glt   ‘He is coming to pick him up later.’ \hfill  (Cruschina 2013: 266)
\z

Varieties displaying DICs always also have their infinitival counterparts:

\ea \label{ac99}\gll Vaju    a  piggjari u     pani.    (Marsala, Sicily)\\
go.PRS.1SG to  take.INF the    bread \\
\z
   
\ea \label{ac100}\gll U   veni      a   piʎʎari   dopu.   (Mussomeli, Sicily)\\
him  come.PRS.3SG to   collect.INF later\\
\z

In their analysis, \cite{cardinaletti2001a} compare DICs with the Infinitival MVCs and, on the basis of a number of syntactic and semantic tests (see section 2.1), convincingly show that DICs (the inflected construction, in their terminology) are monoclausal. It follows that the DICs correspond to the Reduced MVCs, and the Infinitival MVCs to the Campiota Full-Fledged ones.

DICs are in fact restructuring configurations in which the higher motion verb behaves as a functional head. This can account for the different properties of DICs with respect to the Infinitival MVC first examined in \cite{cardinaletti2001a}, including obligatory clitic climbing, single event interpretation, indivisibility, and incompatibility with the arguments and adjuncts typically associated with motion verbs (see \cite{cardinaletti2001a, cardinaletti2003a, manzini2005a, cruschina2013a, caro2019a}). 

As proposed in \cite{cruschina2021a}, double inflection arises independently of restructuring. What is special about this set of constructions is the presence of agreement within the extended vP. In other words, DICs involve the assignment of explicit pronominal agreement features to the lower verbal X$^0$-complex. I have already postulated the presence of an  AGR$_v$ element in this constituent: it is introduced by the rule in (\ref{ac75}).  As postulated earlier, this AGR$_v$ is usually assigned the feature [+anaphoric], or left unspecified, and is hence realized as an infinitive, since this constituent lacks a deictic (non-anaphoric) Tense (or lacks this node entirely) (see (\ref{ac90}). If we assume this, then the main feature that characterizes DIC is the fact that the lower AGR$_v$ is actually assigned the feature [-anaphoric]. DICs thus display special morphological behaviour  --  a [-anaphoric] AGR$_v$ in the lower X$^0$-complex of the structure in \figref{ac93}, that is, the rule in (\ref{ac101}) which is characteristic of these dialects. I assume that the rule in (\ref{ac101}) applies cyclically when the lower complex has been constructed but the GO$_{[+and]}$ has not moved upwards yet:
 
\ea \label{ac101}Ø $\rightarrow$ [-anaphoric] /  GO$_{[+and]}$ [[ \underline{\hspace{3em}} ]$_{ AGR_v}$ ]$_{v^0}$
\z

\section{Conclusions} 
 
In this paper, I attempted to capture the syntactic and morphological properties and processes that account for Full-Fledged and Reduced MVC in Campiota, the Salentino variety of Campi Salentina. I showed that the Reduced MVCs are monoclausal and that the Full-Fledged ones necessarily biclausal. It follows that same motion verb root displays a lexical use and an affixal one:  in its lexical use it may select argument structure and a full clause; when used as an affix, it is part of the full extended projection of the lower verb, and has special morphological behavior: it can be reduplicated and is attached to the participle in participial compound tenses. I argued that the relation between the lexical verb GO and its bleached counterpart in Campiota MVCs is better understood if semantic bleaching may trigger Syntactic Truncation  in terms of \citealt{wurmbrand2014a, wurmbrand2015, wurmbrand2017verb}, in which the higher motion verb selects a vP constituent and therefore all of the projections of the lower verb are prevented from being projected. 
 
I also investigated the characteristic properties of MVCs in other Italo-Romance varieties: restructured and non-restructured Infinitival MVC and the MVC with double inflection (DIC) and showed how they correlate to the Full-Fledged and Reduced MVC in Campiota.  Restructured MVCs and DIC can be simply analysed in terms of Wurmbrand's Syntactic Truncation, while the non-restructured infinitival MVCs correspond to the Salentino Full-Fledged MVCs.  I proposed that the infinitive is the default morphological realization of AGR (which occurs when AGR is [+anaphoric], or T$^0$ missing).  This accounts for infinitival MVC.  DIC arises from an identical structure in which AGR is assigned the feature [-anaphoric], thus agreeing with V1 in person and number. 

\section*{Abbreviation}
\begin{multicols}{2}
\begin{tabbing}
{\sc imper}\hspace{.5ex} \= Imperative \kill
{\sc cl} \> Clitic \\
{\sc imper} \> Imperative \\
{\sc ipf} \> Imperfect\\
{\sc prf} \> Perfect\\
{\sc ptcp} \> Participle\\
{\sc tv} \> Thematic Vowel\\
\end{tabbing}
\end{multicols}

\section*{Acknowledgement}

I thank Giampaolo Calabrese, Serena Palazzo, Giulia Calabrese, and especially Maria Luisa e Maria Teresa Rapanà for their help with Campiota judgments and forms. The article is also based on a detailed investigation of the constructions with the motion verb GO as used in Tatiana Todaro, Cunti e storie. La tradizione orale a Campi Salentina. Regione Puglia. 1989, a collection of texts in the Campiota vernacular. I am also grateful to Guglielmo Cinque, Silvio Cruschina, and two anonymous reviewers for insightful comments and suggestions that enormously improved the first draft of this paper. I dedicate this article to Susi Wurmbrand whose fundamental work on restructuring verbs has inspired my research on these matters.

{\sloppy
    \printbibliography[heading=subbibliography,notkeyword=this]
}
%\printbibliography[heading=subbibliography,notkeyword=this]
\end{document}
