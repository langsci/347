\documentclass[output=paper,colorlinks,citecolor=brown,
% hidelinks,
% showindex
]{langscibook}

\newcommand{\citeapos}[1]{\citeauthor{#1}'s \citeyearpar{#1}}
\newcommand{\laszAsp}[0]{[asp]}
\newcommand{\laszLoc}[1]{[loc:#1]}
\newcommand{\laszDir}[2]{[dir:#1$\to$#2]}
\newcommand{\laszHs}[1]{[hs:#1]}
\newcommand{\laszClaw}[0]{{\fontspec{aslfontgithubio.ttf} S}}
\newcommand{\laszPlain}[0]{[plain]}

\author{
    Sabine Laszakovits\affiliation{University of Connecticut; Austrian Academy of Sciences} and
    Ronice Müller de Quadros\affiliation{Universidade Federal de Santa Catarina} and 
    {Emily Jo} Noschese\affiliation{University of Hawai'i} and 
    Diane Lillo-Martin\affiliation{University of Connecticut; Haskins Laboratories}
}
\title{Object shift in ASL and Libras}
\abstract{
    ASL and Libras have an object-shift construction 
    by which the canonical SVO order is changed to SOV. 
    In both sign languages, this ordering is 
    mandatory for V marked with durative/continuative aspect (reduplicated movement), 
    optional for V that agrees with O in locus,
    and not allowed with plain V. 
    When V agrees with O in handshape, 
    ASL requires OV ordering whereas Libras allows both OV and VO ordering. 
    We present an analysis that derives these data 
    with a combination of syntactic movement of O 
    and violable, equally-ranked PF-constraints as proposed by \citet{BW.2012}. 
    Unlike \citeapos{Matsuoka.1997} and \citeapos{Braze.2004} proposals, 
    we do not move V to a head on the right in violation of the Final-over-final constraint
    \citep{BiberauerHR.2014,SheehanBRH.2017}.
}

\begin{document}
\maketitle

% ===================================================================
\section{Introduction}

The underlying word order in modern American Sign Language (ASL)
and Brazilian sign language (Língua Brasileira de Sinais, Libras) has
often been argued to be subject-verb-object (SVO; 
\citealp[see]{Fischer.1975,Liddell.1980,Padden.1988} for ASL; 
\citealp{Quadros.1999,Quadros.2003} for Libras), 
which we adopt here. However, in both languages, word order
variations are possible. In this paper, we focus on the construction with
the word order SOV, which has been termed ``object shift''. This
construction differs from topicalization structures like O'SV, VO'S,
S'VO, or S'O'V in that the latter contain a prosodic break after the
topic (indicated here by an apostrophe) and may contain nonmanual
marking (raised brows and a slight upward chin tilt; 
\citealp{Liddell.1977,Liddell.1980,Padden.1988}) during the utterance of the topic.%
\footnote{
    \citet{Fischer.1990} argues that object-shift constructions in ASL involve
    ``mini-topicalization'' of O to some intermediate specifier position. She corroborates
    this analysis by reporting a definiteness effect found in SOV orders, which is typical
    of topics. However, unlike typical topicalizations, SOV constructions lack a prosodic
    break after O and cannot mark O with nonmanual topicalization marking. Our
    analysis is similar to \citeauthor{Fischer.1990}'s in that we think O moves to some intermediate
    specifier position. However, we have not been able to reproduce this definiteness
    effect for ASL, and we see no reason to call this movement topicalization.
}
We restrict our attention to verbs that take a direct object (such as `buy X') and
possibly an indirect object (such as `send X to Y'), but do not discuss
verbs that denote movement to or from a location (such as `put X on Z').%
\footnote{
    Furthermore, we distinguish object-shift from locative constructions of the
    form location-subject-predicate, which have been taken as instances of the word
    order OSV because, like object-shift, they lack the prosodic break of topicalization
    structures \citep{Liddell.1980}. Locative constructions differ from object-shift
    constructions at least in the following properties: 
    \begin{inparaenum}[(i)]
        \item word order is OSV (vs. SOV),
        \item the predicate is restricted to `is located at' (vs. any verb), and 
        \item there is no minimally different derivation with the word order SVO (which there is for
    object-shift; see e.g.~\REF{lasz:ex:4}--\REF{lasz:ex:7} and throughout the paper
    \end{inparaenum}).
}

There are three main triggers for object-shift in ASL and
Libras: 
\begin{inparaenum}[1.]
    \item durative/continuative aspect on the verb, which makes object-shift obligatory; 
    \item verbal agreement in handshape with the object, which makes object-shift 
    optional in Libras and obligatory in ASL; and 
    \item verbal agreement in locus with the object, which makes object-shift optional in both languages. 
\end{inparaenum}
The effect of durative aspect on word order is illustrated in \REF{lasz:ex:1}--\REF{lasz:ex:2}.%
\footnote{
    Following standard practice in sign linguistics, signs are glossed using
    English words in all caps. Most of our examples represent sign sequences that are
    grammatical in both ASL and Libras, though the actual signs are of course different.
    `IX' is the gloss used for a pointing indexical sign which serves
    pronominal functions. `\laszAsp{}' indicates an aspectual marker involving reduplication
    of the sign root. `\laszHs{\_}' indicates the use of a classifier handshape. `\laszLoc{\_}'
    indicates that a sign is produced using a spatial locus other than default (`a', `b',
    etc. indicate distinct loci but not physical location). `\laszDir{\_}{\_}' indicates that a sign
    moves from one locus to another. \par 
    In the English translation, past tense is used although neither ASL nor
    Libras has grammatical tense marking; similarly, we often use `a/the' for nouns
    because in both of these sign languages either translation is possible.
}%
\footnote{
    When a source is not provided, the judgments primarily come from the
    2\textsuperscript{nd} and 3\textsuperscript{rd} co-authors, who are native signers of Libras and ASL respectively.
}

\ea
    \label{lasz:ex:1}
    \ea[]{
        \label{lasz:ex:1a}
        IX1 WINE DRINK\laszAsp{}. 
        \hfill 
        \cmark ASL, \cmark Libras \\ 
        `I drank wine continuously.'
    }
    \ex[*]{
        \label{lasz:ex:1b}
        IX1 DRINK\laszAsp{} WINE. 
        \hfill 
        *ASL, *Libras
    }
    \z 
\ex
    \label{lasz:ex:2}
    \ea[]{
        \label{lasz:ex:2a}
        MY SISTER LETTER SEND\laszAsp{}. 
        \hfill 
        \cmark ASL, \cmark Libras \\ 
        `My sister repeatedly sent a/the letter(s).'
    }
    \ex[*]{ 
        \label{lasz:ex:2b}
        MY SISTER SEND\laszAsp{} LETTER. 
        \hfill 
        *ASL, *Libras 
    }
    \z 
\z

Verbal agreement can target the object's noun-class%
\footnote{
    In this sense we follow \citeapos{Benedicto.Brentari.2004} proposal, whereby
    classifiers are treated as a type of agreement between V and O/S.
}, 
whereby V
changes its handshape (`hs') to a classifier as illustrated in~\REF{lasz:ex:3}, or it
can target the object's referent, whereby V changes the end-point of its
movement (`dir') to the object's locus (`loc'), as seen in~\REF{lasz:ex:4}.%
\footnote{
    \citet{Fischer.1975} proposed that object-shift is also licensed if the relation
    between V and O is established in the semantics, such that SOV is possible only if it
    is clear from the semantics of V, S, and O which argument is the agent and which is
    the patient. However, \citet{Liddell.1980} showed that this approach makes incorrect
    predictions, and that syntactic agreement is a better way to go.
}

\ea 
    \label{lasz:ex:3}
    The handshape of `GIVE' changes to the \laszClaw{} handshape indicative of handling an apple. 
    \ea[]{ 
        SALLY APPLE GIVE\laszHs{\laszClaw}. 
        \hfill 
        \cmark ASL, \cmark Libras 
    }
    \ex[]{ 
        SALLY GIVE\laszHs{\laszClaw} APPLE. 
        \hfill
        \cmark Libras 
    }
    \ex[*]{
        SALLY GIVE\laszHs{\laszClaw} APPLE. 
        \hfill 
        *ASL \\ 
        all: `Sally gave someone a/the apple.'
    }
    \z 
\ex
    \label{lasz:ex:4}
    The direction of `HELP' changes to the locus of `ANA'. 
    \ea 
        IX\laszLoc{a} MARIA\laszLoc{a} IX\laszLoc{b} ANA\laszLoc{b} HELP\laszDir{a}{b}. \\
        \citep[from][6]{Quadros.etal.2004}
        \hfill 
        \cmark Libras, \cmark ASL 
    \ex 
        IX\laszLoc{a} MARIA\laszLoc{a} HELP\laszDir{a}{b} IX\laszLoc{b} ANA\laszLoc{b}. \\
        \citep[from][5]{Quadros.etal.2004} 
        \hfill 
        \cmark Libras, \cmark ASL \\
        both: `Maria helped Ana.'
    \z 
\z 
 
For some verbs, locus agreement involves signing the verb at the
object's locus as illustrated in \REF{lasz:ex:5} (termed ``locationality'' by \citealp{Fischer.Gough.1978}, 
``spatialization'' by \citealp{Quadros.etal.2004}, 
and ``co-localization'' by \citealp{Lourenco.Wilbur.2018}; 
see also \citealp{Bergman.1980,Liddell.1980,Costello.2015,Smith.1990}, i.a.; ASL \& Libras),
or by adding an auxiliary as in \REF{lasz:ex:6} (Libras only; \citealp{Quadros.1999}). We
will treat all of these options as locus-agreement strategies and
postulate a uniform syntax.

\ea 
    \label{lasz:ex:5}
    Spatialized plain verbs
    \ea 
        MAN BICYCLE\laszLoc{a} BUY\laszLoc{a}. 
        \hfill 
        \cmark ASL, \cmark Libras \\ 
        \citep[from][9]{Quadros.etal.2004}
    \ex 
        MAN BUY\laszLoc{a} BICYCLE\laszLoc{a}. 
        \hfill 
        \cmark ASL, \cmark Libras \\ 
        \citep[from][9]{Quadros.etal.2004} \\ 
        both: `The man bought a/the bicycle.'
    \z 
\ex 
    \label{lasz:ex:6}
    Auxiliary 
    \ea 
        IX\laszLoc{a} JOÃO\laszLoc{a} IX\laszLoc{b} MARIA\laszLoc{b} AUX\laszDir{a}{b} SUPPORT. 
        \hfill 
        \cmark Libras 
        \\ 
        \citep[from][7]{Quadros.etal.2004} 
    \ex 
        IX\laszLoc{a} JOÃO\laszLoc{a} SUPPORT IX\laszLoc{b} MARIA\laszLoc{b}. 
        \hfill 
        \cmark Libras 
        \\ 
        \citep[from][5]{Quadros.etal.2004}
    \z 
\z 

The object-shift construction is not possible if neither trigger is
present, i.e., when the verb is ``plain'', as in \REF{lasz:ex:7}.

\ea 
    \label{lasz:ex:7}
    \ea[*]{
        MAN NUMBER FORGET\laszPlain{}. 
        \hfill 
        *ASL
        \\ 
        \citep[from][]{Liddell.1980}
    }
    \ex[*]{
        IX JOHN SOCCER LIKE\laszPlain{}. 
        \hfill 
        *Libras
        \\ 
        \citep[from][61]{Quadros.1999}
    }
    \ex[]{
        MAN FORGET\laszPlain{} NUMBER. 
        \hfill 
        \cmark ASL 
        \\ 
        `The man forgot a/the number.'
        \citep[from][]{Liddell.1980}
    }
    \ex[]{
        IX JOHN LIKE\laszPlain{} SOCCER. 
        \hfill 
        \cmark Libras 
        \\ 
        `John likes soccer.'
        \citep[from][61]{Quadros.1999}
    }
    \z 
\z 

Table~\ref{lasz:tab:1} presents a summary of the acceptability judgments for the
basic order SVO and for the object-shift construction SOV for both
ASL and Libras.

\newcommand{\laszTabColA}[0]{\cellcolor{gray!30}}
\newcommand{\laszTabColB}[0]{\cellcolor{gray!10}}
\newcommand{\laszTabColC}[0]{\cellcolor{gray!30}}

\newcolumntype{C}[1]{>{\centering\let\newline\\\arraybackslash\hspace{0pt}}m{#1}} % aligns in the middle vertically

\begin{table}
    \centering 
    \begin{tabularx}{r C{3em} c c c}
        \hline
        \hline 
        & V\laszPlain{} & V\laszLoc{\_} & V\laszHs{\_} & V\laszAsp{} \\ 
        \hline 
        ASL & 
            \laszTabColA{} & 
            \laszTabColB{} & 
            \laszTabColC{} & 
            \laszTabColC{} 
            \\ 
        Libras & 
            \multirow{-2}{*}{\laszTabColA{} \cmark SVO \newline \xmark SOV} & 
            \multirow{-2}{*}{\laszTabColB{} \shortstack{\cmark SVO \\ \cmark SOV}} & 
            \laszTabColB{} & 
            \multirow{-2}{*}{\laszTabColC{} \shortstack{\xmark SVO \\ \cmark SOV}} \\
        \hline 
    \end{tabularx}
    \caption{Comparison of judgments for SVO and OSV}
    \label{lasz:tab:1}
\end{table}

\printbibliography[heading=subbibliography,notkeyword=this]

\end{document}