\documentclass[output=paper,colorlinks,citecolor=brown,
% hidelinks,
% showindex
]{langscibook}
\author{Mark Baker\affiliation{Rutgers University} and Livia Camargo Souza\affiliation{Rutgers University}\orcid{}}
\title{On the size of same subject complements in two Panoan languages}
\abstract{Constituents with same subject marking in the Panoan languages Yawanawa and Shipibo normally function as adjunct CPs. In  this  work,  we  consider  the  fact  that they can also be used as complements of a restricted class of verbs including ‘begin’ and ‘know’. In this use, they can only be “reduced” clauses, which are less than full CPs. We explain this restriction using the idea that  same  subject markers need to enter into Agree with the matrix subject. When they occur in complements, this Agree relation is threatened by the matrix v being an intervening phase head. However, Agree can still go through if and only if no C is present as a second phase head. This analysis helps to explain the typological fact that same subject marking is more common in adjunct clauses and auxiliary constructions than in standard (full CP) clausal complementation.}
  
%move the following commands to the "local..." files of the master project when integrating this chapter
% \usepackage{tabularx}
% \usepackage{langsci-basic}
% \usepackage{langsci-optional}
% \usepackage{langsci-gb4e}
% \usepackage{pst-node}
% \bibliography{localbibliography}
% \newcommand{\orcid}[1]{}
% \pagenumbering{roman}
% \setlength{\footheight}{24.1pt}
\renewcommand{\eachwordone}{\itshape}
\renewcommand{\sc}[1]{\textsc{#1}}
\setlength{\glossglue}{5pt plus 2pt minus 1pt}

% \IfFileExists{../localcommands.tex}{%hack to check whether this is being compiled as part of a collection or standalone
%    % add all extra packages you need to load to this file
\usepackage{tabularx,multicol}
\raggedcolumns

\usepackage{url}
\urlstyle{same}

\usepackage{listings}
\lstset{basicstyle=\ttfamily,tabsize=2,breaklines=true}

\usepackage{langsci-basic}
\usepackage{langsci-optional}
\usepackage{langsci-lgr}

% Wood's chapter
\usepackage{jimsstyle}

\usepackage{subcaption}
\usepackage{color,colortbl}
\definecolor{Gray}{gray}{0.8}%to get a color for the table
\usepackage{expex}
\usepackage[linguistics]{forest}
\forestset{qtree edges/.style={for tree={parent anchor=south, child anchor=north}}}
\usepackage{graphbox}
\usepackage{leipzig}
\usepackage{multirow}
\usepackage{needspace}
\usepackage{newunicodechar}
\usepackage{pifont}
\usepackage{pgf}
\usepackage{pst-node}
\usepackage{qtree}
  \qtreecenterfalse
\usepackage{rotating}
% \usepackage{skak}
\usepackage{soul}
\usepackage{tabto}
\usepackage{textgreek}
\usepackage{tikz}
\usetikzlibrary{calc,decorations,shapes.misc,decorations.pathreplacing,arrows}
\usetikzlibrary{backgrounds, matrix, positioning}
\usetikzlibrary{}
\usetikzlibrary{}
\usepackage{tikz-qtree,tikz-qtree-compat}
\usepackage{tipa} % must be imported before linguex, but we do not use linguex here
\usepackage{todonotes}
\usepackage{tree-dvips}
\usepackage[normalem]{ulem}%for strikeout
\usepackage[Symbolsmallscale]{upgreek} 
\usepackage{verbatim}
\usepackage{xcolor}
\usepackage{xspace}
% \usepackage{xtab}

\usepackage{langsci-gb4e}

%    \newcommand{\orcid}[1]{}


% pesetskycommands.tex
%ExPex stuff

\newcommand{\pexcnn}{\pex[exno={~},exnoformat={X}]}  % pexc with no examplenumber and no increment
\lingset{everygla=\normalfont,{aboveglftskip=0pt}}
\newcommand{\ptxt}[1]{#1\par}
\let\expexgla\gla
\AtBeginDocument{\let\gla\expexgla\gathertags}

%other stuff
\newunicodechar{⟶}{{\symbolfont{⟶}}}
%\newunicodechar{✓}{{\Checkmark}} % checkmark
\newunicodechar{✓}{{\ding{51}}} % checkmark
\newcommand{\gap}{\ \underline{\hspace{10pt}\phantom{x}}\ }
\newcommand{\ix}[1]{\textsubscript{\it{#1}}}  % subscript

\def\gethyperref#1{\hyperlink{#1}{\getref{#1}}}%
\def\getfullhyperref#1{\hyperlink{#1}{\getfullref{#1}}}%

\newcommand{\denote}[1]{⟦{#1}⟧}

%    %% hyphenation points for line breaks
%% Normally, automatic hyphenation in LaTeX is very good
%% If a word is mis-hyphenated, add it to this file
%%
%% add information to TeX file before \begin{document} with:
%% %% hyphenation points for line breaks
%% Normally, automatic hyphenation in LaTeX is very good
%% If a word is mis-hyphenated, add it to this file
%%
%% add information to TeX file before \begin{document} with:
%% %% hyphenation points for line breaks
%% Normally, automatic hyphenation in LaTeX is very good
%% If a word is mis-hyphenated, add it to this file
%%
%% add information to TeX file before \begin{document} with:
%% \include{localhyphenation}
\hyphenation{
affri-ca-te
affri-ca-tes 
agree-ment
anaph-o-ra
an-ti-caus-a-tive
an-ti-caus-a-tives
an-te-ced-ent
Bak-er
caus-a-tive
Christo-poulos
clas-si-fi-er
claus-al
Co-lum-bia
com-ple-ment-izer
com-ple-ments
con-tin-u-a-tive
de-di-ca-ted
De-mo-cra-tas
Dor-drecht
du-ra-tive
Ex-folia-tion
ex-tra-gram-mat-ical
fi-nite-ness
ger-und
ger-unds
Gro-ninger
in-trac-ta-ble
Jap-a-nese
judg-ment
Judg-ment
Lysk-awa
Ma-khach-ka-la
Ma-rantz
Mat-thew-son
Max-Share
merg-er
Pap-u-an
Per-elts-vaig
post-ver-bal
phe-nom-e-non
pre-dic-tion
Pre-dic-tion
Rich-ards
Sa-len-ti-na
to-pic
Ya-wa-na-wa
Wil-liam-son
}

\hyphenation{
affri-ca-te
affri-ca-tes 
agree-ment
anaph-o-ra
an-ti-caus-a-tive
an-ti-caus-a-tives
an-te-ced-ent
Bak-er
caus-a-tive
Christo-poulos
clas-si-fi-er
claus-al
Co-lum-bia
com-ple-ment-izer
com-ple-ments
con-tin-u-a-tive
de-di-ca-ted
De-mo-cra-tas
Dor-drecht
du-ra-tive
Ex-folia-tion
ex-tra-gram-mat-ical
fi-nite-ness
ger-und
ger-unds
Gro-ninger
in-trac-ta-ble
Jap-a-nese
judg-ment
Judg-ment
Lysk-awa
Ma-khach-ka-la
Ma-rantz
Mat-thew-son
Max-Share
merg-er
Pap-u-an
Per-elts-vaig
post-ver-bal
phe-nom-e-non
pre-dic-tion
Pre-dic-tion
Rich-ards
Sa-len-ti-na
to-pic
Ya-wa-na-wa
Wil-liam-son
}

\hyphenation{
affri-ca-te
affri-ca-tes 
agree-ment
anaph-o-ra
an-ti-caus-a-tive
an-ti-caus-a-tives
an-te-ced-ent
Bak-er
caus-a-tive
Christo-poulos
clas-si-fi-er
claus-al
Co-lum-bia
com-ple-ment-izer
com-ple-ments
con-tin-u-a-tive
de-di-ca-ted
De-mo-cra-tas
Dor-drecht
du-ra-tive
Ex-folia-tion
ex-tra-gram-mat-ical
fi-nite-ness
ger-und
ger-unds
Gro-ninger
in-trac-ta-ble
Jap-a-nese
judg-ment
Judg-ment
Lysk-awa
Ma-khach-ka-la
Ma-rantz
Mat-thew-son
Max-Share
merg-er
Pap-u-an
Per-elts-vaig
post-ver-bal
phe-nom-e-non
pre-dic-tion
Pre-dic-tion
Rich-ards
Sa-len-ti-na
to-pic
Ya-wa-na-wa
Wil-liam-son
}

%     \bibliography{localbibliography}
%     \togglepaper[23]
% }{}

%custom footer for preprints
% \papernote{\scriptsize\normalfont
%     Change author.
%     Change title.
%     To appear in:
%     Change Volume Editor.  
%     Change volume title.  
%     Berlin: Language Science Press. [preliminary page numbering]
% }

\begin{document}
\maketitle

\section{Introduction: A gap in the range of Panoan complementation structures}
A seminal insight of Susi Wurmbrand’s rich research program investigating clausal complementation and the so-called restructuring phenomenon is that complements can come in a variety of different “sizes” (\citealt{wurmbrand2001infinitives}, etc.). These range from fully articulated CPs down to bare VPs, with several intermediate sizes in between. Of special importance is whether the complement contains one or more phase heads: full CPs do (C and v/Voice) and bare VPs do not. However, one cannot always tell how big a complement is simply by inspecting its superficial morphology. For example, infinitives with no overt subject in Spanish come in at least two sizes: a big kind with a phase head that blocks locality-sensitive processes like object clitic climbing, and a small kind without a phase head that allows clitics to climb into the matrix clause. This classic “restructuring” alternation is seen in (1) (see also \citealt{rizzi1982issues}, \citealt{burzio1986italian}, etc.). According to Wurmbrand’s very influential account, the complement XP in (\ref{BakerX1a}) is (something like) a CP, whereas in (\ref{BakerX1b}) it is (approximately) a VP.

\begin{exe}
    \ex Spanish (personal knowledge)
	    \begin{xlist}
			\ex \label{BakerX1a}
			\gll Quiero [\textsubscript{XP }comer-la].\\
			        want-1.\sc{sg} eat.\sc{inf}-it.\sc{f.sg}\\
			    \glt `I want to eat it.'
			\ex \label{BakerX1b}
			\gll La quiero [\textsubscript{XP }comer].\\
			        it.\sc{f.sg} want-1.\sc{sg} eat.\sc{inf}\\
			    \glt `I want to eat it.'
		\end{xlist}
\end{exe}

This approach has been very fruitful within a wide range of languages. In this chapter, we apply it to a particular issue in two Panoan languages: Yawanawa (YW) spoken in the Brazilian state of Acre, and Shipibo-Konibo (SK), spoken in the Peruvian Amazon. (Our explicit examples are taken mostly from Yawanawa, for uniformity.)\\

Some aspects of applying the Wurmbrandian approach to these languages are quite straightforward. For example, the desiderative morpheme \textit{kas} `want’ in SK takes a complement headed by a morphologically bare verb stem, not marked with any tense-like affix or switch-reference (SR) marking. Like Spanish infinitives, these can be “big” (phasal) or “small” (nonphasal), as shown by whether or not the object of the complement of `want’ triggers ergative on the subject of `want’ (\citealt[371-376]{baker2014dependent}). Similarly, many verbs take nominal/infinitival complements in which the embedded verb bears the affix \textit{-ti} in SK. With a verb like `know’, the nonfinite complement is big, but with the verb \textit{atipanti} `be able to’ the complement is small by the same criterion (\citealt[17]{baker2020agree}, B\&CS).\\

The specific issue we consider here is the status of verbal projections bearing the so-called imperfective same-subject (SS) markers -i/-kinin YW and SK.These forms are most commonly and canonically used in adjunct clauses,as in (\ref{BakerX2}). When the matrix subject is ergative, the form -kinis used; when it is nominative, -iis used.

\begin{exe}
    \ex Yawanawa (fieldwork) \label{BakerX2}
	    \begin{xlist}
			\ex \label{BakerX2a}
			\gll [(pro) mãnĩa tsisna-\textbf{i}], Shaya pake-a.\\
			     (she) banana carry-\sc{ss.nom} Shaya.\sc{nom} fall-\sc{pfv}\\
			    \glt `While she\textsubscript{i} was carrying bananas, Shaya\textsubscript{i} fell.'
			\ex \label{BakerX2b}
			\gll [Shaya-N pitxã-pai-ki-N], (pro) mai  keti  hi-a.\\
    			 Shaya-\sc{erg} cook-\sc{des-ss-erg} (she) clay pot get-\sc{pfv}\\
			    \glt `Because Shaya\textsubscript{i} was wanting to cook, she\textsubscript{i} got a clay pot.'
		\end{xlist}
\end{exe}

These adjunct clauses are clearly “big” by the relevant metrics.For example, they can contain an overt subject with ergative case, as in (\ref{BakerX2b}), and DPs inside the adjunct clause do not trigger ergative case on the subject of the matrix clause, as in (\ref{BakerX2a}). There is, however, another use of SS-marked constituents in these languages: they can also be used as complements of a restricted class of verbs. Both SK and YW allow SS-marked constituents as the complement of an aspectual verb like ‘begin’, ‘stop’, and ‘finish’ \citep[319]{valenzuela2003transitivity}. These have the character of raising constructions, in that the matrix verb does not assign a thematic role to the subject of the embedded verb (see \citealt{camargosouza2020switch} (CS) for discussion)

\begin{exe}
    \ex Yawanawa (fieldwork) \label{BakerX3}
	    \begin{xlist}
			\ex \label{BakerX3a}
			\gll Shukuvena-N [wixi ane-ki-N] tae-wa.\footnotemark\\
			     Shukuvena-\sc{erg} book read-\sc{ss-erg} begin-\sc{caus.pfv}\\
			    \glt `Shukuvena began reading a book.'
			\ex \label{BakerX3b}
			\gll Shukuvena [raya-i] tae-a.\\
    			 Shukuvena.\sc{nom} work-\sc{ss.nom} begin-\sc{pfv}\\
			    \glt `Shukuvena began working.'
		\end{xlist}
		\footnotetext{Aspectual verbs like `begin' in Panoan languages undergo morphological changes to agree in transitivity with the embedded predicate \citep{valenzuela2003transitivity,}. This is why the verb `begin' bears the causative suffix \textit{wa} in (\ref{BakerX3a}) but not (\ref{BakerX3b}). This phenomenon parallels the voice matching studied by \citet{wurmbrand2017features}. See CS for an account of this factor.}
\end{exe}

In addition, CS discovers that YW also allows SS complements with certain attitude verbs like `know', `dream', `think', `hope', and `forget'. These have the character of control constructions, where the matrix verb does assign a thematic role to the subject in addition to the thematic role that the lower verb assigns to its understood subject.\footnote{These languages also use SS marking in purer auxiliary constructions. For example, `go’ plus a verb marked with SS forms a periphrastic future construction in SK \citep[306]{valenzuela2003transitivity, zariquieybiondi2011}. We think that our analysis of the `begin’ construction can serve as a first-pass analysis of these constructions as well, but there are some morphological differences to consider.}

\begin{exe}
    \ex Yawanawa (fieldwork) \label{BakerX4}
	    \begin{xlist}
			\ex \label{BakerX4a}
			\gll Shaya-N [yuma pitxaN-ki-N] tapiN-a.\\
			     Shaya-\sc{erg} fish cook-\sc{ss-erg} know-\sc{pfv}\\
			    \glt `Shaya knows how to cook fish.'
			\ex \label{BakerX4b}
			\gll Shaya [saik-i] tapiN-a.\\
    			 Shaya.\sc{nom} sing-\sc{ss.nom} know-\sc{pfv}\\
			    \glt `Shaya knows how to sing.'
		\end{xlist}
\end{exe}

At one level, it is not to surprising that SS morphology is used in these environments, since both subject-to-subject raising constructions and subject control constructions have the property that the subject of the embedded constituent is referentially dependent on the subject of the matrix clause. But what we find striking is that the SS-marked projections serving as complements always test out as being “small”, never “big”. Thus, in both (\ref{BakerX3}) and (\ref{BakerX4}) the subject of the matrix clause bears ergative case if and only if the embedded verb has a direct object. This suggests that there is no phase head associated with the embedded constituent. But why should this be? SS-marked constituents do behave like full phasal CPs when they serve as adjuncts. Other morphological verb forms can vacillate between big and small status. Why then should verbs in the SS form require a small construction when and only when they appear in complement position? This is the puzzle that we consider here.\\

Our proposal is as follows. What is special about SS clauses in Panoan according to B\&CS is that the functional heads associated with the embedded clause enter into two relationships of Agree: one with the subject of the embedded clause, and one with the subject of the matrix clause. These instances of Agree create pointers from the functional heads to the two subjects, which LF then interprets as some type of referential dependency. Now when this kind of phrase appears in complement position, it is separated from the matrix subject by an additional phase boundary: the one induced by the v of the matrix clause. This phase boundary threatens to block the upward Agree relation with the matrix subject, which is an essential ingredient of SS constructions. Therefore SS-marked complements are required to be small, without an additional C-type phase head of their own. The sentence as a whole then counts as a single locality domain. A theoretical consequence of this is that it steers us toward Chomsky’s \citeyearpar{chomsky2001derivation} conception of the phase, in which dependencies can cross one phase boundary but not two. We claim that this analysis gives a partial explanation of the typological fact that switch reference marking is more common across languages in adjunct clauses than in complement clauses, and when it is possible in complement clauses it is often limited to auxiliary-like constructions.\\

We develop our analysis as follows: Section 2 establishes the basic structural properties of the two SS complement constructions. Section 3 shows that they both count as a single locality domain by two tests: ergative case assignment and object=subject switch-reference. Section 4 reviews B\&CS’s idea that SS involves Agree relations, and then sketches the outline of our analysis in these terms. Section 5 refines the analysis in certain ways, arguing that the “small” SS complements are in fact FinP projections, not ForcePs \citep{rizzi1997fine}. Section 6 puts our results in a broader typological context and concludes.

\section{The structure of SS complements}

In this section, we argue briefly for three points that support our basic claims about SS complements. We show that they are complements, not adjuncts. We show that the overt subject is in the matrix clause, not the embedded clause. And we show that with `know’-class verbs the subject gets a thematic role from the matrix verb. If all this is true, then YW and SK have the sort of structure that could have been a full-CP control complement, instead of or alongside the reduced structure that they actually do have. The fact that they do not have this familiar and theoretically innocuous structure with SS-marked verbs then becomes interesting and worth trying to explain.\\

Both `begin’ and `know’ can occur with arguments other than an SS-marked constituent. `Begin’ can take an event-denoting DP complement, as in (\ref{BakerX5a}). It cannot, however, take any verb-headed complement other than SS-complements. Like `begin’, `know’ can take a DP direct object, but it can also take a nominalized clause as internal argument along with a sentient DP as its subject as in (\ref{BakerX5b}).\footnote{\textit{TapiN} is naturally glossed as `know how’ in (\ref{BakerX3}) but as `know that’ in (\ref{BakerX5b}). We assume it is essentially the same lexical item in both cases, as in English.}\\
\newpage
\begin{exe}
    \ex Yawanawa (fieldwork) \label{BakerX5}
	    \begin{xlist}
			\ex \label{BakerX5a}
			\gll Vari tae-a.\\
			     summer begin-\sc{pfv}\\
			    \glt `Summer Began.'
			\ex \label{BakerX5b}
			\gll [Shukuvena-N yuma itxapa atxi-ai-tuN] Shaya-N tapiN-a\\
    			 Shukuvena-\sc{erg} fish many catch-\sc{ipfv-nmlz} Shaya-\sc{erg} know-\sc{pfv}\\
			    \glt `Shaya knows that Shukuvena is catching many fish.'
		\end{xlist}
\end{exe}

(\ref{BakerX5b}) is a typical example of YW and SK’s only other form of clausal complementation, in which the embedded verb is marked with a nominalizing suffix: -tuN (YW), –ti (SK), -a (perfective “participle”, both), or -ai (imperfective, SK); see also (\ref{BakerX11}), (\ref{BakerX14a}) and (\ref{BakerX17}). The internal syntax of these nominalized clauses is like root clauses, with ergative and absolutive arguments, but their external syntax is that of DPs: they appear in DP positions and trigger ergative on the subject.\footnote{A reviewer asks why `begin’ class verbs select only SS-complements, whereas `know’ class verbs allow nominalized complements as well as SS-complements. Our tentative answer is that this is because (as in English) aspectual verbs combine semantically with event-denoting expressions, whereas cognitive verbs can combine with fact- and proposition-denoting expressions. Nominalized clauses presumably denote facts or propositions, rather than events.}\\

Comparison with (\ref{BakerX5a}) suggests that the SS-constituent in (\ref{BakerX3}) is the internal argument of `begin’, parallel to `summer’, and that `begin’ has no other argument—like canonical raising verbs. Similarly, the SS-constituent in (\ref{BakerX4}) is plausibly the internal argument of `know’, parallel to the nominalized clause in (\ref{BakerX5b}). In addition, `know’ does take a distinct external argument, like canonical subject control verbs. These differing thematic properties are confirmed by (\ref{BakerX6}), where the `begin’ construction is compatible with a verb that does not have a thematic subject, whereas the `know’-type construction is not.

\begin{exe}
    \ex Yawanawa (fieldwork) \label{BakerX6}
        \begin{xlist}
        \ex \label{BakerX6a}
			\gll [Uik-i] ene-a.\\
			     rain-\sc{ss.nom} stop-\sc{pfv}\\
			    \glt `It stopped raining.'
			\ex \label{BakerX6b}
			\gll \#[Uik-i] tapiN-a.\\
    			 rain-\sc{ss.nom} know-\sc{pfv}\\
			    \glt (`It knows how to rain.')
        \end{xlist}
\end{exe}

Evidence that the SS-constituents in these constructions are complements of the matrix verb comes from wh-movement. (\ref{BakerX7c}) shows that it is not possible to extract a question word from an SS-marked adjunct clause, due to the adjunct island condition. However, CS shows that it is possible to extract a question word from the SS-constituent in the ‘begin’ and ‘know’ constructions (\ref{BakerX7a},\ref{BakerX7b}). The contrast shows that these constituents are complements, not adjuncts.

\begin{exe}
    \ex Yawanawa (fieldwork) \label{BakerX7}
        \begin{xlist}
        \ex \label{BakerX7a}
			\gll Awea=meN Shukuvena-N [\hspace{3pt}-\hspace{0pt}-\hspace{3pt}wa-ki-N] tapiN-a?\\
			     what=\sc{int} Shukuvena-\sc{erg} make-\sc{ss-erg} know-\sc{pfv}\\
			    \glt `What does Shukuvena know how to make?'
		\ex \label{BakerX7b}
			\gll Awea=meN Shukuvena-N [\hspace{3pt}-\hspace{0pt}-\hspace{3pt}ane-ki-N] tae-wa?\\
    			 what=\sc{int} Shukuvena-\sc{erg} read-\sc{ss-erg} begin-\sc{caus.pfv}\\
			    \glt `What did Shukuvena begin to read?'
		\ex \label{BakerX7c}
		    \gll *Awea=meN [\hspace{3pt}-\hspace{0pt}-\hspace{3pt}pitxaN-pai-ki-N] Shaya-N mai keti hi-a?\\
		        what=\sc{int} cook-\sc{des-ss-erg} Shaya-\sc{erg} clay pot buy-\sc{pfv}\\
		        \glt ('What did Shaya buy a clay pot wanting to cook (it)?')
        \end{xlist}
\end{exe}

The last basic property of the SS-complement constructions to affirm is that the overt subject is really a constituent of the matrix clause on the surface, not part of the SS-marked constituent. This is especially an issue for the `begin’ construction, which we claim to be an instance of subject-to-subject raising, since one could imagine that the thematic subject of the lower verb remains in the lower clause. But this turns out to be impossible. Word order evidence for this is in (\ref{BakerX8}). (\ref{BakerX8a}) is the neutral order. (\ref{BakerX8b}) shows that a constituent consisting of the SS-marked verb and its object can be extraposed to the right. However, (\ref{BakerX8c}) shows that it is impossible for the subject to be included in this rightward-moved constituent; rather it is part of the matrix clause.

\begin{exe}
    \ex Yawanawa (fieldwork) \label{BakerX8}
        \begin{xlist}
        \ex \label{BakerX8a}
			\gll Shukuvena-N wixi ane-ki-N tae-wa.\\
			     Shukuvena-\sc{erg} book read-\sc{ss-erg} begin-\sc{caus.pfv}\\
			    \glt `Shukuvena began to read the book.'
		\ex \label{BakerX8b}
			\gll Shukuvena-N tae-wa, [wixi ane-ki-N].\\
    			 Shukuvena-\sc{erg} begin-\sc{caus.pfv} book read-\sc{ss-erg}\\
			    \glt `Shukuvena began to read the book.'
		\ex \label{BakerX8c}
		    \gll *Tae-wa, [Shukuvena-N wixi ane-ki-N].\\
		        begin-\sc{caus.pfv} Shukuvena-\sc{erg} book read-\sc{ss-erg}\\
		        \glt ('Shukuvena began to read the book.')
        \end{xlist}
\end{exe}

This word order restriction applies to examples with ‘know’ as well. Converging evidence comes from second position clitics in YW and SK. These can appear immediately after the subject in an example like (\ref{BakerX8a}), but they cannot appear between ‘read’ and ‘begin’. This also shows that the subject-object-verb+SS sequence is not a single constituent in this construction (this order is fine with SS adjuncts).\\

Putting this all together, we have evidence that (\ref{BakerX9}) is a possible syntactic structure in YW and SK.

\begin{exe}
    \ex \label{BakerX9} \normalfont[\textsubscript{TP }Shukuvena.\sc{erg} [\textsubscript{VP }[\textsubscript{XP } PRO/t fish cook-\sc{SS}] know/begin]]
\end{exe}

The next question, then, is what is XP: a big (phasal) constituent like CP, or a small (nonphasal) constituent? In the next section we argue that XP can only be small in this construction.

\section{The size of SS complements}

We have two sources of evidence that the SS-complements in Panoan are small, the sentence as a whole counting as a single locality domain. Our fancier evidence comes from these languages’ unusual object=subject (O=S) switch-reference construction, analyzed in detail in B\&CS. The basic description of this construction is that the verb in an adjunct clause bears the suffix -a when it has an object that is coreferential with the subject of the main clause. Normally this is only possible if the DP equated with the matrix subject is the verb’s very own object. However, the SS-complement constructions are systematic exceptions to this generalization: they allow -a to appear on the ‘know’-class verb (here ‘forget’) or the ‘begin’-class verb when the object of the complement of ‘forget’ or ‘begin’ is coreferential with the matrix subject. This is seen in (\ref{BakerX10a},\ref{BakerX10b}). In contrast, this is not possible when a verb like ‘think’ takes an infinitival complement rather than an SS-marked complement, as shown in (\ref{BakerX11}) from SK. (YW does not have exactly this sort of complement.)

\begin{exe}
    \ex Yawanawa (fieldwork) \label{BakerX10}
        \begin{xlist}
        \ex \label{BakerX10a}
			\gll [E-N [\textbf{kaNmaN} nesha-ki-N] xinavenu-\textbf{a}] (pro) itxu-a.\\
			     I-\sc{erg} dog tie-\sc{ss-erg} forget-\sc{os} (it) run-\sc{pfv}\\
			    \glt `Because I forgot to tie up the dog\textsubscript{i}, it\textsubscript{i} ran away.'
		\ex \label{BakerX10b}
			\gll [Shukuvena-N [\textbf{wixi} ane-ki-N] tae-wa-hi-\textbf{a} \textbf{aweN} \textbf{wixi} venu-a.\\
    			 Shukuvena-\sc{erg} book read-\sc{ss-erg} begin-\sc{caus-conc-os} his book disappear-\sc{pfv}\\
			    \glt `Although Shukuvena started reading the book\textsubscript{i}, it\textsubscript{i} got lost.'
        \end{xlist}
\end{exe}

\begin{exe}
    \ex Shipibo (Baker and Camargo Souza 2020) \label{BakerX11}\\
        \gll ??[Jose-kan [(pro) oin-ti] shinan-a]=ra, Rosa-n e-a kena-ke.\\
            José-\sc{erg} \textbf{(her)} see-\sc{inf} think-\sc{OS=EV} \textbf{Rosa-\sc{erg}} me-\sc{acc} call-\sc{pfv}\\
        \glt (`When José thought to see her\textsubscript{i}, Rosa\textsubscript{i} called me.')
\end{exe}

There is a similarity here with the clitic climbing seen in Spanish in (\ref{BakerX1b}): in both cases what is thematically the object of another verb behaves like it is the object of the restructuring verb for syntactic purposes. B\&CS’s official analysis is in terms of Agree: a v node associated with ‘begin’/‘forget’ is able to probe downward into its complement to find the object inside that complement as its goal. This is possible because the complement is nonphasal, so the Agree relation does not violate the Phase Impenetrability Condition (PIC).\\

Converging evidence that the SS complement is small comes from the assignment of ergative case. Both YW and SK have morphological ergative case (underlyingly \textit{-n}, with allomorphs) on the subjects of transitive verbs. Baker (2014, 2014) analyzes this as a dependent case in the tradition of Marantz (1991); it is assigned by the rule in (\ref{BakerX12}).

\begin{exe}
    \ex \label{BakerX12} Assign ergative to DP1 at the spell out of the complement of a C head if DP1 c-commands another DP in the same domain.
\end{exe}

For example, a verb that takes both an external argument and an internal argument has ergative case on the external argument, whereas a verb with only one argument has nominative case on that argument.
\newpage
\begin{exe}
    \ex Yawanawa (fieldwork) \label{BakerX13}
        \begin{xlist}
        \ex \label{BakerX13a}
			\gll Shaya saik-i.\\
    			 Shaya sing-\sc{ipfv}\\
			    \glt `Shaya is singing.'
		\ex \label{BakerX13b}
			\gll Shaya-N nami pitxã-i.\\
    			 Shaya-\sc{erg} meat cook-\sc{ipfv}\\
			    \glt `Shaya is cooking meat.'
        \end{xlist}
\end{exe}

The relevance of the domain restriction in (\ref{BakerX12}) is seen in examples with embedded full CP clauses like (\ref{BakerX14}). Here whether the matrix subject is ergative or not does not depend on whether the lower verb has an object, but only on the categorical features of the embedded clause as a whole. If the embedded clause is nominal, the matrix subject is uniformly ergative, even if there is no embedded object, as in (\ref{BakerX14a}) from SK. If the embedded clause is not nominal, the matrix subject is not ergative, even when there is an embedded object, as in (\ref{BakerX14b}).

\begin{exe}
    \ex Shipibo and Yawanawa (fieldwork) \label{BakerX14}
        \begin{xlist}
        \ex \label{BakerX14a}
			\gll Maria-nin=ra [bewa-ti] shinan-ke. (SK)\\
    			 Maria-\sc{erg=ev} sing-\sc{inf} think-\sc{pfv}\\
			    \glt `Maria thought to sing.'
		\ex \label{BakerX14b}
			\gll [(pro) Shaya kena-pai-i], Shukuvena ka-i. (YW)\\
    			 [(he) Shaya call-\sc{des-ss.nom}] Shukuvena.\sc{nom} go-\sc{ipfv}\\
			    \glt `Wanting to call Shaya, Shukuvena is leaving.'
        \end{xlist}
\end{exe}

The `begin' and `know' constructions are markedly different in this respect. For them, the case marking of the matrix subject does depend on whether the verb in the SS complement takes an object: if it does, the matrix subject is ergative; if it does not, the matrix subject is not ergative. This was seen in (\ref{BakerX3}) and (\ref{BakerX4}); the latter is repeated here.

\begin{exe}
    \ex Yawanawa (fieldwork) \label{BakerX15}
	    \begin{xlist}
			\ex \label{BakerX15a}
			\gll Shaya-N [yuma pitxaN-ki-N] tapiN-a.\\
			     Shaya-\sc{erg} fish cook-\sc{ss-erg} know-\sc{pfv}\\
			    \glt `Shaya knows how to cook fish.'
			\ex \label{BakerX15b}
			\gll Shaya [saik-i] tapiN-a.\\
    			 Shaya.\sc{nom} sing-\sc{ss.nom} know-\sc{pfv}\\
			    \glt `Shaya knows how to sing.'
		\end{xlist}
\end{exe}


The lack of ergative case on the subject in (\ref{BakerX15b}) shows that the SS complement as a whole is not nominal, the way -ti complements are. The presence of ergative on the subject in (\ref{BakerX15a}) shows that the SS complement is small/nonphasal, so that ‘Shaya’ and ‘fish’ are in the same domain at the point of spelling out the complement of the matrix C. Indeed, ergative case is obligatory on the matrix subject here. Therefore, the SS complement \textit{must} be small, and cannot be big.\\

And that is something worth trying to explain. It is not that Panoan syntax is adverse to optional restructuring across the board. Baker (2014: 371-376) shows that restructuring is optional in a ‘want’ construction in SK, using the same two tests discussed in this section. In that construction, the matrix subject is optionally ergative when the embedded verb has an object. But this familiar sort of optionality, seen also in (1), is not seen with SS complements in SK or YW. The theoretical question that arises, then, is why not? We turn to this next.

\section{The leading idea for an analysis}

What is special syntactically about SS constituents, which might cause them to have a different range of possibilities than bare verbs and
infinitival verbs? Recall that the core use of SS-marked verbs in Panoan languages is in adjunct clauses, as in (\ref{BakerX2}). B\&CS argue that the key to this construction is that a functional head at the periphery of the CP adjunct (a fusion of T and C) undergoes Agree twice: once with the closest DP searching downward, i.e. the subject of the embedded clause, and once with the closest DP searching upward, i.e. the subject of the matrix clause (see also \citet{arregi2019switch} for essentially the same idea). The result of these two Agree processes is pointers from the functional head(s) to the two subjects (cf. \citealt{arregi2012morphotactics}). LF then interprets these pointers as coconstrual holding between the two subject positions (see CS for discussion). So the structure of a typical SS adjunct like (\ref{BakerX16a}) is (\ref{BakerX16b}). (The adjunct clause then usually extraposes to the sentence’s right or left edge.)

\begin{exe}
    \ex Yawanawa (fieldwork) \label{BakerX16}
	    \begin{xlist}
			\ex \label{BakerX16a}
			\gll Shaya-N [(pro) mixki-ki-N] ixixiwã atxi-a.\\
			     Shaya-\sc{erg} (she) fish\sc{ss-erg} catfish catch-\sc{pfv}\\
			    \glt `Shaya, while fishing, caught a catfish.'
			\ex \label{BakerX16b}
			    [\textsubscript{TP }Sh\rnode{shaya}aya\textsubscript{i} [\textsubscript{CP }[\textsubscript{TP }pro\textsubscript{i} [\textsubscript{vP } \rnode{t2}t fish v]-\rnode{t}T] \rnode{c}C] [\textsubscript{vP } t fish catch v] T]
			    \psset{linewidth=0.5pt, linejoin=1, arrows=->, arrowscale=1.5, arrowinset=0.15, angle=90, nodesep=1pt, arm=2ex}
			    \ncbar[angle= -90]{t}{t2}
			    \ncbar[angle= -90, arm=3ex]{c}{shaya}
			    \ncbar[angle= -90, arm=1ex, arrows=-]{t}{c}
		\end{xlist}
\end{exe}
\vspace{14pt}
B\&CS present more detailed evidence for the downward Agree relationship than for the upward one. But there is adequate evidence that upward Agree holds too. For example, DP-movement to Spec TP in the matrix clause crucially feeds SS marking, showing that the tracked DP in the matrix clause must c-command the SS morpheme in the adjunct clause. Even more to the point, an SS constituent adjoined to one particular clause—to the complement of `see’ in (\ref{BakerX17}), for instance—can only track the subject of that very clause, namely Meni, not the subject of a still higher clause, in this case eN, the subject of `see’. This shows that the SS heads cannot enter into a relationship with a DP that is too far away, with distance measured in terms of clauses, which correspond to phase boundaries within our framework.

\begin{exe}
    \ex Yawanawa (fieldwork) \label{BakerX17}\\
        \gll [[Meni-N Shukuvena vetxi-\textbf{ashe}] \textbf{(pro)} inĩmai-tu] e-N ũi-a.\\
            Meni-\sc{erg} Shukuvena find-\sc{SS.PFV.NOM} be.happy-\sc{NMLZ} I-\sc{erg} see-\sc{pfv}\\
        \glt `I saw that she\textsubscript{i} was happy when Meni\textsubscript{i} found Shukuvena.'
\end{exe}

This last point can provide an entry into understanding the puzzle that is before us now. What is special about SS-marked constituents as opposed to others is that they enter into upward Agree with a DP, the matrix subject. Given this, what could be the difference between SS-marked adjuncts (which are freely available) and SS marked complements (which are quite restricted, both in Panoan languages and crosslinguistically)? An answer is that adjuncts are closer to the subject than complements are. In particular, adjuncts can be adjoined to vP (or higher), so that they are outside the spell out domain of the phase head v.\footnote{In this work, we follow B\&CS in not making a distinction between v and Voice, for simplicity. See CS for a refinement that does distinguish v from Voice.} In contrast, complements are by definition the sisters of the V head, so they are inside the complement of v and they are necessarily spelled out on the v-phase cycle. In other words, there is a phase boundary associated with v that intervenes between heads inside the SS-complement and the matrix subject but not between the SS adjunct and the matrix subject. This phase boundary could prevent the SS heads from entering into the necessary Agree relationship with the matrix subject. This is sketched in (\ref{BakerX18}).

\newpage

\begin{exe}
    \ex Leading idea: (phrases in \{ \} are spell out domains) \label{BakerX18}
	    \begin{xlist}
			\ex \label{BakerX18a}
			    [S\rnode{subj}ubj [\rnode{h}H clause] [\textsubscript{vP V} \textbf\{V....\}]] \textit{Adjunct clause}
			        \psset{linewidth=0.5pt, linejoin=1, arrows=->, arrowscale=1.5, arrowinset=0.15, nodesep=1pt, arm=2.5ex}
		           	\ncbar[angle=-90, arm=3ex]{h}{subj}\\
		           	\hspace*{3.5em} \textit{Agree possible}
		           	%HOW DO YOU LABEL THESE DEAR GOD HELP ME
			    \vspace{8pt}
			\ex \label{BakerX18b}
			    [S\rnode{subj}ubj v \{V [\rnode{h}H clause]\} \textit{Argument clause}
		        	\psset{linewidth=0.5pt, linejoin=1, arrows=->, arrowscale=1.5, arrowinset=0.15, nodesep=1pt, arm=2ex}
		        	\ncbar[angle=-90, arm=3ex]{h}{subj}
		        	\ncput*{X}\\
		        	\hspace*{5.3em} \textit{Agree blocked}
		    	\vspace{12pt}
		\end{xlist}
\end{exe}

This is a start, but we need another layer of the analysis to explain why “small”, restructuring-type SS-complements are possible in Panoan, whereas “large” SS-complements are not. At this point, it evidently matters whether the SS-complement counts as a full CP, with its own phase head, or not. When the SS-complement is a full CP, Agree fails, whereas when it is less than a full CP, Agree can succeed—even though the v-phase boundary is there. We find our way through to a full analysis if we adopt a view in which two phase boundaries block upward Agree but one phase boundary does not. This amounts to adopting \Citet[13-14]{chomsky2001derivation} view of the PIC, as stated in (\ref{BakerX19}), rather than his \citeyearpar{chomsky2000minimalist} version.

\begin{exe}
    \ex Elements in the complement of a phase head H are accessible to the computation until the introduction of the next phase head Z. \label{BakerX19}
\end{exe}

The other key assumption we make concerns the finer structure of the CP in YW and SK. B\&CS assume that T probes downward to find the
5 In this work, we follow B\&CS in not making a distinction between v and Voice, for simplicity. See CS for a refinement that does distinguish v from Voice.
embedded subject, and C probes upward to find the matrix subject, T then fusing with C to form a single head, as in (\ref{BakerX16}). Following CS, we revise this by distinguishing Force from Fin \citep{rizzi1997fine}. The higher head Force is the phase head, whereas the lower head Fin is the upward probe and the head that fuses with T. These assumptions allow us to derive the three-way contrast at hand, where SS is possible in ForceP adjuncts and FinP complements, but not in ForceP complements, as outlined in (\ref{BakerX20}). This is the core of our analysis.

\begin{exe}
    \ex Leading idea expanded: (\{ \} = complement of phase head) \label{BakerX20}
        \begin{xlist}
        \ex \label{BakerX20a}
			 [S\rnode{subj}ubj [\textsubscript{ForceP} Force \{F\rnode{fin}in TP\}] [\textsubscript{v} \{\textsubscript{VP} V....\}]] ForceP adjunct
			        \psset{linewidth=0.5pt, linejoin=1, arrows=->, arrowscale=1.5, arrowinset=0.15, nodesep=1pt, arm=2.5ex}
		           	\ncbar[angle=-90, arm=3ex]{fin}{subj}\\
		        	\hspace*{9em} \textit{Agree OK by (\ref{BakerX19})}
		\vspace{16pt}
		\ex \label{BakerX20b}
			[S\rnode{subj}ubj v \{\textsubscript{VP} know/begin [\textsubscript{ForceP} Force \{F\rnode{fin}in TP\}]\}] ForceP
			        \psset{linewidth=0.5pt, linejoin=1, arrows=->, arrowscale=1.5, arrowinset=0.15, nodesep=1pt, arm=2.5ex}
		           	\ncbar[angle=-90, arm=3ex]{fin}{subj}
		           	\ncput*{X}\\
		        	\hspace*{16.5em} \textit{Agree * by (\ref{BakerX19})}
		\vspace{16pt}
		\ex \label{BakerX20c}
		    [S\rnode{subj}ubj v \{\textsubscript{VP} know/begin [\textsubscript{FinP} F\rnode{fin}in TP]\}] FinP complement
			        \psset{linewidth=0.5pt, linejoin=1, arrows=->, arrowscale=1.5, arrowinset=0.15, nodesep=1pt, arm=2.5ex}
		           	\ncbar[angle=-90, arm=3ex]{fin}{subj}\\
		        	\hspace*{13em} \textit{Agree OK by (\ref{BakerX19})}
		\vspace{16pt}
        \end{xlist}
\end{exe}

\section{Refining the analysis}

There are, however, some details to work out to realize this analysis in a consistent way. These concern details of phase boundaries, whether the same phase boundaries affect dependent case and upward Agree, and so on. Part of the challenge here is harmonizing this analysis with previous research. We proceed now to these refinements, to the degree that they are of some general interest and fit in a work of this size.\\

First, we acknowledge that the “small” version of the SS complement in (\ref{BakerX20c}) is a FinP, hence not all that small. The clearest Wurmbrandian opposition is between full CPs, which definitely have phase heads, and bare VPs, which definitely do not. But Wurmbrand’s work shows that “small” constituents can often be somewhat larger than VP, with some inflectional material as well. In the typology of \citet{wurmbrand2001infinitives}, YW and SK have “reduced non-restructuring complements”, which include tense/aspect information (T) and an internal subject position, rather than restructuring proper. A more detailed structure of (\ref{BakerX4a}) then is (\ref{BakerX21b}), which shows FinP and TP in the SS complement. Fin finds the controlling subject in the case of `know’ and the higher copy of the raised subject in the case of `begin’. T finds the controlled PRO embedded subject in the case of `know’ and the lower copy of the raised subject in the case `begin’. In both cases, the two subjects are interpreted as instances of the same bound variable.\footnotetext{A reviewer asks whether allowing SS markers to point to two copies in the same movement chain opens the door to unwanted structures in which SS morphology is licensed in a single clause by SR heads agreeing downward with the copy of the subject in Spec vP and upward with the copy of the subject in Spec TP. In fact, SS couldn’t do this in YW/SK because of our lexical stipulation that one of the probes is Fin and Fin probes upward. As a result, it cannot find a DP in the Spec TP, which is below Fin. Some other head—a double-Agreeing T, say—might be able to find these two copies. But a morpheme that did only this would not be recognized as an SS morpheme at all.}

\begin{exe}
    \ex Yawanawa (fieldwork) \label{BakerX21}
	    \begin{xlist}
			\ex \label{BakerX21a}
			\gll Shaya-N yuma pitxaN-ki-N tapiN-a.\\
			     Shaya-\sc{erg} fish cook-\sc{ss-erg} know-\sc{pfv}\\
			    \glt `Shaya knows how to cook fish.'
			\ex \label{BakerX21b}
			    \small[\textsubscript{TP }Sh\rnode{shaya}aya [\textsubscript{vP }t [\textsubscript{VP }[\textsubscript{FinP }[\textsubscript{TP }[\textsubscript{vP } P\rnode{pro}RO [\textsubscript{VP }fish cook] v ]  \rnode{t}T\textsubscript{SS} \rnode{fin1}F\rnode{fin}in\textsubscript{SS}] know] v] T]\\
			    \textit{Agree \hspace{7.5em} Agree \hspace{11em}head mov't}
			    \psset{linewidth=0.5pt, linejoin=1, arrows=->, arrowscale=1.5, arrowinset=0.15, angle=90, nodesep=1pt, arm=3ex}
			    \ncbar[angle= -90]{t}{pro}
			    \ncbar[angle= -90, arm=4ex]{fin}{shaya}
			    \ncbar[nodesep=1pt, angleA=-50, armA=2.5ex, armB=0]{t}{fin1}
		\end{xlist}
\end{exe}
\vspace{16pt}
There is some converging evidence that SS complements do contain TPs, as CS discusses in detail. First, “T” is clearly semantically present in the SS complement. What we call T in Panoan primarily expresses the perfective/imperfective distinction, which concerns whether two eventualities overlap or not. The SS marking in the `begin’ and `know’ constructions is always the imperfective form \textit{-i/-kiN}, not the perfective form \textit{-ashe/-shuN}. And indeed it is the imperfective form that is semantically appropriate here, given that the event of (say) reading a book necessarily coincides with the event of beginning in an example like (\ref{BakerX3a}).\\

Further evidence that SS complements contain TP comes from the fact they can contain \textit{derived} subjects—NPs that become subjects by way of movement to Spec TP (or perhaps Spec SubjectP, as in B\&CS). This movement internal to the SS complement is clearest in the applicative of unaccusative construction \citep[see][]{baker2014dependent}. The affectee argument of an applicative construction is generated in Spec ApplP, above the base position of a theme argument inside VP. When the verb root is transitive, this hierarchical order is maintained. But when the verb is unaccusative, without an agent argument, the affectee argument is blocked from moving to Spec TP to satisfy the EPP property in these languages. Instead, the theme argument moves to Spec TP, becoming the subject for purposes of case and agreement. Now this applicative-of-unaccusative structure can be embedded in a `begin’ construction, as in (\ref{BakerX22}). In this case, the theme argument further raises from the embedded Spec TP to the matrix Spec TP, and SS marking succeeds. If there were no TP inside the SS complement, NP-movement would have to go straight from inside VP to the matrix Spec TP. Then the downward-looking SR probe at the edge of the complement would find the affectee argument in Spec ApplP rather than the theme argument in VP and SS marking would fail, since the affectee argument is not the same as the subject of the matrix clause.

\begin{exe}
    \ex Yawanawa (fieldwork) \label{BakerX22}\\
		\small	\gll Ewẽ ketxa {mixti-hãui [\textsubscript{TP} t\textsubscript{i}} {[\textsubscript{ApplP} e-a} {[\textsubscript{VP} t\textsubscript{i} muxi]-shun]-i]}] tae-a-hu.\\
			    my plate little-\sc{PL.ERG} me-\sc{acc} break-\sc{appl-ss.nom} begin-\sc{pfv-pl}\\
			    \glt `My little plates began to break on me.'
\end{exe}

The last refinement we consider is a closer look at the role of v heads in these constructions. If SS complements are FinPs, then they must contain vP projections as well, that being a lower projection in the clausal spine. But v is also a phase head, at least in active/agentive clauses like (\ref{BakerX4}) and (\ref{BakerX4}). The question is whether this spoils our aimed-for result, that ‘begin’ and ‘know’ constructions count as a single domain for processes like O=S switch reference and ergative case assignment. If so, there is a contradiction within our analysis.\\

For simple O=S switch reference, there is no real difficulty, given that we have adopted the PIC in (\ref{BakerX19}), where syntactic dependencies can cross over one phase boundary but not two. The crucial probe in the O=S construction is in the v head associated with ‘begin’ or ‘know’. Probing downward, it needs to see into the VP headed by ‘begin’ or ‘know’, its complement FinP, TP, vP, and the lower VP to find the direct object. Only one of these is a phase head, the embedded v. Therefore, there is no PIC violation in this structure, given (\ref{BakerX19}).\footnote{A different locality issue is how the v associated with the matrix verb can look past the null subject of the SS complement in order to find the object inside that complement. Perhaps the null subject is rendered invisible to further probing once it becomes the goal of the SS probe in T. We do not pursue this here.}\\

The role of v-induced phase boundaries is more of an issue for the assignment of ergative case. Dependent ergative case assignment happens at the spell out of the complement of matrix Force in our framework (see (\ref{BakerX12})). By that time, the matrix subject is in Spec TP, so it is separated from an in situ lower object by two v heads, the matrix one and the embedded one. Moreover, at least in the case of a `know’ construction like (\ref{BakerX4a}), both vs are active and agentive. So the matrix subject getting ergative case under the influence of the lower object should count as a PIC violation according to (\ref{BakerX19}).\\

Fortunately, we have a tool already in hand to address this issue, namely Baker’s \citeyearpar{baker2015case} claim that in some languages v is a soft phase head rather than a hard phase head. A hard phase head is the normal kind, which triggers the spell out of its complement \textit{and} removes it from the syntactic representation. In contrast, a soft phase head triggers the spell out of its complement, fixing many of the PF-oriented properties of elements inside the complement (e.g., word order, morphosyntactic features), but the complement is not actually removed from the representation. This distinction was introduced with languages like YW and SK in mind, where a direct object always triggers ergative on the subject, without the object having to leave the VP by a process of object shift. (In contrast; object shift is required for ergative case assignment in languages like Niuean and Nez Perce.) But two soft phase heads are no different from one soft phase head in this respect. We now interpret (\ref{BakerX19}) as saying that when a higher phase head Z is inserted, the phase head H triggers the spell out and removal of its complement if H is a hard phase head, and it triggers only spell out if H is a soft phase head. So in the structure in (\ref{BakerX21b}), when the matrix v is merged, the embedded v triggers spell out but not removal of its VP complement. Therefore the object DP inside that VP is still present to trigger ergative case on the matrix subject.\\

The view that v is a soft phase head also has a pay-off when it comes to more complex O=S structures. An anonymous reviewer points out that the soft-phase head analysis predicts that O=S in YW and SK should be able to reach into more than one SS-complement, in a potentially unbounded fashion. This is because these complements have only v-headed phases, and those are always soft phases, not hard ones. And indeed the prediction is correct, as (\ref{BakerX23}) is possible.

\begin{exe}
    \ex Yawanawa (fieldwork) \label{BakerX23}\\
			\gll [Ẽ [[wixi ane-kĩ] tae-wa-kĩ] xinãvenu-a], wixi venu-a.\\
			    I book read-\sc{ss} begin-\sc{ss} forget-\sc{OS} book disappear-\sc{pfv}\\
			\glt `I forgot to begin reading the book, and it disappeared.'
\end{exe}

Finally, we need to make sure that having a soft phase head v in the matrix clause is still enough to block the FinSS inside a full ForceP complement from finding the matrix subject as its goal, as our analysis requires. This result follows straightforwardly from the fact that C/Force is always a hard phase head \citep[149]{baker2015case}. Therefore, when the matrix v is merged, the FinP complement of Force is removed as well as spelled out. Therefore FinSS disappears from the structure before the matrix subject comes in. It never finds a goal, so SS marking is impossible in a full ForceP complement, as desired.\\

This analysis raises plenty of other questions, especially our move to the PIC in (\ref{BakerX19}). Much of our previous work was cast in a \citet{chomsky2000minimalist} type system, where one phase boundary gives impenetrability. For example, Baker’s \citeyearpar{chomsky2000minimalist} analysis of languages like Niuean and Nez Perce in which object shift is needed to trigger ergative case assignment on the subject needs to be recast. The same is true for B\&CS’s claim that O=S switch reference in Panoan cannot pick the object of an adposition or the possessor of a DP as its goal, equating it with the matrix subject, because P and D are phase heads which hide DPs in their domain from Agree coming from outside. Perhaps now it can be argued that there are really two phase heads in more articulated PP and DP structures. If so, little else will need to be changed. But we leave exploration of this to future research.\\

\section{General discussion and conclusion}

We close with a few more general remarks about the implications of this analysis for those who are not primarily concerned with the details of Panoan clause structure. There are important typological patterns concerning the interaction of switch-reference and complementation which our analysis begins to explain, we claim. It is well-known that SS-marking is more common on adjunct clauses than on complement clauses—an asymmetry that Finer (\citeyear{finer1984formal}, \citeyear{finer1985syntax}) already wrestled with. This is seen clearly in McKenzie’s (\citeyear{mckenzie2015survey}) survey of almost 70 North American languages with switch-reference systems. He lists some 29 languages that have SR marked on adjunct clauses but not complement clauses, but only one (Mikasuki) that might have SR marked on complement clauses but not on adjunct clauses. There seems to be a robust implicational universal here: a language has SR on complements only if it has SR on adjuncts. We have a general strategy for explaining this: SS involves Agree with the matrix subject, and it is easier for adjunct clauses outside of vP to do that than for complement clauses inside vP, given the PIC.\\

We have also made progress on a second order effect in this domain. For some languages in which SR is primarily a property of adjuncts, it creeps into the realm of complements a little bit, but not very far. McKenzie thus observes that SS marking can be used in auxiliary constructions as well. This is where SK fits in, if ‘begin’ constructions are counted as a type of auxiliary construction (see also note 2). YW is similar but expands the domain of SS somewhat further, to ‘know’ type constructions. Other languages that allow SS on adjuncts and in auxiliary constructions, but not on complements more broadly, are the Yuman languages Cocopa, Hualapai, and Yavapai, according to McKenzie’s survey. In B\&CS, we conjectured that the Panoan generalization is that SS constituents cannot receive thematic roles, perhaps because they are not nominal enough to do so. But that now seems wrong, given CS’s discovery of the ‘know’ construction in YW, where an SS-constituent is an alternative to a nominal clause (see (\ref{BakerX4}) and (\ref{BakerX5b})). These SS-constituents presumably receive the same thematic role that the nominal clauses do. We now suggest that a more accurate generalization is that SS-constituents can be complements only in reduced (non-ForceP) constructions. And we claim that this follows from the need of the SS head to Agree with the matrix subject.\\

There is of course more to do in order to fully explain the typological distribution of SR constructions. For example, there are languages that have SR marked on a fuller range of complement clauses as well as on adjunct clauses. \citet{mckenzie2015survey} lists about 13 of these in North America, including Muskogean languages, some Yuman languages, and the Ute/Paiute cluster. Take for example the Muskogean language Choctaw. It differs from YW and SK in two ways, illustrated in (\ref{BakerX23}). First, an SS complement is possible even with a verb like ‘think’, as in (\ref{BakerX24a}), where ‘think’ is not capable of taking a small complement in most languages \citep{wurmbrand2001infinitives}. Second, Choctaw allows different subject (DS) complement clauses as well as SS complements, as in (\ref{BakerX23}). This is not attested in YW and SK: there is nothing like [Shaya [\textsubscript{CP} Shukuvena fish cook-DS] know], meaning “Shaya knows how/that Shukuvena can/should cook fish.” A fuller account wants to understand these differences as well.

\begin{exe}
    \ex Choctaw \citep[269]{broadwell2006choctaw} \label{BakerX24}
	    \begin{xlist}
			\ex \label{BakerX24a}
			\gll John-at anokfilli-h [pisachokma-ka-t].\\
			    John-\sc{nom} think-\sc{tns} good.looking-\sc{comp-ss}\\
			    \glt `John\textsubscript{i} thinks he\textsubscript{i} is goodlooking.'
			\ex \label{BakerX24b}
			\gll John-at anokfilli-h [pisachokma-ka-N].\\
    			 John-\sc{nom} think-\sc{tns} good.looking-\sc{comp-ds}\\
			    \glt `John\textsubscript{i} thinks he\textsubscript{k} is goodlooking.'
		\end{xlist}
\end{exe}

As to why (\ref{BakerX24a}) is possible in Choctaw but not in Panoan, several hypotheses come to mind. CS proposes one that fits the details of Choctaw well, saying that the upward probing head in SS constructions is Force in Choctaw, whereas in Panoan it is Fin+T. This coheres with the fact that the SS marker \textit{-t} does not replace other T and C morphology, the way that it does in Panoan; rather, it attaches outside of \textit{ka}, arguably a Fin head. Now if the SS probe is at the edge of CP phase in Choctaw, then it can agree with the matrix subject only crossing one phase boundary (the matrix v). This is consistent with the PIC in (\ref{BakerX19}). There are other possibilities as well, and future research will need to sort out which ones might work for which languages.\footnote{Another language with SS marked on complement clauses is Washo, and \citet{arregi2019switch} develop an analysis that is very similar to ours of Choctaw. However, the SS marker in Washo is not right at the edge of the complement, as it is in Choctaw; rather the CP is embedded in a DP layer. Arregi and Hanink claim that D is not a phase head (nor is v) so this has no effect, but we are not entirely comfortable with this assumption and might entertain alternatives.}\\

As to why a DS complement is not possible in Panoan, that needs a different story, and it may not be a very deep one. According to B\&CS, DS clauses do not enter into the same kind of Agree relationship with the higher and lower subjects as SS clauses do. They are just ordinary ForcePs which fail to receive a certain kind of interpretation when an SS clause is possible and is dedicated to expressing that interpretation—a pragmatic blocking account. If that is right, then an extra phase boundary here or there should not be too relevant to the distribution of DS clauses. Here we would just give a surface morphological account for Panoan: C/Force happens to be spelled out as \textit{-kẽ} and \text{-nũ} (the so-called DS forms) in adjunct clauses, but not in complement clauses, as a kind of contextually determined allomorphy. (This might in turn be related to category features: full clausal complements need to be nominal to get a thematic role and clausal adjuncts need to not be. Then the adjectival Force is \textit{-kẽ/-nũ}, but the nominal Force is \textit{-tũ}.) How well this line of thinking holds up crosslinguistically is yet another topic for future research.\\

In conclusion, one important lesson we have learned from Susi Wurmbrand’s career to date is how much there is find out about the topic of complementation if one investigates it crosslinguistically and with a high attention to detail. Here we have shown that the same is true for the topic of same subject clauses. There is even an important interaction between the two topics, such that SS constructions can be complements in reduced clause constructions but not otherwise.

\is{Cognition} %add "Cogntion" to subject index for this page

\section*{Abbreviations}

The following abbreviations are used in glosses: \sc{ACC} = accusative, \sc{APPL} = applicative, \sc{CAUS} = causative, \sc{COMP} = complementizer, \sc{DES} = desiderative, \sc{DS} = different subject, \sc{ERG} = ergative, \sc{EV} = evidential, \sc{F} = feminine, \sc{INF} = infinitive, \sc{INT} = interrogative, \sc{IPFV} = imperfective, \sc{NMLZ} = nominalizer, \sc{NOM} = nominative, \sc{OS} = object-to-subject SR, \sc{PFV} = perfective, \sc{PL} = plural, \sc{SG} = singular, \sc{SS} = same subject.

\printbibliography[heading=subbibliography,notkeyword=this]

\end{document}