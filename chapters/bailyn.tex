\documentclass[output=paper,colorlinks,citecolor=brown,
% hidelinks,
% showindex
]{langscibook}
\author{John Frederick Bailyn\affiliation{Stony Brook University}\orcid{}}
\title{How strict should Cartography be? A view from Slavic}
\abstract{This article addresses one of the core questions in syntactic theory – the architecture of the (universal) tree. The question is crucial when we consider the recent rise of strict syntactic Cartography (\citealt{Rizzi1997}, \citealt{Cinque1999}, \citealt{CinqueRizzi2010}), a research program that seeks to map syntactic structure in as highly an atomized and universal a fashion as possible. In this article, I present two different arguments from Slavic syntax that the strictest current hypothesis about syntactic Cartography is insufficiently flexible to account for the various phenomena under consideration. In particular, I look at multiple WH-movement in Slavic, and the representation of Topic/Focus through word order and prosody. Both areas pose significant problems for strict Cartography, and the case studies should be taken seriously as important empirical tests for a theory that posits a fixed universal and extensive set of syntactic primitives with the additional restrictions imposed by considerations of anti-symmetry, such as single specifier positions and no free process of adjunction. In conclusion I discuss the appropriate place for Cartography in syntax, and the connection to syntactic features.}
  
%move the following commands to the "local..." files of the master project when integrating this chapter
\usepackage{tabularx}
\usepackage{langsci-basic}
\usepackage{langsci-optional}
\usepackage{langsci-gb4e}
\bibliography{localbibliography}
\newcommand{\orcid}[1]{}
\pagenumbering{roman}
\usepackage[linguistics]{forest}
\usepackage{xcolor}
\usepackage{multicol}
\usepackage{multirow}

%%tikz
\usepackage{textgreek}
\usepackage{tikz}
\usepackage{pgf}
\usetikzlibrary{backgrounds, matrix, positioning}
\usepackage{tipa}

\forestset{
fairly nice empty nodes/.style={
delay={where content={}
{shape=coordinate, for siblings={anchor=north}}{}},
for tree={s sep=4mm}
}
}

\newcommand{\rcommentg}[1]{\hfill\raisebox{1.9\baselineskip}[0pt][0pt]{#1}}


\IfFileExists{../localcommands.tex}{%hack to check whether this is being compiled as part of a collection or standalone
   % add all extra packages you need to load to this file
\usepackage{tabularx,multicol}
\raggedcolumns

\usepackage{url}
\urlstyle{same}

\usepackage{listings}
\lstset{basicstyle=\ttfamily,tabsize=2,breaklines=true}

\usepackage{langsci-basic}
\usepackage{langsci-optional}
\usepackage{langsci-lgr}

% Wood's chapter
\usepackage{jimsstyle}

\usepackage{subcaption}
\usepackage{color,colortbl}
\definecolor{Gray}{gray}{0.8}%to get a color for the table
\usepackage{expex}
\usepackage[linguistics]{forest}
\forestset{qtree edges/.style={for tree={parent anchor=south, child anchor=north}}}
\usepackage{graphbox}
\usepackage{leipzig}
\usepackage{multirow}
\usepackage{needspace}
\usepackage{newunicodechar}
\usepackage{pifont}
\usepackage{pgf}
\usepackage{pst-node}
\usepackage{qtree}
  \qtreecenterfalse
\usepackage{rotating}
% \usepackage{skak}
\usepackage{soul}
\usepackage{tabto}
\usepackage{textgreek}
\usepackage{tikz}
\usetikzlibrary{calc,decorations,shapes.misc,decorations.pathreplacing,arrows}
\usetikzlibrary{backgrounds, matrix, positioning}
\usetikzlibrary{}
\usetikzlibrary{}
\usepackage{tikz-qtree,tikz-qtree-compat}
\usepackage{tipa} % must be imported before linguex, but we do not use linguex here
\usepackage{todonotes}
\usepackage{tree-dvips}
\usepackage[normalem]{ulem}%for strikeout
\usepackage[Symbolsmallscale]{upgreek} 
\usepackage{verbatim}
\usepackage{xcolor}
\usepackage{xspace}
% \usepackage{xtab}

\usepackage{langsci-gb4e}

   \newcommand{\orcid}[1]{}


% pesetskycommands.tex
%ExPex stuff

\newcommand{\pexcnn}{\pex[exno={~},exnoformat={X}]}  % pexc with no examplenumber and no increment
\lingset{everygla=\normalfont,{aboveglftskip=0pt}}
\newcommand{\ptxt}[1]{#1\par}
\let\expexgla\gla
\AtBeginDocument{\let\gla\expexgla\gathertags}

%other stuff
\newunicodechar{⟶}{{\symbolfont{⟶}}}
%\newunicodechar{✓}{{\Checkmark}} % checkmark
\newunicodechar{✓}{{\ding{51}}} % checkmark
\newcommand{\gap}{\ \underline{\hspace{10pt}\phantom{x}}\ }
\newcommand{\ix}[1]{\textsubscript{\it{#1}}}  % subscript

\def\gethyperref#1{\hyperlink{#1}{\getref{#1}}}%
\def\getfullhyperref#1{\hyperlink{#1}{\getfullref{#1}}}%

\newcommand{\denote}[1]{⟦{#1}⟧}

   %% hyphenation points for line breaks
%% Normally, automatic hyphenation in LaTeX is very good
%% If a word is mis-hyphenated, add it to this file
%%
%% add information to TeX file before \begin{document} with:
%% %% hyphenation points for line breaks
%% Normally, automatic hyphenation in LaTeX is very good
%% If a word is mis-hyphenated, add it to this file
%%
%% add information to TeX file before \begin{document} with:
%% %% hyphenation points for line breaks
%% Normally, automatic hyphenation in LaTeX is very good
%% If a word is mis-hyphenated, add it to this file
%%
%% add information to TeX file before \begin{document} with:
%% \include{localhyphenation}
\hyphenation{
affri-ca-te
affri-ca-tes 
agree-ment
anaph-o-ra
an-ti-caus-a-tive
an-ti-caus-a-tives
an-te-ced-ent
Bak-er
caus-a-tive
Christo-poulos
clas-si-fi-er
claus-al
Co-lum-bia
com-ple-ment-izer
com-ple-ments
con-tin-u-a-tive
de-di-ca-ted
De-mo-cra-tas
Dor-drecht
du-ra-tive
Ex-folia-tion
ex-tra-gram-mat-ical
fi-nite-ness
ger-und
ger-unds
Gro-ninger
in-trac-ta-ble
Jap-a-nese
judg-ment
Judg-ment
Lysk-awa
Ma-khach-ka-la
Ma-rantz
Mat-thew-son
Max-Share
merg-er
Pap-u-an
Per-elts-vaig
post-ver-bal
phe-nom-e-non
pre-dic-tion
Pre-dic-tion
Rich-ards
Sa-len-ti-na
to-pic
Ya-wa-na-wa
Wil-liam-son
}

\hyphenation{
affri-ca-te
affri-ca-tes 
agree-ment
anaph-o-ra
an-ti-caus-a-tive
an-ti-caus-a-tives
an-te-ced-ent
Bak-er
caus-a-tive
Christo-poulos
clas-si-fi-er
claus-al
Co-lum-bia
com-ple-ment-izer
com-ple-ments
con-tin-u-a-tive
de-di-ca-ted
De-mo-cra-tas
Dor-drecht
du-ra-tive
Ex-folia-tion
ex-tra-gram-mat-ical
fi-nite-ness
ger-und
ger-unds
Gro-ninger
in-trac-ta-ble
Jap-a-nese
judg-ment
Judg-ment
Lysk-awa
Ma-khach-ka-la
Ma-rantz
Mat-thew-son
Max-Share
merg-er
Pap-u-an
Per-elts-vaig
post-ver-bal
phe-nom-e-non
pre-dic-tion
Pre-dic-tion
Rich-ards
Sa-len-ti-na
to-pic
Ya-wa-na-wa
Wil-liam-son
}

\hyphenation{
affri-ca-te
affri-ca-tes 
agree-ment
anaph-o-ra
an-ti-caus-a-tive
an-ti-caus-a-tives
an-te-ced-ent
Bak-er
caus-a-tive
Christo-poulos
clas-si-fi-er
claus-al
Co-lum-bia
com-ple-ment-izer
com-ple-ments
con-tin-u-a-tive
de-di-ca-ted
De-mo-cra-tas
Dor-drecht
du-ra-tive
Ex-folia-tion
ex-tra-gram-mat-ical
fi-nite-ness
ger-und
ger-unds
Gro-ninger
in-trac-ta-ble
Jap-a-nese
judg-ment
Judg-ment
Lysk-awa
Ma-khach-ka-la
Ma-rantz
Mat-thew-son
Max-Share
merg-er
Pap-u-an
Per-elts-vaig
post-ver-bal
phe-nom-e-non
pre-dic-tion
Pre-dic-tion
Rich-ards
Sa-len-ti-na
to-pic
Ya-wa-na-wa
Wil-liam-son
}

    \bibliography{localbibliography}
    \togglepaper[23]
}{}

%custom footer for preprints
\papernote{\scriptsize\normalfont
    Change author.
    Change title.
    To appear in:
    Change Volume Editor.  
    Change volume title.  
    Berlin: Language Science Press. [preliminary page numbering]
}

\begin{document}
\maketitle

\section{Introduction}
Syntactic Cartography “is a research topic asking the question: what are the right structural maps for natural language syntax?” (\citealt{CinqueRizzi2010}: 51). The answer given by Cinque, Rizzi and others (esp. in \citealt{Cinque1999}, \citealt{CinqueRizzi2010}, Rizzi \& Cinque 2016), is what I refer to as “Strict Cartography”, and is comprised of five basic tenets, given directly below (the order of presentation is for expository purposes only). The primarily focus of this article is on the third, fourth and fifth of these, though the first two are also important and issues they raise have been discussed elsewhere in the literature, (see the articles in van \citealt{Craenenbroeck2009}). 

First, Strict Cartography proposes that every piece of morphology has unique syntactic status, and corresponds to a single category head in syntactic structure, typically with unique interpretation. This is referred to as the \textit{One Feature One Head} principle: 


\begin{exe}
\ex \label{bai1}
\emph{One Feature One Head} (OFOH) \\
\emph{Each morphosyntactic feature corresponds to an independent  syntactic head with a specific slot in its functional hierarchy.} (\citealt{CinqueRizzi2010}: 52)
\end{exe}

Second, the base order of the syntactic categories is universal and their underlying order fixed. The evidence in favor of universal fixed order takes the form of transitivity arguments: If A > B and B > C, then A > C, (as argued to hold for adverbs in a range of languages in \citealt{Cinque1999} for the TP-internal adverbial domain). For arguments that call into question the universality of a single underlying order, see \citealt{Bobaljik1999} and \citealt{Nilsen2003} among others. 

Third, multiple specifier positions associated with a single syntactic head are disallowed. \\ 

Fourth, there is no adjunction. Combined with the unavailability of multiple specifiers, this entails that every left edge position is associated with a unique functional head in the functional hierarchy. 

Fifth, surface word order patterns that do not reflect the universal hierarchy are derivable by a limited set of derivational options – (i) head movement, (ii) phrasal movement into specifier positions (including remnant movement of phrases with sub-extraction “gaps” and “roll-up” movements). Movements into such positions are typically motivated by feature requirements of the relevant heads, though it is possible to accommodate non feature-driven movements \footnote{A reviewer points out that nothing about the postulation of an extensive set of “cartographic” heads entails disallowing multiple specifiers or adjunction; rather, these restrictions follow from a strong anti-symmetric version of cartography (following Kayne 1994). However, \citealt{CinqueRizzi2010} explicitly motivate the minimalist and anti-symmetric nature of the cartographic program, and these extensions have become integral to Strict Cartography. Naturally, any theory involving functional heads is in some sense a cartographic mapping of syntactic structure; what distinguishes Strict Cartography are the universality of the posited syntactic heads, and the limitation of phrasal positions to being complements or unique specifiers of each head. This version of Cartography, without multiple specifiers or adjuncts, is what I am concerned with in this article.}.  
Cartographic and non-cartographic approaches to the (traditional) CP domain are shown in (\ref{bai2})-(\ref{bai3}): 

%TREE 2
\begin{exe}
\ex \label{bai2} 
Cartographic: (\citealt{rizzi1997}) \\
\begin{forest}
[ForceP 
[Force] [TopicP
[XP_{[\textsc{top}]}][Topic'
[Topic] [FocusP
[XP_{[\textsc{foc}]}][Focus'
[Focus] [TopicP
[XP_{[\textsc{top}]}][Topic'
[Topic] [FinitenessP
[XP_{[\textsc{fin}]}][Fin'
[Fin] [IP/TP, 
name=t[\textcolor{white}{triangle??}, roof]  ]]]]]]]]]]
    
\end{forest}
\end{exe}

%TREE 3 
\begin{exe}
\ex \label{bai3} 
Cartographic: (\citealt{Rizzi1997}) \\
\begin{forest}
[CP 
[XP] [C'
[C$^{0}$] [TP
[(adjunct)] [TP [YP] [T' [T^{0}] [vP]]] [(adjunct)] ]]]
\end{forest}
\end{exe}

\section{Multiple wh-movement}
Many Slavic languages show multiple WH-fronting (\cite{Rudin1988}). As was observed by \citeauthor{Rudin1988} and discussed in much subsequent work (esp. \citealt{Richards.Norvin1997}, \citealt{́Boskovic1997}, \citeyear{Boskovic2002}, ao), Slavic (and other) MWF languages seem to pattern into two kinds – those such as Bulgarian, in which the WH elements appear in fixed order, obey superiority, and act as an uninterrupted cluster, and those such as Serbo-Croatian or Russian, in which the order appears free, but the first WH element stands apart, with clitics, parentheticals and other elements able to follow it, preceding the subsequent wh-elements. We will take Bulgarian to represent the former type, and Serbo-Croatian, the latter type. (\ref{bai4})-(\ref{bai5}) show the basic properties of Bulgarian and Serbo-Croatian possibilities respectively: \\

\textbf{Multiple overt WH-movement}

\begin{exe}
\ex \label{bai4}
\begin{xlist}
\ex \label{bai4a}
\gll \emph{Koj}		\emph{koga}		\emph{vižda}? \jambox{}(BG) \\
who_{\textsc{nom}} whom_{\textsc{acc}} sees\\
\glt `Who sees who?'


\ex \label{bai4b}
\gll *\emph{Koga}		\emph{koj}		\emph{vižda}? \jambox{}(BG) \\
who_{\textsc{acc}} whom_{\textsc{nom}} sees\\
\end{xlist}
\end{exe}

\begin{exe}
\ex \label{bai5}
\begin{xlist}
\ex \label{bai5a}
\gll \emph{Ko}		\emph{koga}		\emph{vidi}? \jambox{}(SC)\\
who_{\textsc{nom}} whom_{\textsc{acc}} sees\\

\ex \label{bai5b}
\gll \emph{Koga}		\emph{ko}		\emph{vidi}? \jambox{}(SC) \\
who_{\textsc{nom}} whom_{\textsc{acc}} sees\\
\end{xlist}
\end{exe}

In (\ref{bai4a})-(\ref{bai4b}) we see that the Bulgarian type obeys superiority among its fronted wh-elements, whereby fronted wh-subjects must precede fronted wh-objects, something the acceptability of both orders in (\ref{bai5}) shows is not the case in Serbo-Croatian. The two language types also differ in terms of the cluster-like behavior of the fronted wh-elements in the Bulgarian type, but not the Serbo-Croatian type, as shown in (\ref{bai6})-(\ref{bai7}):

\begin{exe}
\ex \label{bai6}
\begin{xlist}
\ex \label{bai6a}
\gll \emph{Koj}		\emph{koga}	\emph{e}	\emph{vidjal}? \hspace{5.5cm}(BG) \\
who_{\textsc{nom}} whom_{\textsc{acc}} aux seen\\
\glt `Who saw whom?'

\ex \label{bai6b}
\gll *\emph{Koj}	\textbf{\emph{e}}	\emph{koga}		\emph{vidjal}? \hspace{5.5cm}(BG) \\
who_{\textsc{nom}} aux whom_{\textsc{acc}} seen\\
\end{xlist}
\end{exe}

\begin{exe}
\ex \label{bai7}
\begin{xlist}
\ex \label{bai7a}
\gll \emph{Ko}	\emph{\textbf{je}} \emph{koga}		\emph{vidio}? \jambox{}(SC) \\
who_{\textsc{nom}} aux whom_{\textsc{acc}} seen\\

\ex \label{bai7b}
\gll *\emph{Ko}		\emph{koga}	 \emph{\textbf{je}}	\emph{vidio}? \jambox{}(SC) \\
who_{\textsc{nom}} whom_{\textsc{acc}} seen\\
\end{xlist}
\end{exe}

In (\ref{bai6a}) we see that Bulgarian auxiliary clitics (or any other material, including pronominal clitics) must follow all wh-phrases, and cannot intervene among them as in (\ref{bai6b}). The opposite holds in Serbo-Croatian where such elements follow the first wh-element (\ref{bai7a}) and cannot follow all of them (\ref{bai7b}). 

\begin{exe}
\ex \label{bai8}
\begin{xlist}

\ex \label{bai8a}
\gll \emph{Ko}		\emph{šta}		\emph{gdje}	  \emph{kupuje}? \hspace{5cm}(SC) \\
who_{\textsc{nom}} what_{\textsc{acc}} where buys\\

\ex \label{bai8b}
 *\emph{Ko-Nom}	    \emph{kupuje}   \emph{šta-Acc}	\emph{gdje}	 \\

\ex \label{bai8c}
 *\emph{Ko-Nom}  \emph{šta-Acc} \emph{kupuje}  	\emph{gdje}	 \\

\ex \label{bai8d}
*\emph{Ko-Nom}  	\emph{gdje}	\emph{kupuje}   \emph{šta-Acc} \\

\end{xlist}
\end{exe}

Both properties (superiority and clustering) have been attributed in the Bulgarian case to the ability (indeed the necessity) to have either a multiply filled SpecCP position (\citealt{Rudin1988}, \citealt{Grewendorf2001}, Bailyn 2018) or multiple specifiers of single C$^{0}$ head (\citealt{Rudin1988}, \citealt{Bošković1997}, \citeyear{Bošković2002}).  \\
\citeauthor{Rudin1988}’s initial proposal for the various kinds of WH-fronting languages, including English, is shown here:

%TREE 9 
\begin{exe}
\ex \label{bai9}
\citet{Rudin1988}'s structure for multiple WH (here "K") languages:

\begin{multicols}{3}


\begin{forest}
[S'
[Comp
[Comp
[Comp[K]][K]
][K]
][S]
]
\caption{Bulgarian type}
\end{forest}





\vfill\null
\columnbreak

\begin{forest}
[S'
[Comp[K]]
[S
[K][S
[K][S, 
name=t[\textcolor{white}{tri}, roof] ]
]
]
]
\caption{Serbo-Croatian type}

\end{forest}

\vfill\null
\columnbreak

\begin{forest}
[S'
[Comp[K]
][S, 
name=t[KK, roof] ]
]
\caption{English type }

\end{forest}



\end{multicols}


\end{exe}

The Bulgarian type, labeled by \citeauthor{Rudin1988} as [+Multiply Filled Spec] or [+MFS], is shown on the left, with a multiply-filled Comp position (the equivalent of SpecCP). Lack of intervening clitics and other cluster-like behavior derives from the [+MFS] property. In the Serbo-Croatian type, conversely, the first wh-phrase fills SpecCP, as it does in English, while all of the other wh’s obligatorily cluster on the left edge of S (IP/TP), as shown above. This accounts for the position of clitics after the first wh-phrase (presumably being located in the C$^{0}$ position), while parentheticals also intervene being on the far left edge of IP.  
The distinction between the two kinds of multiple wh-movement languages was later recast by \citeauthor{Richards.Norvin1997} (\citeyear{Richards.Norvin1997}, 2001) and much subsequent work as the difference between a set of multiple Spec positions on the CP edge, for Bulgarian, and on the IP edge, for Serbo-Croatian, as shown here:


%TREE10 
\begin{exe}
\ex
\begin{xlist}
\ex "CP-absorption" (Bulgarian, Chinese) 

\begin{forest}
[CP
[WH$_1$] [CP
[WH$_2$] [C'
[C$^{}0$] [IP] 
]]]
\end{forest}

\ex "IP-absorption" (Serbo-Croatian, Japanese) 

\begin{forest}
for tree=nice empty nodes
[CP 
[{}] [C' 
[C$^0$] [IP
[WH$_1$] [IP
[WH$_2$] [I
[I$^0$] [ 
{}]  
]]]]]
\end{forest}
\end{xlist}
\end{exe}

Thus \citeauthor{Rudin1988}’s initial intuition of CP-level and IP-level left attachment is maintained prominently by \citet{Richards.Norvin1997}, who labels the types CP-absorption” and “IP-absorption”.  Crucially, both language-types involve multiple specifiers \footnote{ Various complex analyses have been developed in the syntactic literature to account for the properties of the two kinds of Multiple wh movement languages. The accounts vary and have varying advantages, but all share the conclusion that we cannot analyze the surface position of the Bulgarian wh-elements as being specifiers of distinct heads, as would be required by Strict Cartography. The head-like nature of auxiliary and pronominal clitics, on a strict cartographic approach, indicates that the elements to the left must be located in multiple specifiers of a single head.}.   

Besides cluster interruption, the Serbo-Croatian/Russian kind has another well-known property, namely, lack of superiority among all the wh-elements. Thus, in Russian, as we have already seen for Serbo-Croatian, any order of the multiple wh elements is essentially equally acceptable, here seen with 3 elements:

\begin{exe}
\ex \label{bai11}

\begin{xlist}
\ex \label{bai11a}
\gll \emph{Kto}		\emph{kogo}		\emph{komu}		\emph{predstavil}? \hspace{4.5cm}(Rus) \\
who_{\textsc{nom}} who_{\textsc{acc}} who_{\textsc{dat}} introduced \\
\glt `Who introduced who to whom?'

\ex \label{bai11b}
\gll \emph{Kto}			\emph{komu}	\emph{kogo}		\emph{predstavil}? \\
who_{\textsc{nom}}  who_{\textsc{dat}} who_{\textsc{acc}} introduced \\

\ex \label{bai11c}
\gll \emph{Kogo}	\emph{kto}			\emph{komu}		\emph{predstavil}? \\
who_{\textsc{acc}} who_{\textsc{nom}}  who_{\textsc{dat}}  introduced \\

\ex \label{bai11d}
\gll \emph{Kogo}		\emph{komu}		\emph{kto}		\emph{predstavil}? \\
who_{\textsc{acc}} who_{\textsc{dat}}  who_{\textsc{nom}}  introduced \\

\ex \label{bai11e}
\gll \emph{Komu}	\emph{kto}		\emph{kogo}		\emph{predstavil}? \\
who_{\textsc{dat}}  who_{\textsc{nom}} who_{\textsc{acc}}  introduced \\

\ex \label{bai11f}
\gll \emph{Komu}	\emph{kogo}	\emph{kto}		\emph{predstavil}? \\
who_{\textsc{dat}}  who_{\textsc{acc}}   who_{\textsc{nom}} introduced \\

\end{xlist}
\end{exe}

Lack of superiority is accounted for, typically, by the equidistant property of multiple specifier positions. All wh’s cluster on the left edge of IP, as multiple adjuncts or specifiers, and are thus all equidistant from the true wh position, SpecCP, to which any one of them can raise. Were each wh-phrase to occupy a distinct Specifier of a distinct head, we would not expect equidistance to apply, and would therefore expect ordering asymmetries that we do no find. In Bulgarian, on the other hand, where there seems to clearly be a strong requirement that subjects (canonical agents) precede all other wh-phrases, the structural superiority of the subject is maintained derivationally, either as an order-preservation effect (Bailyn 2018), or through the device of Tucking-in (Richards 1997).  In either account multiple specifiers are needed to account for the strict clustering properties. 

Thus, all analyses that cover the wide range of facts involved in these two kinds of languages rely on multiple specifiers or multiple adjunction as a crucial component of the analysis. (\citealt{Rudin1988}, \citealt{Richards.Norvin1997}, \citealt{Bošković1997}, \citeyear{Bošković2002}). 

For strict cartographic approaches these construction appear to cause intractable problems. Naturally, a left-peripheral approach would allow for multiple landing sites for these elements. Thus, on such an approach, we might expect something like (\ref{bai12}):

%TREE 12 
\begin{exe} \label{bai12}
\ex possible cartographic distribution of landing sites for 3 fronted WH phrases:


\begin{forest}
%fairly nice empty nodes,
%for tree={inner sep=0, l=0}
for tree=nice empty nodes
[XP
[\textcolor{white}{space}][
[X$^0$][TopP
[Spec,name=spec1][
[Top$^0$]
[FocP
[Spec,name=spec2][
[Foc$^0$]
[FinP
[Spec,name=spec3][
[Fin$^0$]
[TP
[<WH$_1$>,name=wh1] [
[T$^0$] 
[vP
[<WH$_2$>,name=wh2][v'
[v$^0$] [VP
[V$^{0}$][<WH_3>,name=wh3]]]]]]]]]]]]]]]
\draw[->] (wh1) to[out=west,in=south] (spec1);
\draw[->] (wh2) to[out=west,in=south] (spec2);
\draw[->] (wh3) to[out=west,in=south] (spec3);


\end{forest}


\end{exe}

Any such analysis would face strong empirical hurdles. (The following difficulties do not constitute definitive evidence against such an approach, however they do shift the burden of proof to the cartographic analyses, to both descriptively handle the situation and motivate the derivational steps required to make it work.) First, such an approach would predict specific interpretations to be associated with the various fronted wh elements. The literature does not report any such effects, other than the general tendency for d-linked wh-phrases or precede non d-linked ones, discussed below. However, that зis a binary division, and yet multiple wh-fronting of more than two elements is common, and there is no evidence of any consistent semantic distinction among them. Second, it would not predict any clustering effects (contrary to fact). Third, it would have no obvious explanation for the lack of superiority generally found in the Russian/Serbo-Croatian type \footnote{The story would also have difficulty with the well-known wh-island-voiding behavior  seen in some dialects of Bulgarian (\citealt{Rudin1988}, \citealt{Richards.Norvin1997}).}.  

One attempt to apply a cartographic approach to the multiple wh situation in Slavic is found in \citeauthor{KrapovaCinque2005} (\citeyear{KrapovaCinque2005}, \cite{krapova2008order}) who seek distinctions among fronted wh-phrases that might be clues to their positioning. Their results clearly show that D-linked and non D-linked elements are often uniquely ordered with respect to each other, resulting in the tendencies listed here:

%TABLE how do i even read this table wtf 
\begin{exe}
\ex \label{bai13}
\citeauthor{KrapovaCinque2005} (\citeyear{KrapovaCinque2005}:189) summary of Bulgarian WH orders: 
\begin{table}[]
\begin{tabular}{|l|l|l|l|l|l|l|l|}
\hline
\multicolumn{1}{|c|}{\textbf{D-linked wh-$^12$}}                                                                                                                                             & \multicolumn{7}{c|}{\textbf{Non-D-linked wh-phrases}}                                                                                                                                                                                                                                                                                                                                                                        \\ \hline
\multirow{4}{*}{\textit{\begin{tabular}[c]{@{}l@{}}koj/koja/koe/koi (N)\\ (kogo)\\ (na kogo)\\ (marked) kakvo\\ (marked) k\textbackslash{}\textasciicircum{}ade/koga\end{tabular}}} & \multirow{4}{*}{\textit{kogo}} & \multirow{4}{*}{\textit{\begin{tabular}[c]{@{}l@{}}na\\ kogo\end{tabular}}} & \multirow{4}{*}{\textit{koga}} & \multirow{4}{*}{\textit{k\textbackslash{}\textasciicircum{}ade}} & \multirow{4}{*}{\textit{\begin{tabular}[c]{@{}l@{}}kakvo\\ kolko N\end{tabular}}} & \multirow{4}{*}{\textit{\begin{tabular}[c]{@{}l@{}}kakvo\\ (na)kolko N\end{tabular}}} & \multirow{4}{*}{\textit{kak}} \\
                                                                                                                                                                                    &                                &                                                                             &                                &                                                                  &                                                                                   &                                                                                       &                               \\
                                                                                                                                                                                    &                                &                                                                             &                                &                                                                  &                                                                                   &                                                                                       &                               \\
                                                                                                                                                                                    &                                &                                                                             &                                &                                                                  &                                                                                   &                                                                                       &                               \\ \hline
\end{tabular}
\end{table}
\end{exe}

\citeauthor{KrapovaCinque2005} do not provide any individual derivations, but for a case of two wh-phrases, one of which is D-linked and the other not, it  is clear they assume distinct functional category specifier positions as landing sights, consistent with discourse roles of certain phrases in the left-periphery. They demonstrate clearly that there are discourse preference associated with the orders found, in line with general information structure principle of given > new, to be discussed in Section 3 below. However, it is not possible to determine if C&K are committed to a fully cartographic analysis in which the various wh elements within each general zone (D-linked; non-D linked), that is, whether each lands in a distinct specifier position, or if they assume multiple specifiers. In either case, the analysis would still have to account for the freedom of ordering among non-initial WH elements, and the inability for the entire WH-cluster to be broken up (something usually attributed to sharing of a single specifier position). \\
Furthermore, some sort of Superiority seems to be at play among non-D-linked wh elements – a structure-preserving effect. Once D-linking is controlled for, \citeauthor{KrapovaCinque2005} conclude that “various clues seem to suggest that [surface wh] ordering reflects the order of wh-phrases prior to wh-movement” (p. 189). They analyze such structure preservation as resulting a chain-sensitive form of Relativized Minimality, but do not address the issue of the landing site of non D-linked wh elements. Clearly, for surface positions to line up with base-positions, the lower left periphery would either have to replicate the argument domain (a highly unlikely state of affairs, and one that is inconsistent with cartographic views of the left-periphery since \citealt{Rizzi1997}). Rather, it is generally assumed (Richards 1997 and much subsequent work) that there exist multiply filled or multiple specifiers, which are equidistant, allowing apparent lack of superiority effects. Even if Bulgarian shows a higher rigid left wh order, either multiple specifiers would have to be allowed, or a highly flexible set of functional categories would have to exist, both contradicting basic tenets of Strict Cartography. \\
Finally, \citealt{KrapovaCinque2005} also do not address how the wh elements form a syntactic cluster, which is well-known (\citealt{Grewendorf2001}, \citealt{Bailyn2017}). If, to account for cluster-like behavior, such an account would posit multiple specifiers of a single head, then even this partly cartographic approach would be forced to abandon the strong cartographic requirement of one head - one specifier to accommodate multiple occurrences in varying order, within each larger set. If not, clustering appears to remain a mystery within Strict Cartography. \\
Either way, a fully cartographic account of the landing sites of multiple wh movement structures without recourse to adjuncts or multiple specifiers has yet to be provided.  And this is for the one Slavic language that shows some degree of rigidity in the wh left periphery. For those with none, such as Serbo-Croatian/Russian, a cartographic approach remains feasible, but would require extensive manipulation of the functional hierarchy, since each attested order, with identical semantics, (other than scope), would have to be derived through a highly intricate set of (unmotivated) remnant movements. The burden of proof thus remains with the strict cartographic approaches to provide an analysis that maintains the tenets of Strict Cartography discussed above, while at the same time accounting for the clustering and superiority properties of these well-known multiple-wh language types.

\section{Topic/Focus structure}

As is well-known, Slavic free word order sentences often follow traditional “Theme-Rheme” structure, now sometimes equated with Topic>Focus structure. Traditional analyses go back to the Prague School of the early 20th century under the notion of Functional Sentence Perspective (FSP). 

\begin{exe} 
\ex \label{bai14}
\begin{xlist}

\ex \label{bai14a}
\uline{Functional Sentence Perspective} (FSP) (\citealt{Mathesius1939}, \citealt{Adamec1966}) = the essentially bipartite division of every sentence into \emph{Theme} before \emph{Rheme}

\ex \label{bai14b}
\emph{Theme}:  (or \emph{Topic} or \emph{Departure Point}):  "what is known in   the given situation ... and from which the speaker departs"

\ex \label{bai14c}
\emph{Rheme}:  (or \emph{Focus} or \emph{Comment} or \emph{Core}):  "what the speaker expresses about the departure point or with attention to it"

\end{xlist}
\end{exe}

Syntacticization of Theme-Rheme structure within generative grammar found new support within \citeauthor{Rizzi1997}’s \citeyear{Rizzi1997} Left-periphery, whereby TopP and FocP projections are part of the CP-level functional hierarchy. Given that thematic/topical/given information indeed precedes rhematic/focal/new information, a cartographic account of such basic discourse divisions would look something like (\ref{bai15}):

%TREE15
\begin{exe}
\ex \label{bai15} 
Cartographic approach to Topic > Focus order: \\
\begin{forest}
fairly nice empty nodes,
for tree={inner sep=0, l=0}
[TopicP
[XP$_{TOP}$][
[Top$^0$]
[FocusP
[XP$_{FOC}$][
[Foc$^0$]
[TP (S), 
name=t[\textcolor{white}{triangle??}, roof] ]]]]]
\end{forest}
\end{exe}

There are various problems with (\ref{bai15}) as a derivational approach to surface word order in such languages. First, it is often the case that non-Topic non-Focus material comes between Topic and Focus, and that Focus often appears at the right edge (especially with neutral intonation patterns), as in the (b) answer to the (a) question in (\ref{bai16}):

\begin{exe}
\ex \label{bai16}
\begin{xlist}

\ex \label{bai16a}
\gll \emph{Kto}	\emph{čitaet}	\emph{gazetu?}				\jambox{} (SVO) \\ 
Who reads newspaper \\
\glt `Who reads newspaper?' \\

\ex \label{bai16b}
\gll \emph{[Gazetu]}	[X]	\emph{čitaet}		[X]	\emph{Ivan.}		\hspace{3.15cm}(OVS) \\
[newspaper]_{\textsc{top}} {} reads {} [Ivan]_{\textsc{foc}} \\
\glt `IVAN is reading the paper.' \\

\end{xlist}
\end{exe}

The simple OVS answer in (\ref{bai16b}) is generally judged to allow neutral material, the verb itself as well as other elements, parentheticals included, between the topic gazetu and the focal answer Ivan:  

\begin{exe}
\ex \label{bai17}
Discourse division of (16)b:  Topic > X (neutral) > Focus 
\end{exe}

Non-cartographic approaches to the OVS derivation that recognize the discourse structure in (\ref{bai17}) issue have utilized right adjunction, Extraposition, and even right-Specifiers (see \citealt{Bailyn2012} for an overview). All such mechanisms are explicitly unavailable within strict cartographic approaches. 

There are two possible derivations of the OVS order in (\ref{bai16b}) under Strict Cartography, each of them requiring positing an otherwise unattested functional category in addition to TopP and FocP.  \\ 
In the first type of derivation, the elements Object > V > Subject are all independent constituents, as in the derivation outlined here:

\begin{exe}
\ex \label{bai18}
Strict cartographic derivation of O$_{\textsc{top}}$ P-V-S$_{\textsc{foc}}$ order: (1st attempt) \\
STEP 1:  The (focal) subject raises to SpecFoc. \\
STEP 2:  the remnant vP containing the verb and its Object (in that order) undergoes remnant movement to an XP between FocP and TopP. \\
STEP 3.  The topical Object sub-extracts from the raised vP into SpecTopP.\\

\end{exe}

The derivation in (\ref{bai18}) involves several problematic steps. First, Step 2 involves both an unmotivated functional category and an unmotivated movement step. Second, Step 3 involves sub-extraction from an already A’-moved element, which violates \citealt{Rizzi2004} Criterial Freezing, and is generally not thought to be possible for already moved elements, (\citealt{Stepanov2007}). Furthermore, at no point in the derivation does the topical Object move through a high A-position, despite significant evidence of its A-properties (\citealt{Bailyn2004}, \citealt{Antonyuk2021}, \citealt{Pereltsvaig2021}). \\

The second type of derivation first derives an [Obj > V] extended constituent, via fronting into a low TopicP within TP, out of which the subject moves to SpecFoc before remnant movement into SpecTop.  This is sketched in (\ref{bai19}): 

\begin{exe}
\ex \label{bai19}
Strict cartographic derivation of O$_{\textsc{top}}$P-V-S$_{\textsc{foc}}$ order: (2nd attempt) \\
STEP 1:  The (topical) object raises to SpecXP, X being perhaps a low Topic phrase just outside vP. \\
STEP 2:  The (focal) subject raises to SpecFoc. \\
STEP 3.  the remnant vP containing the Object and the verb (in that order) undergoes remnant movement to SpecTopP \\
\end{exe}

This version does not require subextraction, and thus avoids one major issue with (\ref{bai18}). Furthermore, the XP proposed is in a plausible position for a low TopicP (somewhere between TP and vP). However, in this case again, the Object does not pass through a high A-position. And as before, the required remnant movement is under-motivated. 

The next major problem with attempts to derive Slavic discourse-oriented surface word order under Strict Cartography involves the fact that intonationally marked Focus can in fact be anywhere: (\citealt{Bailyn2012})

\begin{exe}
\ex \label{bai20}
(Context: What did Jacob bring?) \\
\gll \emph{POSYLKU}	\emph{Jakov}	\emph{(POSYLKU)} \emph{prines}		\emph{(POSYLKU)} \\ 
Parcel Jacob {} bought \\
\glt `Jacob bought a PARCEL.'\\
\end{exe}

This is obviously problematic for cartographic approaches. (See also \citealt{wagner2009focus} for other arguments from Information structure against Cartography.) Now, it is possible that Theme-Rheme partitioning occurs at a post-syntactic level of representation, as proposed in \citealt{Bailyn2012}, in the spirit of Zubizaretta 1998, and some sort of sentence partitioning, in the spirit of \citealt{Diesing1992}, rather than (only) the utilization of universal categories in a fixed functional hierarchy. This would require intonational cues to be relevant to the proper alignment of the two main parts of the structure before some sort of discourse-based partitioning at a distinct level of representation, a possibility not countenanced within Strict Cartography. Without that, the functional hierarchy alone is simply too rigid to handle discourse-oriented word order patterns.

\section{Conclusion}

Any theory that posits even a TP and a CP is already “cartographic”.  Some sort of functional hierarchy, however minimal, is assumed in most modern syntactic theories. Distinct schools of thought have developed around varying ideas of how much functional structure there is, how rigid it is, and the extent to which it is universal. It is also well-known that there are extensive word ordering preferences, asymmetries, and regularities, among similar elements, such as attributive adjectives that must show the underlying orderings they do for a reason. And while it is possible the “big tree” functional hierarchy assumed by strict versions of Cartography are descriptively accurate, and have helped with analyses of various word order phenomena, we have seen just in this short excursus into some basic tendencies in Slavic syntax, that the “big tree” hypothesis, especially given its “strict” ban on adjunction and on multiple specifiers,  does not enough to allow us to succinctly analyze the phenomena.  And to the extent that functional hierarchies do emerge, we still do not have a deep explanation – their nature, universality and in particular the order in which they appear, are still very much in question.   

There are viable, empirically sound alternatives to Strict Cartography. Indeed, several current lines of research are specifically focused on question deeper than those that Strict Cartography can ask, attempting to derive whatever functional hierarchies are observed from extra-linguistic considerations. In one prominent approach,. \citealt{ramchand2014deriving} provide plausible semantic and real-world justification for the major domains of functional space (CP > TP > VP) and proposes approaches to deriving the rest of the needed hierarchies from extra-linguistic factors. Somewhat similarly, and certainly compatibly, Larson (in press) proposes an account of fixed and less-fixed adjective order through the interaction with the syntax of a scale of subjectivity of the adjectives in question. The observed rigidity of adjective order result from their distinct location on a relative scale of subjectivity, mediated by features on the head being modified. \\
Such systems imagines a flexible functional hierarchy, projected from the heads of a small set of core functional elements (D, C, T, etc), with intricate feature bundles whose requirements determine the categories projected above them. This allows us to imagine a theory in which appearance of moved elements at the edges of domains would result from the usual feature-driven agree relations; their order at times determined by ordering within the relevant feature-bundling, but not pre-determined by functional templates. The syntactic devices available would include adjunction, mediated by featural and scopal consideration as envisioned by \citet{Ernst2007}, as well as multiple specifiers (a pure side-effect, expected within Bare Phase Structure,  of complex heads and their features) exactly as envisioned by \citealt{Chomsky2001} for the vP edge domain. In so far as constructions show apparent asymmetries of functional hierarchy, the system would derive those, and future research, then, should focus on the nature of the feature bundling and how the projection system serves the needs of the core functional heads, deriving, rather than assuming, those aspects of Strict Cartography that are empirically motivated. And the kinds of problems we have encountered here would be remedied through the availability of multiple specifiers, adjunction and perhaps a discourse-oriented level of representation for Information Structual considerations. 
	
Multiple wh movement would remain driven by the movement requirements of the elements themselves, and multiple specifiers account for the lack of superiority and clustering effects.  For Topic Focus structures, a late level of Information Structure would organize sentences into their basic discourse-related structure, an effect that may or may not turn out to b part of the core syntax of the sentence, but would be represented not through the mediation of fixed categories within an extensive tree, but at a distinct level of organization. This is the direction future research into Slavic word order phenomena should take. 


%%%%%%%%%%%%%%%%%%%%%%%%%%%%%%%%%%%%%%%%%%%%%%%%%%%%%%%%%%%


\section*{Acknowledgements}
...

\printbibliography[heading=subbibliography,notkeyword=this]

\end{document}