\documentclass[output=paper,colorlinks,citecolor=brown,
% hidelinks,
% showindex
]{langscibook}
\author{Ryosuke Hattori \affiliation{Kobe Gakuin University}}
\title{Size of Op in \textit{Tough}-Constructions}
\abstract{Following \citep{Hicks2009}, the availability of the complex operator with D layer is the pre-requisite for English-like \textit{tough} constructions. Based on the NP/DP parameter, I claim that the size of the null operator (Op) is bigger in languages with articles (so called DP languages) with D layer while Op in languages without articles (so called NP languages) are missing the D layer. Hence English-like \textit{tough} constructions should be available only in DP-languages. Through a cross-linguistic survey of 13 languages, I will show that this is in fact borne out.}

\begin{document}
\maketitle

\section{Complex Null Operator}
\subsection{Problems in Analyses of \textit{Tough} Construction}
The analyses of the \textit{tough} constructions have encountered difficulties with at least one of the core theoretical concepts of Case, locality constraints, and θ-role assignment. For example, the raising analysis of the \textit{tough} subject from the embedded object position by A-movement (e.g. \citeauthor{Rosenbaum1967} \citeyear{Rosenbaum1967}; an A-movement account) leads to a problem with respect to Case assignment, i.e. the \textit{tough} subject should not be able to avoid accusative Case assignment by the infinitive verb in the embedded clause.

\begin{exe}
\ex \label{1ha}
He$_{i}$ is easy [$_{CP}$ [$_{TP}$ PRO to please t$_{i}$]].
\end{exe}

On the other hand, \citeauthor{Chomsky1977}'s (\citeyear{Chomsky1977}) account based on A'-movement of a null operator (Op) assumes that the \textit{tough} subject is base-generated in situ. This analysis, however, appears to leave the matrix subject without a θ-role, since the \textit{tough} predicate is claimed to not assign a θ-role to its subject. This is indicated by the grammaticality of the \textit{tough} constructions with expletive/sentential subjects in (\ref{2ha}), which is contrasted with other complement object deletion configurations as with \textit{pretty} in (\ref{3ha}) 

\begin{exe}
\ex \label{2ha}
\begin{xlist}
\ex \label{2aha}
It is tough to please linguists.

\ex \label{2bha}
To please linguists is tough.

\end{xlist}

\ex \label{3ha}
\begin{xlist}
\ex \label{3aha}
*It is pretty to look at these flowers.

\ex \label{3bha}
*To look at these flowers is pretty.
\end{xlist}
\end{exe}

Thus, this A’-movement analysis has to explain how a single θ-role assigned by the embedded verb is apparently “shared” between two arguments, i.e. the null operator in the infinitival clause and the \textit{tough} subject. 

\citet{Postal1971}, \citet{PostalRoss1971}, \citet{Rosenbaum1967} and \citet{Brody1993}, among others propose a composite A/A'-movement analysis by claiming that A’-movement of the \textit{tough} subject is followed by A-movement as shown below.

\begin{exe}
\ex \label{4ha}
John$_{i}$ is easy [$_{CP}$ t$_{i}$ [$_{TP}$ PRO to please t$_{i}$]].
\end{exe}

However, the problem of this approach is the Case mismatch of the subject (Accusative vs. Nominative). Another issue is that movement from an A position to an A’-position that is followed by A-movement, referred to as Improper Movement, is typically assumed to be disallowed (See 
\citeauthor{Chomsky1973} \citeyear{Chomsky1973,Chomsky1981}; \citeauthor{May1979} \citeyear{May1979}).

\subsection{The CNO Analysis} \label{s1.2ha}
\citet{Hicks2009} proposes a new analysis which incorporates both A-movement and A’-movement but without the problems of the previous approaches noted above, using smuggling (\citeauthor{Collins2005a} \citeyear{Collins2005a,Collins2005b}). He claims that a null operator in \textit{tough} constructions is a wh-phrase with a more complex internal structure than is typically assumed, i.e. a complex DP with an internal DP as the \textit{tough} subject (e.g. \textit{John}) as shown below.

\begin{exe}
\ex \label{5ha}

\begin{forest}
[DP $^{[i\phi, uCase, iQ, uWH]}$
[D][NP
[N\\Op][DP $^{[i\phi, uCase]}$\\John]]
]
\end{forest}

\end{exe}

Based on this complex null operator (henceforth, CNO) analysis, the derivation of the \textit{tough} construction \textit{John is easy to please}, for example, proceeds as follows. First, the CNO merges with the V \textit{please} as an object and the patient θ-role from \textit{please} is assigned to the whole complex DP. Second, the derived VP is merged with v, and the complex null operator enters into ϕ-feature agreement with v, [uϕ] (uninterpretable ϕ-feature) on v being the relevant probe. As a reflex of ϕ-feature agreement, v checks [uCase] on the CNO, i.e. the whole DP at this point.

\begin{exe}
\ex \label{6ha}
\begin{forest}for tree=nice empty nodes
[v'
[v $^{[u\phi]}$,name=clitic][VP
[V\\please]
[\hspace{10mm} DP $^{[i\phi, uCase, iQ, uWH]}$, name=ep
[D]
[NP[N \\Op][DP $^{[i\phi, uCase]}$\\John]]]
]
]
\draw[->,dotted] (clitic) to[out=south west,in=south west] ( ep);
\end{forest}

\end{exe}

After V-to-v movement of please and the merger of PRO as the external argument, the CNO must move to the phase edge (outer vP-spec) since it bears [iQ, uWH] feature, where crucially, the operator pied-pipes the inner DP \textit{John}, allowing [uCase] on it to escape. The null operator therefore serves to "smuggle" (\citeauthor{Collins2005a} \citeyear{Collins2005a,Collins2005b}) the \textit{tough} subject.

\begin{exe}
\ex \label{7ha}

\begin{forest}for tree=nice empty nodes
[CP
[C$^{[uQ, EPP]}$, name=clitic][TP
[DP$_{i}$\\PRO][T'
[T\\to][vP
[DP$_{k}$$^{[i\phi, iQ, uWH]}$, name=ep
[D][NP
[N\\Op][DP\\John $^{[i\phi, uCase]}$]]
]
[vP
[t$_{i}$][v'
[v\\please$_{j}$][VP
[t$_{j}$][t$_{k}$]]]]]]]]
\draw[->,dotted] (clitic) to[out=south west,in=south west] ( ep);
\end{forest}
\end{exe}

The PRO, then, moves into Spec, TP of the embedded clause, and the C is merged with [uQ] which is checked with [iQ] on the CNO while the [uWH] is checked as a reflex. The [EPP] on C then drives movement of the CNO into the phase-edge position, allowing the unchecked [uCase] on \textit{John} to escape. At this point the remaining interpretable features in the CNO are now inactive. In other words, the phrase (i.e. the full CNO) is frozen in place and thus is not accessible to further movement, following \citet{Rizzi2006,Rizzi2007}.\footnote{The details of the feature checking relations assumed by \citet{Hicks2009} will actually not be important below.}

\begin{exe}
\ex \label{8ha}
\begin{forest}
[F
[DP$_{k}$$^{[i\phi, iQ]}$
[D][NP
[N\\Op][DP\\John $^{[i\phi, uCase]}$]
]]
[CP
[C\\{[\sout{EPP}]}][TP
[DP\\PRO][T'
[T\\to][vP]]]]]
\end{forest}
\end{exe}
\hspace{80mm} t$_{k}$ \hspace{10mm} ...

\vspace{3mm}
Finally, when the main clause T merges into the structure, T, which has [uϕ], probes for [iϕ]. As a reflex of ϕ-agreement, a nominative case value is assigned to the goal John, which moves to Spec, TP to satisfy [EPP], and its [uCase] is checked.  

\begin{exe}
\ex \label{9ha}
\begin{forest}for tree=nice empty nodes
[TP
    [DP2$_{j}$ $^{[\sout{uCase}]}$,name=specvP]
    [F[DP1$_{i}$ $^{[F]}$\\..t$_{i}$...,name=specTP, tikz={\node [draw,black,dotted, inner sep=0,fit to=tree]{};}]
    [CP$^{[F]}$[..t$_{i}$..., roof, name=VP]]
    ]
]
\draw[->] (VP) to[out=south,in=south] (specTP);
\draw[->] (specTP) to [out= west,in=south]  (specvP);
$
$
$
$
$
$
\end{forest}
\end{exe}   

In short, based on this analysis, when the CNO merges with the V as an object, the patient θ-role is assigned to the whole complex DP1, and after the CNO merges with a CP, the inner DP2 is smuggled \citep{Collins2005a,Collins2005b} into the matrix subject position without being assigned an accusative Case prior to that movement. The shared feature F is projected here (based on the Labeling Algorithm in \citeauthor{Chomsky2013} \citeyear{Chomsky2013}), which I assume is a D-related feature.

This CNO analysis avoids the problems of the previous analyses in that (a) the CNO shields the \textit{tough} subject from Case assignment in the lower clause by the infinitival verb, and that (b) it does not involve improper movement. Crucially, there has to be a DP which embeds Op within it, smuggling the \textit{tough} subject from the complement position of the Op in (\ref{5ha}).

\subsection{The NP/DP Parameter and a Prediction} \label{s1.3ha}

The crucial issue here is that languages without articles have been argued not to have the category D, hence the DP projection \citep[among others]{Corver1992,Zlatic1997,Boskovic2005,Boskovic2012,Despic2013,takahashi2011}. For example, \citet{Boskovic2012} establishes a number of generalizations based on wide-ranging syntactic and semantic phenomena that correlate with the presence or absence of articles in the languages, based on which Bošković argues that languages without articles lack the DP layer. Furthermore, he proposes an NP/DP parameter where languages with articles like English are DP languages and languages like Japanese which do not have articles are NP languages.

Given this, I claim that the Op has a more complex structure in DP-languages while it does not have the DP layer in NP-languages, as shown in (\ref{10ha}). In other words, the size of Op is different among languages.

\begin{exe}
\ex \label{10ha}
Null Operators in \textit{tough} constructions
\begin{multicols}{2}
\begin{xlist}
\ex  DP-languages:\\
\label{10aha}
\begin{forest}
[DP$^{[uF]}$ ($\equal$CNO)
[D][NP
[N\\Op][DP\\John\\(\textit{tough}) subject]
]
]
\end{forest}


\vfill\null
\columnbreak 

\ex NP-languages:\\ 
\label{10bha}

\begin{forest}
[NP[N\\Op]]
\end{forest}

\end{xlist}
\end{multicols}
\end{exe}

Based on the CNO analysis of \textit{tough} constructions \citep{Hicks2009} in (\ref{9ha}), the \textit{tough} subject is smuggled out of the lower infinitive clause by the complex DP (CNO). If this is the case, then it is predicted that English-like \textit{tough} constructions would not be available in NP-languages, since in NP languages Op is not complex and the uninterpretable [F] feature, which is necessary for the smuggling to take place, is missing. 

In order to check this prediction, I will look at the cross-linguistic variation in “\textit{tough} constructions”, and conduct a cross-linguistic survey of the availability of “\textit{tough} constructions” in 13 languages in the following sections, which establishes a correlation between the availability of the “\textit{tough} construction” and being a DP-language.

\section{\textit{Tough} Constructions without the CNO}
Japanese, an NP language, appears to allow \textit{tough} constructions, as in (\ref{11ha}). However, \citet{takezawa1987} claims that (\ref{11ha}) should not be analyzed in accordance with the English \textit{tough} construction \citep{Chomsky1977}, as there is no island effect, which is shown by (\ref{12ha}). (As the English translation here shows, (\ref{12ha}) involves a complex NP configuration and should be ruled out due to movement out of the complex NP.) 

\begin{exe}
\ex \label{11ha}
\gll John$_{i}$ -ga [$_{AP}$ [$_{S’}$ Op$_{i}$ [$_{S}$ PRO t$_{i}$ yorokobase]] yasu -i]]\\
{} -\textsc{nom} {} {} {} {} {} {} please easy -\textsc{pres}\\\\
‘John is easy to please’

\ex \label{12ha}
\begin{xlist}
\ex \label{12aha}
\gll [kono te-no hanzai]$_{i}$ –ga (keisatu-nitotte) [$_{NP}$ [$_{S’}$ e$_{j}$ e$_{i}$ okasi-ta] ningen$_{j}$]-o  sagasi-yasu-i\\
This kind-of crime -\textsc{nom} police-for {} {} {} {} commit-\textsc{pst} man-\textsc{acc} search-easy-\textsc{pres}\\\\
`*[This kind of crime]$_{i}$ is easy (for the police) to search [$_{NP}$ a man [$_{S'}$ who committed e$_{i}$ ]]'

\ex \label{12bha}
\gll [kooitta     itazura]$_{i}$ -ga (senseigata-nitotte) [$_{NP}$ [$_{S'}$ e$_{j}$ e$_{i}$ sita] seito$_{j}$]-o    mituke-yasu-i\\
This-kind-of trick -\textsc{nom} teachers-for {} {} {} {} do-\textsc{pst} pupil-\textsc{acc} find-easy-\textsc{pres}\\\\
 `*[This kind of trick]$_{i}$ is easy (for the teachers) to find [$_{NP}$ a pupil [$_{S'}$ who played e$_{i}$]]'



\ex \label{12cha}
\gll [Sooiu ronbun]$_{i}$ -ga (watasi-nitotte) [$_{NP}$[$_{S'}$ e$_{j}$ e$_{i}$ kai-ta] gakusei$_{j}$]-o  hyookasi-niku-i\\
That-kind-of paper  -\textsc{nom}  me-for  {} {} {}   write-\textsc{pst} student-\textsc{acc} evaluate-difficult-\textsc{pres}\\\\
`*[That kind of paper]$_{i}$ is difficult (for me) to evaluate [$_{NP}$ a student [$_{S'}$ who wrote e$_{i}$]]’ \hspace{30mm} \citep[203]{takezawa1987}                   
\end{xlist}
\end{exe}

Takezawa explains this difference by claiming that Japanese \textit{tough} constructions do not involve movement of Op but involve an empty pronominal (Japanese independently allows empty pronominals) in the gap position and the “aboutness relation” which correlates the pronominal and its antecedent, just as claimed for the derivation of relativization and topicalization by \citet{Saito1985} based on \citeauthor{kuno1973}’s (\citeyear{kuno1973}) observation. He further points out that when \textit{tough} constructions have PP subjects, which cannot be coindexed with an empty pronominal, they observe Subjacency, as shown in (\ref{13ha}). Thus, Takezawa concludes that only \textit{tough} constructions with PP subjects must be derived by movement of a null operator as in their English counterparts.

\begin{exe}
\ex \label{13ha}
\begin{xlist}
\ex \label{13aha}
\gll *[$_{PP}$ Anna taipu -no zyosei-to]$_{i}$ -ga (John-nitotte) [$_{NP}$[$_{S'}$ e$_{j}$ e$_{i}$ kekkon-site-i-ru] otoko$_{j}$]-to hanasi-niku-i.\\
{} that  type of woman-with -\textsc{nom} John-for {} {} {} marry-\textsc{pres} man-with talk-hard-\textsc{pres}\\\\
(lit.) ‘[With that type of woman]$_{i}$ is hard (for John) to talk to [$_{NP}$ the man [$_{S’}$ who marry e$_{i}$ ]].’\\ 
cf. \gll [$_{PP}$ Anna taipu -no zyosei-to]$_{j}$ -ga (John$_{i}$-nitotte) [$_{S'}$ pro$_{i}$ e$_{j}$] kekkonsite-mo-i-i  to] tomodachi-ni ii-niku-i.\\
{} that  type of woman-with -\textsc{nom} John-for {} {} {} marry-may-\textsc{pres} \textsc{comp} friend-to say-hard-\textsc{pres}\\\\
(lit.) ‘[With that type of woman]$_{i}$ is hard (for John$_{j}$) to say to his friends [$_{S’}$ that he$_{j}$ may marry e$_{i}$]’


\ex \label{13bha}
\gll ?*[$_{PP}$ Sooiu kin'yuukikan-kara]$_{i}$  -ga     (John-nitotte) [$_{NP}$ [$_{S'}$ e$_{j}$ itumo e$_{i}$ okane-o      takusan karite-i-ru]    hito$_{j}$]-o  sin'yoosi-niku-i.\\
{} such financial.agency-from -\textsc{nom} John-for {} {} {} always {} money-\textsc{acc} a.lot borrow-\textsc{pres} person-\textsc{acc} trust-hard-\textsc{pres}\\\\
(lit.) ‘[From such a financial agency]$_{i}$ is hard (for John) to trust [$_{NP}$ a person [$_{S’}$ who always loans a lot of money t$_{i}$]].’\\
cf. \gll [$_{PP}$ Sooiu kin'yuukikan-kara]$_{j}$ -ga (John$_{j}$-nitotte) [$_{S'}$ pro$_{i}$ e$_{j}$ okane-o takusan karite-i-ru to] ii-niku-i\\
{} such financial.agency-from -\textsc{nom} John-for {} {} {} money-\textsc{acc} a.lot borrow-\textsc{pres} \textsc{comp} say-hard-\textsc{pres}\\\\
(lit.) ‘[From such a financial agency]$_{i}$ is hard (for John) to say [$_{S'}$ that he has loaned a lot of money e$_{i}$]’ 

\end{xlist}
\end{exe}

I will argue that this PP subject \textit{tough} construction is irrelevant to our expectation that NP languages do not have a \textit{tough} construction since PP itself may bring in richer structure for the Op, enabling the smuggling of the subject, regardless of the presence of DP layer here. 

\begin{exe}
\ex \label{14ha}
[Annna taipu –no zyosei-to]$_{j}$ –ga [$_{CP}$ [$_{PP}$ Op t$_{j}$]$_{i}$ John-nitotte t$_{i}$ kekkon si yasui]
\end{exe}

Thus, I will focus on nominal \textit{tough} constructions where NP/DP distinction is crucial for the availability of \textit{tough} construction. Recall that the Op does not have any uninterpretable features in \textit{tough} construction; a DP above the Op is necessary for smuggling the subject in DP-languages. The availability of \textit{tough} construction with PP subject in Japanese then is explained by saying that PP functions as the DP and has an uninterpretable feature [uF] that is needed for the smuggling of the \textit{tough} subject. 

\begin{exe}
\ex \label{15ha}
\begin{forest}
[PP$^{[uF]}$[P][NP[N\\Op][DP\\John (\textit{tough} subject)]]]
\end{forest}
\end{exe}

The necessity of the CNO analysis comes from the nominative Case marking on the \textit{tough}\textit{} subject in English. I.e. the subject needs to be smuggled into the TP spec position in order to avoid getting assigned the accusative Case in the complement position of the infinitive, instead getting the nominative Case from the higher T. If there are languages where the apparent subject of \textit{tough} construction is assigned a Case other than nominative, CNO will then not be needed. I will therefore focus on nominative subjects of \textit{tough} constructions below.

\section{Cross-linguistic Survey of Availability of CNO in \textit{tough}} \label{s3ha}

\subsection{Diagnostics} \label{s3.1ha}
Before looking at the data, we need to clarify the diagnostics a little more. Regarding the Case marker of the \textit{tough} subject, as noted above, it is crucial to check if it is a Nominative or another Case such as Accusative/Dative (or the Case normally assigned by the infinitive verb). If the matrix subject has a Nominative Case, then in that language the CNO can be involved in the derivation. However, there is another possibility when the language has no island effect (thus no \textit{tough}- movement) because of a resumptive pronoun as in the case of Japanese \textit{tough} constructions. If the \textit{tough} subject has the Case assigned by the lower verb, it is an indication that the CNO analysis is not necessary since there is no need for the subject to avoid Case assignment by being smuggled; this also suggests that the subject was base-generated in the object position of the infinitive, and moved to the surface position without any Op movement. There should, however, still be an island effect here.\footnote{We could be dealing here either with quirky subject movement to Spec TP or movement of the object to a position above TP for topicalization/focalization. Either way, the movement does not result in Case assignment.} The diagnostics are then summarized below.

\begin{exe}
\ex \label{16ha}
Diagnostics to follow
\begin{enumerate}
    \item The subject has a nominative Case or a Case assigned by the embedded infinitive verb?
    \item If nominative Case, then check subjacency effects; if yes, smuggling of the subject with the CNO as in (i); if no, base-generated subject with a null resumptive pronoun in the gap position without Op movement as in (ii).\\
    (i) Subj(\textsc{nom})$_{j}$ is tough [$_{CNO}$…t$_{j}$… ]$_{i}$ to please t$_{i}$ \hfill  e.g. English\\
    (ii) Subj(\textsc{nom})$_{i}$ is tough to please pro$_{i}$ \hfill e.g. Japanese
    \item If no nominative, with Case assigned by the infinitive verb, then the object of the infinitive verb is moved as in (iii) by e.g. focalization; and there is no need for Complex Op analysis, but there should be a subjacency effect for the movement.\\
    (iii) Subj(\textsc{dat/acc})$_{i}$ is tough to please t$_{i}$                        
\end{enumerate}
\end{exe}

In short, there are three types, i.e. English-like \textit{tough} construction with a nominative subject with the CNO, Japanese-like \textit{tough} construction with a nominative subject without the CNO, and the one without a nominative subject or the CNO. 

In order to check the subjacency effect, I will use the translation of \citeauthor{Chomsky1977}’s (\citeyear{Chomsky1977}) examples regarding the locality in English \textit{tough} constructions, i.e. (\ref{17cha}).

\begin{exe}
\ex \label{17ha}
\begin{xlist}
\ex \label{17aha}
John$_{i}$ is easy (for us) to please t$_{i}$

\ex \label{17bha}
\begin{xlist}
\ex \label{17biha}
John$_{i}$ is easy (for us) [to convince Bill [to do business with t$_{i}$]]

\ex \label{17biiha}
John$_{i}$ is easy (for us) [to convince Bill [that he should meet t$_{i}$]]

\end{xlist}
\ex \label{17cha}
\begin{xlist}
\ex \label{17ciha}
*John$_{i}$ is easy (for us) [to describe to Bill [a plan [to assassinate t$_{i}$]]]    \hfill (Complex NP)

\ex \label{17ciiha}
*Which sonatas$_{i}$ are the violin$_{j}$ easy [to play t$_{i}$ on t$_{j}$] \hfill (Wh-island)\footnote{Here, \textit{which} sonatas is moving past a null \textit{wh} operator (i.e. CNO in our analysis), resulting in a \textit{wh}-island constraint violation.
\begin{exe}
\ex 
\begin{xlist}
\ex 
The violin$_{j}$ is easy [$_{CP}$ [$_{CNO}$ Op t$_{j}$ ]$_{k}$ \sout{for} PRO to play sonatas on t$_{k}$].

\ex *Which sonatas$_{i}$ are the violin$_{j}$ easy [$_{CP}$ [$_{CNO}$ Op t$_{j}$ ]$_{k}$ \sout{for} PRO to play t$_{i}$ on t$_{k}$].
\end{xlist}
\end{exe}
} 
\end{xlist}

\end{xlist}
\end{exe}

Based on this, I have conducted a cross-linguistic survey of the availability of “\textit{tough} constructions” in 13 languages. I will show some examples (of each of the three types) below.

\subsection{Example of type (i): German} \label{s3.2ha}
There are \textit{tough} constructions with a nominative subject in several languages. Thus, the literature discusses the \textit{tough} construction (also often referred to as the \textit{easy-to-please} construction) in German or some Romance languages (e.g. see \citeauthor{MontalbettiMarioTravis1982} \citeyear{MontalbettiMarioTravis1982}; \citeauthor{Cinque1990} \citeyear{Cinque1990}; \citeauthor{Roberts1993} \citeyear{Roberts1993}; \citeauthor{Wurmbrand2001} \citeyear{Wurmbrand2001}).

In German,\footnote{German sentences in this subsection were checked by a consultant, Sabine Laszakovits.} \textit{tough} constructions have the subject that is nominative-marked but it is interpreted as an object of the infinitival verb as in (\ref{18aha}).

\begin{exe}
\ex \label{18ha}
\begin{xlist}
\ex \label{18aha}
\gll Dieser Konflikt ist  leicht zu lösen t$_{i}$\\
This.\textsc{nom} conflict.\textsc{nom} is easy to solve\\\\
‘This confict is easy to solve’                             

\ex \label{18bha}
\gll Es ist leicht, diesen Konflikt zu lösen.\\
it is easy this.\textsc{acc} conflict.\textsc{acc} to solve\\\\
‘It is easy to solve this conflict.'

\ex \label{18cha}
\gll John hat den/diesen  Konflikt gelöst.\\
John has the.\textsc{acc}/this.\textsc{acc} conflict.\textsc{acc} solved\\\\
‘John solved the conflict.’

\end{xlist}
\end{exe}

Here, crucially the verb lösen ‘solve’ used in the infinitival clause in (\ref{18bha}) and in the main clause in (\ref{18cha}) normally takes an accusative Case object, which means that the subject \textit{dieser} \textit{Konflikt} ‘this conflict’ in the \textit{tough} construction in (\ref{18aha}) is not assigned a Case by the infinitival verb.

When an inherent Case assigning verb is used as the infinitive in \textit{tough} constructions in German, however, the \textit{tough} subject seems to retain the inherent Case from the infinitives, as shown below.

\begin{exe}
\ex \label{19ha}
\begin{xlist}
\ex \label{19aha}
\gll Ihm ist leicht zu helfen\\
he.\textsc{dat} is easy to help\\\\
‘He is easy to help.’

\ex \label{19bha}
\gll Es ist leicht, ihm zu helfen.\\
it is easy he.\textsc{dat} to help\\\\
‘It is easy to help him.’

\end{xlist}

\ex \label{20ha}
\gll Bitte   hilf   mir\\
Please help me.\textsc{dat}\\\\
 ‘Please help me.’
\end{exe}

Here I assume that the preverbal oblique NP \textit{Ihm} ‘he.\textsc{dat}’ is not a grammatical subject and thus not in spec TP position, following \citet{ZaenenThrainsson1985}, who show that German does not have quirky subjects. Thus, for example, the sentence-initial oblique NP in German passives cannot be deleted under identity with a (nominative) subject, which is contrasted with the oblique NP in Icelandic, which has quirky subjects.

\begin{exe}
\ex \label{21ha}
\begin{xlist}
\ex \label{21aha}
\gll Er kam und (er) besuchte die Kinder. \hspace{25mm} German\\ 
he.\textsc{nom} came and (he) visited the children\\\\ 

\ex \label{21bha}
\gll Er kam und (er) wurde verhaftet.\\
he came and (he) was arrested \hspace{2mm} \citep[477]{ZaenenThrainsson1985}\\\\


\ex \label{21cha}
\gll *Er kam und \underline{\hspace{6mm}} wurde geholfen.\\
he came and {} was helped\\\\

\end{xlist}

\ex \label{22ha}
\begin{xlist}
\ex \label{22aha}
\gll ϸeir  fluttu  líkið og   ϸeir grófu   ϸað. 
\hspace{15mm} Icelandic\\
they.\textsc{nom} moved the-corpse and they buried it\\\\

\ex \label{22bha}
ϸeir fluttu líkið og \underline{\hspace{6mm}} grófu ϸað 

\ex \label{22cha}
\gll Hann segist vera duglegur, en \underline{\hspace{6mm}} finnst  verkefnið of ϸungt.\\
he.\textsc{n} says-self to-be diligent, but \underline{\hspace{6mm}}.\textsc{d} finds the-homework too hard\\\\           
'He says he is diligent, but finds the homework too hard' \citep[453-454]{ZaenenThráinsson1985}              
\end{xlist}

\end{exe}
For this subjecthood test, the sentence-initial oblique DP in German \textit{tough} construction behaves similarly, which is contrasted with the nominative DP in (\ref{24ha}) as shown below.\footnote{It is still not clear, though, what is blocking the derivation where the CNO gets the inherent case and the matrix subject gets smuggled to the specifier of TP to get nominative, in the case of e.g. (\ref{19aha}).}

\begin{exe}
\ex \label{23ha}
\gll *Er hat  überlebt und \underline{\hspace{6mm}} war leicht zu helfen.\\
he.\textsc{nom} has survived and {} was easy to help\\\\
'He survived and \underline{\hspace{6mm}} was easy to help.'       

\ex \label{24ha}
\gll Dieser Konflikt verschlechtert sich und \underline{\hspace{6mm}} ist schwierig zu lösen.\\
this.\textsc{nom} conflict worsened \textsc{refl} and {} is difficult to solve\\\\
'This conflict worsened and is difficult to solve.'    

\end{exe}

Also, as in English, German \textit{tough} constructions observe the island effect, as shown below (p.c. Sabine Laszakovits and Roman Reitschmied). 

\begin{exe}
\ex \label{25ha}
\begin{xlist}
\ex \label{25aha}
\gll Es ist leicht den Plan zu beschreiben, John zu töten\\
It is easy the.\textsc{acc} plan to describe John to kill\\\\
‘It is easy to describe a plan to kill John’


\ex \label{25bha}
\gll *Der John ist leicht den Plan zu beschreiben, \underline{\hspace{3mm}} zu töten. \\
the.\textsc{nom} John is easy the.\textsc{acc} plan to kill {} to describe\\\\
‘*John is easy (for us) to describe a plan to kill’

\end{xlist}
\end{exe}

Therefore, German is categorized as type (i) in our diagnostic where the CNO movement is involved with the smuggling of the subject which gets nominative Case in the matrix TP spec position. In other words, German has the relevant \textit{tough} construction.

\subsection{Example of type (ii): Thai} \label{s3.3ha}
As another example of Japanese-like \textit{tough} construction with base-generated subject and a null resumptive pronoun in the gap position without Op movement, I now turn to Thai.\footnote{Thai sentences are checked with two consultants, Panat Taranat and Sidney Mao.} As shown below, there are morphemes \textit{–gnai/–yak} ‘-easy/-difficult’ corresponding to Japanese \textit{-yasui/-nikui} ‘-easy/-tough’. 

\begin{exe}
\ex \label{26ha}
\gll nang  sue  nian -yak.\\
book this read difficult\\\\
‘This book is difficult to read’

\ex \label{27ha}
\gll khao deejai -ngai.\\
he happy easy\\\\
‘he is easy to make happy’
\end{exe}

Another similarity is that there is no island effect, as in its Japanese counterpart.

\begin{exe}
\ex \label{28ha}
\gll achyakrrm ni  jab [khon  [t$_{i}$   tam  \textit{e}]]  -ngai.\\
crime this arrest person who did {} easy\\\\
‘This (type of) crime is easy to arrest the person who did t.’

\end{exe}

Also, Thai can have resumptive pronouns in e.g. relative clauses. A pronoun referring to the head noun may appear in some relative clauses. Here the resumptive pronoun /kháw/ is associated with the head nouns /khon/ and /nák-lian/.  

\begin{exe}
\ex \label{29ha}
\gll khon  [thîi \underline{kháw} pay yùu  kan taam roŋrian]. \\
people C they go stay \textsc{rec}\footnotemark{} at school\\\\
‘People who want to stay at school…’ \citep{IwasakiIngkaphirom2005}


\ex \label{30ha}
\gll mây-chây pen acaan kháp, pen náklian [thîi kháw fùk maa].\\
\textsc{neg} is teacher \textsc{slp} is student C they train come/\textsc{asp}\\\\
‘(Dorm directors) are not teachers. They are students who have been trained.’
\end{exe}
\footnotetext{\textsc{rec} = reciprocal, \textsc{slp} = Speech Level Particle} 

I assume the island effect is voided by the presence of a null resumptive pronoun in (\ref{28ha}), which enables the aboutness relation between the fronted element and the gap, just as in the case of its Japanese counterpart. 

Now, as the following sentences show, when a PP subject is used for the \textit{tough} construction, the island effect is observed. This is another similarity with Japanese.

\begin{exe}
\ex \label{31ha}
\begin{xlist}
\ex \label{31aha}
\gll ?? [jak tanakhan ni]  waijai [khon [ti gu ngen yeu t$_{i}$]]  yak.\\
{} from bank this trust person who loans money much {} hard\\\\
‘[from this bank] is hard to trust a person who loans a lot of money \textit{t$_{i}$}’

\ex \label{31bha}
\gll waijai [khon  [ti gu ngen yeu jak tanakhan ni]] yak.\\
trust person who loans money much from bank this hard\\\\


\end{xlist}
\end{exe}

In short, Thai \textit{tough} constructions pattern with Japanese, i.e. type (ii) in the diagnostics (\ref{16ha}), in that there is no island effect despite the subject being nominative Case-marked, because of the existence of a null pronoun in the infinitival object position.  

\subsection{Example of type (iii): Serbo-Croatian} \label{s3.4ha}
The survey found that some languages have \textit{tough} constructions with the noun in the apparent subject position being assigned a Case other than nominative. This means that the CNO is not needed in their derivations. In examples corresponding to the \textit{tough} construction in Serbo-Croatian (SC)\footnote{Serbo-Croatian data in this subsection are from two consultants, Aida Talić and Ivana Jovović.}  in (\ref{32ha}), the element in the apparent subject position has the Case which is assigned by the infinitival verb \textit{ugoditi} ‘please’/\textit{otpustiti} ‘fire’. 

\begin{exe}
\ex \label{32ha}
\begin{xlist}
\ex \label{32aha}
\gll Njemu/*On je lako ugoditi.\\
him.\textsc{dat}/he.\textsc{nom} is easy.\textsc{adv} please.\textsc{inf}\\\\
`He is easy to please’

\ex \label{32bha}
\gll Njega/*On je lako otpustiti.\\
him.\textsc{acc}/he.\textsc{nom} is easy.\textsc{adv} fire.\textsc{inf}\\\\
‘He is easy to fire’

\end{xlist}

\ex \label{33ha}
\begin{xlist}
\ex \label{33aha}
\gll Ivan je ugodio  njemu.\\
Ivan is pleased him.\textsc{dat}\\\\
‘Ivan pleased him (but not her)’

\ex \label{32bha}
\gll Šef je otpustio njega.\\
boss is fired him.\textsc{acc}\\\\
`The boss fired him (but not her)’
\end{xlist}
\end{exe}

The pronouns can also be placed in the canonical object position as shown below, where the matrix subject is phonologically null. 

\begin{exe}
\ex \label{34ha}
\begin{xlist}
\ex \label{34aha}
\gll Lako je ugoditi njemu.\\
easy.\textsc{adv} is please him.\textsc{dat}\\\\
‘It is easy to please him (but not her)’

\ex \label{34bha}
\gll Lako je otpustiti njega\\
easy.\textsc{adv} is fire.\textsc{inf} him.\textsc{acc}\\\\
‘It is easy (for the boss) to fire him (but not her)’

\end{xlist}

\end{exe}

All this suggests that in the “\textit{tough}” constructions in (\ref{32ha}), the sentence initial object of the infinitive verb undergoes topicalization/focalization/scrambling into the matrix clause, the real subject being null.  

\begin{exe}
\exi{(32')} \label{32'ha}
\begin{xlist}
\ex \label{32'aha}
\gll Njemu$_{i}$ [ je  lako  ugoditi t$_{i}$]\\
him.\textsc{dat} {} is easy.\textsc{adv} please.\textsc{inf}\\\\
‘Him, it is easy to please’

\ex \label{32'bha}
\gll Njega$_{i}$  [ je  lako otpustiti t$_{i}$]\\
him.\textsc{acc} {} is easy.\textsc{adv} fire.\textsc{inf}\\\\
‘Him, it is easy to fire’

\end{xlist}
\end{exe}

Furthermore, the movement of the object is island-sensitive, as shown below. 

\begin{exe}
\ex \label{35ha}
\begin{xlist}
\ex \label{35aha}
\gll Lako nam je Borisu prepričati trač    da   su  ubili \textbf{njega}. \\
easy us.\textsc{dat} is Boris.\textsc{dat} retell gossip that are kill him.\textsc{acc}\\\\
'It is easy for us to retell to Boris a gossip that they killed him.'

\ex \label{35bha}
*\textbf{Njega}$_{i}$ je nama lako Borisu prepričati trač da su ubili t$_{i}$.
\end{xlist}
\end{exe}

Therefore, in Serbo-Croatian, the object moves directly from the complement of the infinitive without involving smuggling and CNO. In sum, the sentences that correspond to the \textit{tough} constructions in SC are classified as type (iii) in the diagnostics (16), i.e. Serbo-Croatian does not have the relevant \textit{tough} construction. Through the survey, I found that other languages like Slovenian, Russian and Polish all follow the same pattern as SC. 

\subsection{Summary} \label{s3.5ha}
Based on the diagnostics (\ref{16ha}), the \textit{tough} constructions in the 13 languages surveyed are categorized into 3 types in the following way.

\begin{table}[H]
\caption{\textit{Types of} tough \textit{constructions}}
\label{tab:1:types}
 \begin{tabular}{l r}
  \lsptoprule
    \textbf{Languages}  & \textbf{Types}  \\
  \midrule
  English  &   i  \\
  German  &   i\\
  Spanish & i\\
  Italian & i\\
  French & i\\
  Bulgarian & iii\\
  Hungarian & iii\\
  Thai & ii\\
  Japanese & ii\\
  SC & iii\\
  Slovenian & iii\\
  Polish & iii\\
  Russian & iii\\
  \lspbottomrule
 \end{tabular}
\end{table}

As shown above, the type (i) “\textit{tough}” constructions (where the CNO movement is involved) are available in a limited number of languages including English. Recall now that our prediction was that English-like \textit{tough} constructions are available only in DP-languages based on the CNO analysis of \textit{tough} constructions where the presence of the DP layer is crucial for the CNO to smuggle the \textit{tough} subject. In this regard, the NP/DP distinction and the availability of the type (i) \textit{tough} constructions in the languages under consideration are summarized in the following table. 

\begin{table}[H]
\caption{\textit{NP/DP distinction and availability of type (i)} tough \textit{construction}}
\label{tab:2:npdp}
 \begin{tabular}{l rr}
  \lsptoprule
     \textbf{Languages}  & \textbf{NP/DP} & \textbf{\textit{Tough} (i)}\\
  \midrule
  English  &   DP  &    Yes  \\
  German  &   DP &   Yes  \\
    Spanish  &   DP &   Yes  \\
    Italian  &   DP &   Yes  \\
    French  &   DP &   Yes  \\
    Bulgarian  &   DP &   No  \\
    Hungarian  &   DP &   No  \\
    Thai  &   NP &   No  \\
    Japanese  &   NP &   No  \\
    SC  &   NP &   No  \\
    Slovenian  &   NP &   No  \\
    Polish  &   NP &   No  \\
    Russian  &   NP &   No  \\
  \lspbottomrule
 \end{tabular}
\end{table}

Table \ref{tab:2:npdp} confirms that \textit{tough} constructions are indeed allowed only in DP languages. Here, we can establish a one-way correlation, i.e. \textit{tough} constructions with (Complex) Op movement are allowed only in DP languages. This is accounted for under the proposed analysis where only DP languages can have the complex null operator, which is needed for the derivation of \textit{tough} constructions.

Note that the correlation between the availability of \textit{tough} constructions and DP languages is a one-way correlation, because of Hungarian or Bulgarian. A remaining question is, then what makes Hungarian and Bulgarian different among DP languages regarding the availability of \textit{tough} constructions. I suggest here that other independent factors are involved. In the case of Bulgarian, its \textit{tough} formation utilizes a subjunctive complement, as infinitive is rarely used in this language. 

Even in English, \textit{tough}-formation movement is very local, i.e. it can only cross an infinitival clause but not a finite clause, which was pointed out by \citet{Stowell1986}. 

\begin{exe}
\ex \label{36ha}
\begin{xlist}
\ex \label{36aha}
*Betsy$_{i}$ is easy [Op$_{i}$ [ PRO to expect [ t$_{i}$ fixed the car] ] ].

\ex \label{36bha}
*John is easy [Op$_{i}$ [PRO to believe [ t$_{i}$ kissed Mary] ] ]. 

\ex \label{36cha}
??This car is hard [Op$_{i}$ [ PRO to claim [ [ Betsy fixed t$_{i}$ ]] ]].

\ex \label{36dha}
??That language is impossible [Op$_{i}$ [PRO to say [ [ Greg will learn t$_{i}$ ]] ]].  \hfill \citep[477]{Stowell1986}                                        
\end{xlist}
\end{exe}

I suggest then that the movement across a subjunctive clause boundary in Bulgarian is prohibited in the same way, which blocks the possibility of the relevant \textit{tough} constructions.

Turning now to Hungarian, it has been argued that the Op movement in \textit{tough} constructions in some languages like German is more local than in English, in that it is not even allowed out of all infinitives \citep{Wurmbrand2001,Kayne1989,Roberts1997}, more precisely it is allowed only out of “small” infinitives (i.e. restructuring). While I will not address the issue here, it is worth noting that it may be related to Hungarian. \citet{Kenesei2005} and \citet{Dalmi2004} argue that infinitival constructions in Hungarian project a full-fledged CP by pointing out that it has typical left peripheral projections with the strict order that is also found in finite clause. This property of infinitival constructions in Hungarian may be the reason why \textit{tough} construction is not allowed in Hungarian; \textit{tough} formation movement may not be allowed to cross the Hungarian infinitive clause.

\section{Conclusion} \label{s4ha}
In conclusion, I have argued for the CNO analysis \citep{Hicks2009} of \textit{tough} constructions in English, with smuggling of the nominative \textit{tough} subject. This analysis resolves the problems of the previous analyses by blocking the \textit{tough} subject from Case assignment in the infinitival clause, and it also avoids the Improper Movement issue. The smuggling of the \textit{tough} subject is what resolves both issues. Crucially, for the smuggling to take place, there has to be a DP layer above a bare Op. Based on this, a prediction was made that \textit{tough} constructions involving nominative subjects as well as Op movement will be possible only in DP languages. This prediction was borne out through a survey of 7 DP-languages and 6 NP-languages, which showed that \textit{tough} constructions are indeed possible in only DP languages. Under the proposed analysis, the null Op does not have any uninterpretable features that would enable it to smuggle the \textit{tough} subject. In DP languages, there is a DP above the null Op. It is this DP that smuggles the \textit{tough} subject. The only difference between DP languages and NP languages is then that there is a DP above the null Op in DP languages. The lack of (type (i)) \textit{tough} constructions in NP languages was attributed to the inability of Op to smuggle the \textit{tough} subject. It was also noted that Japanese and Thai, which are NP languages, have the relevant \textit{tough} construction when its subject is a PP. This is captured under the proposed analysis because PP itself brings in a richer structure for the Op, enabling the smuggling of the subject, regardless of the presence of the DP layer. 

\section*{Abbreviations}
\begin{tabularx}{.5\textwidth}{lQ}
\textsc{a}    & argument                      \\
\textsc{acc}  & accusative                    \\
\textsc{adv}  & adverb                        \\
\textsc{ap}   & adjective phrase                \\
\textsc{asp}  & aspect                         \\
\textsc{cno}  & complex null operator             \\
\textsc{comp} & complementizer                         \\
\textsc{cp}   & complementizer phrase                   \\
\textsc{d}    & determiner                             \\
\textsc{dat}  & dative                                 \\
\textsc{dp}   & determiner phrase              \\
\textsc{epp}  & extended projection principle                \\
\textsc{ind}  & infinitive                                   \\
\textsc{iQ}   & interpretable question feature             \\
\textsc{n}    & noun                           \\
\textsc{neg}  & negation                            \\     
\end{tabularx}
\begin{tabularx}{.5\textwidth}{lQ}
 \textsc{nom}  & nominative              \\
 \textsc{np}   & noun phrase             \\
 \textsc{op}   & operator                \\
\textsc{pp}   & preposition             \\
 \textsc{pres} & present                 \\
\textsc{pro}  & pronominal anaphor      \\
 \textsc{pst}  & past                    \\
 \textsc{rec}  & reciprocal              \\
 \textsc{s}    & sentence                \\
\textsc{sc}   & Serbo-Croatian          \\
 \textsc{slp}  & speech level particle   \\
 \textsc{subj} & subject                 \\
 \textsc{t}    & tense                   \\
 \textsc{tp}   & tense phrase            \\
 \textsc{uF}   & uninterpretable feature \\
\textsc{vp}   & verb phrase            \\
\end{tabularx}


\section*{Acknowledgements}
I would like to thank Jonathan Bobaljik, William Snyder and especially Željko Bošković for invaluable comments and discussion. My gratitude also goes to my language consultants, Ksenia Bogomolets, Marcin Dadan, Éva Dékány, Ivana Jovović, Pavel Koval, Sabine Laszakovits, Sidney Mao, Emma Nguyen, Roberto Petrosino, Asia Pietraszko, Vesela Simeonova, Adrian Stegovec, Aida Talić, Panat Taranat, Alexandre Vaxman, and Gabrel Martinez Vera. I also thank two anonymous reviewers. The responsibility of any errors is of course my own.

\printbibliography[heading=subbibliography,notkeyword=this]

\end{document}