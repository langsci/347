\documentclass[output=paper]{langscibook} 
\title{Introduction: The size of things, volume~2} 
\author{
    Zheng Shen%
        \affiliation{National University of Singapore}
    and 
    Sabine Laszakovits%
        \affiliation{Austrian Academy of Sciences; University of Connecticut}
}  

%\abstract{Change the  abstract in chapters/prefaceEd.tex Lorem ipsum dolor sit amet, consectetur adipiscing elit. Aenean nec fermentum risus, vehicula gravida magna. Sed dapibus venenatis scelerisque. In elementum dui bibendum, ultricies sem ac, placerat odio. Vivamus rutrum lacus eros, interdum scelerisque est euismod eget. Class aptent taciti sociosqu ad litora torquent per conubia nostra, per inceptos himenaeos.}

\begin{document}
\maketitle


\noindent \textit{Size} in grammar, broadly construed, is the focus of this two-volume collection, \textit{the size of things.}
Under the umbrella term `size' fall the size of syntactic projections, the size of feature content, and the size of the reference sets. 
Papers in Volumes I share the focus of structure building while Volume II presents papers looking into size effects in movement, agreement, and interpretation. 
By bringing together a variety of research projects under the common theme, we hope this collection could inspire new connections and ideas in generative syntax and related fields. 

Contributions in volume 2 are grouped into three parts. The first six papers explore various interaction between size and movement. 
Despić discusses two approaches to Left Branch Extraction in Serbo-Croatian and Japanese and argues that LBE involves movement of a smaller, rather than a larger constituent.
Harðarson compares movement of genitive possessors of two different sizes in Icelandic: one with modifier and one without, and proposes a head movement analysis for the former and an overt quantifier raising analysis for the latter.
Hattori links the availablity of \textit{tough}-movement with the size of nominal domain and argues that NP languages like Japanese and Serbo-Croatian lack English-type \textit{tough} constructions. 
Travis looks into head movement, a movement unique in the size of its moved element, and proposes that it has a natural place in narrow syntax in a feature-based locality system.
Lohninger investigates multiple-wh questions and argues that A' wh-movement, A-wh-movement, and cross-clausal-A dependencies target different CP projections. 
Bailyn argues that multiple-WH-fronting and topic/focus constructions in Slavic languages poe challenges to the strict cartography approach to sentence structure. 

The next group of five papers look into agreement and morphology-related matters like allomorphy. 
Baker and Camargo Souza accounts for the same-subject markers in complement clauses of aspectual verbs in Yawanawa and Shipibo-Konibo by proposing that these complement clauses lack CP layer, making agreement with matrix subject possible.
Wood's contribution provides a feature-bundle-based analysis to an exceptional ameliorative effect in dative-nominative construction involving the syncretic form -st in Icelandic.
Calabrese proposes a syntactic truncation analysis of full-fledged and reduced motion verb constructions in the Campiota vernacular. 
Kalin looks into affixal morphemes in Nancowry with exponents whose distribution is sensitive to the prosodic size of the stem they affix to.
Newell offers a phonological rather than a morphological analysis of Tamil pronominal alternations. 

The last three papers explore semantic-oriented issues, in particular on size of reference domains and NPI licensing.  
Prinzhorn and Schmitt offer a semantic account for a restriction on the reference size involved in embedded predicates of partial control construction in German.
Neubarth discusses several issues with strong and weak NPIs especially under comparatives. 
Sauerland and Yatsushiro's contribution shows that exceptive phrases in Japanese form strong NPIs, different from their English counterpart and account for the contrast by proposing that the two languages select different exhaustification operators. 

All the papers in these two volumes are influenced in various ways by the work of Susi Wurmbrand, who not only pioneers the investigation into clausal complements across languages from the lenses of binding, finiteness, movement, restructuring, tense, and verb clusters, but has also deepened our understanding of Agreement, Case, features, and quantifier raising. Furthermore, Susi has had a direct personal impact on the work of all contributors and editors, and so we dedicate this book to her not only in recognition of her achievements, but also in gratitude of her generosity to us.


{\sloppy\printbibliography[heading=subbibliography,notkeyword=this]}
\end{document}
