% !TEX encoding = IsoLatin9
% !TEX TS-program = XeLaTeX
\documentclass[output=paper,colorlinks,citecolor=brown,
% hidelinks,
% showindex
]{langscibook}
\author{Martin Prinzhorn\affiliation{University of Vienna} and Viola Schmitt\affiliation{Humboldt University Berlin}\orcid{}}
\title{Partial control and plural predication}
\abstract{Partial control is known to exhibit constraints on which predicates can be embedded: While it allows for collective predicates like \textit{assemble}, it blocks predicates with the VP-modifier \textit{each}, for instance. This restriction is surprising if the subject of the embedded predicate is a plurality, and existing accounts appeal to syntactic explanations. Based on German data, we argue that the constraint is semantic in nature: Embedded predication in PC cannot ascribe properties to parts of a plurality that do not contain the matrix subject -- essentially because, following (a simplified version of) \cite{Pearson:2016}, PC involves a property attribution \textit{de se}. We sketch an implementation of this idea using the “plural projection” system (\citealt{Schmitt:2019, Haslinger:2018a, Haslinger:2018b}): It lets us “divide” certain predicates -- e.g., those involving \textit{each} -- into “parts”. The afore-mentioned constraint must hold for each such predicate part, which rules out certain types of predication in PC.}


\begin{document}
\maketitle

\section{Introduction}\label{prinzhornsec:1}

Some control verbs in English allow for so-called  {\it partial control} (PC): The matrix subject is semantically singular, but the embedded predicate can only hold of semantic pluralities (\cite{Wilkinson:1971, Landau:2000, Wurmbrand:2001, Wurmbrand:2002, Pearson:2016} a.o.). {\it assemble} requires a semantically plural subject, as shown in (\ref{prinzhornex1a}). Yet, when the relation between such a semantically singular subject and the predicate is “mediated” by the control verb {\it want}, as in (\ref{prinzhornex1b}), the result is acceptable. (\ref{prinzhornex2a}) and (\ref{prinzhornex2b}) make the same point for {\it go on vacation together}.

\todo[inline]{Please check source of examples: are they both (1a) in Pearson 2016?}
\ea  
\ea The children\slash\#Ada assembled in the hall. \label{prinzhornex1a}
\ex	Ada wanted to assemble in the hall. \hfill \citep[(1a)]{Pearson:2016} \label{prinzhornex1b}
\ex The children\slash\#Ada went on vacation together. \label{prinzhornex2a}
\ex Ada expected to go on vacation together. \hfill \citep[(1a)]{Pearson:2016} \label{prinzhornex2b}
\z\z
	
It thus seems that the embedded subject can introduce a plurality of individuals which the denotation of the matrix subject is a proper part of. This is schematized in the simplified rendering of (\ref{prinzhornex1b}) in (\ref{prinzhornex3}).\footnote{Unless noted otherwise, we draw on some basic notions of plural semantics: We assume a set $A \subseteq D_e$ of atomic individuals, a binary operation $+$ on $D_e$ and a function $f\colon (\mathcal{P}(A)\setminus \{\emptyset\}) \to D_e$ such that: 1. $f(\{x\}) = x$ for any $x \in A$ and 2. $f$ is an isomorphism between the structures $(\mathcal{P}(A)\setminus \{\emptyset\}, \cup)$ and $(D_e, +)$. Hence there is a one-to-one correspondence between plural individuals and nonempty sets of atomic individuals. We will use the notation in (\ref{prinzhornDD}).

\ea \label{prinzhornDD} For any $x, y \in D_e$, $S \subseteq D_e$:
\ea $x \leq y \Leftrightarrow x + y = y$ (``$x$ is a part of $y$'')
 \ex $x \leq_{a} y \Leftrightarrow x \leq y \land x \in A$ (``$x$ is an atomic part of $y$'')
\ex  $\scalerel*{+}{S}S = f(\bigcup\{f^{-1}(x)\ |\ x \in S\})$ (the sum of all individuals in $S$)
\z\z}

\ea	\label{prinzhornex3} \sem{g}{[Ada wanted [\textsc{pro}$_{1}$ to assemble in the hall]]} = 1 iff \textbf{Ada} $< g(1)$ $\&$ Ada wanted $g(1)$ to assemble in the hall. \z

PC is known to be subject to two kinds of restrictions. First, there are constraints w.r.t. the {\it matrix predicate} (MP): Control verbs like {\it want}/{\it expect} license PC, but control verbs like {\it  manage} don't, as shown in (\ref{prinzhornex4}) (\citealt{Landau:2000, Pearson:2016} a.o.). We will write “MPs\textsubscript{PC}” for matrix predicates that license PC.

\ea[\#]{\label{prinzhornex4}
  Ada managed  to go on vacation together.\jambox*{\cite[(2b)]{Pearson:2016}}}
\z
	
We focus on the second restriction, which concerns the {\it embedded predicate} (EP): While collective predicates like  {\it assemble} can occur in PC, predicates with what \cite{Schwarzschild:1996} calls “plurality seekers” -- e.g., the distributive VP-mod\-i\-fi\-er {\it each}, the reciprocal  {\it each other} -- can't (\citealt{Landau:2000} a.o.), as shown by (\ref{prinzhornex5b}), (\ref{prinzhornex6b}).

\ea\label{prinzhornex5}
\ea[\#]{\label{prinzhornex5b}Ada told her friends that she wanted to each donate at least \$100.\\ \jambox*{adapted from \citet[48 (61a)]{Landau:2000}}}
\ex[\#]{\label{prinzhornex6b}Ada told Bea that she expected to meet each other at 6 today.\\\jambox*{adapted from \citet[59 (61a)]{Landau:2000}}}
\z\z

This contrast is unexpected if the embedded subject is taken to introduce a plurality of individuals: Expressions containing plurality seekers cannot combine with semantically singular subjects like {\it Ada}, but are perfectly fine with semantically plural subjects like {\it the children}, as shown in (\ref{prinzhornex7}).

\ea \label{prinzhornex7}
\ea[]{The children\slash\# Ada will meet each other at 6 today.}
\ex[]{The children\slash\# Ada each donated at least \$100.}
\z\z

To our knowledge, it is usually assumed that this restriction has {\it syntactic} reasons (see in particular \citealt{Landau:2000}). Here, we will argue, based on German data, that it is {\it semantic} in nature: The EP cannot attribute properties to parts of a plurality that don't contain the matrix subject --  as the meaning of the MP requires “self-attribution” of a property. (So, \textit{Size matters!} -- not in syntactic terms for us, but in terms of the parts of the pluralities that are attributed the EP-properties: They must be “large enough” to contain the matrix subject.) We provide an informal sketch of how this restriction can be implemented compositionally, combining  the “plural projection” account of plural predication (\citealt{Schmitt:2019, Haslinger:2018a, Haslinger:2018b}) with a simplified version of \cite{Pearson:2016} semantics for MPs\textsubscript{PC}.

Note: Our examples/scenarios will mostly be based on a toy model with children Ada, Bea and Carl, dogs Dean, Eric and Fay, and cats Gene, Hans and Ivo. 



\section{Semantic constraints on EPs in PC in German}\label{prinzhornsec:2}

In the following, we describe the situation in German\footnote{The judgements reported here are our own and those of 10 speakers we consulted. There was minor variation in this pool of speakers, which we note when discussing the relevant examples. This is not to say there might not be more variation: A reviewer notes that their judgements differ from those reported here, also in terms of the availability of PC with the type of matrix predicates considered here. The root of this variation is unclear to us at this point.} and give a first semantic characterization of the type of EP blocked in PC.

\subsection{The EP-restriction in German}\label{prinzhornsec:2.1}

German, like English, allows for PC (with some MPs\footnote{\cite{Pitteroff:2017} distinguish between “fake” and “true” PC in German, where the former involves a silent comitative (but see \citealt{Landau:2016}). Our examples would be cases of “true” PC.}) if the EP is a collective predicate: The examples in (\ref{prinzhornex8}) are parallel, in this respect, to (\ref{prinzhornex1b}) and  (\ref{prinzhornex2b}). Further, as in English, the EP cannot contain the overt distributivity operator \textit{jeweils} ($\approx$ \textit{each}), as shown in (\ref{prinzhornex9a}),\footnote{The relevant reading is the one where \textit{jeweils} distributes over an individual plurality. Two of our consultants were unsure about their judgments in this case. This might be connected to  the fact (discussed by \citealt{Zimmermann:2002}) that \textit{jeweils} also permits distribution over parts of an event plurality.  See \sectref{prinzhornsec:5}.} or reciprocal \textit{einander}, as illustrated in (\ref{prinzhornex9b}). (\ref{prinzhornex9c}) shows that such elements are not excluded \textit{per se} in control constructions -- they are fine with a plural matrix subject.



\ea \label{prinzhornex8} {\sc context: } Ada called Bea and Carl. Bea reports:
\ea \gll {Ada} {hat} {vor}, {sich} {morgen} {auf} {dem} {Platz} {zu} {treffen}.\\
   Ada intends {} \textsc{refl} tomorrow on the square to meet\\
\glt `Ada intends to meet in the square tomorrow.'\label{prinzhorncoll-pred}
\ex   \gll {Ada} {hat} {vor}, {gemeinsam} {in} {den} {Urlaub} {zu} {fahren}.\\
   Ada intends {} together in the vacation to go\\
\glt `Ada intends to go on vacation together.'\label{prinzhornvaca}
\z 
\ex \label{prinzhornex9} {\sc context: } Ada called Bea and Carl ahead of the trip to the pet shop:
\ea[\#]{ \gll {Ada} {hat}  {vor}, {jeweils} {zwei} {Tiere} {zu} {kaufen}.\\
    Ada intends {} each two dogs to buy \\
\glt `Ada intends to buy two dogs each.' \label{prinzhornex9a}}
\ex[\#]{\gll  {Ada} {hat} {vor}, {einander} {mit} {Katzenfutter} {zu} {bewerfen}.\\
    Ada intends {} each-other with cat-food to throw-at\\
\glt `Ada intends to throw cat food at each other'  \label{prinzhornex9b}}
\ex[]{ \gll {Ada} {und} {Bea} {haben} {vor}, {jeweils} {zwei} {Tiere} \texttt{zu} {kaufen}\\
         Ada and Bea intend {} each two animals to buy\\
\glt `Ada and Carl intend to buy two pets each.'\label{prinzhornex9c}}
\z \z

A further restriction in German, which, to our knowledge, has not been noted in the literature, is that the EP cannot be construed as {\it cumulative}. Cumulative readings -- namely, certain weak truth-conditions -- can be observed for sentences with two or more semantically plural expressions (see \citealt{Langendoen:1978} a.o.): The sentence in (\ref{prinzhornex10a}) is true in  scenario (\ref{prinzhornex10b}), and generalizing over the verifying scenarios, we can paraphrase its truth-conditions as in (\ref{prinzhornex10c}). Thus, descriptively, the children-plurality and the dog-plurality stand in a cumulative feeding-relation.

\ea \label{prinzhornmyex}
\ea \gll {Die} {drei}  {Kinder} {haben} {die} {drei} {Hunde} {gefüttert}.\\
    The three children have the three dogs fed \\
\glt `The three children fed the three dogs.' \label{prinzhornex10a}
\ex   \textsc{scenario}: Ada fed Dean. Bea fed Eric. Carl fed Fay.  \label{prinzhornex10b}
\ex \sem{}{(\ref{prinzhornex10a})} = 1 iff each A,B,C fed at least one of  D, E,F $\&$ each of D,E,F was fed by at least one of A,B,C.\label{prinzhornex10c}
\z \z

In PC, the embedded subject cannot stand in cumulative relation with another plurality inside the EP. The fact that (\ref{prinzhornex11b}) is false in the `cumulative' scenario (\ref{prinzhornex11a}) shows that it cannot express that a plurality including Ada (e.g., \textbf{Ada+Bea+Carl}) stands in a cumulative feeding relation with the dog-plurality.\footnote{While all our consultants reject the cumulative reading for this sentence, a reviewer considers this reading possible. As stated above, the reason for this variation is unclear to us.} The sentence can only express that Ada's intention is that she herself feeds all three dogs.\footnote{Likewise, (\ref{prinzhornex12b}) is false in the `cumulative' scenario (\ref{prinzhornex12a}) -- it can only express that Ada intends to drink exactly 30 beers herself. 

\ea 
\ea \label{prinzhornex12a} {\sc scenario: } It's Ada's 12th birthday. She invited Bea and Carl. According to Ada, each of the three is supposed to drink 10 beers.
\ex \gll {Ada} {hat} {vor}, {genau} {30} {Bier} {zu} {trinken}.\\
   Ada intend {} exactly 30 beers to drink \\
   \glt `Ada intends to drink exactly 30 beers.' \hfill \textbf{$\%$ false} in (\ref{prinzhornex12a})  \label{prinzhornex12b}
\z \z

}

\ea \label{prinzhornex11}
\ea \label{prinzhornex11a} {\sc scenario: } Ada is assigning jobs at the animal shelter. She assigns herself the job of feeding Dean,  Bea that of feeding  Eric,  Carl that of feeding  Fay. She wants everyone to be done by 11 am. 
\ex \gll {Ada} {hat} {vor}, {die} {drei} {Hunde} {am} {Morgen} {zu} {füttern}.\\
   Ada intends {} the three dogs in-the morning to feed \\
   \glt `Ada intends to feed the three dogs in the morning.' \hfill \textbf{false} in (\ref{prinzhornex11a}) \label{prinzhornex11b}
\z \z

Again, cumulative readings are not ruled out \textit{per se} in control constructions, as witnessed by the fact that (\ref{prinzhornex120b}) is true in the `cumulative' scenario (\ref{prinzhornex120a}).

\ea \label{prinzhornex120}
\ea \label{prinzhornex120a} {\sc scenario: }  Ada intends to feed dog Dean, Bea intends to feed dog Eric, Carl intends to feed dog Fay.
\ex \gll {Ada}, {Bea} {und} {Carl} {haben} {vor}, {die} {drei} {Hunde}  {zu} {füttern}.\\
   Ada, Bea and Carl intend {} the three dogs  to feed \\
   \glt `Ada, Bea and Carl intend to feed the three dogs.' \hfill \textbf{true} in (\ref{prinzhornex120a}) \label{prinzhornex120b}
\z \z

In summary, we can thus state the restriction on the EPs in PC as in (\ref{prinzhornex-ep}).  


\ea EP-restriction (German):  The EP in PC cannot contain overt distributivity operators or reciprocals and prohibits a cumulative reading of an embedded plurality relative to the subject of the embedded clause. \label{prinzhornex-ep} \z

\subsection{Collective predicates vs. “part-predicates”}\label{prinzhornsec:2.2}

Semantically, the difference between collective predication, on the one hand,  and distributive/cumulative predication, on the other, is that the latter, but not the former, must access the {\it part-structure} of its plural argument(s): A collective predicate like {\it assemble} can attribute a property to a plurality as whole. This behavior differs from that of the other class of predicates, which we subsume under the name \textit{part predicates}:  Predicates modified by {\it each} (or German {\it jeweils}) must access the part-structure of the argument plurality in the sense that they ascribe the property denoted by the predicate modified by \textit{each} to each atomic part of the plurality they modify.  For instance, following \citealt{Link:1987} a.o., the EP in (\ref{prinzhornex9a}) has the denotation in (\ref{prinzhornex13a}) (we will ignore tense throughout). Cumulative readings of predicates like \textit{feed the three dogs} resemble distributive predication in the sense that they, too, must access the part-structure of the pluralities involved. This is made explicit by the simplified denotation in (\ref{prinzhornmy13}).\footnote{(\ref{prinzhornmy13}) ignores the potential presence of cumulation operators and makes the simplifying assumption that {\bf feed} primitively holds of pairs of atomic individuals.} 
As most existing analyses of reciprocals as in (\ref{prinzhornex9b} either assume that they involve distributivity (e.g., \citealt{Heim:1991}) or cumulativity, \citep{Beck:2001}, any analysis that explains why overt distributivity operators and cumulative readings are blocked should extend to reciprocals.

\ea	 
\ea \sem{}{each buy two pets} = $\lambda x_{e}. \forall y \le_{a} x \,(\textbf{buy two pets}(y))$\label{prinzhornex13a}
\ex	\sem{}{feed the three dogs} = $\lambda x_{e}. \forall y \le_{a} x \,(\exists z \le_{a} \sem{}{the three dogs}$\\$ \&\,\textbf{feed}(z)(y)) \,\&\, \forall z \le_{a} \sem{}{the three dogs} (\exists y \le_{a} x \&\, \textbf{feed}(z)(y))$\label{prinzhornmy13}
\z\z

In light of this distinction, we posit our preliminary hypothesis 1.


\ea H1 (preliminary): EPs in PC cannot access the part-structure of the plurality corresponding to the subject of the EP.\label{prinzhornex14}
\z 




\section{A syntactic account of the EP-restriction?}\label{prinzhornsec:3}

But maybe H1 is on the wrong track -- maybe the EP-restriction doesn't resort to semantic properties of the EP but has a syntactic explanation. 

\cite{Landau:2000} notes that the EP-restriction in English extends to plural morphology that would have to be triggered by  the embedded subject. This observation extends to German, as shown in (\ref{prinzhornex15}) (adapted from English examples in \citealt{Landau:2000}).


\ea \label{prinzhornex15}
\ea \textsc{scenario:} Ada wants herself, Bea and Carl to join the club.
\ex[*]{\gll {Ada} {hat} {vor}, {Mitglieder} {von} {diesem} {Verein} {zu} {werden}.\\
   Ada intends {}  member-\textsc{pl} of this club to become\\
   \glt `Ada intends to become members of the club.'} 
   \z\z



\cite{Landau:2000} concludes from this that while the subject of the embedded clause doesn't inherit semantic number from the matrix subject (otherwise, PC shouldn't be possible), it inherits its syntactic number (singular), as sketched in (\ref{prinzhornex16}).\footnote{\cite{Pearson:2016} makes an analogous claim, but submits that the \textsc{sg}-feature on \textsc{pro} is deleted. 
}

\ea [Ada$_{sg}$ plant [\textsc{pro}$_{sg}$ Mitglieder von diesem Verein zu werden]]\label{prinzhornex16}
\z

Landau claims that this explains the EP-restriction, arguing that elements like \textit{each} and reciprocals can only be licensed by syntactically plural expressions. To support his point, he provides  examples like (\ref{prinzhornex17}), which shows that a semantically plural, but syntactically singular expression cannot license the VP-modifier \textit{each}.


\ea * \textit{The class each submitted a different paper.} \phantom{.}\hfill \cite[49 (66d)]{Landau:2000} \label{prinzhornex17}
\z

Crucially, this account is insufficient for German: Syntactic plurality is not necessary to license the elements  blocked in EPs in PC. (\ref{prinzhornex18a}) shows that the singular collective DP {\it  das Paar} (`the couple') licenses reciprocals,  (\ref{prinzhornex18b}) shows that it can combine with predicates containing \textit{jeweils}.\footnote{While German  \textit{jeweils} is not identical in its behavior to English binominal  \textit{each} (see \citealt{Stowell:2013} a.o. for binominal \textit{each} and \citealt{Zimmermann:2002} for the differences), it resembles the latter more closely than English VP-\textit{each}, as only VP-\textit{each} is incompatible with singular collective nouns, as shown by the contrast between (\ref{prinzhornex17}) and (\ref{prinzhornfnnex5}).

\ea    \textit{The couple have drunk five beers each.} \hfill Tim Stowell (pc)  \label{prinzhornfnnex5} \z  

} Moreover, it can partake in cumulative readings, as witnessed by the fact that (\ref{prinzhornex19b}) is true in scenario (\ref{prinzhornex19a}) \footnote{Singular generics also seem to license cumulative readings,  and also predicates modified by \textit{jeweils}, as in (\ref{prinzhornexfn45}) (example due to Magdalena Roszkoswki (pc)).

\ea 
%\ea 
%\gll Der B\"{a}r lebt in Nordamerika und Osteuropa. \\
%The.\textsc{sg} bear lives in North-American and East-Europe\\
%\glt `Bears live in North America and Eastern Europe.'
%\label{prinzhornexfn44} 
%\ex 
\gll \textit{Der} \textit{Spatz} \textit{hat} \textit{jeweils} \textit{bis} \textit{zu} \textit{150} \textit{Zecken} \textit{auf} \textit{seinem} \textit{Federkleid}.\\
The.\textsc{sg} sparrow has \textit{jeweils} up to 150 ticks on his feather-dress\\
\glt `Sparrows have up to 150 ticks each in their feathers.'
\label{prinzhornexfn45} 
\z

}

\ea \label{prinzhornex18} 
\ea 
\gll \textit{Das} \textit{Paar} \textit{sah} \textit{einander} \textit{vor} \textit{Gericht}.\\
  The couple saw each-other in-front of  court \\
\glt `The couple saw each other in court.' \label{prinzhornex18a}
\ex   \gll \textit{Dieses} \textit{Paar} \textit{da} \textit{drüben} \textit{hat} \textit{jeweils} \textit{10} \textit{Bier} \textit{getrunken}! \\
  This couple there over has jeweils 10 beers drunk\\
\glt `This couple over there drank 10 beers each'  \label{prinzhornex18b}
\ex  \textsc{scenario:} Sue and Zoe are a couple, living in different households. Sue, a banker, earned 1000 Euros today. Zoe, a designer, made 2000 Euros today. Mary is impressed and tells me: \label{prinzhornex19a}
\ex \gll \textit{Dieses} \textit{Paar} \textit{hat} \textit{heute} (\textit{insgesamt}) \textit{genau} \textit{3000} \textit{Euro} \textit{verdient}!\\
This couple has today (in-total) texactly 3000 Euros earned \\
\glt `The couple earned 3000 Euros (in total) today' \label{prinzhornex19b}\phantom{.}\hfill \textbf{true} in (\ref{prinzhornex19a})
\z\z

Such semantically plural, syntactically singular elements also license distributive/cumulative EP-predicates in PC, as shown in (\ref{prinzhornex20}). So while we follow \cite{Landau:2000} in assuming that the lack of a plural feature on \textsc{pro} and \todo{Something missing in this sentence!} on the matrix subject plays a role in the lack of plural morphology shown in (\ref{prinzhornex15}),\footnote{Syntactic plurality by itself, however, is not sufficient to license plural morphology in the embedded structure in (\ref{prinzhornex15}). This is witnessed by examples like (\ref{prinzhorndame}), which were brought to our attention by a reviewer: A  semantically singular matrix subject with syntactic plural marking (i.e., the polite plural) cannot license plural morphology on the embedded noun.


\ea[*]{\gll  \textit{Meine} \textit{Dame}, \textit{Sie} \textit{haben vor} \textit{Mitglieder} \textit{von} \textit{diesem} \textit{Verein} \textit{zu} \textit{werden}.\\
   my lady \textsc{3rd}.\textsc{pl}  intend.\textsc{pl} members of this club to become \\
\glt  `My lady, you intend to become members of this club'\label{prinzhorndame}}
\z

} we submit that it does not explain the EP-restriction in German: The modifier {\it jeweils}, reciprocals and cumulative readings don't require a \textit{syntactically} plural subject.



\ea \label{prinzhornex20}
\gll \textit{Dieses} \textit{Paar} \textit{hat} \textit{vor}, \textit{jeweils} \textit{10} \textit{Bier} \textit{zu} \textit{trinken}.\\
this couple intends {} each 10 beers to drink\\
\glt `The couple intends to drink 10 beers each' 
\z



\section{ A tentative semantic account of the EP-restriction}\label{prinzhornsec:4}

Having ruled out a syntactic account, we now link our preliminary H1 to the meaning of MPs\textsubscript{PC} and sketch a compositional implementation. It will turn out that EPs in PC \textit{can} access the part-structure of their subject -- but that MPs\textsubscript{PC} impose certain restrictions on what these parts can look like.



\subsection{Spelling out the underlying intuition, Part I}\label{prinzhornsec:4.1}

To get an intuitive grasp of the status of H1 in PC, we first need a broad conception of the semantics of MPs\textsubscript{PC}. \cite{Pearson:2016} argues convincingly that all MPs\textsubscript{PC} are attitude verbs.\footnote{But not all attitude verbs license PC, see \sectref{prinzhornsec:5}.} Thus, they require us to relativize the parameters of evaluation for the embedded structure to the matrix subject. Take \textit{expect}: When evaluating the sentence in (\ref{prinzhornex31-a}) at world $w$, $S$ is evaluated w.r.t. Ada's belief worlds (or their future states) in $w$ -- the sentence can be true if no zombies exist in $w$, but Ada must believe in $w$ that they do. Thus, we evaluate $S$ relative to worlds  Ada considers candidates for $w$ (which we will call \emph{world-candidates}).

\ea  \label{prinzhornex31-a} Ada expects [$_{S}$ that she will meet a zombie at my house]. \z

Following  \cite{Chierchia:1989} treatment of control, \cite{Pearson:2016} argues that MPs\textsubscript{PC} involve an attitude \textit{de se}: While (\ref{prinzhornex31-d}) with a finite complement is true in scenario (\ref{prinzhornex31-b}), the analogous control sentence  in (\ref{prinzhornex31-c}) is not: In (\ref{prinzhornex31-c}) Ada must attribute the property of becoming rich and famous to an individual she considers a candidate for herself (which we will call \emph{self-candidates}). Informally speaking, MPs\textsubscript{PC} thus always involve a relation between the matrix subject $x$ and a self-candidate of $x$.


\ea 
\ea  \textsc{scenario:} Ada is an amnesiac. She reads a linguistics article that she herself wrote, but she has forgotten this fact. Impressed, she remarks, “The author of this paper will become rich and famous, but I won't”. \label{prinzhornex31-b}
\ex  \textit{Ada expects to become rich and famous}. \hfill \textbf{false} in (\ref{prinzhornex31-b}) \label{prinzhornex31-c}
\ex \textit{Ada expects that she will become rich an famous}.  \hfill \textbf{true} in (\ref{prinzhornex31-b})
\newline \phantom{.}\hfill cf. \cite[(9)]{Pearson:2016} \label{prinzhornex31-d}
\z\z 

Omitting a crucial part of Pearson's account (see \sectref{prinzhornsec:5}), she submits that MPs\textsubscript{PC} “expand” the self-candidates of the matrix subject in that subject's world-candidates: MPs\textsubscript{PC} combine with properties (of worlds and individuals in our simplified rendering, see \citealt{Chierchia:1989}) and these properties are evaluated with respect to pairs $\langle w, x \rangle$, where $w$ is a world-candidate of the matrix subject and $x$ an expansion of the self-candidate of the matrix subject in $w$ (so $x$ will be the self-candidate, or a plurality properly containing that self-candidate).\footnote{\cite{Pearson:2016} notes that this analysis does not straightforwardly extend to (partial) object control. We here do not discuss how to expand our own analysis to object control: \cite{Pearson:2016} does not present a definitive account of the matrix predicates in such cases (although she probes some possibilities), so we would have to make a proposal regarding their semantics, which is beyond the confines of this paper.}

Take the sentence in (\ref{prinzhornex32a}) (≙\,\ref{prinzhornvaca}): \textit{intend} expresses the relation in (\ref{prinzhornex32b}), where  $\mathcal{I}_{x,w}$ is the set of all pairs $\langle w',y \rangle$, s.th. $w'$ is an world-candidate of $x$ in $w$ and $y$ a self candidate of $x$ in $w'$.\footnote{This is a simplification: We here equate the \textsc{intend}-relation with the \textsc{belief}-relation.} Combining this predicate with the denotation of its complement in (\ref{prinzhornex32c}) and with the subject yields (\ref{prinzhornex32d}): (\ref{prinzhornex32a}) is correctly predicted true in scenarios where Ada's intention is that she and other people  go on vacation together.\footnote{A reviewer wonders whether it is sufficient to require \textit{some} expansion of the self-candidate in each world candidate. We think Pearson's treatment here is correct: It seems to us that an example like (\ref{prinzhornex2b}) can be true in a context where Ada expects to go on vacation with Bea or Carl.}


\ea 	\label{prinzhornhazels}
\ea Ada intends to go on vacation together \label{prinzhornex32a}
\ex   \sem{}{\textit{intend}} = $\lambda w_{s}. \lambda P_{\langle s,\langle e,t \rangle \rangle}. \lambda x_{e}. \forall \langle w' ,y \rangle \in \mathcal{I}_{x,w} (\exists \langle w'', z \rangle (w'' = w' \, \& \, y \le z\, \& \,P(w'')(z)))$ \label{prinzhornex32b}
\ex \sem{}{\textsc{pro} go on vacation together} = $\lambda w_{s}. \lambda x_{e}: \exists y (y < x). x$ goes on vacation together in $w$\label{prinzhornex32c}
\ex   \sem{}{(\ref{prinzhornex32a})} = $\lambda w_{s}. \forall \langle w' ,y \rangle \in \mathcal{I}_{A,w} (\exists \langle w'', z \rangle (w'' = w' \, \& \, y \le z\, \& \,\textbf{go on vacation together} (w'')(z)))$ \\
$[$simplified: $\lambda w. \forall \langle w' ,y \rangle \in \mathcal{I}_{A,w} (\exists z (y \le z\, \& \,\textbf{g.o.v. together} (w')(z)))]$ \label{prinzhornex32d}
\z\z

(\ref{prinzhornex32c}) encodes the intuition that MPs\textsubscript{PC} always involve a relation between the matrix subject and a self-candidate of this subject: It requires that  the property  the EP introduces is attributed to an entity containing a self-candidate of the matrix subject, as captured in H2. The intuitive connection to H1 is that part-predicates somehow access parts of the plurality the subject has in mind that don't contain its self-candidate. We will now spell out this intuition more concretely.


\ea H2: In PC, the property introduced by the EP must apply to a plurality containing a \textsc{self}-candidate of the matrix subject. \label{prinzhornexr2} \z




\subsection{Spelling out the intuition, part II}\label{prinzhornsec:4.2}

In order to explain the EP-restriction via H2, we must show two things: First,  that collective predicates don't cause any problems regarding H2. And second, that part-predicates do. The first point is straightforward: As (\ref{prinzhornhazels}) illustrates, collective predicates can apply to the expansion the subject has in mind \textit{as a whole} -- so  the self-candidate of the subject will  be a part of the plurality that the property (e.g., \textbf{go on vacation together})  is ascribed to. The second point -- showing why part-predicates do \textit{not}  comply with H2 -- is more difficult. Above, we established an \textit{intuitive} link: As part-predicates access the part structure of the expansion the subject has in mind, they might access proper parts of this plurality that \textit{don't} include the self-candidate of the subject. Spelling out this vague intuition in a plausible way will in fact require a non-standard take on plural predication.


\subsubsection{The problem}

Let us first outline why it is hard to spell out this intuition \textit{qua} the traditional accounts of part predicates used in \sectref{prinzhornsec:2}: (\ref{prinzhornex33-b}) repeats the denotation for a 
 predicate with \textit{jeweils/each}, (\ref{prinzhornex33-c}) that for a cumulative predicate.

\ea \label{prinzhornexden}
\ea \sem{}{each buy two pets} = $\lambda x_{e}. \forall y \le_{a} x \,(\textbf{buy  two pets}(y))$\label{prinzhornex33-b}
\ex	 \sem{}{feed the three dogs} = $\lambda x_{e}. \forall y \le_{a} x (\exists z \le_{a} \sem{}{the three dogs} \&\, \textbf{feed}(z)(y)) \&\, \forall z \le_{a} \sem{}{the three dogs} (\exists y \le_{a} x \& \,\textbf{feed}(z)(y))$ \label{prinzhornex33-c}
\z\z

Both predicates denote properties that can hold of pluralities -- just as the collective predicate in (\ref{prinzhornex32c}). For example, the sentence in (\ref{prinzhornex40a}) will involve applying the property in (\ref{prinzhornex33-c}) to the plurality \textbf{A+B+C} as whole and  for the sentence in (\ref{prinzhornex40b}) we apply the property in (\ref{prinzhornex33-b}) to \textbf{A+B+C} as a whole. As opposed to collective predicates, part predicates access the parts of that plurality for their \textit{truth-conditional component}, but this trait is invisible at the level of semantic composition: For purposes of the latter, part predicates and collective predicates “look the same”. 



\ea \label{prinzhornex40}
\ea \textit{Die drei Kinder haben die drei Hunde gefüttert}.\\
`The three children fed the three dogs.' \label{prinzhornex40a}
\ex \textit{Die drei Kinder haben jeweils zwei Tiere gekauft}.\\
`The three children bought two pets each.' \label{prinzhornex40b}
\z\z

The traditional view thus does not let us explain the EP-restriction via H2: If \sem{}{intend} = \sem{}{vorhaben},  (\ref{prinzhornex37a})  (=\,\ref{prinzhornex11b}) will involve applying (\ref{prinzhornex33-c}) to the expansion that Ada has in mind \textit{as a whole} and in (\ref{prinzhornex37b}) (=\,\ref{prinzhornex9b}) we apply (\ref{prinzhornex33-b}) to that expansion, again \textit{as a whole}.\footnote{Note that for purposes of simplicity, and because it has no bearing on our conceptual points, we sometimes use extensional versions of predicates.} Both  predications comply with H2 and we falsely predict that (\ref{prinzhornex37b}) should be fine and that (\ref{prinzhornex37a}) should have a cumulative reading.




\ea  \label{prinzhornpc}
\ea[]{Ada hat vor, die drei Hunde zu füttern\\
	`Ada intends die drei Hunde zu füttern' \jambox*{$\#$ cumulative reading}\label{prinzhornex37a}}
\ex[\#]{Ada hat vor, jeweils zwei Tiere zu kaufen\\
	`Ada intends to buy two pets each' \label{prinzhornex37b}}
	\z\z

But intuitively, there is another way  to conceive of part predicates -- one where we divide the predicates themselves into parts. Take first cumulative predication as in (\ref{prinzhornex40a}): Instead of viewing it as involving a cumulated relation between individuals (which is what underlies the denotation in (\ref{prinzhornex33-b})) we could view it as a cumulative relation between a plurality of individuals and a plurality of predicates with the parts in  (\ref{prinzhornex46a}). Requiring that it “holds cumulatively” of \textbf{A+B+C} would then mean that at least one of these properties must hold of \textbf{A}, at least one of them  of \textbf{B} and at least one of them of \textbf{C} -- and each of the properties must hold of \textbf{A}, \textbf{B} or \textbf{C}. By “dividing up” the predicate in this way, we would  end up with different properties that could apply to different parts of the individual plurality.

\ea  parts of \textit{die drei Hunde füttern} = $\{$ \textbf{feed D, feed E, feed F} $\}$.\label{prinzhornex46a}
\z

Returning to PC in (\ref{prinzhornex37a}), the expansion Ada has in mind -- say, {\bf A+B+C} -- could thus be “divided up” in a way such that each part -- {\bf A}, {\bf B}, {\bf C} -- could be attributed a different property. This would mean that we should run into a problem according to H2: For instance, in scenario (\ref{prinzhornex46b}) ($\widehat{=}$ (\ref{prinzhornex11a})), the property {\bf feed D} would apply to \textbf{A},  {\bf feed E} to \textbf{B}, \textbf{feed F} to \textbf{C}. So for some of these properties, Ada's self-candidate (\textbf{A}) would \textit{not}  be part of the plurality that is assigned that property.

\ea  {\sc scenario: } Adas intentions: A feeds D, B feeds E, C feeds F. \label{prinzhornex46b} \z

For predication with \textit{jeweils /each}, as in (\ref{prinzhornex40b}), the denotation of  the predicate could consist of parts like those in (\ref{prinzhornex46c}) -- each part essentially corresponds to the buying of two pets. (\ref{prinzhornex40b}) would then require “mapping” each part of \textbf{A+B+C} to one of the properties in (\ref{prinzhornex46c}) (e.g.., \textbf{A} to \textbf{buy D+E}, \textbf{B} to \textbf{buy F+G}, etc.). In a PC-context like (\ref{prinzhornex37b}),  the “expansion” the subject has in mind would be divided into parts that can be ascribed different properties -- which H2 would rule out .

\ea 
Parts of \textit{jeweils zwei Tiere kaufen} = $\{$ \textbf{buy D+E, buy D+F, buy E+F,  buy F+G, buy H+I, buy G+I, buy G+I, \dots} $\}$.\label{prinzhornex46c}
\z

\subsubsection{Plural projection: Background}\label{prinzhornsec:4.2.1}

The plural projection system \citep{Schmitt:2019, Haslinger:2018a, Haslinger:2018b} is an independently motivated mechanism for plural predication that allows us to encode these intuitions about “dividing up” predicates properly: We first give a very informal overview of the system,\footnote{For reasons of space, we omit the independent motivation for the mechanism, and proper definitions. Both are discussed at length in \citealt{Schmitt:2019, Haslinger:2018a, Haslinger:2018b}.} and then (again very informally) describe its view of part predicates and collective predicates. For each case, we show that combining these “new” predicate denotations with the semantics of MPs\textsubscript{PC} will give us the correct predictions concerning the EP-restriction if H2 is assumed.



The system  is based on two core ideas (we here follow \cite{Haslinger:2018b}): First, all semantic domains contain pluralities -- we have pluralities of individuals, but also pluralities of predicates, pluralities of propositions etc. For every type $a$, pluralities stand in a one-to-one correspondence with non-empty sets of atoms of $D_{a}$, e.g., the domain $D_{\langle e,t \rangle}$ would look like (\ref{prinzhornex-d}) (“+” indicates plurality formation)\footnote{Note that this means, crucially, that pluralities of predicates of type $\langle a,t \rangle$ are not reducible to predicates of  pluralities of type $a$.}: 



\ea	 $\{ \lambda x. $smoke$(x), \lambda x. $drink$(x), \lambda x. $smoke$(x) + \lambda x.  $drink$(x), \dots\}$ \label{prinzhornex-d} \z

Second, semantic plurality “projects” up in the syntactic tree (in analogy to versions of alternative semantics, e.g., \citealt{Rooth:1985}):
 Any constituent containing a semantically plural subexpression will itself be semantically plural, unless “projection”  is blocked by an intervening operator: e.g., as the VP in (\ref{prinzhornex50}) contains the semantically plural expression \textit{the three dogs} it will itself be semantically plural.
 
 \ea   \textit{fed the three dogs} \label{prinzhornex50}
  \z

This “projection” behavior is implemented by a special composition rule (“CC” for “cumulative composition”). We start with the idea that if a function combines with an argument plurality, or a function plurality  with an argument, we obtain a plurality of values. In the former case, the function applies to each atomic part of the argument and the resulting values are summed up. In the latter case, each atomic part of the function applies to the argument and the resulting values are summed up.  (\ref{prinzhornpp1}) shows that the mereological structure introduced by the embedded plural expression is preserved in the denotation of the node dominating it.

\ea\begin{forest}
   [,phantom
      [{$f(a)+f(b)$} 
         [$f$]  
         [{$a+b$}]
      ]   
      [{$f(a)+g(a)$} 
         [{$f+g$}]   
         [{$a$}]
      ]
   ] \label{prinzhornpp1} 
   \end{forest}\z 

We need a slightly more complex system to generalize this idea, since if both the functor and the argument denote pluralities, a single plurality of values will be insufficient: Cumulative truth conditions (see (\ref{prinzhornmyex})) are compatible with various ways of matching up the functor-parts and the argument-parts. Plural expressions are thus assumed to denote sets of pluralities --  \textbf{plural sets} (indicated here by square brackets) -- rather than single pluralities. These plural sets can be targeted by specific compositional rules: They form the input for the CC-rule that yields  “projection” and also encodes cumulativity. This rule combines a set of function pluralities and a set of argument pluralities as follows: It returns the set of all value pluralities obtained by applying atomic function parts to atomic argument parts in such a way that all the parts of some plurality in the function set are “covered”, and all the parts of some plurality in the argument set are “covered”.\footnote{For instance, $\{ \langle f,a \rangle, \langle g,b \rangle \}$ would be a cover for the two pluralities $f+g$, $a+b$, but , $\{ \langle f,a \rangle, \langle g,a \rangle \}$ wouldn't be a cover, because we are “missing” a part of the $a+b$-plurality.} This is schematized in (\ref{prinzhornpp2}). As in (\ref{prinzhornpp1}),  the denotation of the mother node preserves the part structure introduced by the plural expressions it dominates.  ((\ref{prinzhornpp1}) is what we obtain if one of the two plural sets is a singleton containing a non-plural denotation.) Crucially, this operation is repeated at any syntactic node that dominates at least one plural expression. So sentences containing plurality-denoting expressions denote plural sets of propositions: They count as true if at least one plurality in the set consists exclusively of true propositions. Accordingly, if (\ref{prinzhornpp2}) were the representation of a full sentence, the top level plural set would be a plural set of propositional pluralities and (\ref{prinzhornpp2}) would be mapped to true if at least one of the elements in the set were such that all of its atoms were true, e.g., if both $f(a)$ and $g(b)$ were true, or if both $f(b)$ and $g(a)$ were true etc.


\ea \begin{forest}
   [{$[f(a)+g(b), f(b)+g(a), f(a)+g(a)+g(b), f(b)+g(a)+g(b),$} \\ 
    {$f(a)+f(b)+g(a), f(a)+f(b)+g(b), f(a)+f(b)+g(a)+g(b)]$}
      [{$[f+g]$}] [{$[a+b]$}]
   ]
   \end{forest}
\label{prinzhornpp2} 
\z


Based on this rough sketch, we will now consider the denotations of part-predicates and collective-predicates in light of H2.

\subsubsection{Cumulative predicates}\label{prinzhornsec:cum}  
\citet{Haslinger:2018a, Haslinger:2018b}, assume that plural definites and indefinites denote plural sets containing individuals from the NP-extension: The plural set containing the sum of all such elements, (\ref{prinzhornex51a}), and the one containing all such pluralities of the “right size”, (\ref{prinzhornex51b}), respectively.


\ea 
\ea \sem{}{the three dogs} = $[ \textbf{D+E+F} ]$\label{prinzhornex51a}
\ex \sem{}{two pets} = $[ \textbf{D+E, D+F, E+F, G+H, G+I, H+I, D+G,  \dots} ]$\label{prinzhornex51b}
\z\z

We now derive the denotation of (\ref{prinzhornex50}) (cf. (\ref{prinzhornex40a})) on the basis of (\ref{prinzhornex51a}), the CC-rule illustrated in (\ref{prinzhornpp2}) (represented by “$\bullet$”) , and the assumption that the denotations of expressions that don't contain plural expressions like (\textit{feed}) can be mapped to singleton plural sets (i.e., \textbf{[feed]}). This yields the plural set of predicates in (\ref{prinzhornex52}): As the only element of [\textbf{feed}] is atomic, it applies to each atom of the only element of  $[ \textbf{D+E+F} ]$; we obtain a plural set containing the sum of all the resulting values. 

\ea \sem{}{feed the three dogs} = \textbf{[f(eed)]} $\bullet$ [ \textbf{D+E+F} ] =  [ \textbf{f(D) + f(E) + f(F)} ]\label{prinzhornex52}
\z

If we combine this plural set with a plural subject as in (\ref{prinzhornex40a}), the result will be a plural set of propositions: We consider all the possible “covers” of the plurality in the plural set introduced by the subject and the plurality contained in the plural set in (\ref{prinzhornex52}); for each such “cover” we sum up the results of applying the functor-part to its respective argument part; we then collect the results in a plural set. The sentence will be true iff at least one of the pluralities in this set consists only of true atoms, e.g., in a scenario where Ada fed Dean, Bea fed Eric and Carl fed Fay.

\ea \sem{}{The three children fed the three dogs} = \textbf{[A+B+C]} $\bullet$  [ \textbf{f(D) + f(E) + f(F)} ] =  [ \textbf{f(D)(A) + f(E)(B) + f(F)(C)}, \textbf{f(D)(B) + f(E)(A) + f(F)(C)},  \textbf{f(D)(C) + f(E)(B) + f(F)(A)}, \dots ]
\label{prinzhornex53}
\z

This is exactly the type of system needed to spell out our intuitions behind H2: Cumulative predication now involves applying parts of the predicate (obtained by “projection” of an embedded plurality) to parts of the subject. Let's see how this plays out in PC, i.e., (\ref{prinzhornex37a}) with the LF in (\ref{prinzhornex60}).\footnote{We don't want to discuss the  plural projection version of binding, \citep{Haslinger:2020d}, so we assume that \textsc{pro} is semantically vacuous and semantically inert, i.e., we do not assume here (\textit{contra} \citealt{Heim:1998}) that it induces abstraction over a variable.} 

\ea  \label{prinzhornex60} [$_{S3}$ Ada [$_{S2}$ hat vor [$_{S1}$ \sout{\textsc{pro}} drei Hunde zu füttern ]]]  \z

We assume that \sem{}{vorhaben} = \sem{}{intend} (see (\ref{prinzhornex32b})). In order to derive \sem{}{$S2$},  \textbf{[intend]} combines with the denotation of the embedded predicate (i.e., (\ref{prinzhornex52})) as in (\ref{prinzhornex62a}) via the CC-rule: \textbf{intend} applies to every atomic part of the predicate plurality, which yields another plural set of predicates.\footnote{This requires an expansion of the CC-rule to  intensions, \citep{Schmitt:2019a}.} For \sem{}{$S3$}, we combine this latter set-- again via the CC-rule -- with the plural set introduced by the subject, (\ref{prinzhornex62b}). We obtain a plural set containing a single plurality of propositions. In order for the sentence to be true, each atom of this plurality must be true -- and each atom involves one property (e.g, {\bf feed Dean}), which, in all of Ada's world-candidates $w$, must hold of an expansion of Ada's self candidate in $w$.




\ea
\ea  \sem{}{S2} = \textbf{[intend]} $\bullet$ \textbf{[f(D) + f(E) + f(F)]} = 
 [$\lambda w. \lambda x. \forall \langle w' ,y\rangle \in \mathcal{I}_{x,w} (\exists  z \,(y \le z\, \& \,\textbf{f(D)}(w')(z)))$ +
 $\lambda w. \lambda x. \forall \langle w' ,y\rangle \in \mathcal{I}_{x,w} (\exists  z \,(y \le z\, \& \,\textbf{f(E)}(w')(z)))$ +
 $\lambda w. \lambda x. \forall \langle w' ,y \rangle \in \mathcal{I}_{x,w} (\exists  z \,(y \le z\, \& \,\textbf{f(F)}(w')(z)))$]
\label{prinzhornex62a}
\ex \sem{}{S3} = \textbf{[Ada]} $\bullet$ \sem{}{S2} = \\
 $[\lambda w.  \forall \langle w' ,y \rangle \in \mathcal{I}_{A,w} (\exists  z \,(y \le z\, \& \,\textbf{f(D)}(w')(z)))$ +
 $\lambda w.  \forall \langle w' ,y\rangle \in \mathcal{I}_{A,w} (\exists  z \,(y \le z\, \& \,\textbf{f(E)}(w')(z)))$ +
 $\lambda w.  \forall \langle w' ,y \rangle \in \mathcal{I}_{A,w} (\exists  z \,(y \le z\, \& \,\textbf{f(F)}(w')(z)))$ ]
\label{prinzhornex62b}
\z\z



Note that this analysis doesn't assume that cumulation predication in PC is ruled out \textit{by the grammar}. Rather, if the EP contains a plurality (e.g., \textit{fed the three dogs}), the denotation of the MP and of the EP will conspire: For each of the predicate's atomic parts,  the matrix subject's self-candidate must be part of the individual that this property applies to. (\ref{prinzhornex32b}) is thus correctly predicted false in scenario (\ref{prinzhornex46b}). Now, if we assumed that the atomic properties (like \textbf{feed Dean}) can also hold of pluralities collectively (see \sectref{prinzhornsec:coll}), then  two types of scenarios that should make (\ref{prinzhornex32b}) true: Those where Ada's intension is that she herself feeds each dog, and those where Ada's intension is that for each dog, she and possibly other people feed that dog together. We believe this prediction to be correct.








\subsubsection{Distributive predicates with \textit{jeweils/each}}\label{prinzhornsec:dist}  Let's turn to distributive predication with \textit{jeweils/each}, (\ref{prinzhornex40b}). We first consider the denotation of \textit{zwei Tiere kaufen}. As the object is indefinite, the CC-rule requires the following: We pick a plurality from the set in (\ref{prinzhornex51b}),  \textbf{buy} must apply to both of its atomic parts, then we sum up the results. We do this for each plurality of two pets and collect the results in a plural set, (\ref{prinzhornex55}).

\ea \sem{}{buy two pets} = $\textbf[\textbf{b(uy)}] \bullet [ \textbf{D+E, D+F, E+F, G+H, G+I, H+I, D+G,  \dots} ]$\\ = 
$[$\textbf{b(D)+b(E), \dots, b(E)+b(F), b(G)+b(H), \dots, b(H)+b(I), b(D)+b(G), \dots} $]$\label{prinzhornex55} \z

This plural set combines with \sem{}{each} (= \sem{}{jeweils}). \cite{Haslinger:2018b} argue that such distributive elements are operators that block application of the CC-rule: They directly manipulate plural sets by  taking such sets as their arguments, and  yield plural sets of their values. \sem{}{each} takes the plural set of predicates in (\ref{prinzhornex55}). It returns another plural set of predicates, (\ref{prinzhornex56}), whose elements each consist of atomic functions  mapping an individual to some predicate plurality in the argument set (e.g., one such atom is the function $\lambda x.\textbf{buy D}(x)+\textbf{buy E}(x)$).


\ea  \sem{}{each buy two pets} = $[\lambda x_{e}. \textbf{b(D)}(x)+ \textbf{b(E)}(x), (\lambda x_{e}.\textbf{b(D)}(x)+\textbf{b(E)}(x))+ (\lambda x_{e}.\textbf{b(E)}(x) +\textbf{b(F)}(x)),  (\lambda x_{e}.\textbf{b(D)}(x)+\textbf{b(G)}(x))+(\lambda x_{e}.\textbf{b(E)}(x)+\textbf{b(F)}(x)) + (\lambda x_{e}. \textbf{b(H)}(x)+\textbf{b(I)}(x)), \dots]$ \label{prinzhornex56} \z

(\ref{prinzhornex56}) is again a plural set of predicate pluralities, so it combines with the denotation of the subject in (\ref{prinzhornex40b}), \textbf{[A+B+C]}, via the CC-rule: For each plurality in (\ref{prinzhornex56}), we form all the possible “covers” relative to \textbf{A+B+C}, and then, for each such cover, we let the functions apply to their respective arguments and sum up the results for this cover. We do this for each cover of each plurality in (\ref{prinzhornex56}) and \textbf{A+B+C}, which yields the plural set of propositions in (\ref{prinzhornex57}). The sentence will be true iff one of the pluralities  in (\ref{prinzhornex57}) consists only of true atoms, e.g., in a scenario where Ada bought Dean and Gene, Bea bought Eric and Fay, and Carl bought Harry and Ivo.


\ea  \sem{}{(\ref{prinzhornex40b})} = (\ref{prinzhornex56}) $\bullet$ \textbf{[A+B+C]} = $[\textbf{b(D)(A)}+\textbf{b(E)(A)}+\textbf{b(D)(B)} +\textbf{b(E)(B)} + \textbf{b(D)(C)} +\textbf{b(E)(C)}, 
\textbf{b(D)(A)}+\textbf{b(E)(A)}+\textbf{b(E)(B)} +\textbf{b(F)(B)} + \textbf{b(E)(C)} +\textbf{b(F)(C)}, 
  \textbf{b(D)(A)}+\textbf{b(G)(A)}+ \textbf{b(E)(B)}+\textbf{b(F)(B)} +  \textbf{b(H)(C)}+\textbf{b(I)(C)}, \dots]$ \label{prinzhornex57} \z


Again, we have “divided up” a predicate into parts (in our case, each part is a property of buying to particular pets). Let's consider the consequences for PC, i.e. example (\ref{prinzhornex37b}) 
with the LF in (\ref{prinzhornex63a}). The derivation of node \sem{}{$S2'$} involves combining the singleton plural set \textbf{[intend]} via the CC-rule with (\ref{prinzhornex56}) -- and combining this with the matrix subject yields the plural set of propositions sketched in (\ref{prinzhornex63b}). 

\ea
\ea  \label{prinzhornex63a} [$_{S3'}$ Ada [$_{S2'}$ hat vor [$_{S1'}$ \sout{\textsc{pro}} jeweils zwei Tiere zu kaufen]]].
\ex \sem{}{S3'} = $[\lambda w.  \forall \langle w' ,y \rangle \in \mathcal{I}_{A,w} (\exists  z \,(y \le z\, \& \,\textbf{b(D)}(w')(z)+\textbf{b(E)}(w')(z))),$\\$
\lambda w.  \forall \langle w' ,y \rangle \in \mathcal{I}_{A,w} (\exists  z \,(y \le z\, \& \,\textbf{b(D)}(w')(z)+\textbf{b(E)}(w')(z))) $\\$
+ \lambda w.  \forall \langle w' ,y \rangle \in \mathcal{I}_{A,w} (\exists  z \,(y \le z\, \& \,\textbf{b(E)}(w')(z)+\textbf{b(F)}(w')(z))), $\\$
\lambda w.  \forall \langle w' ,y \rangle \in \mathcal{I}_{A,w} (\exists z\,( y \le z\, \& \,\textbf{b(D)}(w')(z) + \textbf{b(G)}(w')(z)))$\\$ + \lambda w.  \forall \langle w' ,y \rangle \in \mathcal{I}_{A,w} (\exists z \, (y \le z\,\& \,\textbf{b(E)}(w')(z)+\textbf{b(F)}(w')(z))) $\\$ + \lambda w.  \forall \langle w' ,y \rangle \in \mathcal{I}_{A,w} (\exists z \,(y \le z\, \& \,\textbf{b(H)}(w')(z)+\textbf{b(I)}(w')(z))), \dots]$\label{prinzhornex63b}
\z\z

Look at the propositional pluralities: Each part of the subject is mapped  to the buying of \textit{two} pets (e.g., {\bf D} and {\bf E}) -- and each such part must be an expansion of Ada's self-candidate. Now, crucially, if the properties can only hold of atomic individuals (which would mean, essentially, that the variable $z$ above is restricted to atomic individuals),\footnote{\cite{Schwarzschild:1996} notes that VP-\textit{each} can sometimes distribute to non-atomic elements. We leave this issue to future research.} then Ada's intention will be that she herself buys two (or more) pets. The sentence is thus interpretable in our system, so why is it unacceptable? Our account makes (\ref{prinzhornex37b}) parallel to (\ref{prinzhornex80}) -- so whatever explains the unacceptability of (\ref{prinzhornex80}) (which can't be syntactic, see \sectref{prinzhornsec:3}) should carry over to (\ref{prinzhornex37b}).




\ea 
\ea[]{\textsc{scenario:} Ada bought two pets. That's all she did.}
\ex[\#]{\gll {Ada} {hat} {jeweils} {zwei} {Tiere} {gekauft}.\\
	Ada has each two animals bought.\\
	\glt `Ada bought two animals each.' \label{prinzhornex80}}
	\z\z


\subsubsection{Collective predicates}\label{prinzhornsec:coll}  The current version of the plural projection system omits collective predicates, so we add the assumption that predicate denotations specify whether they apply to atomic individuals, proper pluralities of individuals, or both. This specification can be a property of lexical predicates, as in (\ref{prinzhornex71}), but modifiers like \textit{together} can manipulate the specification of complex predicates (e.g., restrict them to proper pluralities).\footnote{We leave a compositional treatment of such complex expressions (see e.g., \citealt{Lasersohn:1990}) and the question how to combine it with the plural projection mechanism to future research. We furthermore omit a more thorough discussion of the semantics of collective predication (see e.g. \citealt{Landman:2000}).} To implement this, we loosen the restrictions on the “covers” via which we match functor-parts with argument parts. So far, we matched atomic functor-parts with atomic argument parts: For a functor plurality $f+g$ and an argument plurality $a+b$, we would for instance get the cover $\{ \langle f,a \rangle, \langle g,b \rangle\}$, but not the cover $\{ \langle f, a+b \rangle, \langle g, a \rangle \}$. Such atomic covers are suitable for predicates like \textit{smoke}, but in order to include collective predicates like \textit{meet}, we must allow covers matching functor parts with \textit{non-atomic} argument-parts, as in (\ref{prinzhornpp4}).\footnote{This is a simplification: We should always allow every possible cover. So for predicates with a domain restriction, not all of the covers are suitable -- which raises the general question of how to deal with cases where some members of a plural set will involve a presupposition failure. }

\ea \label{prinzhornex71}
\ea \sem{}{smoke} = $\lambda x: \forall y \le x (y =x).\,$\textbf{smoke}$(x) \label{prinzhornex71a}$
\ex  \sem{}{meet} = $\lambda x:  \exists y (y < x) .\,$\textbf{meet}$(x)$ \label{prinzhornex71b}
\z\z

\ea \begin{forest}
    [,phantom
    [{$[meet(a+b)]]$}
        [{$[meet]$}]   
        [{$[a+b]$}] 
     ]
     [{$[meet(a+b), meet(a+c), meet(b+c), meet (a+b+c)]$}
      [{$[meet]$}]   
      [{$[a+b+c]$}] 
     ]]  \label{prinzhornpp4} 
    \end{forest}
    \todo[inline]{Please verify these trees}
 \z

Accordingly, a sentence like (\ref{prinzhorncoll-pred}) above, with the LF in (\ref{prinzhornex72a}) comes out as in (\ref{prinzhornex72b}): We obtain a plural set of propositions, the only element of which is an atomic proposition. Accordingly, the sentence is true, for instance, in a scenario where Ada's intentions are that she, Bea and Carl meet in the park.

\ea
\ea   \label{prinzhornex72a} [$_{S3'}$ Ada [$_{S2'}$ hat vor [$_{S1'}$ \sout{\textsc{pro}} sich auf dem Platz zu treffen]]]
\ex \sem{}{S3'} =  $[\lambda w.  \forall \langle w' ,y \rangle \in \mathcal{I}_{A,w} (\exists z \, (y \le z\, \& \,\textbf{meet}(w')(z)))]$\label{prinzhornex72b}
\z\z

As \textit{meet in the square} doesn't contain any semantically plural expressions or a distributivity operator, it doesn't denote a predicate plurality -- so as opposed to our part-predicates, it doesn't have different parts that could apply different parts of its argument plurality. But this property is independent of it being a collective predicate: If \textit{meet} first combines with a semantically plural expression,  like \textit{in den zwei Parks} in (\ref{prinzhornex73a}), the result will be a predicate with a part structure (i.e., \textbf{[meet in park 1 + meet in park 2]}). The only effect of collectivity is that each of the parts of this predicate plurality can only hold of proper pluralities. We  thus correctly predict the sentence in (\ref{prinzhornex73a}) to be true in the scenario in (\ref{prinzhornex73b}), where each predicate part holds of a different sub-plurality of the argument (see e.g. \citealt{Schwarzschild:1996} for related discussion). So this is where we should observe an effect for PC: Each part of the predicate should  be required to hold of a plurality containing a self-candidate of the matrix subject. This  seems to be the case: The sentence in (\ref{prinzhornex73c}) is false in scenario (\ref{prinzhornex73d}) where this constraint is violated. (On the other hand, we would predict the sentence to be true in a scenario where Ada's intentions are: Ada and Bea meet in park 1, Ada and Carl meet in park 2. It seems to us that these predictions are correct.)

\ea 
\ea \textsc{scenario}  D, E F meet in park 1, G, H, I in  park 2.  \label{prinzhornex73b}
\ex \gll \textit{Die} \textit{Tiere} \textit{treffen} \textit{sich} \textit{morgen} \textit{in} \textit{den} \textit{zwei} \textit{Parks}.\\
The animals meet \textsc{refl} tomorrow in the two parks.\\
\glt `The animals will meet in the two parks tomorrow.'  \phantom{.}\hfill \textbf{true} in (\ref{prinzhornex73b})\label{prinzhornex73a}
\ex Ada's intentions: A, B meet in park 1. B,C meet in park 2. \label{prinzhornex73d}
\ex \gll \textit{Ada} \textit{hat} \textit{vor}, \textit{sich} \textit{morgen} \textit{in} \textit{den} \textit{zwei} \textit{Parks} \textit{zu} \textit{treffen}.\\
Ada intends {} \textsc{refl} tomorrow in the two parks to meet.\\
\glt `Ada intends to meet in the two parks tomorrow.' \label{prinzhornex73c} \phantom{.}\hfill \textbf{false} in (\ref{prinzhornex73d})
\z\z


\subsection{Interim Summary}

In sum, our story for the EP-restriction boils down to the following: Predicates that themselves embed semantically plural expressions or involve overt distributivity operators are “divided up” into different parts. The semantics of MPs\textsubscript{PC}, in turn, require its property argument (the denotation of the embedded clause) to hold of individuals that a self-candidate of the matrix subject is a part of. Accordingly, if a predicate is divided up into parts, all these parts must meet this requirement. The effect of this is that cumulative readings become “invisible” and that the use of overt distributivity operators is odd because we are attributing properties to a semantically singular self-candidate of the subject. Collective predicates are fine as long as all their parts (in those cases where they exhibit a part-structure) apply to pluralities that contain a self candidate of the subject.\footnote{Event-based analyses of plural predication (e.g. \citealt{Kratzer:2000}) would blur the distinction between predicates with and those without a part-structure: We would always relate an individual plurality to an event with a part-structure. (Thanks to Nina Haslinger (pc) for this point.)}

\section{Discussion and outlook}\label{prinzhornsec:5}

We gave a semantic description of the EP-restriction in German PC and argued that it follows from two assumptions: (i) MPs\textsubscript{PC} involve an attitude \textit{de se}, \citep{Pearson:2016}, so the property denoted by the EP must hold of pluralities that include a self-candidate of the matrix subject. (ii) Predicates can denote (sets of) pluralities of properties, and the parts of these pluralities can apply to different parts of their semantically plural arguments, \citep{Haslinger:2018a,Haslinger:2018b}.

This sketch leaves much room for further research (apart from the technical issues mentioned above). On the one hand, we took an overly simple view of collective predicates, treating them as a homogenous class. Yet, they are known to exhibit a divergent behavior in various respects (e.g. \citealt{Dowty:1986, Landman:2000}) and it is unclear, both empirically and analytically, how this is reflected in PC (and whether the behavior of collective predicts PC can inform us about  semantic differences between them).
On the other hand,  we used a very simplified semantics of MPs\textsubscript{PC}. We omitted a crucial feature of the proposal by \citet{Pearson:2016}: She actually 
 argues \textit{two} properties are correlated in  MPs\textsubscript{PC}: An expansion of the attitude subject's self-candidate, and an expansion of her “now-candidates” in her world-candidates. Pearson's lexical entry for a MP\textsubscript{PC} $C$ thus looks like (\ref{prinzhornhazel}) (where \textbf{C} is the set of triples of C-accessible worlds, times and individuals):

\ea \sem{}{C} = $\lambda P_{\langle s, \langle e, \langle i,t \rangle \rangle \rangle}. \lambda x_{e}. \lambda t_{i}. \lambda w_{s}. \forall \langle w',t',y \rangle \in \textbf{C}_{x,w,t} \rightarrow (\exists\, \langle w'', t'', z \rangle (w'' = w \, \& \, y \le z \& (t''$ precedes $t'$ or $t'$ precedes $t'') \& P(w'')(z)(t'')))]$
 \label{prinzhornhazel}\z  
 
In light of our approach, this raises the question whether the EP-restriction expands to predicates that access the part-structure of the temporal intervals expanding the subject's “now-candidates”. At first sight, this doesn't seem to be the case:   (\ref{prinzhornexfn1+2}), where \textit{jeweils} distributes over events, is fine. We leave this matter to future work -- just as the more general question  whether different MPs\textsubscript{PC} vary with respect to the EP-restriction and, if so, to what extent this variation correlates with independently attested semantic differences between them.
 
\ea \label{prinzhornexfn1+2}
\ea \label{prinzhornexfn1} {\sc context: } Ada wants to go to three different pet stores.
\ex \gll {Ada} {hat vor}, {jeweils} {genau} {zwei} {Tiere} {zu} {kaufen}.\\
   Ada intends {} each exactly two pets to buy \\
\glt `Ada intends to buy exactly two pets at each pet store visit.' \label{prinzhornexfn2}
\z \z




\section*{Acknowledgements}
\begin{sloppypar}
We thank Nina Haslinger, Magdalena Roszkowski, Tim Stowell and Valerie Wurm for discussion, comments and judgements, and two anonymous reviewers for very helpful comments and suggestions. We furthermore thank the students in Martin's class on modality for their judgements. Viola Schmitt's research was funded by the Austria Science Fund (FWF), project  P\,32939 (“The typology of cumulativity”). Cheers to Susi!
\end{sloppypar}

\printbibliography[heading=subbibliography,notkeyword=this]

\end{document}
