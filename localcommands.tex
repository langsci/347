\newcommand{\orcid}[1]{}

\newcommand{\LaConcept}[1]{\textsc{\oldstylenums{\MakeLowercase{#1}}}}

%other stuff
\newunicodechar{⟶}{{\symbolfont{⟶}}}
%\newunicodechar{✓}{{\Checkmark}} % checkmark
\newunicodechar{✓}{{\ding{51}}} % checkmark
\newcommand{\gap}{\ \underline{\hspace{10pt}\phantom{x}}\ }
\newcommand{\ix}[1]{\textsubscript{\it{#1}}}  % subscript

\newcommand{\cmark}{\ding{51}}%
\newcommand{\xmark}{\ding{55}}%

\def\gethyperref#1{\hyperlink{#1}{\getref{#1}}}%
\def\getfullhyperref#1{\hyperlink{#1}{\getfullref{#1}}}%

\newcommand{\denote}[1]{⟦{#1}⟧}
\newcommand{\sem}[2]{\mbox{\denote{#2}$^{#1}$}}

\newcommand{\hardArt}{\textsc{art}}
\newcommand{\hardGen}{\textsc{gen}}
\newcommand{\hardNum}{\textsc{num}}
\newcommand{\hardAdj}{\textsc{adj}}
\newcommand{\hardN}{\textsc{n}}
\newcommand{\hardProp}{\textsc{prop}}
\newcommand{\hardWk}{\textsc{wk}}
\newcommand{\hardStr}{\textsc{str}}

%\newcommand*\circled[1]{\tikz[baseline=(char.base)]{
%            \node[shape=circle,draw,inner sep=2pt] (char) {#1};}}%to circle text

% Newell

\newcommand{\rotatecharone}[1]{\rotatebox[origin=c]{180}{#1}}

% Laszakovits et al. 
\newcommand{\citeapos}[1]{\citeauthor{#1}'s \citeyearpar{#1}}
\newcommand{\laszAsp}[0]{[asp]}
\newcommand{\laszLoc}[1]{[loc:\ensuremath{#1}]}
\newcommand{\laszDir}[2]{[dir:\ensuremath{#1}$\to$\ensuremath{#2}]}
\newcommand{\laszHs}[1]{[hs:#1]}
\newcommand{\laszClaw}[0]{{\fontspec{aslfontgithubio.ttf} S}}
\newcommand{\laszPlain}[0]{[plain]}
\newcommand{\laszTabColA}[0]{\cellcolor{gray!30}}
\newcommand{\laszTabColB}[0]{\cellcolor{gray!10}}
\newcommand{\laszTabColC}[0]{\cellcolor{gray!30}}
% \newcommand{\laszTabRowA}[0]{\rowcolor{gray!30}}
% \newcommand{\laszTabRowB}[0]{\rowcolor{gray!10}}
\newcommand{\laszTabRowA}[0]{}
\newcommand{\laszTabRowB}[0]{}
\newcommand{\laszLB}[1]{[\textsubscript{#1}}
\newcommand{\lasztO}[0]{t\textsubscript{O}}
\newcommand{\lasztV}[0]{t\textsubscript{V}}

\setlength{\footexindent}{\parindent}


% First, we need counters to hold temporary values:
\newcounter{nexttmp}    % Counter for the Next example
\newcounter{nnexttmp}   % Counter for the NNext example
\newcounter{lasttmp}    % Counter for the Last example
\newcounter{llasttmp}   % Counter for the LLast example
% These replicate the linguex commands. I can't promise they won't break things.
\newcommand{\Next}{\setcounter{nexttmp}{\value{equation}}\stepcounter{nexttmp}(\thenexttmp)\xspace}
\newcommand{\NNext}{\setcounter{nnexttmp}{\value{equation}}\addtocounter{nnexttmp}{2}(\thennexttmp)\xspace}
\newcommand{\Last}{\setcounter{lasttmp}{\value{equation}}(\thelasttmp)\xspace}
\newcommand{\LLast}{\setcounter{llasttmp}{\value{equation}}\addtocounter{llasttmp}{-1}(\thellasttmp)\xspace}
